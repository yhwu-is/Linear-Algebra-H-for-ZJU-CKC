\section{2024-2025学年线性代数I(H)期中}

\begin{center}
    任课老师:吴志祥\hspace{4em} 考试时长:120分钟

    考试时间:2024年11月8日(星期五)19:00-21:00
\end{center}

\begin{enumerate}
    \item (10 分)解下列线性方程组:
          \[ \begin{cases}
                x_1 + 2x_2 - x_3 + 3x_4 +x_5 = 2     \\
                2x_1 + 4x_2 - 2x_3 + 6x_4 + 3x_5 = 6 \\
                -x_1 - 2x_2 + x_3 - x_4 + 3x_5 = 4
            \end{cases} \]

    \item (10 分)已知线性空间\(V\)中\(m > 1\)个向量\(\alpha_1,\alpha_2,\ldots,\alpha_{m}\)线性相关. 证明:存在\(m\)个不全为\(0\)的数\(c_1,c_2,\ldots,c_{m}\), 使对\(V\)中任一向量\(\beta\),向量组\(\{ \beta,\alpha_1 + c_1\beta,\ldots,\alpha_{m} + c_{m}\beta \}\)都线性相关.

    \item (10 分)设\(a \in \mathbf{F}\)为常数, 证明:
          \[\{ f_1 = ( a^{n - 1},a^{n - 2},\ldots,a,1 ),f_2 = ( a^{n - 2},a^{n - 3},\ldots,1,0 ),\ldots,f_{n} = (1,0,\ldots,0,0)\}\]
          是\(\mathbf{F}^{n}\)中的一组基, 并求向量\(\alpha = ( a_1,a_2,\ldots,a_{n} )\)在这组基下的坐标.

    \item (10 分)设
          \begin{align*}
              W_1 &= \{ ( x_1,x_2,\ldots,x_{n} ) \in  \mathbf{R}^{n} \mid x_1 = x_2 = \cdots = x_{n} \}, \\
              W_2 &= \{ ( x_1,x_2,\ldots,x_{n} ) \in  \mathbf{R}^{n} \mid x_1 + x_2 + \cdots + x_{n} = 0 \}.
          \end{align*}
          证明:\(\mathbf{R}^{n} = W_1 \oplus W_2\).

    \item (10 分)设\(\mathbf{R}^3\)中两个子空间为
          \begin{align*}
              W_1 &= \{ ( x_1,x_2,x_3 ) \in \mathbf{R}^3 \mid x_1 + x_2 + x_3 = 0,x_1 - x_2 + x_3 = 0 \}, \\
              W_2 &= \spa( a_1,a_2,a_3 ),a_1 = (1,2,1),a_2 = (1,1, - 1),a_3 = (1,3,3),
          \end{align*}
          \begin{enumerate}
              \item 求\(W_1\)的一组基, 并将其扩充为\(\mathbf{R}^3\)中的基;
              \item 分别求出\(W_1 + W_2\)以及\(W_1 \cap W_2\)的维数及其一组基.
          \end{enumerate}

    \item (10 分)
        \begin{enumerate}
              \item 设\(a_1 = (1,0,0,0,1),a_2 = (1, - 1,0,1,0),\alpha_3 = (2,1,1,0,0) \in \mathbf{R}^{5}\), 用
                    Schmidt 正交化的方法将\(\alpha_1,\alpha_2,\alpha_3\)化为
                    \(\mathbf{R}^3\)的一组单位正交基;

                    \item 若\(W_1\)是\(n\)维欧氏空间\(V( \mathbf{R} )\)的一个子空间, 证明:
                    \[W_2 = \{ \alpha \mid \alpha \in V,\alpha\perp W_1 \}\]
                    是\(W_1\)的正交补, 即\(W_1\perp W_2\), 且\(W_1 \oplus W_2 = V\).
          \end{enumerate}

    \item (10 分)设\(V\)是实数域上二阶矩阵关于矩阵加法和数乘组成的实数域上的线性空间, 定义\(V\)上的变换为
          \[\varphi(A) =
              \begin{pmatrix}
                  1 & 1 \\
                  1 & 1
              \end{pmatrix} A \begin{pmatrix}
                  2 & 0 \\
                  0 & 1
              \end{pmatrix} = \begin{pmatrix}
                  2(a + c) & b + d \\
                  2(a + c) & b + d
              \end{pmatrix},\ A = \begin{pmatrix}
                  a & b \\
                  c & d
              \end{pmatrix},\]
          证明 \(\varphi\)是\(V\)上的线性变换, 并求\(\varphi\)的秩和核空间\(\ker\varphi\)的维数.


    \item (10 分)设\(\sigma,\tau \in \mathcal{L}(V,V),\sigma^2 = \sigma,\tau^2 = \tau\). 证明:\((\sigma + \tau)^2 = \sigma + \tau\)的充分必要条件是\(\sigma\tau = \tau\sigma = 0\).

    \item (20 分)下列命题是否正确?若正确, 请证明;若错误, 请给出反例或证明.
          \begin{enumerate}
              \item 设\(V_1,V_2,V_3\)是\(V\)的子空间, 则\(V_1 \cap ( V_2 + V_3 ) = V_1 \cap V_2 + V_1 \cap V_3\).

              \item 设向量\(\beta\)可以由\(\alpha_1,\alpha_2,\ldots,\alpha_{m}\)线性表示, 但不能由其中任何一个个数小于\(m\)的部分向量线性表示, 则这\(m\)个向量线性无关.

              \item 若\(\sigma\colon V \to W\)是线性映射, 且\(\dim W > \dim W\), 则\(\ker\sigma \neq \{ 0 \}\).

              \item 存在线性映射\(\sigma\colon \mathbf{R}^3 \to \mathbf{R}^2\), 使得\(\sigma(1, - 1,1) = (1,0),\enspace \sigma(1,1,1) = (0,1)\).
          \end{enumerate}
\end{enumerate}

\clearpage
