\section{2021-2022学年线性代数I(H)期末}

\begin{center}
    任课老师:统一命卷\hspace{4em} 考试时长:120分钟
\end{center}

\begin{enumerate}
    \item (12分)定义实数域上的线性空间$\mathbf{R}^n$到自身的映射$T$如下:
    \[\forall X=(x_1,x_2,\ldots,x_n)\in\mathbf{R}^n,\enspace T(X)=(x_1-x_2,x_2-x_3,\ldots,x_{n-1}-x_n,x_n-x_1).\]
    \begin{enumerate}
        \item 验证$T\in\mathcal{L}(\mathbf{R}^n)$;

        \item 求$T$的像空间,和$T$核空间的维数.
    \end{enumerate}

    \item (12分)设
    \[A=\begin{pmatrix}
        1 & 1 & -1 & -1 \\ 2 & 2 & 1 & 0 \\ 3 & 3 & 0 & -1 \\ 1 & 1 & 2 & 0
    \end{pmatrix},\enspace b=\begin{pmatrix}
        0 \\ 1 \\ 1 \\ 1
    \end{pmatrix},\enspace X=\begin{pmatrix}
        x_1 \\ x_2 \\ x_3 \\ x_4
    \end{pmatrix}.\]
    求线性方程组$AX=b$的一般解.

    \item (12分)设三元二次齐次实多项式如下:
    \[f(x,y,z)=x^2+2xy+y^2-6yz-4xz.\]
    \begin{enumerate}
        \item 求实对称矩阵$A$,使得$f(x,y,z)=(x, y, z)A(x, y, z)^\mathrm{T}$;

        \item 求一个与$A$合同的对角矩阵;

        \item 求$f(x,y,z)$的正惯性指数和负惯性指数.
    \end{enumerate}

    \item (12分)设$V=\mathbf{R}^{4\times 1}$,$W=\mathbf{R}^{3\times 1}$,定义映射$T\colon V\to W$如下:
    \[T(X)=AX=\begin{pmatrix}
        1 & -1 & 0 & 1 \\ 1 & 1 & 2 & 3 \\ 2 & 2 & 3 & 4
    \end{pmatrix}\begin{pmatrix}
        x_1 \\ x_2 \\ x_3 \\ x_4
    \end{pmatrix},\enspace\forall X=\begin{pmatrix}
        x_1 \\ x_2 \\ x_3 \\ x_4
    \end{pmatrix}\in V.\]
    \begin{enumerate}
        \item 证明$T$的秩为3;

        \item 求$V$和$\im T$的基$\{\varepsilon_1,\varepsilon_2,\varepsilon_3,\varepsilon_4\}$和$\{\eta_1,\eta_2\}$,使得
        \[T(\varepsilon_1)=\eta_1,\enspace T(\varepsilon_2)=\eta_2,\enspace T(\varepsilon_3)=T(\varepsilon_4)=(0,0,0)^\mathrm{T}.\]
    \end{enumerate}

    \item (12分)设 $V=\mathbf{R}^{2 \times 2}$ 是按矩阵加法和数乘构成的实数域上的线性空间.
    \begin{enumerate}
        \item 验证下列向量组构成 $V$ 的一组基:
        \[B=\left\{\begin{pmatrix}
        1 & 0 \\ 0 & 0 \end{pmatrix},\begin{pmatrix}
        0 & 1 \\ 0 & 0 \end{pmatrix},\begin{pmatrix}
        1 & 1 \\ 1 & 0 \end{pmatrix},\begin{pmatrix}
        1 & 1 \\ 1 & 1 \end{pmatrix}\right\};\]

        \item 在 $V$ 上定义运算
        \[\sigma\left((a_{ij})_{2 \times 2},\left(b_{ij}\right)_{2 \times 2}\right)=a_{11} b_{11}+a_{12} b_{12}+a_{21} b_{21}+a_{22} b_{22}.\]
        验证 $\sigma$ 是 $V$ 上一个内积,使得 $V$ 成为一个欧氏空间;

        \item 将 Schmidt 正交化过程用于 $B$ 求出 $V$ 的一组单位正交基.
    \end{enumerate}

    \item (8分)求矩阵
    \[A=\begin{pmatrix}
    0 & -1 & 1 \\
    -1 & 0 & 1 \\
    1 & 1 & 0
    \end{pmatrix}\]
    的所有特征值,对应的特征子空间,以及与 $A$ 相似的一个对角矩阵.

    \item (16分)设 $V=\mathbf{R}^3$ 是具有自然内积的欧氏空间,$T \in \mathcal{L}(V)$. 设
    \begin{gather*}
        \alpha_1=(1,2,0), \enspace \alpha_2=(0,1,2), \enspace \alpha_3=(2,0,1); \\
        T(\alpha_1)=-(1,0,2), \enspace T(\alpha_2)=-(2,1,0), \enspace T(\alpha_3)=-(0,2,1).
    \end{gather*}
    \begin{enumerate}
        \item 求 $T$ 关于 $V$ 的自然基的矩阵;

        \item 证明 $T$ 是一个正交变换;

        \item 证明 $T$ 是一个镜面反射变换.(存在 $V$ 的单位正交基 $\{\eta,\beta,\gamma\}$ 使得 $T(\eta)=-\eta,\enspace\allowbreak T(\beta)=\beta,\enspace\allowbreak T(\gamma)=\gamma$,或等价地,存在单位向量 $\eta$ 使得 $T(\alpha)=\alpha-2\langle\alpha, \eta\rangle \eta,\enspace\allowbreak \forall \alpha \in V$)
    \end{enumerate}

    \item (16分)判断下列命题的真伪,若它是真命题,请给出简单的证明;若它是伪命题,给出理由或举反例将它否定.
    \begin{enumerate}
        \item 设 $A$ 是实数域上 $m \times n$ 阶矩阵,则矩阵秩 $r\left(A^{\mathrm{T}} A\right)=r(A)$;

        \item 设 $A$ 是复数域上 $m \times n$ 阶矩阵,则矩阵秩 $r\left(A^{\mathrm{T}} A\right)=r(A)$;

        \item 设 $V, W$ 是数域 $\mathbf{F}$ 上的线性空间,则 $V \cup W$ 是线性空间;

        \item 实矩阵的下列性质有其二必有其三:
        \begin{enumerate}
            \item $A^{\mathrm{T}}=A$;

            \item $A^{\mathrm{T}} A=E$(单位矩阵);

            \item $A^2=E$.
        \end{enumerate}
    \end{enumerate}
\end{enumerate}

\clearpage
