\section{2024-2025学年线性代数I(H)期中}

\begin{center}
    任课老师:吴志祥\hspace{4em} 考试时长:120分钟

    考试时间:2024年11月9日(星期六)16:15-18:15
\end{center}

\begin{enumerate}
    \item (10分)解下列线性方程组:
    \[
        \begin{cases}
        x_1 + x_2 - x_3 + 2x_4 = 3, \\
        2x_1 + x_2 - 3x_4 = 1, \\
        -2x_1 - 2x_3 + 10x_4 = 4.
        \end{cases}
    \]

    \item (10分)设在线性空间 \(V(\mathbf{F})\) 中,向量 \(\beta\) 是 \(\{\alpha_1,\alpha_2,\ldots,\alpha_r\}\) 的线性组合,但不是 \(\{\alpha_1,\alpha_2,\ldots,\alpha_{r-1}\}\) 的线性组合,
    证明:\(\spa(\alpha_1,\alpha_2,\ldots,\alpha_r)=\spa(\alpha_1,\alpha_2,\ldots,\alpha_{r-1},\beta)\).

    \item (10分)设 \(\spa(\alpha_1,\alpha_2,\ldots,\alpha_r)\) 是一个 \(n\) 维向量空间组 (\(r\geqslant 2\)),且向量组在向量组
    \(
    \beta_1 = \alpha_2 + \alpha_3 + \cdots + \alpha_r, \enspace\allowbreak
    \beta_2 = \alpha_1 + \alpha_3 + \cdots + \alpha_r, \enspace\allowbreak
    \cdots, \enspace\allowbreak
    \beta_r = \alpha_1 + \alpha_2 + \cdots + \alpha_{r-1}
    \),
    证明:\(\alpha_1,\alpha_2,\ldots,\alpha_r\) 线性无关的充分必要条件是 \(\beta_1,\beta_2,\ldots,\beta_r\) 线性无关。

    \item (10分)设 \(\alpha_1=(1,1,1),\alpha_2=(0,1,1),\alpha_3=(1,0,1)\in\mathbf{R}^3\)
    \begin{enumerate}
        \item 求 \(\alpha_1,\alpha_2\) 的夹角;

        \item 用 Schmidt 正交化方法将 \(\alpha_1,\alpha_2,\alpha_3\) 化为 \(\mathbf{R}^3\) 的一组标准正交基。

        \item 求 \(\mu=(3,1,2)\) 在这组基下的坐标。
    \end{enumerate}

    \item (10分)\(\mathbf{R}^4\) 的子空间 \(W_1\) 和 \(W_2\) 为:
    \begin{align*}
        W_1 = \{ (x_1,x_2,x_3,x_4) \mid x_1 - x_2 + x_3 - x_4 = 0 \}, \\
        W_2 = \{ (x_1,x_2,x_3,x_4) \mid x_1 + x_2 + x_3 + x_4 = 0 \}.
    \end{align*}
    \begin{enumerate}
        \item 求 \(W_1\) 的基,并扩充为 \(\mathbf{R}^4\) 的基;

        \item 分别求出 \(W_1+W_2\) 和 \(W_1\cap W_2\) 的基和维数。
    \end{enumerate}

    \item (10分)证明:\(\{1,x-2,(x-2)^2\}\) 是 \(\mathbf{R}_3[x]\) 的一组基,并求 \(f(x)=a+bx+cx^2\) 关于这组基的坐标。

    \item (10分)设 \(V\) 是实数域上的 \(2\times2\) 矩阵关于矩阵加法和数乘所构成的实数域上的线性空间,定义 \(V\) 上的变换
    \(\sigma\left(\begin{matrix}
        a & b \\
        c & d
    \end{matrix}\right)= \begin{pmatrix}
        a + b & b - c \\
        a + c & 0
    \end{pmatrix}
    \),
    证明\(\sigma\) 是 \(V\) 上的线性变换,并求 \(\sigma\) 的核 \(\ker \sigma\) 、像 \(\im \sigma\) 和 \(\sigma\) 的秩。

    \item (10分)设 $\mathbf{R}[x]_3 = \{a + bx + cx^2 \mid a, b, c \in \mathbf{R}\}$,$\lambda$ 是待定常数,$\sigma$ 是 $\mathbf{R}[x]_3$ 上的线性变换,$\sigma(p(x)) = x^2 p''(x) + 1$,求所有可能的 $f(x) \in \mathbf{R}[x]_3$ 满足 $\sigma(f(x)) = \lambda f(x)$ 及相应的待定常数 $\lambda$。

    \item (20分)下列命题是否正确?若正确,请证明;若错误,请给出反例或证明。
    \begin{enumerate}
        \item \(V_1 = \{(x_1,x_2,x_3,x_4)\in \mathbf{R}^4 \mid x_1 = x_4,\enspace x_2 + x_3 = 0\}\) 是 \(\mathbf{R}^4\) 的子空间。

        \item 设 \(\alpha_1,\alpha_2,\ldots,\alpha_s,\beta_1,\beta_2,\ldots,\beta_t\) 都是 \(n\) 维实向量,且 \(\langle \alpha_i,\beta_j \rangle=0\enspace\allowbreak (i=1,2,\ldots,s,\enspace\allowbreak j=1,2,\ldots,t)\),则 \(r(\{\alpha_1,\alpha_2,\ldots,\alpha_s\})+r(\{\beta_1,\beta_2,\ldots,\beta_t\}) \leqslant n\)。

        \item 若 \(\sigma\colon V\rightarrow W\) 是线性映射,且 \(\dim V \geqslant \dim W\) ,则 \(\sigma\) 的核为 \(\{0\}\)。

        \item 存在 \(\mathbf{R}^2\) 上的线性变换,使得该变换分别将 \((0,1),(0,-1),(3,2)\) 映射为 \((0,1),\allowbreak(2,0),\allowbreak(-4,2)\)。
    \end{enumerate}
\end{enumerate}

\clearpage
