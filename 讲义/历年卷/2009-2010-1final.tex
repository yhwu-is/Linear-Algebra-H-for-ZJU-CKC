\section{2009-2010学年线性代数I(H)期末}

\begin{center}
    任课老师:统一命卷\hspace{4em} 考试时长:120分钟
\end{center}

\begin{enumerate}
    \item (10分)记 $C([0,2\pi],\mathbf{R})$ 是区间 $[0,2\pi]$ 上全体连续函数作成的实线性空间,对 $f,g \in C([0,2\pi],\mathbf{R})$,用
    \[\langle f,g\rangle = \int_0^{2\pi}f(x)g(x)\,\mathrm{d}x\]
    来定义内积. 如果
    \[f,g\colon [0,2\pi] \to \mathbf{R},\enspace f(x)=x,\enspace g(x)=\sin x\]
    求 $f$ 与 $g$ 的夹角 $\theta$.

    \item (10分)设 $V$ 是次数 $\leqslant 2$ 的实多项式线性空间,$T\colon V\to V$,
    \[T(f(x)) = f(x) + xf'(x).\]
    求 $T$ 的特征值. 对于每个特征值,求属于它的特征子空间.

    \item (10分)设 $B$ 是 $3\times 1$ 矩阵,$C$ 是 $1\times 3$ 矩阵,证明:$r(BC)\leqslant 1$;反之,若 $A$ 是秩为 1的 $3\times 3$ 矩阵,证明:存在 $3\times 1$ 矩阵 $B$ 和 $1\times 3$ 矩阵 $C$,使得 $A=BC$.

    \item (10分)设矩阵 $A=\begin{pmatrix}a & -1 & 1 \\ -1 & a & -1 \\ 1 & -1 & a\end{pmatrix},\beta =\begin{pmatrix}0 \\ 1 \\ 1\end{pmatrix}$. 假设线性方程组 $AX=\beta$ 有解但解不唯一.
    \begin{enumerate}
        \item 求 $a$ 的值;

        \item 给出 $AX=\beta$ 的一般解.
    \end{enumerate}

    \item (10分)设 $A$ 是可逆实矩阵.
    \begin{enumerate}
        \item 证明 $A^{\mathrm{T}}A$ 是对称矩阵;

        \item 证明 $A^{\mathrm{T}}A$ 是正定的.
    \end{enumerate}

    \item (10分)令 $A = \begin{pmatrix}0 & 1 & 1 \\ 1 & -2 & 2 \\ 1 & 2 & -1 \end{pmatrix} \in M_{3\times 3}(\mathbf R)$.
    \begin{enumerate}
        \item 求可逆矩阵 $Q\in M_{3\times 3}(\mathbf R)$ 使 $Q^{\mathrm{T}}AQ$ 是对角矩阵;

        \item 给出 $A$ 的正惯性指数、负惯性指数,并确定 $A$ 的定性.
    \end{enumerate}

    \item (10分)设 $\beta=\{v_1,v_2,\ldots,v_n\}$ 是 $V$ 的一组基,$T\colon V\to V$ 是线性变换,
    \[ T(v_1)=v_2,T(v_2)=v_3,\ldots,T(v_{n-1})=v_n,T(v_n)=a_1v_1+a_2v_2+\cdots+a_nv_n.\]
    求 $T$ 关于 $\beta$ 的矩阵表示. 以及,在什么条件下 $T$ 是同构?

    \item (10分)设 $A\in M_{n\times n}(\mathbf{F})$ 有两个不同的特征值 $\lambda_1,\lambda_2$,且属于 $\lambda_1$ 的特征子空间的维数是$n-1$,证明:$A$ 是可对角化的.

    \item (20分)判断下面命题的真伪. 若它是真命题,给出一个简单证明;若它是伪命题,举一个具体的反例将它否定.
    \begin{enumerate}
        \item 给定线性空间 $V$ 的非零向量 $v$ 和线性空间 $W$ 的向量 $w$,总存在线性映射 $T\colon V\to W$ 使得 $T(v)=w$.

        \item 若线性方程组有 $m$ 个方程,$n$ 个变量,且 $m < n$,则这个方程组一定有非零解.

        \item 若 $n$ 阶方阵 $A$ 的秩是 $n$,则 $A$ 是可逆的.

        \item 正交变换是可对角化的.
    \end{enumerate}
\end{enumerate}

\clearpage
