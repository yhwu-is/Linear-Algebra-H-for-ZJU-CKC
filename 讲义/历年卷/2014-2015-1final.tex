\section{2014-2015学年线性代数I(H)期末}

\begin{center}
    任课老师:统一命卷\hspace{4em} 考试时长:120分钟
\end{center}

\begin{enumerate}
    \item (10分)求过点$(2,0,-1)$且垂直于平面$x+2y-z=1$的直线标准方程和参数方程,以及该直线方向矢量的方向余弦.
	\item (10分)求参数 $a, b$  的值,使得 $\begin{vmatrix}1 & 1 & 1 \\ x & y & z \\u & v & w\end{vmatrix}=1,\enspace \begin{vmatrix}1 & 2 & -5 \\ x & y & z \\u & v & w\end{vmatrix}=2,\enspace \begin{vmatrix}2 & 3 & b \\ x & y & z \\u & v & w\end{vmatrix}=a$ 都成立,并求$\begin{vmatrix}x & y & z \\ 1 & -1 & 5 \\u & v & w\end{vmatrix}$.
	\item (10分)计算矩阵 $B=\begin{pmatrix}1 & 0 & 0 & 0 & 0 \\ -1 & 1 & 0 & 0 & 0 \\ 0 & 0 & 1 & -1 & -1 \\ 0 & 0 & -2 & -2 & 2 \\ 0 & 0 & -3 & -3 & 3\end{pmatrix}^n$,其中 $n$ 是自然数.
	\item (10分)设 $W$ 是线性方程组 $\begin{cases}
        x_1-x_2+4x_3-x_4=0 \\ x_1+x_2-2x_3+3x_4=0
    \end{cases}$ 的解空间,求 $W$ 的一组单位正交基,并将其扩充成 $\mathbf{R}^4$ 的单位正交基,这里 $\mathbf{R}$ 是实数域.
	\item (10分)设$\alpha_1,\alpha_2,\alpha_3$是线性空间$V$的一组基,$T\in\mathcal{L}(V)$,且$T(\alpha_1)=\alpha_1+\alpha_2$,$T(\alpha_2)=\alpha_1-\alpha_2$,$T(\alpha_3)=\alpha_1+2\alpha_2$. 求$T$的像空间和核空间,以及$T$的秩.
	\item (10分)设 $T$ 是次数小于等于 2 的实多项式线性空间 $V$ 上的变换,对任意 $f(x) \in V$,定义
    \[T(f(x))=\frac{\,\mathrm{d}((x-2)f(x))}{\,\mathrm{d}x}.\]
    证明 $T$ 是 $V$ 上的线性变换,并求 $T$ 的特征值以及对应的特征子空间.
	\item (10分)设在$\mathbf{F}[x]_3$中有两组基:
    \begin{gather}
	    \alpha_1=1-x,\enspace\alpha_2=-x+x^2,\enspace\alpha_3=3x-2x^2; \tag{A} \\
        \beta_1=4x+5x^2,\enspace\beta_2=-1,\enspace\beta_3=3x+4x^2. \tag{B}
    \end{gather}
    \begin{enumerate}
        \item 求基 (A) 到基 (B) 的过渡矩阵;

        \item 设$\alpha$在基 (A) 下的坐标为$(1,1,-1)^{\mathrm{T}}$,求$\alpha$在基 (B) 下的坐标.
    \end{enumerate}

\item (10分)已知实对称矩阵 $A$ 有两个特征值 1 和 $-1$,对应的特征向量分别是 $(1,-1,0)^\mathrm{T}$ 和 $(1,1,-2)^\mathrm{T}$,假如该矩阵与对角矩阵 $\diag\{1,2,-1\}$ 相似,求 $A^n$,其中 $n$ 为自然数.
	\item (20分)判断下列命题的真伪,若它是真命题,请给出简单的证明;若它是伪命题,给出理由或举反例将它否定.
    \begin{enumerate}
        \item 设 $A$ 和 $B$ 都是正定矩阵,那么矩阵 $AB$ 也是正定矩阵;

        \item 设 $A$ 和 $B$ 都是可逆矩阵,那么矩阵 $\begin{pmatrix}0 & B \\ A & C\end{pmatrix}$ 也是可逆矩阵;

        \item 若 $M$ 表示区间 $[0,1]$ 上所有可积实值函数全体关于通常函数加法和数乘所构成的实线性空间,在 $M$ 上定义 $\langle f(x), g(x)\rangle=\ds\int_0^1f(x)g(x)\,\mathrm{d}x$,那么 $M$ 关于该运算成为欧氏空间;

        \item 对任何 $m \times n$ 实矩阵 $A$ 和实列向量 $b$,方程组 $A^\mathrm{T}AX=A^\mathrm{T}b$ 总有解.
    \end{enumerate}
\end{enumerate}

\clearpage
