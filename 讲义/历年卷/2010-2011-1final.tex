\phantomsection
\section*{2010-2011学年线性代数I(H)期末}
\addcontentsline{toc}{section}{2010-2011学年线性代数I(H)期末}

\begin{center}
    任课老师:统一命卷\hspace{4em} 考试时长:120分钟
\end{center}

\begin{enumerate}
    \item [一、](10分)求全部的实数 $a$,使线性方程组$\begin{cases} 3x_1+2x_2+x_3=2 \\ x_1-x_2-2x_3=-3 \\ ax_1-2x_2+2x_3 = 6\end{cases}$ 的解集非空.

    \item [二、](10分)设 $M_{3\times2} (\mathbf{F})$ 是数域 $\mathbf{F}$ 上全体 $3\times 2$ 矩阵构成的线性空间,
    \[P = \begin{pmatrix}1 & 1 & 0 \\ 0 & 1 & 0 \\ 0 & 0 & 0\end{pmatrix},Q=\begin{pmatrix}0 & 0 \\ 1 & 0\end{pmatrix}.\]
    定义 $T:M_{3\times 2}(\mathbf{F}) \to M_{3\times2} (\mathbf{F})$ 如下,对任意的 $A\in M_{3\times2}(\mathbf{F})$ 有 $T(A) = PAQ$.
    \begin{enumerate}[label=(\arabic*)]
        \item 证明 $T$ 是线性映射.

        \item 求出 $T$ 的像空间和核空间.

        \item 验证关于 $T$ 的维数公式.
    \end{enumerate}

\item [三、](10分)设 $A$ 和 $B$ 是 $n$ 阶方阵,其中 $n$ 是奇数. 若 $AB=-BA$,证明:$A$ 是不可逆的或者 $B$ 是不可逆的.

    \item [四、](10分)设 $V$ 是欧氏空间, $\mathbf{u,v}\in V$ 且 $\mathbf{v} \neq \mathbf{0}$.  证明
    \[\lvert(\mathbf{u,v}) \rvert =\lvert \mathbf{u} \rvert \lvert \mathbf{v} \rvert \]
    当且仅当存在 $\lambda \in \mathbf{R}$,使 $\mathbf{u} = \lambda \mathbf{v}$.

    \item [五、](10分)设 $A=(a_{ij})_{n\times n}$ 是实正交矩阵.
    \begin{enumerate}[label=(\arabic*)]
        \item 证明 $\lvert\sum\limits_{i=1}^n{a_{ii}}\rvert \leqslant n$.

        \item 在什么条件下等式成立?
    \end{enumerate}

\item [六、](10分)求 $2\times 2$ 实矩阵 $A$,使得 $A$ 的特征值是 2 和 1,而对应于 2 的特征子空间由 $\begin{pmatrix}1 \\ 1\end{pmatrix}$生成,对应于 1 的特征子空间由 $\begin{pmatrix}2 \\ 1\end{pmatrix}$ 生成.

    \item [七、](10分)设 $n$ 阶方阵 $A$ 和 $B$ 都可对角化,并且它们有相同的特征子空间(但不一定有相同的特征值),证明 $AB=BA$.

    \item [八、](10分)实三元二次多项式 $f:\mathbf{R^3}\to \mathbf{R}$ 的定义是
    \[f(x,y,z) = 2x^2-8xy+y^2-16xz+14yz+5z^2.\]
    \begin{enumerate}[label=(\arabic*)]
        \item 给出 $3\times 3$ 实对称矩阵 $A$,使 $f(x,y,z) = (x,y,z)A(x,y,z)^{\mathbf{T}}$.

        \item 给出一个与 $A$ 相合的对角矩阵.

        \item 给出 $A$ 的秩,正惯性指数和负惯性指数.
    \end{enumerate}

\item [九、](20分)判断下面命题的真伪. 若它是真命题,给出一个简单证明;若它是伪命题,举一个具体的反例将它否定.
    \begin{enumerate}[label=(\arabic*)]
        \item 若线性映射 $T_1,T_2:V \to W$ 对 $V$ 的一组基中的每一个基向量 $v$ 满足 $T_1(v)=T_2(v)$,则 $T_1=T_2$.

        \item 若对于任何正整数 $n$,方阵 $A$(阶数大于 1)的 $n$次乘积 $A^n$ 都是非零方阵,则 $A$ 可逆.

        \item 若线性映射 $T:V\to W$ 的核是 $K$,则 $\mathrm{dim}V=\mathrm{dim}W+\mathrm{dim}K$.

        \item 若方阵 $A$ 相似于方阵 $B$,则 $A$ 与 $B$ 有相同的特征向量.
    \end{enumerate}
\end{enumerate}

\clearpage
