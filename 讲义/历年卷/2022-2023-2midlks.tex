\phantomsection
\section*{2022-2023学年线性代数II(H)期中}
\addcontentsline{toc}{section}{2022-2023学年线性代数II(H)期中(刘康生老师)}

\begin{center}
    任课老师:刘康生\hspace{4em} 考试时长:45分钟 \\
    注:上午班考察1-3题,下午班考察2-4题
\end{center}

\begin{enumerate}
	\item[一、]设$V=\mathbf{R}[x]_4$(即次数不超过4的实系数多项式全体构成的线性空间),$T\in\mathcal{L}(V)$,$T'$是$T$的对偶映射. 已知$\ker T'=\spa(\varphi)$,$\varphi\in V'$,$\varphi(p)=p(18),\enspace\forall p\in V$. 求$\im T$.
	\item[二、]设$e_1,e_2,e_3,e_4$是欧式空间$\mathbf{R}^4$的标准正交基,设$\alpha_1=e_1-e_2,\alpha_2=e_2-e_3,\alpha_3=e_3-e_4,\beta_1=e_4,\beta_2=e_1+e_2+e_3,\beta_3=e_4-e_1-2e_2-3e_3$. 求$w\in\mathbf{R}^4$使得$\langle\alpha_j,w\rangle<0$且$\langle\beta_j,w\rangle>0,\enspace j=1,2,3$.
	\item[三、]已知平面方程
    \[\pi_1:x-2y+2z+d=0,\enspace \pi_2:-2x+4y+cz+1=0.\]
    分别求$c$,$d$使分别满足
    \begin{enumerate}[label=(\arabic*)]
        \item $\pi_1$与$\pi_2$平行;

        \item $\pi_1$与$\pi_2$重合;

        \item $\pi_1$与$\pi_2$垂直;

        \item $\pi_1$与$\pi_2$相交,并求交线的参数方程;

        \item 原点到交线的最短距离为1.
    \end{enumerate}
	\item[四、]对$x=(x_1,x_2,\ldots,x_n)^\mathrm{T}\in\mathbf{C}^n$,定义1范数为$||x||_1=\sum\limits_{i=1}^n|x_i|$,设$A=(a_{ij})_{n\times n}\in\mathbf{C}^{n\times n}$.
	\begin{enumerate}[label=(\arabic*)]
        \item 求$A$关于1范数的矩阵范数,即$||A||_1=\max\{||AX||_1\mid ||X||_1=1\}$;

        \item 已知$B=(b_{ij})_{n\times n}\in\mathbf{C}^{n\times n}$,$|a_{ij}|\leqslant b_{ij},\enspace 1\leqslant i,j\leqslant n$. 证明:对任何正整数$m$,有$||A^m||_1\leqslant||B^m||_1$;

        \item 设$|a_{ii}|<1,\enspace 1\leqslant i\leqslant n$,$a_{ij}=0(i>j)$. 证明:$||A^m||_1\to 0(m\to\infty)$.(提示:若$a_{ij}=0(i>j)$,则$A^n=O$)
    \end{enumerate}
\end{enumerate}

\clearpage
