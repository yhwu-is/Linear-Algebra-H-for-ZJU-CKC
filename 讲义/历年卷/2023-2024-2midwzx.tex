\phantomsection
\section*{2023-2024学年线性代数II(H)期中}
\addcontentsline{toc}{section}{2023-2024学年线性代数II(H)期中(吴志祥老师)}

\begin{center}
    任课老师:吴志祥\hspace{4em} 考试时长:120分钟
\end{center}

\begin{enumerate}
	\item[一、]设 $V$ 和 $W$ 都是有限维线性空间,$T \in \mathcal{L}(V, W)$,证明 $\im T' = (\ker T)^{0}$.

    \item[二、]设 $T$ 是 $\mathbf{F}^3$ 上的算子,它关于标准基的矩阵为 $\begin{pmatrix}
        2 & 2 & -2 \\ 2 & 4 & -3 \\ -2 & -3 & 4
    \end{pmatrix}$,求 $\mathbf{F}^3$ 的一个由 $T$ 的本征向量组成的规范正交基.

	\item[三、]求多项式 $q \in \mathbf{R}[x]_3$,使得 $\forall p \in \mathbf{R}[x]_3, \int_{0}^{1}p(x)(\sin \pi x)\,\mathrm{d}x = \int_{0}^{1}p(x)q(x)\,\mathrm{d}x$.

	\item[四、]设 $T \in \mathcal{L}(V)$,证明 $T$ 是可逆的当且仅当存在唯一的等距同构 $S \in \mathcal{L}(V)$ 使得 $T = S \sqrt{T^{*}T}$.

    \item[五、]设 $T \in \mathcal{L}(V, W)$,证明 $T$ 是单射当且仅当 $T^{*}$ 是满射.

    \item[六、]设 $V$ 是实内积空间,$T$ 是 $V$ 上的可逆线性变换,满足 $\forall x, y \in V, \langle T(T(x)), y \rangle = \langle x, T(y) \rangle$,证明 $T$ 是等距同构.

    \item[七、]设 $T \in \mathcal{L}(V)$,对 $u, v \in V$,定义 $\langle u, v \rangle_{T} = \langle T(u), v \rangle$,证明 $\langle \cdot, \cdot \rangle_{T}$ 是 $V$ 上的内积当且仅当 $T$ 是关于原内积 $\langle \cdot, \cdot \rangle$ 的可逆正算子.

    \item[八、]设 $T$ 是内积空间 $V$ 上的正规算子,证明 $\forall k \in \mathbf{Z}^{+}, \ker T^{k} = \ker T$.

    \item[九、]判断下列命题的真伪,若它是真命题,请给出简单的证明;若它是伪命题,给出理由或举反例将它否定.
    \begin{enumerate}[label=(\arabic*)]
        \item $V$ 是线性空间,$V$ 的 $2$ 个仿射子集的交也是 $V$ 的仿射子集或者空集.

        \item $V$ 是有限维线性空间,$U$ 是 $V$ 的真子空间,则一定存在非零的 $f \in V'$,使得 $f(U) = 0$.

        \item $V$ 是有限维线性空间,$U$ 是 $V$ 的子空间,则 $U = {0}$ 当且仅当 $U^{0} = V'$.

        \item $\mathbf{R}^{2}$ 上存在一个内积,使得该内积确定的范数 $\lVert(x, y)\rVert = \max {|x|, |y|}$.
    \end{enumerate}
\end{enumerate}

\clearpage
