\section{2018-2019学年线性代数I(H)期末}

\begin{center}
    任课老师:统一命卷\hspace{4em} 考试时长:120分钟
\end{center}

\begin{enumerate}
    \item (10分)设 $\mathbf{R}[x]_4$ 是数域 $\mathbf{R}$ 上次数小于 4 的多项式所构成的线性空间(约定零多项式次数为 $-\infty$). $\mathbf{M}_2(\mathbf{R})$ 是 $\mathbf{R}$ 上 2 阶方阵所构成的线性空间,定义 $T \colon \mathbf{R}[x]_4 \to \mathbf{M}_2(\mathbf{R})$ 如下,对 $f(x) \in \mathbf{R}[x]_4$,
    \[T(f(x))=\begin{pmatrix}f(0) & f(1) \\ f(-1) & f(0)\end{pmatrix}.\]
    \begin{enumerate}
        \item 求出 $T$ 的核空间 $N(T)$ 和像空间 $R(T)$;

        \item 验证关于 $T$ 的维数公式.
    \end{enumerate}

    \item (10分)已知矩阵 $A$ 与 $B=\begin{pmatrix}2 & 8 & 6 & 5 \\ 0 & 0 & 3 & 2 \\ 0 & 0 & 1 & 7 \\ 0 & 0 & 0 & 9\end{pmatrix}$ 相似,求:
    \begin{enumerate}
        \item 行列式 $|A^2-9A+4E_4|$ 的值;

        \item $r(A^*)+r(9E_4-A)$,其中 $A^*$ 是 $A$ 的伴随矩阵.
    \end{enumerate}

    \item (10分)设 $\mathbf{R}^4$ 是 4 维欧氏空间(标准内积),$\alpha=(1,1,1,1),\enspace\allowbreak\beta=(-1,-1,0,2),\enspace\allowbreak\gamma=(1,-1,0,0) \in \mathbf{R}^4$,求:
    \begin{enumerate}
        \item 与 $\alpha,\beta,\gamma$ 都正交的一个单位向量 $\delta$;

        \item $\lVert \alpha+\beta+\gamma+\delta \rVert$.
    \end{enumerate}

    \item (10分)设实二次型 $f(x_1,x_2,x_3)=ax_1^2+2x_2^2-2x_3^2+2bx_1x_3\enspace(b > 0)$,二次型对应的矩阵 $A$ 的特征值之和为 1,特征值之积为 $-12$.
    \begin{enumerate}
        \item 求参数 $a,b$;

        \item 用正交变换将二次型 $f$ 化为标准形,并写出所用的正交变换及标准形;

        \item 判断此二次型是否是正定二次型.
    \end{enumerate}

    \item (10分)设 $A$ 是数域 $\mathbf{F}$ 上一个秩为 $r$ 的 $n$ 阶方阵,$\beta$ 是一个 $n$ 维非零列向量,$X_0$ 是线性方程组 $AX=\beta$ 的一个解,$X_1,\ldots,X_s$ 是它的导出组 $AX=0$ 的一组线性无关解.
    \begin{enumerate}
        \item 证明:向量组 $\{X_0, X_1,\ldots,X_s\}$ 线性无关;

        \item 求出包含 $AX=\beta$ 解集的最小线性空间 $W$(需写出基和维数).
    \end{enumerate}

    \item (10分)线性变换 $T \colon \mathbf{R}^3 \to \mathbf{R}^3$ 的定义是:
    \[T(x_1,x_2,x_3)=(4x_1+x_3,2x_1+3x_2+2x_3,x_1+4x_3).\]
    \begin{enumerate}
        \item 求出 $T$ 的特征多项式及特征值;

        \item 判断 $T$ 是否可对角化,并给出理由.
    \end{enumerate}

    \item (10分)设 $A \in M_{m \times n}(\mathbf{F})$,$r(A)=r$,$k$ 是满足条件 $r \leqslant k \leqslant n$ 的任意整数,证明存在 $n$ 阶方阵 $B$,使得 $AB=0$,且 $r(A)+r(B)=k$.

    \item (10分)设 $A$ 是数域 $\mathbf{F}$ 上一个 $n$ 阶方阵,$E$ 是 $n$ 阶单位矩阵,$\alpha_1 \in \mathbf{F}^n$ 是 $A$ 的属于特征值 $\lambda$ 的一个特征向量,向量组 $\alpha_1,\alpha_2,\ldots,\alpha_s$ 按如下方式产生:
    \[(A-\lambda E)\alpha_{i+1}=\alpha_i,\quad i=1,2,\ldots,s-1.\]
    证明向量组 $\{\alpha_1,\alpha_2,\ldots,\alpha_s\}$ 线性无关.

    \item (20分)判断下列命题的真伪,若它是真命题,请给出简单的证明;若它是伪命题,给出理由或举反例将它否定.
    \begin{enumerate}
        \item 设 $n$ 阶方阵 $A$ 满足 $A^2-5A+5E_n=0$,则对所有的有理数 $r$,$A+rE_n$ 都是可逆阵;

        \item 在 5 维欧氏空间 $V$ 中,存在两组线性无关向量 $S_1=\{v_1,v_2,v_3\}$ 和 $S_2=\{w_1,w_2,w_3\}$,使其满足内积 $\langle v_i,w_j\rangle=0,\enspace 1 \leqslant i,j \leqslant 3$;

        \item 不存在 2 阶方阵 $A$ 使得 $r(A)+r(A^*)=3$,其中$A^*$ 是 $A$ 的伴随矩阵;

        \item 设 $n$ 阶方阵 $A$ 的每一行元素之和是 10,则 $2A^3+A+9E_n$ 的每一行元素之和是 2019.
    \end{enumerate}
\end{enumerate}

\clearpage
