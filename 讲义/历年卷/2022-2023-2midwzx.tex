\phantomsection
\section*{2022-2023学年线性代数II(H)期中}
\addcontentsline{toc}{section}{2022-2023学年线性代数II(H)期中(吴志祥老师)}

\begin{center}
    任课老师:吴志祥\hspace{4em} 考试时长:90分钟
\end{center}

\begin{enumerate}
	\item[一、](15分)求通过直线$L:\begin{cases}
        2x+y-3z+2=0 \\ 5x+5y-4z+3=0
    \end{cases}$的两个互相垂直的平面$\pi_1$和$\pi_2$,使$\pi_1$过点$(4,-3,1)$.
	\item[二、](15分)求直线$l_1:\begin{cases}
        x-y=0 \\ z=0
    \end{cases}$与直线$l_2:\cfrac{x-2}{4}=\cfrac{y-1}{-2}=\cfrac{z-3}{-1}$的距离.
	\item[三、](15分)设$\mathbf{R}[X]$是实系数多项式构成的线性空间,令$W=\{(x^3+x^2+1)h(x)\mid h(x)\in\mathbf{R}[x]\}$.
	\begin{enumerate}[label=(\arabic*)]
        \item 证明:$W$是$\mathbf{R}[x]$的子空间;

        \item 求$\mathbf{R}[x]/W$的一组基和维数.
    \end{enumerate}
	\item[四、](15分)设$V$和$W$是数域$\mathbf{F}$上的线性空间,$V_1,V_2,\ldots,V_n$是$V$的$n$个子空间且$V=V_1\oplus V_2\oplus\cdots\oplus V_n$. 证明:$\mathcal{L}(V,W)$和$\mathcal{L}(V_1,W)\times\mathcal{L}(V_2,W)\times\cdots\times\mathcal{L}(V_n,W)$同构.
	\item[五、](10分)设$V$是一个有限维线性空间,$T\in\mathcal{L}(V)$是同构映射,记其逆映射为$T^{-1}$. 设$W$是$T$的不变子空间,证明:$W$是$T^{-1}$的不变子空间.
	\item[六、](15分)设$M_n(\mathbf{C})$是$n$阶复矩阵全体构成的线性空间,$U=\{A\in M_n(\mathbf{C})\mid A^{\mathrm{T}}=A\},W=\{B\in M_n(\mathbf{C})\mid B^{\mathrm{T}}=-B\}$. 在$M_n(\mathbf{C})$上定义二元映射$\langle,\rangle: M_n(\mathbf{C})\times M_n(\mathbf{C})\to\mathbf{C}$,使得对于任意的$A,B\in M_n(\mathbf{C})$,有$\langle A,B\rangle=\mathrm{tr}(AB^{\mathrm{H}})$,其中$B^{\mathrm{H}}$表示$B$的共轭转置矩阵.
	\begin{enumerate}
        \item 证明:$(M_n(\mathbf{C}),\langle\rangle)$是复内积空间;

        \item 证明:$U=W^{\perp}$;

        \item 设$A\in M_n(\mathbf{C})$,试求$B\in U$使得$\forall D\in U$,有$||A-B||\leqslant||A-D||$,其中$||A||=\sqrt{\langle A,A\rangle}$.
    \end{enumerate}

	\item[七、](15分)设$\mathbf{R}[x]_3$是由次数小于3的实系数多项式构成的线性空间. 对于$g(x)\in\mathbf{R}[x]_3$,定义$f_1(g(x))=\int_0^1g(x)dx$,$f_2(g(x))=\int_0^2g(x)dx$,$f_3(g(x))=\int_0^{-1}g(x)dx$.
    \begin{enumerate}[label=(\arabic*)]
        \item 证明:$f_1,f_2,f_3$是$\mathbf{R}[x]_3$对偶空间的一组基;

        \item 求$\mathbf{R}[x]_3$的一组基$g_1(x),g_2(x),g_3(x)$,使得$f_1,f_2,f_3$是$g_1(x),g_2(x),g_3(x)$的对偶基.
    \end{enumerate}
\end{enumerate}

\clearpage
