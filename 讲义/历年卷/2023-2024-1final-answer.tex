\section{2023-2024学年线性代数I(H)期末答案}

\begin{enumerate}
    \item 增广矩阵为:
    \begin{align*}
        \begin{pmatrix}[cccc|c]
        2 & 1 & -3 & -5 & 6 \\
        2 & 1 & -2 & -6 & 7 \\
        -3 & 2 & 1 & 4 & 5 \\
        -1 & 3 & 2 & b & 11
        \end{pmatrix} & \to
        \begin{pmatrix}[cccc|c]
        2 & 1 & -3 & -5 & 6 \\
        0 & 0 & 1 & -1 & 1 \\
        -1 & 3 & -2 & -1 & 11 \\
        -1 & 3 & 2 & b & 11
        \end{pmatrix} \\
        & \to
        \begin{pmatrix}[cccc|c]
        2 & 1 & -3 & -5 & 6 \\
        0 & 0 & 1 & -1 & 1 \\
        -1 & 3 & -2 & -1 & 11 \\
        0 & 0 & 4 & b+1 & 0
        \end{pmatrix}
    \end{align*}
    故当 \(b+1=-4\) 即 \(b=-5\) 时,方程组无解;

    当 \(b+1\neq -4\)时,由\((b+5)x_3=b+1,(b+5)x_4=-4\)得
    \[ x_3=\frac{b+1}{b+5},\enspace x_4=-\frac{4}{b+5}, \]
    代回有\( \begin{cases}
        2x_1+x_2=\dfrac{9b+13}{b+5} \\[1.2em]
        -3x_1+2x_2=\dfrac{4b+40}{b+5}
    \end{cases} \),解得 \(x_1=\dfrac{2b-2}{b+5},x_2=\dfrac{5b+17}{b+5}\).

    故当 \(b=-5\) 时,方程组无解. 当 \(b\neq -5\) 时,方程组有唯一解
    \[(x_1,x_2,x_3,x_4)^\mathrm{T}=\left(\frac{2b-2}{b+5},\frac{5b+17}{b+5},\frac{b+1}{b+5},-\frac{4}{b+5}\right)^\mathrm{T}.\]

    \item
    \begin{enumerate}
        \item 由于 \(W_1,W_2\) 中的元素均属于 \(\mathbf{R}^4\),故 \(W_1,W_2\) 关于加法封闭.

        对 \(\forall \alpha=(x_1,-x_1,y_1,z_1),\beta=(x_2,-x_2,y_2,z_2) \in W_1\) 及 \(\forall \lambda,\mu\in \mathbf{R}\),有
        \[\lambda\alpha+\mu\beta=(\lambda x_1+\mu x_2,-\lambda x_1-\mu x_2,\lambda y_1+\mu y_2,\lambda z_1+\mu z_2)\in W_1, \]
        因此 \(W_1\) 关于数乘封闭,故 \(W_1\) 是 \(\mathbf{R}^4\) 的子空间. 同理可证 \(W_2\)是\(\mathbf{R}^4\) 的子空间.

        \item 由条件可知,\(\forall \alpha=(x_1,x_2,x_3,x_4)\),
        \begin{gather*}
            \alpha\in W_1 \iff x_1=-x_2, \\
            \alpha\in W_2 \iff x_1=-x_3,
        \end{gather*}
        故 \(\alpha\in W_1\cap W_2 \iff \alpha=(a,-a,-a,b),\enspace a,b\in \mathbf{R}\),从而 \(W_1\cap W_2\) 维数为 \(2\),一组基为 \((1,-1,-1,0),(0,0,0,1)\).

        对 \(\forall \alpha=(x_1,-x_1,y_1,z_1)\in W_1,\beta=(x_2,y_2,-x_2,z_2) \in W_1\),
        \[ \alpha+\beta=(x_1+x_2,-x_1+y_2,y_1-x_2,z_1+z_2), \]
        由于 \(y_1,y_2\) 均为自由变量,不难知 \(W_1+W_2\) 的维数为 \(4\),一组基为 \((1,0,0,0),\allowbreak(0,1,0,0),\allowbreak(0,0,1,0),\allowbreak(0,0,0,1)\).
    \end{enumerate}

    \item
    \begin{enumerate}
        \item 由条件 \(\sigma(1,0,0)=\sigma(\alpha_3)=(3,7,1),\enspace\sigma(0,1,0)=\sigma(\alpha_2)-\sigma(\alpha_3)=(-1,-4,2),\enspace\allowbreak\sigma(0,0,1)=\sigma(\alpha_1)-\sigma(\alpha_2)=(-1,-1,-2)\),解线性方程组
        \[ \begin{cases}
            3a-b-c=3\\
            7a-4b-c=6\\
            a+2b-2c=2
        \end{cases}\]
        得 \(a=0,\enspace b=-1,\enspace c=-2\),即 \(-\sigma(0,1,0)-2\sigma(0,0,1)=(3,6,2)\), 即所求 \(\alpha=(0,-1,-2)\).

        \item 解线性方程组 \(\begin{cases}
            a+2b+c=0\\
            2a+3b+3c=0\\
            3a+7b-c=0
            \end{cases}\),用增广矩阵:
            \[
            \begin{pmatrix}[ccc|c]
            1 & 2 & 1 & 0 \\
            2 & 3 & 3 & 0 \\
            3 & 7 & -1 & 0
            \end{pmatrix} \to
            \begin{pmatrix}[ccc|c]
            1 & 2 & 1 & 0 \\
            0 & -1 & 1 & 0 \\
            0 & 1 & -4 & 0
            \end{pmatrix} \to
            \begin{pmatrix}[ccc|c]
            1 & 2 & 1 & 0 \\
            0 & -1 & 1 & 0 \\
            0 & 0 & -3 & 0
            \end{pmatrix}
            \]
            解得 \(a=b=c=0\),故 \(\beta_1,\beta_2,\beta_3\) 线性无关.
        \item 由 (1) 知 \(\sigma\) 在自然基下的矩阵表示为
        \[
        A=\begin{pmatrix}
            3 & -1 & -1\\
            7 & -4 & -1\\
            1 & 2 & -2
        \end{pmatrix}
        \]
        由自然基到 \(\beta_1,\beta_2,\beta_3\) 的过渡矩阵为
        \[
        P=\begin{pmatrix}
            1 & 2 & 3\\
            2 & 3 & 7\\
            1 & 3 & 1
        \end{pmatrix}
        \]
        用初等变换法求得 \(P^{-1}=\begin{pmatrix}
            -18 & 7 & 5\\
            5 & -2 & -1\\
            3 & -1 & -1
        \end{pmatrix}\),故 \(\sigma\) 在 \(\beta_1,\beta_2,\beta_3\) 下的矩阵表示为
        \[ P^{-1}AP=\begin{pmatrix}
            1 & 3 & 1\\
            1 & 0 & 6\\
            -1 & -1 & -4
        \end{pmatrix}\]
    \end{enumerate}

    \item
    \begin{enumerate}
        \item 由 \(p'(x)\)也关于加法和数乘封闭,易知 \(\sigma(p(x))=(2x+1)p'(x)+p(1)\) 也关于加法和数乘封闭,故 \(\sigma\) 是线性映射.

        \item 由 \(\sigma(p(x))=(2x+1)p'(x)+p(1)\) 可知 \(\sigma(1)=1,\enspace\sigma(x)=2x+2,\enspace\sigma(x^2)=4x^2+2x+1\),由此知 \(\sigma\) 在 \(\{1,x,x^2\}\) 下的矩阵表示为
        \[
        A=\begin{pmatrix}
            1 & 2 & 1\\
            0 & 2 & 2\\
            0 & 0 & 4
        \end{pmatrix}
        \]
        计算 \(|\lambda E-A|=(\lambda-1)(\lambda-2)(\lambda-4) \) 得三个特征值为 \(1,2,4\),对应的三个特征向量为 \((1,0,0)^\mathrm{T},(2,1,0)^\mathrm{T},(1,1,1)^\mathrm{T}\).
        \item 由 (2) 知自然基到所求基的过渡矩阵为
        \(P=\begin{pmatrix}
            1 & 2 & 1\\
            0 & 1 & 1\\
            0 & 0 & 1
        \end{pmatrix}\),故所求基在 \(\{1,x,x^2\}\) 这组基下的三个坐标
        \begin{align*}
            (\varepsilon_1,\varepsilon_2,\varepsilon_3) &= \left((1,0,0)^\mathrm{T},(0,1,0)^\mathrm{T},(0,0,1)^\mathrm{T}\right) P \\
            &= \left((1,0,0)^\mathrm{T},(2,1,0)^\mathrm{T},(1,1,1)^\mathrm{T}\right)
        \end{align*}
        故所求基为 \(\{1,2+x,1+x+x^2\}\).
    \end{enumerate}

    \item
    \begin{gather*}
    \begin{vmatrix}
    1 & 1 & 1 & 1 \\
    a & b & c & d \\
    a^3 & b^3 & c^3 & d^3 \\
    a^4 & b^4 & c^4 & d^4
    \end{vmatrix}
    =
    \begin{vmatrix}
    1 & 1 & 1 & 1 \\
    0 & b - a & c - a & d - a \\
    0 & b(b - a^2) & c(c^2 - a^2) & d(d^2 - a^2) \\
    0 & b^3(b - a) & c^3(c - a) & d^3(d - a)
    \end{vmatrix} \\
    = (b - a)(c - a)(d - a)
    \begin{vmatrix}
    1 & 1 & 1 \\
    b(b + a) & c(c + a) & d(d + a) \\
    b^3 & c^3 & d^3
    \end{vmatrix}
    \end{gather*}
    对后一个行列式直接展开得
    \begin{align*}
        &\begin{vmatrix}
        1 & 1 & 1 \\
        b(b + a) & c(c + a) & d(d + a) \\
        b^3 & c^3 & d^3
        \end{vmatrix} \\
        ={}& a(bc^3-b^c+cd^3-c^3d+db^3-d^3b)+b^2c^3-b^3c^2+c^2d^3-c^3d^2+d^2b^3-d^3b^2 \\
        ={}& a(b+c+d)(bc^2-b^2c+cd^2-c^2d+db^2-d^2b) \\
           &+(bc+cd+db)(bc^2-b^2c+cd^2-c^2d+db^2-d^2b) \\
        ={}& (ab+bc+cd+bc+cd+db)(c-b)(d-b)(d-c)
    \end{align*}

    故结果为 \((ab+ac+ad+bc+bd+cd)(a-b)(a-c)(a-d)(b-c)(b-d)(c-d)\).

    (本题也可以采用升阶的方式,将原行列式补成\(
        \begin{vmatrix}
            1 & 1 & 1 & 1 & 1 \\
            a & b & c & d & x \\
            a^2 & b^2 & c^2 & d^2 & x^2 \\
            a^3 & b^3 & c^3 & d^3 & x^3 \\
            a^4 & b^4 & c^4 & d^4 & x^4
            \end{vmatrix}
    \)并按最后一列展开成\(x\)的四次多项式,再用韦达定理求解.)

    \item
    \begin{enumerate}
        \item 由条件 \(\left(A-E\right)\left(A-2E\right)=O\),且对\(\forall \lambda \neq 1,2\),有 \(\left(A-\lambda E\right)\left(A-\left(3-\lambda\right)E\right)=\left(-\lambda^2+3\lambda-2\right)E\),由Sylvester不等式得 \(r(A-E)+r(2E-A) \leqslant n\).

        由于右边不为零矩阵,故\(A-\lambda E\) 可逆,即 \(\lambda\) 不是 \(A\) 的特征值,故 \(A\) 的特征值只能为 \(1\) 或 \(2\).

        由秩不等式 \(r(A-E)+r(2E-A)\geqslant r(2E-A + A-E)=n\),故 \(r(A-E)+r(2E-A)=n\), 即 \(A\) 有 \(n\) 个线性无关的特征向量.故 \(A\) 可对角化.

        \item 由于 \(A\) 的特征值只能为 \(1\) 或 \(2\) 且 \(A\) 可对角化,故 \(A\) 的特征值为 \(1\) 或 \(2\) ,且二者的代数重数之和为 \(n\),由 \(\tr A=\sum\limits_{i=1}^n \lambda_i\) 知 \(n=\sum\limits_{i=1}^n 1 \leqslant \tr A \leqslant \sum\limits_{i=1}^n 2=2n\).得证.
    \end{enumerate}

    \item
    设 \(C=P\begin{pmatrix}
        E_r & O\\
        O & O
        \end{pmatrix}Q\),其中 \(P,Q\) 为可逆矩阵,故
    \begin{align*}
        AC=CB &\iff AP\begin{pmatrix}E_r & O\\O & O\end{pmatrix}Q=P\begin{pmatrix}E_r & O\\O & O\end{pmatrix}QB \\
        &\iff P^{-1}AP\begin{pmatrix}E_r & O\\O & O\end{pmatrix}=\begin{pmatrix}E_r & O\\O & O\end{pmatrix}QBQ^{-1}
    \end{align*}

    由于 \(P^{-1}AP\) 与 \(A\) 的特征值相同,\(QBQ^{-1}\) 与 \(B\) 的特征值相同,故可令 \(C=P^{-1}AP,\allowbreak D=QBQ^{-1}\),证明 \(C,D\) 至少有 \(r\) 个特征值相同.

    设 \(C=\begin{pmatrix}
        C_1 & C_2\\
        C_3 & C_4
        \end{pmatrix},
        D=\begin{pmatrix}
        D_1 & D_2\\
        D_3 & D_4
        \end{pmatrix}\) ,其中 \(C\) 与 \(D\) 的分块均与 \(\begin{pmatrix}E_r & O\\O & O\end{pmatrix}\) 相同. 带入上述等式有 \(\begin{pmatrix}
            C_1 & O\\
            C_3 & O
            \end{pmatrix}=
            \begin{pmatrix}
            D_1 & D_2\\
            O & O
            \end{pmatrix}\),由此知 \(C_1=D_1,C_3=O,D_2=O\),故
        \(C=\begin{pmatrix}
        C_1 & C_2\\
        O & C_4
        \end{pmatrix},
        D=\begin{pmatrix}
        C_1 & O\\
        D_3 & D_4
        \end{pmatrix}\).

        由分块三角矩阵的性质知
        \begin{gather*}
            |\lambda E_n-C|=|\lambda E_r-C_1||\lambda E_{n-r}-C_4|,\\
            |\lambda E_n-D|=|\lambda E_r-C_1||\lambda E_{n-r}-D_4|,
        \end{gather*}
        故 \(C,D\) 共同占有 \(C_1\) 的 \(r\) 个特征值,从而 \(C,D\) 至少有 \(r\) 个特征值相同.

    \item
    \[f(x,y,z)=5\left(x+\frac{2}{5}y-\frac15 z\right)^2+\frac15(y-3z)^2+(t-2)z^2. \]

    故 \(t>2\) 时 \(f(x,y,z)\) 为正定二次型,\(t=2\) 时 \(f(x,y,z)\) 为半正定二次型,\(t<2\) 时 \(f(x,y,z)\) 为不定二次型,且负惯性指数为 \(1\).

    \item
    \begin{enumerate}
        \item 错误,\(\alpha=(0, 1)^\mathrm{T},\beta=(1, 0)^\mathrm{T}\) 即为反例.

        \item 正确. 取 \(V\) 上的一组基 \(\alpha_1,\alpha_2,\ldots,\alpha_n\) 与 \(W\) 上的一个非零向量 \(\beta\),令 \(\sigma(\alpha_i)=\beta,\enspace\allowbreak i=1,2,\ldots,n\),则 \(\sigma\) 即满足要求.

        \item 正确. 由于复对称矩阵相合于它的相抵标准型,故两复对称矩阵相合当且仅当它们的秩相等,而复对称矩阵等秩即是相抵.

        \item 错误,取 \(A=1,C=(0, 1)\) 即为反例(\(C^\mathrm{T}AC\)半正定).
    \end{enumerate}
\end{enumerate}

\clearpage
