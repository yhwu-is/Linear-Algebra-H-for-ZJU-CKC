\section{2022-2023学年线性代数II(H)期末}

\begin{center}
    任课老师:统一命卷\hspace{4em} 考试时长:120分钟
\end{center}

\begin{enumerate}
    \item (15分)已知 $ T \in \mathcal{L}(\mathbf{C}^3) $,其对应矩阵为
    \[ A = \begin{pmatrix} 0 & 0 & 0 \\ 2023 & 0 & 0 \\ 6 & 28 & 0 \end{pmatrix} \]
       \begin{enumerate}
        \item 求 $A$ 的 Jordan 标准形(不必求 Jordan 基);

        \item 证明不存在复矩阵 $B$ 使得 $B^2 = A$.
    \end{enumerate}
    \item (15分)已知直线 $ L_1 = \begin{cases} x + y + z - 1 = 0 \\ x - 2y + 2 = 0 \end{cases} $,$ L_2 = \begin{cases} x = 2t \\ y = t + a \\ z = bt + 1 \end{cases} $,试确定 $ a, b $ 满足的条件使得 $ L_1, L_2 $ 是:

    \begin{enumerate*}
           \item 平行直线;\hspace{2em}
        \item 异面直线.
    \end{enumerate*}

    \item (18分)定义在 $ V = \mathbf{R}^3 $ 上的运算
       \[ \langle \vec{x}, \vec{y} \rangle_V = x_1 y_1 + x_2 y_2 + (x_2 + x_3)(y_2 + y_3) \]
    其中 $ \vec{x} = (x_1, x_2, x_3) $,$ \vec{y} = (y_1, y_2, y_3) $.
    \begin{enumerate}
        \item 验证 $ \langle \cdot, \cdot \rangle_V $ 是 $ \mathbf{R}^3 $ 上的一个内积;

        \item 求 $ \mathbf{R}^3 $ 在 $ \langle \cdot, \cdot \rangle_V $ 下的一组标准正交基;

        \item 求 $ \vec{\beta} \in V $ 使得 $ \forall \vec{x} \in V,\enspace x_1 + 2x_2 = \langle \vec{x}, \vec{\beta} \rangle_V $.
    \end{enumerate}

    \item (15分)$ T \in \mathcal{L}(V) $ 在一组基 $ \vec{\varepsilon} = \{\varepsilon_1, \varepsilon_2, \varepsilon_3\} $ 下的矩阵为
       \[ T(\vec{\varepsilon}) = (\vec{\varepsilon}) \begin{pmatrix} 1 & 0 & 0 \\ 0 & 2 & 1 \\ 0 & 0 & 2 \end{pmatrix} \]
    求 $ V $ 所有的 $ T $-不变子空间.

    \item (20分)试给出下列命题的真伪. 若命题为真,请给出简要证明;若命题为假,请举出反例.
    \begin{enumerate}
        \item $ T \in \mathcal{L}(V) $. 若子空间 $ W \in V $ 在 $ T $ 下不变,则其补空间 $ W' $ 在 $ T $ 下也不变;

        \item 定义 $ T \in \mathcal{L}(V, W) \colon Tv = \langle v, \alpha \rangle \beta,\enspace \beta \in W $ 对 $ \forall v \in V $ 成立,则 $ T^* w = \langle w, \beta \rangle \alpha,\enspace \alpha \in V $ 对 $ \forall w \in W $ 成立;

        \item $ T \in \mathcal{L}(V) $ 是非幂零算子,满足 $ \ker T^{n - 1} \neq \ker T^{n - 2} $. 则其极小多项式为
        \[ m(\lambda) = \lambda^{n-1}(\lambda - a) ,\quad 0 \neq a \in \mathbf{R} \]

        \item $ A \in \mathbf{R}^{n \times n} $. $ S_1 = A^{\mathrm{T}} + A $,$ S_2 = A^{\mathrm{T}} - A $. 则 $ A $ 是正规矩阵当且仅当 $ S_1 S_2 = S_2 S_1 $.

        \item $ A \in \mathbf{C}^{n \times n} $ 是正规矩阵,则 $ A $ 的实部矩阵和虚部矩阵是对称矩阵.
    \end{enumerate}

    \item (15分)$ T \in \mathcal{L}(V) $. 有极分解 $ T = S \sqrt{G} $,其中 $ S $ 是等距同构,$ G = T^* T $. 证明以下条件等价:
       \begin{enumerate}
        \item $ T $ 是正规算子;

        \item $ GS = SG $;

        \item $ G $ 的所有特征空间 $ E(\lambda, G) $ 都是 $ S $-不变的.
    \end{enumerate}
\end{enumerate}

\clearpage
