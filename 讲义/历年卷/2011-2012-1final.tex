\section{2011-2012学年线性代数I(H)期末}

\begin{center}
    任课老师:统一命卷\hspace{4em} 考试时长:120分钟
\end{center}

\begin{enumerate}
    \item (10分)求 $a,b,c,d,e,f\in \mathbf{R}$,使 $(1,1,1)^\mathrm{T},(1,0,-1)^\mathrm{T},(1,-1,0)^\mathrm{T}\in \mathbf{R}^3$ 是矩阵 $\begin{pmatrix}1 & -1 & 1 \\ a & b & c \\ d & e & f\end{pmatrix}$ 的特征向量.

    \item (10分)记线性映射 $\sigma$ 的核为 $\ker \sigma$,像为 $\im \sigma$.  设 $\sigma_1,\sigma_2\colon V\to V$ 是线性映射. 证明:
    \begin{enumerate}
        \item $\ker \sigma_1 \subseteq \ker (\sigma_2 \circ \sigma_1)$.

        \item $\im (\sigma_2 \circ \sigma_1) \subseteq \im \sigma_2$.
    \end{enumerate}

    \item (10分)设 $B=\{v_1,v_2,v_3\}$ 是线性空间 $V$ 的一组基,线性映射 $\sigma\colon V\to V$ 定义如下:
    \[\sigma(v_1)=v_2+v_3,\sigma(v_2)=v_3,\sigma(v_3)=v_1-v_2.\]
    \begin{enumerate}
        \item 给出 $\sigma$ 关于基 $B$ 的矩阵表示.

        \item 证明 $B'=\{v_2,v_3+v_1,v_1-v_2\}$ 是 $V$ 的另一组基.

        \item 给出 $\sigma$ 关于基 $B'$ 的矩阵表示.
    \end{enumerate}

    \item (10分)设 $\{v_1,v_2,\ldots,v_n\}$ 是欧氏空间 $V$ 的一组单位正交基. 证明:对于任何 $u\in V$,成立
    \[\lvert u \rvert^2 = \langle u,v_1\rangle^2+\langle u,v_2\rangle^2+\cdots+\langle u,v_n\rangle^2.\]

    \item (10分)记 $\mathcal{P}_2(\mathbf{R})$ 为次数小于等于 2 的实多项式线性空间.
    \begin{enumerate}
        \item 证明:$\langle f,g\rangle =\ds\int_{-1}^1f(x)g(x)\,\mathrm{d}x$ 是 $\mathcal{P}_2(\mathbf{R})$ 的内积.

        \item 将 Schmidt 正交化过程应用于 $S=\{1,x,x^2\}$,求出 $\mathcal{P}_2(\mathbf{R})$ 的一组单位正交基 $B$.
    \end{enumerate}

    \item (10分)设 $\sigma\colon V\to V$ 是有限维线性空间 $V$ 上的一个同构映射. 记 $V(\sigma;\lambda)$ 为 $\sigma$ 的属于特征值 $\lambda$ 的特征子空间.
    \begin{enumerate}
        \item 如果 $\lambda$ 是 $\sigma$ 的特征值,证明:$\lambda \neq 0$.

        \item 证明 $\lambda$ 是 $\sigma$ 的特征值,证明:$V(\sigma;\lambda)=V(\sigma^{-1};\lambda^{-1})$.

        \item 证明 ``$\sigma$ 可对角化'' 的充要条件是 ``$\sigma^{-1}$ 可对角化''.
    \end{enumerate}

    \item (10分)求下面实对称矩阵的秩,正惯性指数和负惯性指数.
    \begin{enumerate}
        \item $\begin{pmatrix}1 & -2 & 3 \\ -2 & 6 & 9 \\ 3 & -9 & 4\end{pmatrix};$

        \item $\begin{pmatrix}1 & 1 & -2 & -3\\ 1 & 2 & -5 & -1 \\ -2 & -5 & 10 & 9 \\ -3 & -1 & 9 & -14\end{pmatrix}$.
    \end{enumerate}

    \item (10分)设 $A,B$ 都是域 $\mathbf{F}$ 上的 $n$ 阶对角矩阵,且 $A$ 的对角元是 $B$ 的对角元的一个置换. 证明:
    \begin{enumerate}
        \item $A$ 相似于 $B$.

        \item $A$ 相合于 $B$.
    \end{enumerate}

    \item (20分)判断下面命题的真伪. 若它是真命题,给出一个简单证明;若它是伪命题,举一个具体的反例将它否定.
    \begin{enumerate}
        \item 若 $S$ 是线性空间 $V$ 的线性相关子集,则 $S$ 的每个向量都是 $S$ 的其他向量的线性组合.

        \item 域 $\mathbf{F}$ 上的全体 $n$ 阶可逆阵构成 $M_{n\times n}(\mathbf{F})$ 的一个子空间.

        \item 若存在正整数 $n$,使得方阵 $A$ 的 $n$ 次幂 $A^n=0$,则 $A$ 的行列式 $\lvert A \rvert = 0$.

        \item 对任意的 $n$ 阶实对称阵 $A$,总存在 $\varepsilon$,使得 $E_n+\varepsilon A$ 是正定矩阵.
    \end{enumerate}
\end{enumerate}

\clearpage
