\section{2022-2023学年线性代数I(H)期中}

\begin{center}
    任课老师:吴志祥\hspace{4em} 考试时长:90分钟
\end{center}

\begin{enumerate}
    \item (10分)讨论当$a$取何值时,下列方程组有解?无解?
	\[\begin{cases}
        x_1+x_2+x_3+x_4+x_5=1 \\
        3x_1+2x_2+x_3+x_4-3x_5=a \\
        x_2+2x_3+2x_4+6x_5=3 \\
        5x_1+4x_2+3x_3+3x_4-x_5=2
    \end{cases}.\]
	\item (10分)证明向量组$\{2\alpha_1+\alpha_2,2\alpha_2+\alpha_3,2\alpha_3+\alpha_1\}$线性无关的充分必要条件是向量组$\{\alpha_1,\alpha_2,\alpha_3\}$线性无关.
	\item (10分)已知向量$\alpha_1=(1,2,4,3)^\mathrm{T},\enspace \alpha_2=(1,-1,-6,6)^\mathrm{T},\enspace \alpha_3=(-2,-1,2,-9)^\mathrm{T},\enspace \alpha_4=(1,1,-2,7)^\mathrm{T},\enspace \beta=(4,2,4,a)^\mathrm{T}$.
    \begin{enumerate}
        \item 求子空间$W=\spa(\alpha_1,\alpha_2,\alpha_3,\alpha_4)$的维数和一组基;

        \item 求$a$的值,使得$\beta\in W$,并求$\beta$在 (1) 中选取的基下的坐标.
    \end{enumerate}
	\item (10分)设$\{\varepsilon_1,\varepsilon_2,\varepsilon_3,\varepsilon_4\}$是欧式空间$V$的一组标准正交基,$W=\spa(\alpha_1,\alpha_2,\alpha_3)$,其中$\alpha_1=\varepsilon_1+\varepsilon_2+\varepsilon_3+\varepsilon_4,\alpha_2=3\varepsilon_1+3\varepsilon_2-\varepsilon3-\varepsilon_4,\alpha_3=-2\varepsilon_1+6\varepsilon_3+8\varepsilon_4$.
	\begin{enumerate}
        \item 求$\alpha_1,\alpha_2$的夹角;

        \item 求$W$的一组标准正交基.
    \end{enumerate}
	\item (10分)已知$f_1=1-x,f_2=1+x^2,f_3=x+2x^2$是$\mathbf{R}[x]_3$中三个元素,$\sigma$是$\mathbf{R}[x]_3$上的线性变换且满足$\sigma(f_1)=2+x^2,\sigma(f_2)=x,\sigma(f_3)=1+x+x^2$.
    \begin{enumerate}
        \item 证明:$f_1,f_2,f_3$构成$\mathbf{R}[x]_3$的一组基;

        \item 求$\sigma$在基$\{f_1,f_2,f_3\}$下的矩阵;

        \item 设$f=1+2x+3x^2$,求$\sigma(f)$.
    \end{enumerate}
	\item (10分)已知$\mathbf{R}^3$的两个线性变换$\sigma,\tau$为
    \begin{gather*}
        \sigma(x_1,x_2,x_3)=(x_1+2x_2+3x_3,-x_1+2x_2-x_3,0), \\
        \tau(x_1,x_2,x_3)=(x_2,x_3,0).
    \end{gather*}
    \begin{enumerate}
        \item 求$r(\sigma+\tau)$和$r(\sigma\tau)$;

        \item 求$\im\sigma+\ker\sigma$.
    \end{enumerate}
	\item (10分)设$M_n(\mathbf{R})$是实数域$\mathbf{R}$上所有$n$阶矩阵组成的集合. 设$W=\{A\in M_n(\mathbf{R})\mid a_{ji}=ka_{ij},\enspace i\leqslant j\}$,求当$k=0,1,2$时,$W$的一组基和维数.

    \item (10分)设$V$是域$\mathbf{F}$上的$n$维线性空间,$\alpha_1,\alpha_2,\ldots,\alpha_n$是$V$的一组基,且
    \begin{gather*}
        V_1=\spa(\alpha_1+2\alpha_2+\cdots+n\alpha_n) \\
        V_2=\left\{k_1\alpha_1+k_2\alpha_2+\cdots+k_n\alpha_n \;\middle|\; k_1+\frac{k_2}{2}+\cdots+\frac{k_n}{n}=0\right\}
    \end{gather*}
    证明:
    \begin{enumerate}
        \item $V_2$是$V$的子空间;

        \item $V=V_1\oplus V_2$.
    \end{enumerate}
	\item (20分)判断下列命题的真伪,若它是真命题,请给出简单的证明;若它是伪命题,给出理由或举反例将它否定.
    \begin{enumerate}
        \item $\forall n\geqslant 2$,不存在非零实线性映射$f\colon M_n(\mathbf{R})\to\mathbf{R}$使得$f(AB)=f(A)f(B)$;

        \item 设$W_1,W_2$是线性空间$V$的两个子空间,$W_1\cup W_2=W_1+W_2$当且仅当$W_1\subseteq W_2$或$W_2\subseteq W_1$;

        \item 设$\alpha,\beta$是欧式空间$V$中两个线性无关向量,且${2\langle\alpha,\beta\rangle}/{\langle\alpha,\alpha\rangle}$和${2\langle\alpha,\beta\rangle}/{\langle\beta,\beta\rangle}$都是不大于零的整数,则$\alpha$和$\beta$的夹角只可能是${\pi}/{2}$,${2\pi}/{3}$,${3\pi}/{4}$,${5\pi}/{6}$;

        \item $n$是一个大于1的整数,$W=\{(x_1,x_2,\ldots,x_n)\in\mathbf{C}^n\mid x_1^2+x_2^2+\cdots+x_n^2=0\}$是复线性空间$\mathbf{C}^n$的一个子空间.
    \end{enumerate}
\end{enumerate}

\clearpage
