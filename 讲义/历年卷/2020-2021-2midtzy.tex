\section{2020-2021学年线性代数II(H)期中}

\begin{center}
    任课老师:谈之奕\hspace{4em} 考试时长:90分钟
\end{center}

\begin{enumerate}
    \item (10分)设$g(x)=ax+b,\enspace a,b\in\mathbf{F},\enspace a\neq 0,\enspace f(x)\in \mathbf{F}[x]$,证明:$g(x)$是$f^2(x)$的因式的充要条件是$g(x)$是$f(x)$的因式.

    \item (10分)设$\lambda$是$n$阶实矩阵$A$的特征值,$\lambda^3=1$且$\lambda\notin\mathbf{R}$,$A$的极小多项式次数为2,证明:矩阵$A+I$可逆.

    \item (15分)设算子$T$的特征值仅为1,代数重数为5,几何重数为3,求$T$的所有可能的若当标准形及相应的极小多项式.

    \item (20分)设$V$为$n$维复向量空间,$T\in \mathcal{L}(V)$,$T$在$V$的一组基$e_1,e_2,\ldots,e_n$ 下的矩阵为对角矩阵$\diag\{d_1,\ldots,d_n\}$,且$d_i\neq d_j\enspace(i\neq j)$.
    \begin{enumerate}
        \item 求$T$的所有一维不变子空间;

        \item 求$T$的所有不变子空间.
    \end{enumerate}

    \item (20分)设$V$为$n$维复向量空间,$S,T\in \mathcal{L}(V)$,$ST=TS$,则
    \begin{enumerate}
        \item $S,T$至少有一个公共的特征向量;

        \item 存在$V$的一组基,使得$S$和$T$在此基下的矩阵均为上三角矩阵.
    \end{enumerate}

    \item (25分)设$A=\begin{pmatrix}
        2 & 1 & 1 \\ -2 & -1 & -2 \\ 1 & 1 & 2
    \end{pmatrix}$,求$A$的若当标准形$J$和矩阵$P$,使得$P^{-1}AP=J$.
\end{enumerate}

\clearpage
