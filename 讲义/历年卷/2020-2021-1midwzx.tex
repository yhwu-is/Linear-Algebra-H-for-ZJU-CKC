\section{2020-2021学年线性代数I(H)期中}

\begin{center}
    任课老师:吴志祥\hspace{4em} 考试时长:120分钟
\end{center}
\begin{enumerate}
    \item (10分)设方程组:
    \[\begin{cases}
           x_1-x_2+x_3-x_4=0 \\ 2x_1+4x_2-5x_3+7x_4=0\\ax_1+3x_2-4x_3+6x_4=0
    \end{cases}\]
    的解空间为 $V_1$,方程组:
    \[\begin{cases}
        4x_1+2x_2-3x_3+bx_4=0 \\ 5x_1+7x_2-9x_3+13x_4=0\\3x_1-3x_2+3x_3-2x_4=0
    \end{cases}\]
    的解空间为 $V_2$,问 $a,b$ 为何值时,$\mathbf{R}^4=V_1 \oplus V_2$.

    \item (10分)设 $V=\{(a_{ij})_{n \times n} \mid \forall i, j, a_{ij}=a_{ji}\}$
    \begin{enumerate}
        \item 证明:$V$ 为 $\mathbf{F}^{n \times n}$ 的子空间;

        \item 求 $V$ 的基和维数.
    \end{enumerate}

    \item (10分)设 $f_1=-1+x,\enspace\allowbreak f_2=1-x^2,\enspace\allowbreak f_3=1-x^3,\enspace\allowbreak g_1=x-x^2,\enspace\allowbreak g_2=x+x^3,\enspace\allowbreak V_1=\spa(f_1, f_2, f_3),\enspace\allowbreak V_2=\spa(g_1, g_2)$,求:
    \begin{enumerate}
        \item $V_1+V_2$ 的基和维数;

        \item $V_1 \cap V_2$ 的基和维数;

        \item $V_2$ 在 $\mathbf{R}[x]_4$ 空间的补.
    \end{enumerate}

    \item (10分)设 $\varepsilon_1, \varepsilon_2$ 为 $n$ 维欧氏空间 $V$ 的两个单位正交向量,定义
    \[\sigma(\alpha)=\alpha-2\langle\alpha,\varepsilon_1\rangle\varepsilon_1-2\langle\alpha,\varepsilon_2\rangle\varepsilon_2\]
    证明:
    \begin{enumerate}
        \item $\sigma$ 是 $V$ 上的线性变换;

        \item $\forall \alpha,\beta \in V,\enspace \langle\sigma (\alpha),\sigma (\beta)\rangle=\langle\alpha,\beta\rangle$.
    \end{enumerate}

    \item (10分)已知 $n$ 阶矩阵 $A$ 的秩为 1 ,证明:$A^k=(\tr A)^{k-1}A$.(注:$\tr$ 为矩阵的迹,即矩阵的对角线元素之和)

    \item (10分)已知矩阵 $A=\begin{pmatrix}a & b & c \\ d & e & f \\ h & x & y\end{pmatrix}$ 的逆是 $A^{-1}=\begin{pmatrix}-1 & -2 & -1 \\ 2 & 1 & 0 \\ 0 & -3 & -1\end{pmatrix}$,且已知矩阵$B=\begin{pmatrix}a-2b & b-3c & -c \\ d-2e & e-3f & -f \\ h-2x & x-3y & -y\end{pmatrix}$. 求矩阵 $X$ 满足:
    \[X+(B(A^\mathrm{T}B^2)^{-1}A^\mathrm{T})^{-1}=X(A^2(B^\mathrm{T}A)^{-1}B^\mathrm{T})^{-1}(A+B).\]

    \item (10分)设 $V(\mathbf{F})$ 是一个 $n$ 维线性空间,$\sigma \in \mathcal{L}(V,V)$,证明:
    \begin{enumerate}
        \item 在 $\mathbf{F}[x]$ 中有一个次数不高于 $n^2$ 的非零多项式 $p(x)$ 使 $p(\sigma)=0$;

        \item $\sigma$ 可逆$\iff$有一常数项不为 0 的多项式 $p(x)$ 使 $p(\sigma)=0$.
    \end{enumerate}

    \item (10分)已知三维线性空间 $V$ 的线性变换 $\sigma$ 关于基 $\{\alpha_1,\alpha_2,\alpha_3\}$ 所对应的矩阵为
    \[\begin{pmatrix}1 & 2 & -1 \\ 2 & 1 & 0 \\ 3 & 0 & 1\end{pmatrix}\]
    \begin{enumerate}
        \item 求 $\sigma$ 在基 $\{\beta_1, \beta_2, \beta_3\}$ 下对应的矩阵 $B$,其中:
        \[\beta_1=2\alpha_1+\alpha_2+3\alpha_3, \beta_2=\alpha_1+\alpha_2+2\alpha_3, \beta_3=-\alpha_1+\alpha_2+\alpha_3\]

        \item 求 $\sigma$ 的值域 $\sigma(V)$ 和核 $\ker \sigma$;

        \item 把 $\sigma(V)$ 的基扩充为 $V$ 的基,并求 $\sigma$ 在这组基下对应的矩阵;

        \item 把 $\ker \sigma$ 的基扩充为 $V$ 的基,并求 $\sigma$ 在这组基下对应的矩阵.
    \end{enumerate}

    \item (20分)判断下列命题的真伪,若它是真命题,请给出简单的证明;若它是伪命题,给出理由或举反例将它否定.
    \begin{enumerate}
        \item 若 $\alpha_1,\alpha_2,\alpha_3$ 线性相关,则 $\alpha_1+\alpha_2,\enspace \alpha_2+\alpha_3,\enspace \alpha_3+\alpha_1$ 也线性相关;

        \item 一个有限维线性空间只包含有限个子空间;

        \item 已知 $\sigma \in \mathcal{L}(V,V)$,$\dim V=n$,则由 $r(\sigma)+\dim\ker\sigma=n$ 可得$\im \sigma+\ker \sigma=V$;

        \item 若对于任何正整数 $n$,方阵 $A$(阶数大于 1)的 $n$ 次乘积 $A^n$ 都是非零方阵,则 $A$ 是可逆的.
    \end{enumerate}
\end{enumerate}

\clearpage
