\section{2023-2024学年线性代数II(H)期末}

\begin{center}
    任课老师:统一命卷\hspace{4em} 考试时长:120分钟
\end{center}

\begin{enumerate}
    \item (10分)
    求 $x$ 使得矩阵 $\begin{pmatrix}
        1 & 1 & 0 \\
        0 & 1 & 1 \\
        1 & 0 & x
    \end{pmatrix}$ 是正规矩阵,当其是正规矩阵时求出相似的对角矩阵.

    \item (10分)
    设 $U = \spa\{(1,1,0,0)^{\mathrm{T}},(1,1,-1,2)^{\mathrm{T}}\}$ 是 $\mathbf{R}^4$ 的子空间,求 $u \in U^\perp$ 使得 $\lVert u - (1,1,2,2)^{\mathrm{T}} \rVert$ 最小.

    \item (10分)
    设 $T$ 为复数域上的 $n$ 维线性空间 $V$ 上的线性变换,$T$ 在某组基下对应的矩阵是 $\begin{pmatrix}
        -6 & 2 & 3 \\
        2 & 0 & -1 \\
        -12 & 4 & 6
    \end{pmatrix}$,是否存在线性变换 $S$ 满足 $S^2 = T$. 假如存在,求 $S$;假如不存在,说明理由,并求 $T$ 的极小多项式以及Jordan标准形.

    \item (10分)
    设 $T$ 是 $\mathbf{R}^3$ 到 $\mathbf{R}^4$ 的线性映射,在自然基下对应的矩阵为 $\begin{pmatrix}
        1 & -1 & 2 \\
        4 & -3 & 1 \\
        3 & 4 & -2 \\
        6 & 2 & -1
    \end{pmatrix}$,$f(x,y,z,w) = x - y + z + 2w$ 是 $\mathbf{R}^4$ 上的线性泛函,求对偶映射 $T'$ 在相应对偶基下的矩阵以及 $T'(f)$.

    \item (10分)
    定义 $T(x,y,z) = (z,2x,3y)$,求 $T$ 的奇异值.

    \item (10分)
    求过直线 $\begin{cases}
        x - y + z + 4 = 0 \\
        x + y - 3z = 0
    \end{cases}$ 和点 $(1,-1,-1)$ 的平面方程,并求该点到直线的距离.

    \item (10分)
    设 $V$ 是由 $1, \cos x, \sin x$ 所张成的线性空间,求 $V$ 中的向量 $f(x)$,使得等式
    \[
    \int_{-\pi}^{\pi} f(x)g(x) \, \mathrm{d}x = \int_{-\pi}^{\pi} (2x-1)g(x) \, \mathrm{d}x
    \]
    对所有 $V$ 中所有 $g(x)$ 都成立.

    \item (10分)
    设 $T$ 是 $n$ 维线性空间 $V$ 到 $m$ 维线性空间 $W$ 的线性变换,证明 $U = \{(v,Tv) \mid v \in V\}$ 是 $V \times W$ 的子空间,并求 $U$ 的维数和 $V \times W/U$ 的维数.

    \item (20分)
    判断正误,并说明理由.
    \begin{enumerate}
        \item 正算子一定可以对角化,即存在一组基,使得该算子在这组基下为对角矩阵.

        \item 对于线性变换 $T$ 以及其伴随 $T^*$,有 $\ker T^* = \im T$.

        \item 设 $T$ 是实线性空间 $V$ 上的线性变换,线性空间 $\ker(T^2 + T + I)$ 的维数都是偶数维的.

        \item 设 $S,T$ 是有限维内积空间上的等距变换,证明 $S$ 相似于 $T$ 当且仅当它们有相同的本征(特征)多项式.
    \end{enumerate}
\end{enumerate}

\clearpage
