\section{2023-2024学年线性代数I(H)期末}

\begin{center}
    任课老师:统一命卷\hspace{4em} 考试时长:120分钟
\end{center}

\begin{enumerate}
    \item (10 分)求线性方程组
    \[ \begin{cases}
        2x_1+x_2-3x_3-5x_4=6 \\
        2x_1+x_2-2x_3-6x_4=7 \\
        -3x_1+2x_2+x_3+4x_4=5 \\
        -x_1+3x_2+2x_3+bx_4=11
    \end{cases} \]
    在\(b\)为多少的时候无解、有唯一解或有无数组解,并在有无数组解的时候求出一般解.

    \item (10 分)\(W_1,W_2\)是\(\mathbf{R}^4\)的子集,满足以下条件:
    \begin{gather*}
        W_1=\{(x,-x,y,z)\mid(x,y,z)\in\mathbf{R}^3\}, \\
        W_2=\{(x,y,-x,z)\mid(x,y,z)\in\mathbf{R}^3\}.
    \end{gather*}

    \begin{enumerate}
        \item 证明:\(W_1,W_2\)是\(\mathbf{R}^4\)的子空间.

        \item 求\(W_1\cap W_2\),\(W_1+W_2\)的维数和一组基.
    \end{enumerate}

    \item (10 分)\(B_1=\{\alpha_1,\alpha_2,\alpha_3\},B_2=\{\beta_1,\beta_2,\beta_3\}\)是\(\mathbf{R}^3\)的向量组,\(\alpha_1=(1,1,1),\alpha_2=(1,1,0),\\\alpha_3=(1,0,0)\),\(\beta_1=(1,2,1),\beta_2=(2,3,3),\beta_3=(3,7,1)\),映射\(\sigma:\mathbf{R}^3\to\mathbf{R}^3\)满足\(\sigma(\alpha_1)=\beta_1,\sigma(\alpha_2)=\beta_2,\sigma(\alpha_3)=\beta_3\).
    \begin{enumerate}
        \item 求\(\alpha\),使得\(\sigma(\alpha)=(3,6,2)\).

        \item 证明:\(B_2\)是\(\mathbf{R}^3\)的一组基.

        \item 求\(\sigma\)在基\(B_2\)下的矩阵表示.
    \end{enumerate}

    \item (10 分)\(p(x)\in\mathbf{R}[x]_3\),映射\(\sigma\)满足\(\sigma(p(x))=(2x+1)p'(x)+p(1)\)。
    \begin{enumerate}
        \item 证明:\(\sigma\)是线性映射.

        \item 求\(\sigma\)的特征值和对应的特征向量.

        \item 求\(\mathbf{R}[x]_3\)的一组基,使得\(\sigma\)在这组基下是对角矩阵.
    \end{enumerate}

    \item (10 分)求下面行列式的值:
    \[ \begin{vmatrix}
        1 & 1 & 1 & 1 \\
        a & b & c & d \\
        a^3 & b^3 & c^3 & d^3 \\
        a^4 & b^4 & c^4 & d^4
    \end{vmatrix} \]

    \item (10 分)\(n\)阶矩阵\(A\)满足:\(A^2=3A-2E\),求证:
    \begin{enumerate}
        \item \(A\)可对角化.

        \item \(\tr A\)在\(n\)和\(2n\)之间.
    \end{enumerate}

    \item (10 分)\(A,B,C\)都是\(n\)阶矩阵,\(C\)的秩为\(r\),满足\(AC=CB\),求证:\(A,B\)至少有\(r\)个特征值相同,从而相似矩阵的特征值相同.

    \item (10 分)\(f(x,y,z)=5x^2+y^2+tz^2+4xy-2yz-2xz\),当\(t\)取何值时,\(f(x,y,z)\)是正定,半正定,不定的?并求不定时的正负惯性系数。

    \item (20 分)
    判断下面命题是否正确,如正确,请给出简要证明,如错误,请举出反例或说明理由:
    \begin{enumerate}
        \item \(n\geqslant 2\),\(\alpha,\beta\)都是\(n\times1\)的列向量,则\(A=\alpha\beta^\mathrm{T}\)可对角化。

        \item \(V,W\)是域\(\mathbf{F}\)上非零的有限维线性空间,则在\(V,W\)之间存在非零线性映射。

        \item 两复对称矩阵相合当且仅当它们相抵。

        \item \(A\)是\(m\)阶实正定矩阵,\(C\)为\(m\times n\)的实矩阵,\(r(C)=m\),则\(C^\mathrm{T}AC\)也正定。
    \end{enumerate}
\end{enumerate}
