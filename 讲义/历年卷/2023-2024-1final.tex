\documentclass{ctexbook}
\usepackage{amssymb,amsmath,geometry,enumitem}
\geometry{left=2.5cm,right=2.5cm,top=3cm,bottom=3cm}

\begin{document}
\centering
\section*{2023-2024学年线性代数I(H)期末}
\begin{enumerate}
    \item[一、]增广矩阵为:
    \[
    \left(\begin{array}{cccc|c}
    2 & 1 & -3 & -5 & 6 \\
    2 & 1 & -2 & -6 & 7 \\
    -3 & 2 & 1 & 4 & 5 \\
    -1 & 3 & 2 & b & 11
    \end{array}\right) \to 
    \left(\begin{array}{cccc|c}
    2 & 1 & -3 & -5 & 6 \\
    0 & 0 & 1 & -1 & 1 \\
    -1 & 3 & -2 & -1 & 11 \\
    -1 & 3 & 2 & b & 11
    \end{array}\right) \to
    \left(\begin{array}{cccc|c}
    2 & 1 & -3 & -5 & 6 \\
    0 & 0 & 1 & -1 & 1 \\
    -1 & 3 & -2 & -1 & 11 \\
    0 & 0 & 4 & b+1 & 0
    \end{array}\right)
    \]
    故当 \(b+1=-4\) 即 \(b=-5\) 时,方程组无解;\par
    当 \(b+1\neq -4\)时 \(\Rightarrow \left(b+5\right)x_3=b+1,\left(b+5\right)x_4=-4 \Rightarrow x_3=\dfrac{b+1}{b+5},x_4=-\dfrac{4}{b+5}\),\par
    代回有 \(
            \begin{cases}
                2x_1+x_2=\dfrac{9b+13}{b+5}\\
                -3x_1+2x_2=\dfrac{4b+40}{b+5}
            \end{cases}
            \),
    解得 \(x_1=\dfrac{2b-2}{b+5},x_2=\dfrac{5b+17}{b+5}\).\par
    
    \begin{enumerate}
        \item[故]当 \(b=-5\) 时,方程组无解.
        \item[] 当 \(b\neq -5\) 时,方程组有唯一解 \((x_1,x_2,x_3,x_4)^{T}=(\dfrac{2b-2}{b+5},\dfrac{5b+17}{b+5},\dfrac{b+1}{b+5},-\dfrac{4}{b+5})^{T}\).
    \end{enumerate}
    \item[二、]
    \begin{enumerate}
        \item[(1)] 由于 \(W_1,W_2\) 中的元素均属于 \(\mathbf{R}^4\),故 \(W_1,W_2\) 关于加法封闭;\par
        对 \(\forall \alpha=(x_1,-x_1,y_1,z_1),\beta=(x_2,-x_2,y_2,z_2) \in W_1\) 及 \(\forall \lambda,\mu\in \mathbf{R}\),有\par
        \(\lambda\alpha+\mu\beta=(\lambda x_1+\mu x_2,-\lambda x_1-\mu x_2,\lambda y_1+\mu y_2,\lambda z_1+\mu z_2)\in W_1\),\par
        \(\Rightarrow W_1\) 关于数乘封闭,故 \(W_1\) 是 \(\mathbf{R}^4\) 的子空间.同理可证 \(W_2\)是\(\mathbf{R}^4\) 的子空间.\par
        \item[(2)] 由条件可知,\(\forall \alpha=(x_1,x_2,x_3,x_4),\alpha\in W_1 \Leftrightarrow x_1=-x_2,\alpha\in W_2 \Leftrightarrow x_1=-x_3\),\par
        故 \(\alpha\in W_1\cap W_2 \Leftrightarrow \alpha=(a,-a,-a,b),a,b\in \mathbf{R}\),\par
        从而 \(W_1\cap W_2\) 维数为 \(2\),一组基为 \((1,-1,-1,0),(0,0,0,1)\).\par
        对 \(\forall \alpha=(x_1,-x_1,y_1,z_1)\in W_1,\beta=(x_2,y_2,-x_2,z_2) \in W_1\),\par
        \(\alpha+\beta=(x_1+x_2,-x_1+y_2,y_1-x_2,z_1+z_2)\),\par
        由于 \(y_1,y_2\) 均为自由变量,不难知 \(W_1+W_2\) 的维数为 \(4\),一组基为 \((1,0,0,0),(0,1,0,0),(0,0,1,0),(0,0,0,1)\).\par
    \end{enumerate}
    \item[三、]
    \begin{enumerate}
        \item[(1)] 由条件 \(\sigma(1,0,0)=\sigma(\alpha_3)=(3,7,1),\sigma(0,1,0)=\sigma(\alpha_2)-\sigma(\alpha_3)=(-1,-4,2)\),\par
        \(\sigma(0,0,1)=\sigma(\alpha_1)-\sigma(\alpha_2)=(-1,-1,-2)\),\par
        解线性方程组 \(\begin{cases}
            3a-b-c=3\\
            7a-4b-c=6\\
            a+2b-2c=2
            \end{cases}\) 得 \(a=0,b=-1,c=-2\), \par
            即 \(-\sigma(0,1,0)-2\sigma(0,0,1)=(3,6,2)\), 即所求 \(\alpha=(0,-1,-2)\).
        \item[(2)] 解线性方程组 \(\begin{cases}
            a+2b+c=0\\
            2a+3b+3c=0\\
            3a+7b-c=0
            \end{cases}\) , 用增广矩阵:
            \[
            \left(\begin{array}{ccc|c}
            1 & 2 & 1 & 0 \\
            2 & 3 & 3 & 0 \\
            3 & 7 & -1 & 0
            \end{array}\right) \to
            \left(\begin{array}{ccc|c}
            1 & 2 & 1 & 0 \\
            0 & -1 & 1 & 0 \\
            0 & 1 & -4 & 0
            \end{array}\right) \to
            \left(\begin{array}{ccc|c}
            1 & 2 & 1 & 0 \\
            0 & -1 & 1 & 0 \\
            0 & 0 & -3 & 0
            \end{array}\right)
            \] \(\Rightarrow\) 解得 \(a=b=c=0\),故 \(\beta_1,\beta_2,\beta_3\) 线性无关.
        \item[(3)] 由 (1) 知 \(\sigma\) 在自然基下的矩阵表示为
        \[
        A=\begin{pmatrix}
            3 & -1 & -1\\
            7 & -4 & -1\\
            1 & 2 & -2
        \end{pmatrix}
        \]
        由自然基到 \(\beta_1,\beta_2,\beta_3\) 的过渡矩阵为
        \[
        P=\begin{pmatrix}
            1 & 2 & 3\\
            2 & 3 & 7\\
            1 & 3 & 1
        \end{pmatrix}
        \]
        用初等变换法求得 \(P^{-1}=\begin{pmatrix}
            -18 & 7 & 5\\
            5 & -2 & -1\\
            3 & -1 & -1
        \end{pmatrix}\),\par
        故 \(\sigma\) 在 \(\beta_1,\beta_2,\beta_3\) 下的矩阵表示为 \(P^{-1}AP=\begin{pmatrix}
            1 & 3 & 1\\
            1 & 0 & 6\\
            -1 & -1 & -4
        \end{pmatrix}\).
    \end{enumerate}
    \item[四、]
    \begin{enumerate}
        \item[(1)] 由 \(p'(x)\)也关于加法和数乘封闭,易知 \(\sigma(p(x))=(2x+1)p'(x)+p(1)\) 也关于加法和数乘封闭,故 \(\sigma\) 是线性映射.
        \item[(2)] 由 \(\sigma(p(x))=(2x+1)p'(x)+p(1)\) 可知 \(\sigma(1)=1,\sigma(x)=2x+2,\sigma(x^2)=4x^2+2x+1\),\par
        由此知 \(\sigma\) 在 \(\{1,x,x^2\}\) 下的矩阵表示为
        \[
        A=\begin{pmatrix}
            1 & 2 & 1\\
            0 & 2 & 2\\
            0 & 0 & 4
        \end{pmatrix}
        \]
        \(|\lambda E-A|=(\lambda-1)(\lambda-2)(\lambda-4) \Rightarrow\) 三个特征值为 \(1,2,4\),\par
        对应的三个特征向量为 \((1,0,0)^{T},(2,1,0)^{T},(1,1,1)^{T}\).
        \item[(3)] 由 (2) 知自然基到所求基的过渡矩阵为
        \(P=\begin{pmatrix}
            1 & 2 & 1\\
            0 & 1 & 1\\
            0 & 0 & 1
        \end{pmatrix}\),\par
        故所求基在 \(\{1,x,x^2\}\) 这组基下的三个坐标 \(\left(\epsilon_1,\epsilon_2,\epsilon_3\right)=\left((1,0,0)^{T},(0,1,0)^{T},(0,0,1)^{T}\right)P\)\par
        \(=\left((1,0,0)^{T},(2,1,0)^{T},(1,1,1)^{T}\right)\).\par
        故所求基为 \(\{1,2+x,1+x+x^2\}\).
    \end{enumerate}
    \item[五、]
    \[
    \begin{vmatrix}
    1 & 1 & 1 & 1 \\
    a & b & c & d \\
    a^3 & b^3 & c^3 & d^3 \\
    a^4 & b^4 & c^4 & d^4
    \end{vmatrix}
    =
    \begin{vmatrix}
    1 & 1 & 1 & 1 \\
    0 & b - a & c - a & d - a \\
    0 & b(b - a^2) & c(c^2 - a^2) & d(d^2 - a^2) \\
    0 & b^3(b - a) & c^3(c - a) & d^3(d - a)
    \end{vmatrix}\] 
    \[
    =(b - a)(c - a)(d - a)
    \begin{vmatrix}
    1 & 1 & 1 \\
    b(b + a) & c(c + a) & d(d + a) \\
    b^3 & c^3 & d^3
    \end{vmatrix}
    \]
    对后一个行列式直接展开得 \(
        \begin{vmatrix}
        1 & 1 & 1 \\
        b(b + a) & c(c + a) & d(d + a) \\
        b^3 & c^3 & d^3
        \end{vmatrix}\)\par
    \(=a(bc^3-b^c+cd^3-c^3d+db^3-d^3b)+b^2c^3-b^3c^2+c^2d^3-c^3d^2+d^2b^3-d^3b^2\)\par
    \(=a(b+c+d)(bc^2-b^2c+cd^2-c^2d+db^2-d^2b)+(bc+cd+db)(bc^2-b^2c+cd^2-c^2d+db^2-d^2b)\)\par
    \(=(ab+bc+cd+bc+cd+db)(c-b)(d-b)(d-c)\).\par
    故结果为 \((ab+ac+ad+bc+bd+cd)(a-b)(a-c)(a-d)(b-c)(b-d)(c-d)\).\par
    (本题也可以采用升阶的方式,将原行列式补成\(
        \begin{vmatrix}
            1 & 1 & 1 & 1 & 1 \\
            a & b & c & d & x \\
            a^2 & b^2 & c^2 & d^2 & x^2 \\
            a^3 & b^3 & c^3 & d^3 & x^3 \\
            a^4 & b^4 & c^4 & d^4 & x^4
            \end{vmatrix}
    \)并按最后一列展开成\(x\)的四次多项式,再用韦达定理求解.)
    \item[六、]
    \begin{enumerate}
        \item[(1)] 由条件 \(\left(A-E\right)\left(A-2E\right)=O\),\par
        且对\(\forall \lambda \neq 1,2\),有 \(\left(A-\lambda E\right)\left(A-\left(3-\lambda\right)E\right)=\left(-\lambda^2+3\lambda-2\right)E\),\par
        由Sylvester不等式得 \(\mathrm{r}(A-E)+\mathrm{r}(2E-A) \leq n\). \par
        由于右边不为零矩阵,故\(A-\lambda E\) 可逆,即 \(\lambda\) 不是 \(A\) 的特征值,
        故 \(A\) 的特征值只能为 \(1\) 或 \(2\).\par
        由秩不等式 \(\mathrm{r}(A-E)+\mathrm{r}(2E-A)\geq \mathrm{r}(2E-A + A-E)=n\). \par
        故 \(\mathrm{r}(A-E)+\mathrm{r}(2E-A)=n\), 即 \(A\) 有 \(n\) 个线性无关的特征向量.故 \(A\) 可对角化.
        \item[(2)] 由于 \(A\) 的特征值只能为 \(1\) 或 \(2\) 且 \(A\) 可对角化,\par
        故 \(A\) 的特征值为 \(1\) 或 \(2\) ,且二者的代数重数之和为 \(n\),\par
        由 \(\mathrm{tr}(A)=\sum\limits_{i=1}^n \lambda_i\) 知 \(n=\sum\limits_{i=1}^n 1 \leq \mathrm{tr}(A) \leq \sum\limits_{i=1}^n 2=2n\).得证.
    \end{enumerate}
    \item[七、]
    设 \(C=P\begin{pmatrix}
        E_r & O\\
        O & O
        \end{pmatrix}Q\), 其中 \(P,Q\) 为可逆矩阵,\par
    故 \(AC=CB \Leftrightarrow AP\begin{pmatrix}E_r & O\\O & O\end{pmatrix}Q=P\begin{pmatrix}E_r & O\\O & O\end{pmatrix}QB\)\par
    \(\Leftrightarrow P^{-1}AP\begin{pmatrix}E_r & O\\O & O\end{pmatrix}=\begin{pmatrix}E_r & O\\O & O\end{pmatrix}QBQ^{-1}\) \par
    由于 \(P^{-1}AP\) 与 \(A\) 的特征值相同,\(QBQ^{-1}\) 与 \(B\) 的特征值相同,\par
    故可令 \(C=P^{-1}AP,D=QBQ^{-1}\),证明 \(C,D\) 至少有 \(r\) 个特征值相同.\par
    设 \(C=\begin{pmatrix}
        C_1 & C_2\\
        C_3 & C_4
        \end{pmatrix},
        D=\begin{pmatrix}
        D_1 & D_2\\
        D_3 & D_4
        \end{pmatrix}\) ,其中 \(C\) 与 \(D\) 的分块均与 \(\begin{pmatrix}E_r & O\\O & O\end{pmatrix}\) 相同.\par
        带入上述等式有 \(\begin{pmatrix}
            C_1 & O\\
            C_3 & O
            \end{pmatrix}=
            \begin{pmatrix}
            D_1 & D_2\\
            O & O
            \end{pmatrix}\),\par
        由此知 \(C_1=D_1,C_3=O,D_2=O\), 故 
        \(C=\begin{pmatrix}
        C_1 & C_2\\
        O & C_4
        \end{pmatrix},
        D=\begin{pmatrix}
        C_1 & O\\
        D_3 & D_4
        \end{pmatrix}\).\par
        由分块三角矩阵的性质知 \(|\lambda E_n-C|=|\lambda E_r-C_1||\lambda E_{n-r}-C_4|,|\lambda E_n-D|=|\lambda E_r-C_1||\lambda E_{n-r}-D_4|\),\par
        故 \(C,D\) 共同占有 \(C_1\) 的 \(r\) 个特征值,从而 \(C,D\) 至少有 \(r\) 个特征值相同.
    \item[八、] \(f(x,y,z)=5(x+\dfrac{2}{5}y-\dfrac15 z)^2+\dfrac15(y-3z)^2+(t-2)z^2\).\par
    故 \(t>2\) 时 \(f(x,y,z)\) 为正定二次型,\par
    \(t=2\) 时 \(f(x,y,z)\) 为半正定二次型,\par
    \(t<2\) 时 \(f(x,y,z)\) 为不定二次型,且负惯性指数为 \(1\).
    \item[九、]
    \begin{enumerate}
        \item[(1)] 错误,\(\alpha=\begin{pmatrix} 0 \\ 1 \end{pmatrix},\beta=\begin{pmatrix} 1 \\ 0 \end{pmatrix}\) 即为反例.
        \item[(2)] 正确。取 \(V\) 上的一组基 \(\alpha_1,\alpha_2,\cdots,\alpha_n\) 与 \(W\) 上的一个非零向量 \(\beta\),\par
        令 \(\sigma(\alpha_i)=\beta,i=1,2,\cdots,n\),则 \(\sigma\) 即满足要求.
        \item[(3)] 正确。由于复对称矩阵相合于它的相抵标准型,故两复对称矩阵相合当且仅当它们的秩相等,而复对称矩阵等秩即是相抵。
        \item[(4)] 错误,取 \(A=1,C=\begin{pmatrix} 0 & 1 \end{pmatrix}\) 即为反例 (\(C^{T}AC\)半正定).
    \end{enumerate}
\end{enumerate}    
\end{document}