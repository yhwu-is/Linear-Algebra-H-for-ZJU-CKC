\documentclass{ctexbook}

\usepackage{geometry}
\geometry{a4paper}
%\usepackage[UTF8, heading = false, scheme = plain]{ctex}%格式
\usepackage{ctex}
\usepackage{bm}
\usepackage{graphicx} %添加图片
% \usepackage{amsthm}
\usepackage{amsmath}
\renewcommand{\vec}[1]{\boldsymbol{#1}} % 生产粗体向量,而不是带箭头的向量
\usepackage{amssymb}
\usepackage{booktabs} % excel导出的大表格
\usepackage{hyperref}
\usepackage{rotating}
\usepackage{extarrows}

\usepackage{indentfirst}
\setlength{\parindent}{2em}

\usepackage{ntheorem}
\theoremheaderfont{\bf\heiti}
\theorembodyfont{\fangsong} 

\usepackage{zhnumber}
% \renewcommand\thechapter{\zhnum{chapter}}

\usepackage[center]{titlesec}  % chapter标题修改为第*讲
\titleformat{\chapter}{\centering\Huge\bfseries}{第\,\thechapter\,讲}{1em}{}
\titleformat{\section}{\raggedright\Large\bfseries}{\,\thesection\,}{1em}{}
\titleformat{\subsection}{\raggedright\large\bfseries}{\,\thesubsection\,}{1em}{}

% \CTEXsetup[name = {第\,\thechapter\,讲}]{chapter}
% \usepackage{titletoc}
% \titlecontents{chapter}[3em]{\bfseries \zihao{5} \vspace{10pt}}{\contentslabel{4em}}{\hspace*{-4em}}{~\titlerule*[0.6pc]{$.$}~\contentspage}

\newtheorem{definition}{定义}[chapter] %中文
\newtheorem{example}{例}[chapter]
\newtheorem{lemma}{引理}[chapter]
\newtheorem{theorem}{定理}[chapter]

%\newenvironment{proof}{{\noindent\it 证明}\quad}{\hfill □ \square□\par}
\DeclareMathOperator{\Ima}{Im}%定义新符号
\DeclareMathOperator{\Rank}{rank}%定义求秩算子

\title{\heiti 浙江大学 2023-2024 学年 \\ 线性代数荣誉课辅学讲义}
\author{2023-2024 学年线性代数 I/II(H)辅学授课 \\ 吴一航\ \ yhwu\_is@zju.edu.cn}

\begin{document}
\maketitle

\songti
\tableofcontents
\newpage
\setcounter{page}{1} % 将页码计数设置为1
\chapter{预备知识}

\indent 线性代数作为大学的第一门数学课,预修要求并不高。我们默认读者具有基本的
高中数学知识,因此关于集合、映射以及向量的基本知识我们不在此赘述。这一讲
我们将介绍本书中常见的概念——等价类,最常用的算法之一——高斯消元法以及为
引入线性空间做铺垫的基本代数结构的内容。

\section{等价类}


\section{高斯消元法}


\section{基本代数结构}


\vspace{2ex} 
\centerline{\heiti \Large 内容总结}
\vspace{2ex} 

在本讲中我们介绍了

\centerline{\heiti \Large 习题}
\vspace{2ex} 
{\kaishu 我这门课很简单,只有简单的加减乘除四则运算,甚至除法都不太需要。}
\begin{flushright}
    \kaishu
	——浙江大学数学科学学院教授吴志祥
\end{flushright}
\centerline{\heiti A组}
\begin{enumerate}
	\item 
\end{enumerate}
\centerline{\heiti B组}
\begin{enumerate}
	\item 
\end{enumerate}
\centerline{\heiti C组}
\begin{enumerate}
	\item 
\end{enumerate}
\chapter{线性空间}

\section{线性空间的定义}
线性空间是我们接触的第一个比较重要的概念,它是定义在非空集合$V$和数域$\mathbf{F}$
上的,并且定义了$V$中的加法和$V\times \mathbf{F}$(即数域中的数与集合中向量之间的)数乘运算.
总结起来就是由非空集合+数域+运算构成的,并且运算满足以下性质:

加法构成交换群$\langle V:+\rangle$(Abel 群,见教材 1.10 节,考试一般不要求)

\begin{enumerate}
    \item 结合律:$a+(b+c)=(a+b)+c$;

    \item 加法单位元:$\exists 0 \in V$,使得$\forall\alpha\in V$ 有 $a+0=0+a$;

    \item 逆元:$\forall\alpha\in V,\enspace \exists b\in V$,有$a+b=b+a=0$,记$b=-a$;

    \item 交换律:$\forall\alpha,\enspace \beta\in V,\enspace \alpha+\beta=\beta+\alpha$.
\end{enumerate}

注意:加法零元和逆元是唯一的,我们可以定义减法运算为加上一个元素的逆.

数乘运算也满足四条性质:$\forall \alpha,\beta \in V,\enspace\forall \lambda,\mu\in\mathbf{F}$以及$\mathbf{F}$
上的乘法单位元1,有:
\begin{enumerate}
    \item $1\cdot \alpha=\alpha$;

    \item $\lambda(\mu\alpha)=(\lambda\mu)\alpha$;

    \item $(\lambda+\mu)\alpha=\lambda\alpha+\mu\alpha$;

    \item $\lambda(\alpha+\beta)=\lambda\alpha+\lambda\beta$.
\end{enumerate}

注意,综合上述性质我们有方程$\lambda\beta+\lambda_1\alpha_1+\lambda_2\alpha_2+\cdots+\lambda_r\alpha_r=0$
在$\lambda\neq 0$时的解为$\beta=-\lambda^{-1}\lambda_1\alpha_1-\lambda^{-1}\lambda_2\alpha_2-\cdots-\lambda^{-1}\lambda_r\alpha_r$.

以上定义请务必牢记于心,考试可能要求你验证线性空间.记忆难度也并不大,Abel 群四条性质都有名称标注,
数乘运算也是结合律和分配律加单位元.

注意线性空间还有一个重要的概念是运算封闭,即线性空间中的元素进行加法或数乘运算后,得到的元素仍然是属于线性空间的.
这一点是定义要求的,加法封闭是 Abel 群的要求,数乘请参考教材定义的要求.

\section{线性子空间}
我们首先看线性子空间的定义:
\begin{definition}
    设$W$是线性空间$V(\mathbf{F})$的非空子集,如果$W$对$V$中的运算也构成域$\mathbf{F}$
    上的线性空间,则称$W$是$V$的\keyterm{线性子空间}[linear subspace](简称\keyterm*{子空间}[subspace]).
\end{definition}
请一定注意定义中的非空子集,建议验证子空间时先验证非空.接下来是验证子空间的一般方法:
\begin{theorem}
    线性空间$V(\mathbf{F})$的非空子集$W$为$V$的子空间的充分必要条件是$W$对于$V(\mathbf{F})$的线性运算封闭.
\end{theorem}
这表明只要子空间中的元素满足对原空间的加法和数乘运算封闭即可.

注意线性空间有两个子空间称为平凡子空间,即仅含零元的子集$\{0\}$和其自身$V$.其它
子空间称为非平凡子空间.
\begin{example}
    \begin{enumerate}[label=(\arabic*)]
    \item 说明$\mathbf{R}[x]_2$是$\mathbf{R}[x]_3$的子空间;

    \item 判断$W_1=\left\{(x,y,z) \mid \dfrac{x}{3}=\dfrac{y}{2}=z\right\},\enspace
        W_2=\{(x,y,z) \mid x+y+z=1,\enspace x-y+z=1\}$是否为$\mathbf{R}^3$的子空间.
    \end{enumerate}
\end{example}
第二小问表明过原点的直线/平面构成三维空间的子空间,不过原点的无法保持线性性.

\section{线性扩张}
接下来我们讨论线性扩张及其性质,我们首先来看线性组合和线性表示的概念:
\begin{definition}
    设$V(\mathbf{F})$是一个线性空间,$\alpha_i\in V,\enspace\lambda_i\in \mathbf{F}\enspace(i=1,2,\ldots,m)$,
    则向量$\alpha=\lambda_1\alpha_1+\lambda_2\alpha_2+\cdots+\lambda_m\alpha_m$
    称为向量组$\alpha_1,\alpha_2,\ldots,\alpha_m$在域$\mathbf{F}$的线性组合,或说$\alpha$
    在域$\mathbf{F}$上可用向量组$\alpha_1,\alpha_2,\ldots,\alpha_m$线性表示.
\end{definition}
基于此,我们给出线性扩张的定义:
\begin{definition}
    设$S$是线性空间$V(\mathbf{F})$的非空子集,我们称
    \[ \spa(S)=\{\lambda_1\alpha_1+\cdots+\lambda_k\alpha_k \mid \lambda_1,\ldots,\lambda_k\in\mathbf{F},\enspace\alpha_1,\ldots,\alpha_k\in S,\enspace k\in\mathbf{N_+}\} \]
    为$S$的\keyterm{线性扩张}[linear span],即$S$中所有有限子集在域$\mathbf{F}$上的一切线性组合组成的$V(\mathbf{F})$的子集.
\end{definition}
下面的定理告诉我们可以通过线性扩张构造子空间:
\begin{theorem}
    线性空间$V(\mathbf{F})$的非空子集$S$的线性扩张$\spa(S)$是$V$中包含$S$的最小子空间.
\end{theorem}
这一定理的证明首先证明线性扩张是子空间,这是容易的,然后说明最小只需要说明$\spa(S)$是$V$中包含$S$的任意子空间的子集即可.

最后我们再说明有限维线性空间和无限维线性空间的定义,本课程研究的内容都在有限维线性空间:
\begin{definition}
    $V(\mathbf{F})$称为有限维线性空间,如果$V$中存在一个有限子集$S$使得$\spa(S)=V$,反之称为无限维线性空间.
\end{definition}
\begin{example}
    证明:$\mathbf{R}[x]_3$是有限维线性空间,$\mathbf{R}[x]$是无限维线性空间.
\end{example}

\vspace{2ex}
\centerline{\heiti \Large 内容总结}

\vspace{2ex}

\centerline{\heiti \Large 习题}
\vspace{2ex}
{\kaishu 1520年以来,全世界只有85个机构存活至今,其中50家是大学.大学依靠梦想、希望生存下去——这就是大学的历史.}
\begin{flushright}
    \kaishu
    ——美国哥伦比亚大学校长L·C·柏林格
\end{flushright}
\centerline{\heiti A组}
\begin{enumerate}
    \item 检验下列集合对指定的加法和数乘运算是否构成实数域上的线性空间.
    \begin{enumerate}[label=(\arabic*)]
        \item 有理数集$\mathbf{Q}$对普通的数的加法和乘法;

        \item 集合$\mathbf{R}^2$对通常的向量加法和如下定义的数量乘法:$\lambda\cdot(x,y)=(\lambda x,y)$;

        \item $\mathbf{R}_+^n$(即$n$元正实数向量)对如下定义的加法和数乘运算:
        \begin{gather*}
            (a_1,\ldots,a_n)+(b_1,\ldots,b_n)=(a_1b_1,\ldots,a_nb_n) \\
            \lambda\cdot(a_1,\ldots,a_n)=(a_1^\lambda,\ldots,a_n^\lambda)
        \end{gather*}

        \item 请继续完成教材P86第二章习题第一题9-11小问关于函数的加法数乘定义线性空间的问题.
    \end{enumerate}
    \item 请完成教材P86-87第二章习题第三题的全部小问.第5小问平常问题较多,实际上就是要判断满足一定条件的
          多项式是否构成子空间.
\end{enumerate}
\centerline{\heiti B组}
\begin{enumerate}
    \item 设$V$是一个线性空间,$W$是$V$的子集,证明:$W$是$V$的子空间$\iff \spa(W)=W$.
    \item \begin{enumerate}[label=(\arabic*)]
        \item 设$\mathbf{R}_+$是所有正实数组成的集合,加法和数乘定义如下:\[ \forall a,b \in \mathbf{R}_+,\enspace k\in \mathbf{R}\colon\enspace a\oplus b = ab,\enspace k\odot a = a^k \] 则 $\mathbf{R}_+$关于这一加法和数乘构成一个实线性空间.求$\mathbf{R}_+$的一组基;

        \item 设$V$是一个$n$维实线性空间,证明:存在$V$中的一个由可列无穷多个向量组成的向量组$\{\alpha_i \mid i\in\mathbf{Z}_+\}$,使得其中任意$n$个向量组成的向量组都是$V$的一组基.
    \end{enumerate}
\end{enumerate}
\centerline{\heiti C组}
\begin{enumerate}
    \item 设$\mathbf{E}$是域$\mathbf{F}$的一个子域.
    \begin{enumerate}[label=(\arabic*)]
        \item 证明:$\mathbf{F}$关于自身的加法和乘法构成一个$\mathbf{E}$上的向量空间,并举一例;

        \item 举例说明:$\mathbf{E}\enspace(\mathbf{E}\neq \mathbf{F})$不是$\mathbf{F}$上的线性空间;

        \item 证明:若$V$是$\mathbf{F}$上的一个线性空间,则$V$也是$\mathbf{E}$上的一个线性空间.
    \end{enumerate}
\end{enumerate}

\input{./专题/第三讲 有限维线性空间.tex}
\chapter{线性空间的运算}

\section{线性空间的交、并、和}
\subsection{线性空间的交与和的概念}
\begin{definition}
    设$W_1,W_2$是线性空间$V(\mathbf{F})$的两个子空间,则
    \begin{align*}
    W_1 \cap W_2&=\{\alpha \mid \alpha\in W_1 \text{ 且 } \alpha\in W_2\} \\
    W_1 \cup W_2&=\{\alpha \mid \alpha\in W_1 \text{ 或 } \alpha\in W_2\} \\
    W_1 + W_2&=\{\alpha_1+\alpha_2 \mid \alpha_1\in W_1,\enspace\alpha_2\in W_2\}
    \end{align*}
    分别称为$W_1$和$W_2$的交、并、和.
\end{definition}
我们要注意,线性空间的交与和仍然是$V$的子空间,请各位同学自行证明.并且$V$的有限个子空间的交与和
仍然是$V$的子空间.

关于线性空间的并,我们必须注意线性空间的并不一定是线性空间,这很容易理解,
因为两个线性空间元素组合在一起,两个线性空间各取一个元素求和显然不一定在并集中,大家
可以自行举反例. 我们给出以下结论:
\[ W_1 \cup W_2 \text{ 为线性空间 } \iff W_1 \subseteq W_2 \text{ 或 } W_2 \subseteq W_1 \iff W_1 \cup W_2=W_1+W_2 \]

这一结论证明并不复杂,希望各位同学掌握. $V$的有限个子空间的并仍为$V$的子空间的充要条件是其中有一个
子空间能包含其他所有子空间.

我们可以从几何直观上理解这些概念,例如三维空间中两个不同的过原点的平面构成的线性空间的交是其交线(交线也过原点)构成的线性空间,
其和为整个三维空间.三维空间中一个平面与不在该平面上的直线的交只有零元,和为整个三维空间.
考试时我们遇到反例问题可以首先考虑这些简单的几何图形,当然无法解决时可以考虑$(1,0),(0,1),(1,1)$此类
简单的向量为基构成的空间.

关于线性空间的并我们还有一个重要的覆盖定理:
\begin{theorem}
    设$V_1,V_2,\ldots,V_s$是线性空间$V$的$s$个非平凡子空间,证明:$V$中至少存在一个向量
    不属于$V_1,V_2,\ldots,V_s$中的任何一个,即$V_1 \cup V_2 \cup \cdots \cup V_s\subsetneq V.$
\end{theorem}
这一定理表明,任何一个线性空间都不能被自身有限个非平凡子空间通过并得到.例如,有限条直线的并不可能是一个平面.
定理的证明可以使用数学归纳法,下面是一个应用的例子:
\begin{example}
    设$V_1,V_2,\ldots,V_s$是线性空间$V$的$s$个非平凡子空间,证明:存在$V$的一组基$\alpha_1,\alpha_2,\ldots,\alpha_n$
    都不在$V_1,V_2,\ldots,V_s$中.
\end{example}

\section{维数公式}
\begin{theorem}
    设$W_1,W_2$是线性空间$V(\mathbf{F})$的两个子空间,则
    \[\dim W_1+\dim W_2=\dim(W_1+W_2)+\dim(W_1\cap W_2).\]
\end{theorem}
上式称为子空间的维数公式,区别于下一专题中的线性映射基本定理的维数公式.这一定理的证明思想
是重要的,利用基的扩张等技巧,需要同学们熟练掌握,下面是一个证明思想类似的例子:
\begin{example}
    已知$A,B$分别是数域$\mathbf{F}$上的$s \times k$和$k \times n$矩阵,$X$是$n \times 1$
    的列向量. 证明:所有满足$ABX=0$的$BX$构成一个线性空间$V$,且$\dim V = r(B) - r(AB)$.
\end{example}

\section{线性空间的直和}
我们证明或者和空间很多时候都是利用和空间定义进行向量分解,这种分解唯一时即为直和.我们有如下定义:
\begin{definition}
    设$W_1,W_2$是线性空间$V(\mathbf{F})$的两个子空间. 若$W_1 \cap W_2=\{0\}$,则$W_1+W_2$叫做
    $W_1$与$W_2$的\keyterm{直和}[direct sum],记作$W_1\oplus W_2$.此时称$W_1,W_2$为\keyterm{互补子空间}[complementary subspaces],或$W_1$是$W_2$的补空间,
    或$W_2$是$W_1$的补空间.
\end{definition}
我们需要注意,一个线性子空间的补空间并不唯一,请同学们给出相应的例子.

直和有以下等价的命题,我们证明或者利用直和都可以任意选择:
\begin{theorem}
    对于子空间$W_1,W_2$,下列命题等价:
    \begin{enumerate}[label=(\arabic*)]
        \item $W_1+W_2$是直和,即$W_1 \cap W_2=\{0\}$;

        \item $W_1+W_2$中的每个向量$\alpha$的分解式$\alpha=\alpha_1+\alpha_2\enspace(\alpha_1\in W_1,\enspace\alpha_2\in W_2)$唯一;

        \item 零向量的分解式$0=\alpha_1+\alpha_2 \enspace(\alpha_1\in W_1,\enspace\alpha_2\in W_2)$仅当$\alpha_1=\alpha_2=0$时成立;

        \item $\dim (W_1+W_2)=\dim W_1+\dim W_2$.
    \end{enumerate}
\end{theorem}
我们也可以定义有限个子空间的直和,即$V=W_1\oplus W_2\oplus\cdots\oplus W_n \iff W_i \cap \sum\limits_{j \neq i}W_j=\{0\}$.
等价命题也是上述定理的推广,例如唯一分解、0的分解以及维数公式推广. 我们有一个与多空间直和相关的定理:
\begin{theorem}
    若$V=V_1\oplus V_2$,$V_1=V_{11}\oplus\cdots\oplus V_{1s}$,$V_2=V_{21}\oplus\cdots\oplus V_{2t}$,则
    \[V=V_{11}\oplus\cdots\oplus V_{1s}\oplus V_{21}\oplus\cdots\oplus V_{2t}\]
\end{theorem}
我们证明直和一般有两种思路,一种是先证和,再证直和,我们来看一个例子:
\begin{example}
    数域$\mathbf{F}$上所有$n$阶方阵组成的线性空间$V=\mathbf{M}_n(\mathbf{F})$,$V_1$表示所有对称矩阵
    组成的集合,$V_2$表示所有反对称矩阵组成的集合. 证明:$V_1,V_2$都是$V$的子空间,且$V=V_1\oplus V_2$.
\end{example}
还有一种证明$V=V_1\oplus V_2$的方式是先令$W=V_1+V_2$,先证明和为直和(即交为$\{0\}$)再证$W=V$即可,
下面是一个例子:
\begin{example}
    设$A$是数域$\mathbf{F}$上的一个$n$阶可逆方阵,$A$的前$r$个行向量组成的矩阵为$B$,后$n-r$个
    行向量组成的矩阵为$C$,$n$元线性方程组$BX=0$与$CX=0$的解空间分别为$V_1,V_2$. 证明:$\mathbf{F}^n=V_1\oplus V_2$.
\end{example}

\vspace{2ex}
\centerline{\heiti \Large 内容总结}

\vspace{2ex}

\centerline{\heiti \Large 习题}
\vspace{2ex}
{\kaishu }
\begin{flushright}
    \kaishu

\end{flushright}
\centerline{\heiti A组}
\begin{enumerate}
    \item
\end{enumerate}
\centerline{\heiti B组}
\begin{enumerate}
    \item
\end{enumerate}
\centerline{\heiti C组}
\begin{enumerate}
    \item
\end{enumerate}

\input{./专题/第五讲 线性映射.tex}
\input{./专题/第六讲 线性映射基本定理.tex}
\input{./专题/第七讲 商空间与对偶.tex}
\chapter{矩阵基本运算}

\section{矩阵基本运算}
\subsection{基本概念}
\begin{enumerate}
    \item 矩阵的加法来源于线性映射的加法,矩阵相加要求两矩阵行列数一致,相加时只需对应位置元素相加即可;
    \item 矩阵的数乘来源于线性映射的数乘,计算只需矩阵的每个元素乘以常数即可;
    \item 矩阵的乘法来源于线性映射的复合,计算时要求前一个矩阵的列数等于后一个矩阵的行数,矩阵$A$与$B$
    相乘结果中第$i$行第$j$列元素为矩阵$A$的第$i$行与矩阵$B$的第$j$列对应位置元素相乘后求和的结果,
    即对于$A=(a_{ij})_{m \times n}$和$B=(b_{ij})_{n \times l}$,矩阵$C=AB=(c_{ij})_{m \times l}$,且
    $c_{ij}=a_{i1}b_{1j}+a_{i2}b_{2j}+\cdots+a_{in}b_{nj}\enspace(i=1,\ldots,m,\enspace j=1,\ldots,l)$.
\end{enumerate}

\subsection{基本性质}
\begin{enumerate}
    \item 回顾上一专题中$m \times n$矩阵构成的线性空间$\mathbf{M}_{m \times n}(\mathbf{F})$;
    \item 回顾矩阵乘法的基本性质:
    \begin{enumerate}[label=(\arabic*)]
        \item $(AB)C=A(BC)$(结合律)
        \item $\lambda(AB)=(\lambda A)B=A(\lambda B),\enspace \lambda \in \mathbf{F}$
        \item $A(B+C)=AB+AC$(左分配律)
        \item $(B+C)P=BP+CP$(右分配律)
        \item $A^kA^m=A^{k+m},\enspace (A^k)^m=A^{km}$,其中$A$为方阵,$k,m$为任意整数. 负整数对应于逆矩阵的情况.
    \end{enumerate}
    \item 回顾矩阵多项式的定义(利用线性映射多项式在基下的矩阵表示定义),
    并注意其交换性以及可因式分解性.
\end{enumerate}
\begin{example}
    展开矩阵多项式$(A+\lambda E)^n$.
\end{example}
\begin{example}
    设$f(x),g(x) \in \mathbf{F}[x],\enspace A,B \in \mathbf{M}_n(\mathbf{F})$. 证明:
    \[f(A)g(A)=g(A)f(A)\]
    \begin{enumerate}
        \item 如果$AB=BA$,则$f(A)g(B)=g(B)f(A)$;

        \item 设$f(x)=1+x+\cdots+x^{m-1}$,$g(x)=1-x$,$A=\begin{pmatrix}
            a & b \\ 0 & a
        \end{pmatrix}$,计算$f(A)g(A)$.
    \end{enumerate}
\end{example}
其他需要注意的性质:
\begin{enumerate}
    \item 矩阵乘法不一定满足交换律(即$AB$不一定等于$BA$).但是注意数量矩阵和任何矩阵相乘都是可交换的,因此求矩阵的幂次时,可以将其转化为$(A+\mu E)^n$(其中$E$为单位矩阵,$\mu$为常数)类型,然后利用二项式展开即可.很多情况下$A$都会是幂零矩阵,此时结果为有限项.
    \item $A\neq O$且$B\neq O$不能推出$AB\neq O$.例如线性方程组$AX = 0$有非零解,若$B$的各列均为方程非零解,则$AB = O$.
    \item 消去律也不一定满足:即$AB = AC$不一定$A = B$.原因在于$AB=AC \implies A(B-C)=O$,由(2)可知不一定$B = C$.
\end{enumerate}

\subsection{矩阵可交换问题}
一般来说在本课程中此类问题直接设可交换矩阵的每一个元素都是未知数即可,一些特殊的技巧
(使用关于一些特殊形状矩阵的结论)以及涉及到之后才能学到的知识的方法我们在这里也不展开了.我们只讨论一个基本的技巧,即
\[\forall t,\enspace AB=BA \iff (A-tE)B=B(A-tE)\]
此处的$t$根据矩阵的对角线上元素来决定,原则是使得其余矩阵与$A-tE$相乘的计算过程更为简单(一般是使得0元素更多),这样解方程也会更轻松.
我们来看一个简单的例子:
\begin{example}
    求与矩阵$A=\begin{pmatrix}
        3 & 0 & 0 \\ -1 & 3 & 0 \\ 0 & -1 & 3
    \end{pmatrix}$可交换的矩阵.
\end{example}

关于可交换我们有以下定理,证明并不是很复杂(教材习题中有出现):
\begin{theorem}
    \begin{enumerate}
        \item 与主对角元两两互异的对角矩阵可交换的方阵只能是对角矩阵;

        \item 准对角矩阵$A$每个对角分块内对角线元素相同,但不同对角块之间不同,则与$A$可交换的矩阵只能是准对角矩阵;

        \item 与所有$n$级可逆矩阵可交换的矩阵为数量矩阵;

        \item 与所有$n$级矩阵可交换的矩阵为数量矩阵.
    \end{enumerate}
\end{theorem}

\section{矩阵转置}
\subsection{基本概念}
实际上,矩阵的转置就是第$i$行变成了第$i$列,或者抽象表达为:
\[A=(a_{ij})_{m \times n},\enspace A^\mathrm{T}=(a'_{ji})_{n \times m},\enspace a_{ij}=a'_{ji}\]
写成矩阵形式就是:
\begin{definition}
    设$A=\begin{pmatrix}
        a_{11} & a_{12} & \cdots & a_{1n} \\
        a_{21} & a_{22} & \cdots & a_{2n} \\
        \vdots & \vdots & \ddots & \vdots \\
        a_{m1} & a_{m2} & \cdots & a_{mn}
    \end{pmatrix}$,称$\begin{pmatrix}
        a_{11} & a_{21} & \cdots & a_{m1} \\
        a_{12} & a_{22} & \cdots & a_{m2} \\
        \vdots & \vdots & \ddots & \vdots \\
        a_{1n} & a_{2n} & \cdots & a_{mn}
    \end{pmatrix}$为矩阵$A$的转置,记作$A^\mathrm{T}$.
\end{definition}

\subsection{基本性质}
\begin{enumerate}
    \item $(A^\mathrm{T})^\mathrm{T}=A$

    \item $(A+B)^\mathrm{T}=A^\mathrm{T}+B^\mathrm{T}$

    \item $(\lambda A)^\mathrm{T}=\lambda A^\mathrm{T},\enspace \lambda \in \mathbf{F}$

    \item $(AB)^\mathrm{T}=B^\mathrm{T}A^\mathrm{T}$,$(A_1A_2\cdots A_n)^\mathrm{T}=A_n^\mathrm{T}\cdots A_2^\mathrm{T}A_1^\mathrm{T}$

    \item $(A^\mathrm{T})^{-1}=(A^{-1})^\mathrm{T}$

    \item $(A^\mathrm{T})^m=(A^m)^\mathrm{T}$
\end{enumerate}

以上证明大都是平凡的,可以自己尝试完成.
\subsection{对阵矩阵与反对称矩阵}
\begin{definition}
    设$A=(a_{ij})_{n \times n}$,如果$\forall i,j=1,2,\ldots,n$均有$a_{ij}=a_{ji}$,
    则称$A$为对称矩阵. 若均有$a_{ij}=-a_{ji}$,则称$A$为反对称矩阵.
\end{definition}
易得$A$为对称矩阵的充要条件为$A=A^\mathrm{T}$,$A$为反对称矩阵的充要条件为$A=-A^\mathrm{T}$.
\begin{example}
    证明以下几点性质:
    \begin{enumerate}
        \item 反对称矩阵主对角元均为$0$;

        \item $AA^\mathrm{T}$和$A^\mathrm{T}A$均为对称矩阵;

        \item 设$A,B$为$n$阶对称和反对称矩阵,则$AB+BA$是反对称矩阵;

        \item 对称矩阵的乘积不一定对称;

        \item 可逆的对称(反对称)矩阵的逆矩阵也是对称(反对称)矩阵.
    \end{enumerate}
\end{example}

\section{初等矩阵}
\subsection{基本概念与性质}
\begin{definition}
    将单位矩阵$E$做一次初等变换得到的矩阵称为初等矩阵,与三种初等行、列变换对应的三类初等矩阵为:
    \begin{enumerate}
        \item 将单位矩阵第$i$行(或列)乘$c$,得到初等倍乘矩阵$E_i(c)$;

        \item 将单位矩阵第$i$行乘$c$加到第$j$行,或将第$j$列乘$c$加到第$i$列,得到初等倍加矩阵$E_{ij}(c)$;

        \item 将单位矩阵第$i,j$行(或列)对换,得到初等对换矩阵$E_{ij}$.
    \end{enumerate}
\end{definition}
请各位同学以矩阵形式写出以上三类矩阵.注意:
\begin{enumerate}
    \item 倍加变化请一定注意$i$和$j$在行列的情况下的不同;

    \item 三类矩阵不是三个矩阵,例如行列选择不唯一,常数选择不唯一;

    \item 注意三种初等矩阵都是可逆的,且$E_i^{-1}(c)=E_i\left(\dfrac{1}{c}\right)$,$E_{ij}^{-1}(c)=E_{ij}(-c)$,$E_{ij}^{-1}=E_{ij}$;

    \item 三种初等矩阵的转置:$E_i^\mathrm{T}(c)=E_i(c)$,$E_{ij}^\mathrm{T}(c)=E_{ji}(c)$,$E_{ij}^\mathrm{T}=E_{ij}$;
\end{enumerate}

初等矩阵大家非常关心为什么左乘代表行变换,右乘代表列变换.以右乘为例,我们来看矩阵$A$和$B$相乘的任一列结果.我们可以将矩阵$A$
按列做分块矩阵得到$\begin{pmatrix}\alpha_1,\ldots,\alpha_n\end{pmatrix}$,$\alpha_i$即表示$A$的第$i$列.然后矩阵$B$的第$j$列为列向量$(x_1,\ldots,x_n)^\mathrm{T}$,
由于矩阵$A$与$B$相乘结果第$j$列就是$A$与$B$的第$j$列相乘结果(回顾矩阵乘法的计算方式),则有$B$的第$i$列等于
$x_1\alpha_1+\cdots+x_n\alpha_n$即为$A$的全部列向量的线性组合,故右乘矩阵$A$得到矩阵的任一列都是$A$的全部列向量的线性组合,
所以右乘可以代表列变换.注意我这里并没有限制矩阵$B$为初等矩阵或可逆矩阵.

实际上左乘表示行变换可以用类似方法说明,只需按行对$B$分块即可.这一思想是特别重要的,在很多时候如果我们意识到左右乘是对被乘矩阵的行列
重新线性组合,思路会清晰很多.

关于初等矩阵还有一个相当重要的定理:

\begin{theorem}
    任意可逆矩阵都可以被表示为若干个初等矩阵的乘积.
\end{theorem}
定理证明只需要回忆高斯消元法可以将可逆矩阵化为单位矩阵即可.

利用矩阵初等变换我们可以获得本学期需要学习的三个矩阵标准形,因此这一内容虽然很基本但是非常重要:
\begin{enumerate}
    \item 相抵矩阵:本章已学习的内容,在之后会详细说明;
    \item 相似矩阵:若$P$为初等矩阵,对矩阵做$P^{-1}AP$变换即可得到与$A$相似的矩阵;
    \item 相合矩阵:两个矩阵,其中一个可以通过做相同的初等行列变换的到另一个矩阵(若$P$为初等矩阵,
    $P^{\mathrm{T}}AP$就是对$A$做了一次相同的初等行列变换).
\end{enumerate}
请同学们思考:如何从线性映射矩阵表示的角度理解初等变换与标准形的关系?在B组习题中将有练习进行体会
(实际上对矩阵表示的基做``初等变换''就是对表示矩阵做了初等变换,这两种变换行列方向不一致且矩阵互逆).

\section{矩阵的逆}
\subsection{基本概念}
\begin{definition}
    设$A \in \mathbf{M}_n(\mathbf{F})$. 若存在$B \in \mathbf{M}_n(\mathbf{F})$使得$AB=BA=E$,则称矩阵$A$可逆,
    并把$B$称为$A$的逆矩阵,记作 $ B = A^{-1} $.
\end{definition}
注意,逆矩阵定义基于方阵,非方阵没有上述逆矩阵.广义逆矩阵允许非方阵,但那是另一个定义,
我们不需要掌握.对于可逆矩阵,注意以下两个定理:
\begin{theorem}
    可逆矩阵$A$的逆矩阵唯一.
\end{theorem}
\begin{theorem}
    $AB=E \iff A$与$B$互为逆矩阵.
\end{theorem}
这两个定理的证明教材中有,特别注意唯一性的证明,反证法的思路一定要掌握,十分经典.
还需要强调的一点是,逆矩阵来源于逆映射.
\subsection{基本性质}
\begin{enumerate}
    \item 注意没有加法性质(请举出反例),对于数乘有$(\lambda A)^{-1}=\lambda^{-1}A^{-1}$;

    \item $(AB)^{-1}=B^{-1}A^{-1},\enspace (A_1A_2\cdots A_k)^{-1}=A_k^{-1}\cdots A_2^{-1}A_1^{-1}$;

    \item $(A^k)^{-1}=(A^{-1})^k,\enspace A^kA^m=A^{k+m},\enspace (A^k)^m=A^{km}$;

    \item 若$A$和$B$可逆,则$A\neq O$且$B\neq O$能推出$AB\neq O$,并且$A$可逆且$AB=O$可以推出$B=O$,除此之外还有消去律成立,即$A$则有$AB=AC \implies B=C$成立.
\end{enumerate}

还需要熟练掌握可逆矩阵的几个等价条件:
\begin{theorem}
    设$A \in \mathbf{M}_n{\mathbf{F}}$,则下列命题等价:
    \begin{enumerate}
        \item $A$可逆;

        \item $r(A)=n$;

        \item $A$的$n$个行(列)向量线性无关;

        \item 齐次线性方程组$AX=0$只有零解;

        \item $|A|\neq 0$.
    \end{enumerate}
\end{theorem}
\begin{example}
    已知矩阵 $A=\begin{pmatrix}a & b & c \\ d & e & f \\ h & x & y\end{pmatrix}$ 的逆是 $A^{-1}=\begin{pmatrix}-1 & -2 & -1 \\ 2 & 1 & 0 \\ 0 & -3 & -1\end{pmatrix}$,

$B=\begin{pmatrix}a-2b & b-3c & -c \\ d-2e & e-3f & -f \\ h-2x & x-3y & -y\end{pmatrix}$.求矩阵 $X$ 满足:

\[X+\left(B(A^TB^2)^{-1}A^T\right)^{-1}=X\left(A^2(B^TA)^{-1}B^T\right)^{-1}(A+B)\]
\end{example}

\subsection{逆矩阵的求解(基本方法)}
\begin{enumerate}
    \item 利用解线性方程组的方法:假设$AX=b$,使用高斯消元法求解;

    \item 利用初等矩阵的方法(初等行变换为常用方法).
\end{enumerate}

注意,基于初等变换的方法是非常重要的,我们很多时候不要被题目吓到去采用其他
偏门的方法,实际上很多时候拿到一个具体的矩阵求逆,使用的方法就是初等行变换.

\begin{example}
    用上述两种方法求矩阵$A=\begin{pmatrix}1 & -1 & 1 \\ 0 & 1 & 2 \\ 1 & 0 & 4\end{pmatrix}$的逆矩阵.
\end{example}

\subsection{矩阵方程}
\begin{enumerate}
    \item 考虑以下情形(其中出现的矩阵除$X$外均可逆,$X$不一定是列向量):
    \begin{enumerate}[label=(\arabic*)]
        \item $AX=B \implies X=A^{-1}B, \enspace XA=B \implies X=BA^{-1}$;
        \item $AXB=C \implies X=A^{-1}CB^{-1}$;
    \end{enumerate}
    \item 考虑以下情形:$AX=B$但$A$不可逆($X$不一定是列向量),直接高斯消元即可;
    \item 考虑以下求解方式的合理性:
    \begin{enumerate}[label=(\arabic*)]
        \item 若求$A^{-1}$,只需对$(A,E)$只做初等行变换,可以得到$(E,A^{-1})$;
        \item 若求$A^{-1}B$,只需对$(A,B)$只做初等行变换,可以得到$(E,A^{-1}B)$;
        \item 若求$BA^{-1}$,只需对$\begin{pmatrix}
            A \\ B
        \end{pmatrix}$只做初等列变换,可以得到$\begin{pmatrix}
            E \\ BA^{-1}
        \end{pmatrix}$;
        \item 对$\begin{pmatrix}
            A & E \\ E & O
        \end{pmatrix}$的前$n$行与$n$列做相同的行列变换,可以得到$\begin{pmatrix}
            P^\mathrm{T}AP & P^\mathrm{T} \\ P & O
        \end{pmatrix}$.
    \end{enumerate}
\end{enumerate}

\begin{example}
    设$A=\begin{pmatrix}1 & 0 & 0 \\ 1 & 1 & 0 \\ 1 & 1 & 1\end{pmatrix},\
    B=\begin{pmatrix}0 & 1 & 1 \\ 1 & 0 & 1 \\ 1 & 1 & 0\end{pmatrix}$,求矩阵$X$满足:
    \[AXA+BXB=AXB+BXA+A(A-B)\]
\end{example}

\vspace{2ex}
\centerline{\heiti \Large 内容总结}

\vspace{2ex}

\centerline{\heiti \Large 习题}
\vspace{2ex}
{\kaishu }
\begin{flushright}
    \kaishu

\end{flushright}
\centerline{\heiti A组}
\begin{enumerate}
    \item
\end{enumerate}
\centerline{\heiti B组}
\begin{enumerate}
    \item
\end{enumerate}
\centerline{\heiti C组}
\begin{enumerate}
    \item
\end{enumerate}

\chapter{矩阵运算进阶}

\section{分块矩阵}
\subsection{运算性质}
\begin{definition}
    一般的,对于$m \times n$矩阵$A$,如果在行的方向分成$s$块,在列的方向分成$t$
    块,就得到$A$的一个$s \times t$分块矩阵,记作$A=(A_{kl})_{s \times t}$,其中
    $A_{kl}\enspace(k=1,\ldots,s,\enspace l=1,\ldots,t)$称为$A$的子块.
\end{definition}
实际上上述表示方法就是将一般矩阵表示$A=(a_{ij})_{m \times n}$中的$a_{ij}$替换为了小块矩阵,
字母含义并无变化,内层代表索引,外层代表总行列数(只是分块矩阵是块索引和块数).
我们接下来考察分块矩阵的运算性质.
\begin{enumerate}
    \item 分块矩阵的加法:设分块矩阵$A=(A_{kl})_{s \times t}$,$B=(B_{kl})_{s \times t}$,如果$A$与$B$
    对应的子块$A_{kl}$和$B_{kl}$都是同型矩阵,则\[A+B=(A_{kl}+B_{kl})_{s \times t}\]
    由此我们看到分块矩阵加法要求小块形状和行列分块数都一致,实际上回顾一般矩阵加法要求矩阵完全同型即可理解这一要求.

    \item 分块矩阵的数乘:设分块矩阵$A=(A_{kl})_{s \times t}$,$\lambda$是一个数,则
    \[\lambda A=(\lambda A_{kl})_{s \times t}\]
    实际上数乘最好理解,因为如此计算的效果相当于一般矩阵数乘的效果,即给每个元素
    都乘以一个常数$\lambda$.

    \item 分块矩阵的乘法:设$A=(a_{ij})_{m \times n}$,$B=(b_{ij})_{n \times p}$,如果
    把$A$,$B$分别分块为$r \times s$和$s \times t$分块矩阵,且$A$的列分块法与$B$的行分块法相同
    (注意这些条件始终保证可乘性成立),则
    $$AB=\begin{pmatrix}
        A_{11} & A_{12} & \cdots & A_{1s} \\
        A_{21} & A_{22} & \cdots & A_{2s} \\
        \vdots & \vdots & \ddots & \vdots \\
        A_{r1} & A_{r2} & \cdots & A_{rs}
    \end{pmatrix}\begin{pmatrix}
        B_{11} & B_{12} & \cdots & B_{1t} \\
        B_{21} & B_{22} & \cdots & B_{2t} \\
        \vdots & \vdots & \ddots & \vdots \\
        B_{s1} & B_{s2} & \cdots & B_{st}
    \end{pmatrix}=C=(C_{kl})_{r \times t}.$$
    其中$C$是$r \times t$分块矩阵,且$C_{kl}$与一般矩阵计算类似,即为$A$第$k$行块$B$的$l$列块对应元素相乘后相加,即
    \[C_{kl}=A_{k1}B_{1l}+A_{k2}B_{2l}+\cdots+A_{ks}B_{sl},\enspace k=1,\ldots,r,\enspace l=1,\ldots,t\]

    \item 分块矩阵的转置:大、小矩阵都要转置,这是分块矩阵与普通矩阵的一大性质差异;即$s \times t$分块矩阵$A=(A_{kl})_{s \times t}$
    转置后$A^\mathrm{T}=(B_{lk})_{t \times s}$为$t \times s$分块矩阵,且$B_{lk}=A_{kl}^\mathrm{T}$.
    例如$\begin{pmatrix}
        A_{11} & A_{12} \\ A_{21} & A_{22}
    \end{pmatrix}^\mathrm{T}=\begin{pmatrix}
        A_{11}^\mathrm{T} & A_{21}^\mathrm{T} \\ A_{12}^\mathrm{T} & A_{22}^\mathrm{T}
    \end{pmatrix}$.
\end{enumerate}

补充以下注意事项:
\begin{enumerate}
    \item 常见的行列分块方法:将矩阵按行/列分块,注意$A(\beta_1,\ldots,\beta_n)=(A\beta_1,\ldots,A\beta_n)$成立,
    但当$A$在右侧时并不可乘,按行分块也有对称的结论;

    \item 注意分块矩阵求逆,可以直接使用设未知数的方式完成,也可以利用下面即将介绍的分块矩阵初等变换进行解决;

    \item 分析分块矩阵与普通矩阵的运算性质的异同:分块矩阵转置需要注意大小都要转置,注意分块矩阵每一块仍为矩阵,所以当普通矩阵元素的求倒数
    对应于小块的求逆,加法乘法一定要块对应等,但实际上其他很多性质都是将单个元素推广为一块.
\end{enumerate}

\begin{example}
    设\[A=\begin{pmatrix}
        1 & 2 & 0 & 0 & 0 \\
        2 & 5 & 0 & 0 & 0 \\
        0 & 0 & -2 & 1 & 0 \\
        0 & 0 & 0 & -2 & 1 \\
        0 & 0 & 0 & 0 & -2
    \end{pmatrix},\enspace B=\begin{pmatrix}
        1 & 0 & 1 & 0 \\
        -1 & 2 & 3 & 0 \\
        1 & 2 & 0 & 4 \\
        0 & 1 & 2 & 4 \\
        0 & 0 & 1 & 4
    \end{pmatrix}\]
    利用分块矩阵的方法,求$A^2,\enspace AB,\enspace A^\mathrm{T},\enspace A^{-1}$.
\end{example}

\subsection{分块矩阵初等变换(打洞法)*}
分块矩阵的初等变换实际上可以视为一般矩阵初等变换的推广,实际上也有三种相应的推广形式,
即交换两行、对某一行乘以一个可逆矩阵以及对某一行左乘矩阵后加到另一行.它们的计算性质
以及可逆性质的证明比较繁琐,我们这里略去,直接应用即可.实际使用的时候,很多时候都是
使用一种将分块矩阵中的小块视为常数来处理.

分块矩阵初等行变换的一个重要的应用就是``打洞法'',常用于分块矩阵求逆的运算,在之后行列式的一些技巧性处理中也很常见.
例如:
\begin{enumerate}
    \item 当$A$可逆时,我们可以通过初等行变换消去$C$:
    \[ \begin{pmatrix}
        E & O \\ -CA^{-1} & E
    \end{pmatrix}\begin{pmatrix}
        A & B \\ C & D
    \end{pmatrix}=\begin{pmatrix}
        A & B \\ O & D-CA^{-1}B
    \end{pmatrix} \]
    可以继续做列变换消去$B$:
    \[ \begin{pmatrix}
        A & B \\ O & D-CA^{-1}B
    \end{pmatrix}\begin{pmatrix}
        E & -A^{-1}B \\ O & E
    \end{pmatrix}=\begin{pmatrix}
        A & O \\ O & D-CA^{-1}B
    \end{pmatrix} \]
    \item 特别地,对于对称矩阵$\begin{pmatrix}A & B \\ B^\mathrm{T} & D\end{pmatrix}$,其中$A$和$D$也是对称方阵,
    则$A$可逆时,可以通过合同变换消除$B$和$B^\mathrm{T}$,即
    \[ \begin{pmatrix}
        E & -A^{-1}B \\ O & E
    \end{pmatrix}^\mathrm{T}\begin{pmatrix}
        A & B \\ B^\mathrm{T} & D
    \end{pmatrix}\begin{pmatrix}
        E & -A^{-1}B \\ O & E
    \end{pmatrix}=\begin{pmatrix}
        A & O \\ O & D-B^\mathrm{T}A^{-1}B
    \end{pmatrix} \]
\end{enumerate}
接下来求取逆矩阵就很容易了,因为分块对角矩阵求逆矩阵就是对每个小对角块求逆,十分简单,所以解决此类问题
首先要利用分块矩阵初等变换进行对角化(一定注意区分行列变换的左右乘),然后如果$PAQ=\Lambda$,
其中$P$和$Q$为分块初等矩阵,$\Lambda$为分块对角矩阵,利用分块对角矩阵的逆容易计算的
特点计算$Q^{-1}A^{-1}P^{-1}=\Lambda^{-1}$,即可得到$A^{-1}=Q\Lambda^{-1}P$.
\begin{example}
    当$D$可逆时,仿照上面的步骤对角化分块矩阵$\begin{pmatrix}A & B \\ C & D\end{pmatrix}$并求逆矩阵.
\end{example}

\subsection{分块矩阵与数学归纳法}
分块矩阵经常运用在数学归纳法中,我们在之后的课程中也会经常用到这样的思想,
这一思想基于以下内容:

对于$\begin{pmatrix}
    A_1 & \alpha \\ \beta & a_{nn}
\end{pmatrix}$,假设$A_1$可逆,我们有
$$\begin{pmatrix}
    E_{n-1} & 0 \\ -\beta A_1^{-1} & 1
\end{pmatrix}\begin{pmatrix}
    A_1 & \alpha \\ \beta & a_{nn}
\end{pmatrix}=\begin{pmatrix}
    A_1 & \alpha \\ 0 & a_{nn}-\beta A_1^{-1}\alpha
\end{pmatrix}$$

\begin{example}
    若$n$阶矩阵$A$的各阶左上角子块矩阵都可逆,则存在主对角元全为$1$的下三角矩阵$L$和上三角矩阵$U$,使得$A=LU$($L$-$U$分解).
\end{example}

\section{特殊矩阵}
本节将会介绍一些常见的特殊矩阵以及它们常用的基本性质,还有一些将在特征值专题中讲解.
\subsection{对角矩阵}
我们一般记主对角矩阵为$\diag(d_1,d_2,\dots,d_n)$,准对角矩阵为$\diag(A_1,A_2,\dots,A_n)$.
下面是对角矩阵的一个基本定理,它很简单,但是很重要:
\begin{theorem}
    设$A$是一个$s \times n$矩阵,把$A$写成列向量与行向量的形式,分别为

    \[ A = \begin{pmatrix}\alpha_1 & \alpha_2 & \cdots & \alpha_n\end{pmatrix} = \begin{pmatrix} \beta_1 \\ \beta_2 \\ \vdots \\ \beta_n \end{pmatrix} \]
    则
    \begin{gather*}
        \begin{pmatrix}\alpha_1 & \alpha_2 & \cdots & \alpha_n\end{pmatrix}
        \begin{pmatrix}
            d_1 & & & \\
            & d_2 & & \\
            & & \ddots & \\
            & & & d_n
        \end{pmatrix} = \begin{pmatrix}d_1\alpha_1 & d_2\alpha_2 & \cdots & d_n\alpha_n\end{pmatrix} \\
        \begin{pmatrix}
            d_1 & & & \\
            & d_2 & & \\
            & & \ddots & \\
            & & & d_n
        \end{pmatrix} \begin{pmatrix} \beta_1 \\ \beta_2 \\ \vdots \\ \beta_n \end{pmatrix} = \begin{pmatrix} d_1\beta_1 \\ d_2\beta_2 \\ \vdots \\ d_n\beta_n \end{pmatrix}
    \end{gather*}

    即$A$右乘对角矩阵$\diag(d_1,d_2,\ldots,d_n)$相当于给$A$的第$i$列元素都乘以$d_i$,
    $A$左乘对角矩阵$\diag(d_1,d_2,\ldots,d_n)$相当于给$A$的第$i$行元素都乘以$d_i$.
\end{theorem}
\begin{theorem}
    (请自行完成以下内容的补充)

    对角矩阵以及准对角矩阵的三则运算、可逆性以及逆运算、乘方运算等规则.
\end{theorem}

\subsection{上(下)三角矩阵}
\begin{theorem}
    已知$A,B$都是上三角矩阵,且设$A$的主对角元素分别为$a_{11},\ldots,a_{nn}$,B的主对角元素分别为
    $b_{11},\ldots,b_{nn}$,则
    \begin{enumerate}
        \item $A^{\mathrm{T}}, B^\mathrm{T}$都是下三角矩阵;

        \item $AB$仍然是上三角矩阵,且$AB$的主对角元素为$a_{11}b_{11},\ldots,a_{nn}b_{nn}$;

        \item $A$可逆的充要条件是其主对角元均不为$0$,且$A$可逆时,$A^{-1}$也是上三角矩阵,并且$A^{-1}$的主对角元素分别为$a_{11}^{-1},\ldots,a_{nn}^{-1}$.
    \end{enumerate}
\end{theorem}

\begin{example}
    已知$A_1,\ldots,A_n$是$n$个对角元都为$0$的上三角矩阵,证明:$A_1A_2\cdots A_n=O$.
\end{example}

\subsection{基本矩阵}
只有一个元素为1,其余元素全为0的矩阵称为基本矩阵,第$i$行第$j$列元素为1的基本矩阵记为$E_{ij}$,
他们具有如下性质(可以回忆左右乘对应行列变换):
\begin{theorem}
    \begin{enumerate}
        \item $AE_{ij}$的结果就是把$A$的第$i$列移到第$j$列的位置,其余元素都为$0$的矩阵;

        \item $E_{ij}B$的结果就是把$B$的第$j$行移到第$i$行的位置,其余元素都为$0$的矩阵;

        \item $E_{ik}E_{kj}=E_{ij}$,当$k \neq l$时,有$E_{ik}E_{lj}=O$.
    \end{enumerate}
\end{theorem}

\subsection{其他矩阵}
其他矩阵如正交矩阵、置换矩阵、幂等矩阵、幂零矩阵等,有的会在稍后介绍部分性质,有的
则会在课程进行中或者后续课程中再见到它们.

\section{矩阵的逆进阶求法}
\subsection{给定多项式求逆矩阵}
此类题目相信大家已经有所见识,实际上就是通过一些初中所学的因式分解等基本变换得到需要求逆的矩阵与另一个矩阵相乘
可以得到单位矩阵(的一个倍数).
\begin{example}
    设$A$为非零矩阵,且$A^3=O$,证明:$E+A$和$E-A$都可逆.
\end{example}

\begin{example}
    若$X$,$Y$是两个列向量,且$X^\mathrm{T}Y=2$,证明:
    \begin{enumerate}
        \item $(XY^\mathrm{T})^k=2^{k-1}(XY^{\mathrm{T}})$;

        \item 如果$A=E+XY^\mathrm{T}$,则$A$可逆,并求其逆矩阵.
    \end{enumerate}
\end{example}

\subsection{利用分块矩阵初等变换*}
我们在前面已经讲解过了打洞法的基础题型,这里再给出一些例子:
\begin{example}
    设$A$、$B$为$n$阶矩阵,证明:若$E\pm AB$可逆,则$E\pm BA$可逆.
\end{example}
\begin{example}
    设$A$为$n$阶矩阵,$B$、$C$分别为$n \times m$、$m \times n$阶矩阵,
    证明:$E_m+CA^{-1}B$可逆$\iff A+BC$可逆.
\end{example}

\subsection{求逆的分式思想*}
虽然矩阵没有除法运算,但是我们如果将$(E-A)^{-1}$写成$\dfrac{E}{E-A}$,再类比泰勒展开
\[\frac{1}{1-x}=1+\sum_{n=1}^\infty x^n \quad x\in (-1,1)\]我们可以得到(不严谨!只能用来解题的时候当作初步的思路!)
\[(E-A)^{-1}=\frac{E}{E-A}=E+A+A^2+\cdots\]

\begin{example}
    已知方阵$A$满足$A^k=O$,其中$k$是一个正整数,求$E-A$的逆.
\end{example}

\begin{example}
    设$A$,$B$分别是$n \times m$和$m \times n$的矩阵,且$E_n \pm AB$可逆,则$E_m \pm BA$可逆.
\end{example}
不难发现这一例是前述4.3.1节中最后一个例题的推广. % TODO 编号

\subsection{提逆思想*}
这一思想的来源是矩阵逆没有加减相关的运算法则,因此我们需要提逆产生一些乘积项来解决问题.
\begin{example}
    设$A$是$n$阶方阵,且$E-A$,$E+A$和$A$都可逆,证明:$(E-A^{-1})^{-1}+(E-A)^{-1}=E$.
\end{example}

\section{矩阵的迹*}
\subsection{基本概念与性质}
\begin{definition}
    一个方阵$A$的所有主对角元素之和称为$A$的迹,记为$\tr(A)$.
\end{definition}
迹的常见性质如下:
\begin{theorem}
    已知$A, B$是两个$n$阶矩阵,$k$是一个常数,则
    \begin{enumerate}
        \item $\tr(kA) = k\tr(A)$;

        \item $\tr(A+B)=\tr(A)+\tr(B)$;

        \item $\tr(AB)=\tr(BA)$;

        \item 如果$A$是实矩阵,则$A=O \iff \tr(A^{\mathrm{T}}A)=0$.
    \end{enumerate}
\end{theorem}

我们先来看一个基本的例子练习一下:
\begin{example}
    证明:不存在方阵$A$、$B$使得$AB-BA=E$.
\end{example}

接下来我们介绍一个性质,这一性质的证明需要利用矩阵的秩中讲到的一些技巧:
\begin{theorem}
    已知 $n$ 阶矩阵 $A$ 的秩为 1,证明:$A^k=\tr(A)^{k-1}A$.
\end{theorem}
这一定理的证明需要用到矩阵的分解,在2020年吴志祥老师班期中考试有出现,可以参考
辅学网站上的解答.

当然我们还会在特征值一章再次见到矩阵的迹,相关内容在最后一个专题会展开讲述.

\subsection{幂零矩阵}
幂零矩阵是一种特殊的矩阵,幂零矩阵$A$存在一个正整数$k$使得$A^k=O$,
它具有如下性质(部分需要用到特征值,所以最后一个专题还会提及):
\begin{theorem}
    若$n$阶矩阵$A$为幂零矩阵,则
    \begin{enumerate}
        \item $A^n=O$;

        \item $A\pm E$均为可逆矩阵;

        \item 幂零矩阵对应的线性变换一定存在一个矩阵表示使得矩阵为上三角矩阵且对角线元素全为0;

        \item $A$ 为幂零矩阵 $\iff \forall k \in \mathbf{N}_+,\enspace\tr(A^k)=0$.
    \end{enumerate}
\end{theorem}

\begin{example}
    若$A$、$B$为两个$n$阶矩阵且满足$AB-BA=A$,证明:
    \begin{enumerate}
        \item $A$不可逆;

        \item $A$是幂零矩阵.
    \end{enumerate}
\end{example}

\section{矩阵的幂}
\begin{enumerate}
    \item 找规律

    在矩阵的转置中我们已经见识了一种找规律的方式,下面是一种类似的题型:
    \begin{example}
        计算$(PAQ)^k$,其中
        \[P=\begin{pmatrix}2 & 3 \\ 1 & 2\end{pmatrix},\enspace A=\begin{pmatrix}2 & 0 \\ 0 & -1\end{pmatrix},\enspace Q=\begin{pmatrix}2 & -3 \\ -1 & 2\end{pmatrix}\]
    \end{example}

    \begin{example}
        设$A=\begin{pmatrix}0 & -1 & 0 \\ 1 & 0 & 0 \\ 0 & 0 & -1 \end{pmatrix},\enspace P^{-1}AP=B$,求$B^{2004}-2A^2$.
    \end{example}

    还有一种找规律基于幂等矩阵,显然幂等矩阵的任意次方都与其平方相等是很好的性质,另一种找规律基于对合矩阵,即平方等于单位矩阵的矩阵,我们这里
    主要与大家分享关于幂零矩阵的方法,例子如下:
    \begin{example}
        求$A=\begin{pmatrix}a & 1 & 0 & 0 \\ 0 & a & 1 & 0 \\ 0 & 0 & a & 0 \\ 0 & 0 & 0 & a \end{pmatrix}^n$.
    \end{example}
    在上例中,我们采用将矩阵分为$A=tE+B$的方法,会发现矩阵$B$为上三角矩阵且对角线上全为0,是典型的幂零矩阵,利用这一性质可以快速解题.
    \item 数学归纳法
    \begin{example}
        求$A=\begin{pmatrix}\cos\alpha & \sin\alpha \\ -\sin\alpha & \cos\alpha\end{pmatrix}^n$.
    \end{example}
    这一问题对应我们常见的旋转变换.
    \begin{example}
        证明$\begin{pmatrix}
            a & c \\ 0 & b
        \end{pmatrix}^n=\begin{pmatrix}
            a^n & (a^{n-1}+a^{n-2}b+\dots+b^{n-1})c \\ 0 & b^n
        \end{pmatrix}$.
    \end{example}
    \item 利用秩为1的矩阵

    在我们有关秩的讨论中已经提到了如果$A$是秩为1的矩阵,那么$A^n=\tr(A)^{n-1}A$,我们可以利用这一性质解决问题.
    \begin{example}
        已知$M$是秩为 1 的矩阵,记$\tr(M)=b$,讨论$(aE+M)^n$的计算结果.
    \end{example}

    \begin{example}
        已知$A$是数域$P$上的一个$2$阶方阵,且存在正整数$l$使得$A^l=O$,证明:$A^2=O$.
    \end{example}
    事实上,我们在幂零矩阵的讨论中已经提及了上例的一般情况.

    \begin{example}
        已知数列$\{a_n\},\enspace\{b_n\}$满足$a_0=-1,\enspace b_0=3$,且
        $$\begin{cases}
            a_n=3a_{n-1}+b_{n-1}+2^{n-1} \\ b_n=2a_{n-1}+4b_{n-1}+2^n
        \end{cases}$$
        求$\{a_n\}\enspace\{b_n\}$的通项公式.
    \end{example}
    \item 利用初等矩阵的性质
    \begin{example}
        设$A$为三阶矩阵,$P$为三阶可逆矩阵,$P^{-1}AP=B$,其中$P=\begin{pmatrix}
            0 & 2 & -1 \\ 1 & 1 & 2 \\ -1 & -1 & -1
        \end{pmatrix}$,$B=\begin{pmatrix}
            0 & 0 & -1 \\ 0 & -1 & 0 \\ -1 & 0 & 0
        \end{pmatrix}$,求$A^{2022}$.
    \end{example}

    \item 利用对角化

    若一个矩阵$A$可对角化,即存在可逆矩阵$P$使得$A=P^{-1}\Lambda P$(其中$\Lambda$为对角矩阵),
    在这种形式下$A$的幂是很好求的.
    \begin{example}
        已知$A=\begin{pmatrix}
            0 & \cfrac{1}{2} & \cfrac{1}{2} \\ 1 & -\cfrac{1}{2} & \cfrac{1}{2} \\ 1 & -\cfrac{1}{2} & \cfrac{1}{2}
        \end{pmatrix}$,求$A^n$.
    \end{example}

\end{enumerate}

\vspace{2ex}
\centerline{\heiti \Large 内容总结}

\vspace{2ex}

\centerline{\heiti \Large 习题}
\vspace{2ex}
{\kaishu }
\begin{flushright}
    \kaishu

\end{flushright}
\centerline{\heiti A组}
\begin{enumerate}
    \item
\end{enumerate}
\centerline{\heiti B组}
\begin{enumerate}
    \item
\end{enumerate}
\centerline{\heiti C组}
\begin{enumerate}
    \item
\end{enumerate}

\chapter{矩阵的秩}
本节内容理解难度较大,事实上这里利用了很多线性空间与线性映射的思想,
也有很多技巧性的内容,因此希望各位同学根据自己实际情况理解掌握.虽然很推荐
这部分内容采用与教材不同的思路去理解,更多利用线性空间与线性映射的抽象知识思考,但是如果
理解起来有一定困难也记住一些结论去解决一些问题.

还有一部分应当属于本节的内容将在专题五线性方程组的部分提及,因此本节不再专门
讲解利用线性方程组的思想解决矩阵的秩相关问题的部分.

\section{矩阵的秩}
我们首先给出矩阵的三个秩的定义:
\begin{definition}
    设$A$是线性映射$\sigma$对应的矩阵,我们把$\sigma$的秩也称为矩阵$A$的秩,
    记为$r(A)$.我们将矩阵$A$的所有行向量组成的秩称为$A$的\keyterm*{行秩}[row rank],
    所有列向量组成的向量组的秩称为$A$的\keyterm*{列秩}[column rank].
\end{definition}
对于以上三个秩我们有重要的定理如下:
\begin{theorem}
    任意矩阵的秩 = 行秩 = 列秩.
\end{theorem}
这一定理的证明,矩阵的秩 = 列秩的部分根据线性映射的相关概念是显然的,行秩的部分
教材中有较为繁琐的证明,在本讲义下面的内容会有更加形象的解释.

实际上,这一定理有两个重要的直接推论,一是将求矩阵的秩的问题转化为求矩阵行/列极大线性无关向量组的问题,
第二是矩阵的秩等于其转置的秩.

可能很多同学对于行秩、列秩相等以及转置的几何意义很感兴趣.实际上我们有两种获得转置矩阵的
方式,第一种来源于我们之前讨论的对偶空间上的线性映射对应的矩阵,这种方式可能不够直观.
另一种获得的方法基于内积,感兴趣的同学可以了解矩阵的伴随(不是行列式中的伴随矩阵).

我们可以研究矩阵及其转置的关系,我们可以用一个图形来表示:

\begin{figure}[h]
    \centering
    \small
    \begin{tikzpicture}
        \tikzset{->-/.style={decoration={
            markings,
            mark=at position .6 with {\arrow{stealth}}},postaction={decorate}}}

        \draw[rotate=45] (0,6) rectangle (-3,3) rectangle (-5,0)
            (-3,3) rectangle(-3.35,3.35)
            coordinate (xr) at (-2,4)
            coordinate (xn) at (-4,2)
            coordinate (x) at (-2,2)
            coordinate (0n) at (-3,3)
            node at (-1,5) {行空间}
            node at (-4,1) {$A$的核空间}
            node at (-1.5,6.5) {$\dim r$}
            node at (-4,4) {$\mathbf{R}^n$}
            node at (-6,3) {$\dim n-r$};

        \draw[rotate=30] (6,2) rectangle (3.5,-2) rectangle (0,-4)
            (3.5,-2) rectangle (3.85,-2.35)
            coordinate (b) at (4.5,0.5)
            coordinate (0m) at (3.5,-2)
            node at (5,1.5) {列空间}
            node at (2,-3) {$A^{\mathrm{T}}$的核空间}
            node at (7,0) {$\dim r$}
            node at (5,-3) {$\mathbf{R}^m$}
            node at (4,-4.5) {$\dim m-r$};

        \foreach \point in {xr, x, xn, 0n, b, 0m}
            \fill[black] (\point) circle (1pt);

        \node [left] at (xr) {$x_r$};
        \node [below right] at (x) {$x=x_r+x_n$};
        \node [left] at (xn) {$x_n$};
        \node [right] at (0n) {0};
        \node [right] at (b) {$b$};

        \draw[->-,very thick] (xr) -- node[above,sloped] {$Ax_r = b$} (b);
        \draw[->-,very thick] (x) -- node[below,sloped] {$Ax = b$} (b);
        \draw[->-,very thick] (xn) -- node[below,sloped] {$Ax_n = 0$} (0m);

        \draw[dashed,thick] (xr) -- (x) -- (xn);

    \end{tikzpicture}
\end{figure}

我们观察到以下几点:
\begin{enumerate}
    \item 矩阵的行空间与解空间(零空间)互为正交补(直观理解两个空间就是互相垂直且互为补空间),这一点应当是在正交的内容中有所提及的;
    \item 矩阵的列空间与其转置矩阵的零空间互为正交补,这一点实际与上一条等价.
\end{enumerate}

接下来我们来看行秩(列秩比较显然,此处不再详细展开).我们首先得到解空间的维数,这可以直接
根据维数公式得到:$\dim \ker A =n-r(A)$,根据正交补的性质,我们的可以得到行秩即为
$n-(n-r(A))=r(A)$.于是我们得到了一个基于正交补的行秩解释.

\section{相抵标准形}
此处我们需要首先回顾一个基本定理:
\begin{theorem}
    初等变换不改变矩阵的秩(包括行变换和列变换).
\end{theorem}
由这一定理我们可以推导出相抵标准形:
\begin{theorem}
    若$r(A_{m \times n})=r$,则存在可逆矩阵$P$和$Q$,使得
    $$PAQ=\begin{pmatrix}
        E_r & 0 \\ 0 & 0
    \end{pmatrix}=U_r,$$
    其中$E_r$表示$r$阶单位矩阵.
\end{theorem}
这一定理证明直接使用定理4以及可逆矩阵可以拆分为初等矩阵的乘积即可.
其中$U_r$称为相抵标准形.我们称两个矩阵相抵即两个矩阵可以通过一系列
初等变换可以互相转化.由此我们得到关于矩阵相抵的两个等价命题:

1. 矩阵$A$与$B$相抵$\iff$存在可逆矩阵$P$和$Q$使得$PAQ=B$;

2. 矩阵$A$与$B$相抵$\iff r(A)=r(B)$.

\begin{example}
    设$A=\begin{pmatrix}
        1 & 0 & 2 & -4 \\ 2 & 1 & 3 & -6 \\ -1 & -1 & -1 & 2
    \end{pmatrix}$. 求
    \begin{enumerate}
        \item $A$的秩$r$和相抵标准形;

        \item 3 阶可逆矩阵$P$和 4 阶可逆矩阵$Q$使得$PAQ=\begin{pmatrix}
            E_r & 0 \\ 0 & 0
        \end{pmatrix}$.
    \end{enumerate}
\end{example}

关于相抵标准形,我们需要在此补充一个常用的技术,即相抵标准形的分解:

我们对$s \times n$矩阵$\begin{pmatrix}
    E_r & O \\ O & O
\end{pmatrix}$有一种很重要的分解:
\[\begin{pmatrix}
    E_r & O \\ O & O
\end{pmatrix}=\begin{pmatrix}
    E_r \\ O
\end{pmatrix}\begin{pmatrix}
    E_r & O
\end{pmatrix}\]
由此我们可以知道任意一个非零矩阵都可以被分解成一个列满秩矩阵和一个
行满秩矩阵的乘积:

$$A=P\begin{pmatrix}
    E_r & O \\ O & O
\end{pmatrix}Q=P\begin{pmatrix}
    E_r \\ O
\end{pmatrix}\begin{pmatrix}
    E_r & O
\end{pmatrix}Q$$
记$P_1=P\begin{pmatrix}
    E_r \\ O
\end{pmatrix}$,$Q_1=\begin{pmatrix}
    E_r & O
\end{pmatrix}Q$,则$A=P_1Q_1$,且$P_1$和$Q_1$分别为列满秩、行满秩矩阵.

我们可以利用相抵标准形解决很多问题,例如下一节中部分秩不等式的证明:
\begin{example}
    \begin{enumerate}
        \item $r\begin{pmatrix}
            A & O \\ O & B
        \end{pmatrix}=r(A)+r(B)$.

        \item $r\begin{pmatrix}
            A & D \\ O & B
        \end{pmatrix}\geqslant r(A)+r(B),\enspace r\begin{pmatrix}
            A & O \\ C & B
        \end{pmatrix}\geqslant r(A)+r(B)$.
    \end{enumerate}
\end{example}

\section{秩不等式}
本节的内容实际上部分内容有一定的技巧性,对于荣誉课程来说还是以理解为主(所以
其实本节中提到的很多内容都只是介绍性的,而非要求大家熟练掌握,但是遇见了要有
一些基本的思路而不能完全不理解),可能下面列出定理的时候显得比较繁冗,但是实
际上我们更重视其中的理解而非硬套结论.

我们首先给出一些常见的秩相关的不等式或等式,这些式子希望各位同学能够理解其含义,
而非机械记忆套用.下面这些等式/不等式的证明方式非常多,实际上可以利用之前所说化为
相抵标准形的方法,也可以利用线性相关性的方法,也可以回到线性映射进行考量.总之
解决的方法非常多,希望各位同学能熟练推导理解.
\begin{enumerate}
    \item $r(A)=r(PA)=r(AQ)=r(PAQ)$,其中$P$、$Q$可逆
    \item $|r(A)-r(B)|\leqslant r(A\pm B) \leqslant r(A)+r(B)$
    \item $r(AB) \leqslant \min\{r(A),\ r(B)\}$
    \item $r(A)=r(A^\mathrm{T})=r(AA^\mathrm{T})=r(A^\mathrm{T}A)$(注意第二个等号需要实矩阵作为前提条件)
    \item $A \in \mathbf{F}^{s \times n}$,$B \in \mathbf{F}^{n \times m}$,
    则$r(AB) \geqslant r(A)+r(B)-n$.(可以视为结论6的推论,特例$AB=O$时有$r(A)+r(B)\leqslant n$)
    \item $r(ABC) \geqslant r(AB)+r(BC)-r(B)$.(还可以考虑$A,B,C$相等的特殊情况的结果)
\end{enumerate}

分块矩阵的相关公式在上一小节的例题中已经书写过,此处不再重复.

一般而言,解决较为复杂的秩的问题时,我们可以采用如下方法:
\begin{enumerate}
    \item 利用(分块)矩阵初等变换;

    \item 利用线性方程组解的一般理论(将在专题五讲解);

    \item 利用向量组线性相关性;

    \item 利用已知的矩阵秩的等式和不等式.实际上等式很多时候基于可逆矩阵变换或者两个不等号夹逼.
\end{enumerate}

相关方法的应用都在本节最后的习题中有所体现,当然首要的任务是掌握上述基本的秩不等式的证明,
很多也利用了上面的思想,并且解法不唯一.

\vspace{2ex}
\centerline{\heiti \Large 内容总结}

\vspace{2ex}

\centerline{\heiti \Large 习题}
\vspace{2ex}
{\kaishu }
\begin{flushright}
    \kaishu

\end{flushright}
\centerline{\heiti A组}
\begin{enumerate}
    \item
\end{enumerate}
\centerline{\heiti B组}
\begin{enumerate}
    \item
\end{enumerate}
\centerline{\heiti C组}
\begin{enumerate}
    \item
\end{enumerate}

\input{./专题/第十一讲 行列式(I).tex}
\input{./专题/第十二讲 行列式计算进阶.tex}
\input{./专题/第十三讲 朝花夕拾.tex}
\input{./专题/第十四讲 多项式.tex}
\input{./专题/第十五讲 不变子空间.tex}
\input{./专题/第十六讲 相似标准形.tex}
\input{./专题/第十七讲 多项式的进一步讨论.tex}
\input{./专题/第十八讲 若当标准形.tex}
\chapter{内积空间}

\section{内积和范数}
\subsection{内积和范数的定义及性质}
前面研究的所有空间都是线性空间,只注重于线性结构,忽视了向量的度量性质,如向量的长度、夹角等。
但度量性质恰是在分析、几何问题中不可缺少的。故从此章起,我们引入度量的概念,将线性空间推广为内积空间。

\vspace{2ex} 

内积的引入始于我们曾在高中研究过的 $\mathbf{R}^{2}$ 与 $\mathbf{R}^{3}$ 上的向量点积,范数则是始于向量的长度概念。
内积即是点积性质的推广,本质上就是一个函数,它把 $ V $ 中元素的每个有序对 $(u, v)$ 都映射成一个数 
$ \langle u, v \rangle \in \mathbf{F}$,并且具有以下性质:

1. 正定性:$\forall v \in V, \enspace \langle v, v \rangle \geqslant 0, \enspace \langle v, v \rangle = 0 \Leftrightarrow v = 0$; 

2. 第一个位置的加性:$\forall u, v, w \in V, \enspace \langle u + v, w \rangle = \langle u, w \rangle + \langle v, w \rangle$;

3. 第一个位置的齐性:$\forall \lambda \in \mathbf{F}, \enspace \forall u, v \in V, 
\enspace \langle \lambda u, v \rangle = \lambda \langle u, v \rangle$;

4. 共轭对称性:$\forall u, v \in V, \enspace \langle u, v \rangle = \overline{\langle v, u \rangle}$.

每个实数都等于它的复共轭,所以在处理实向量空间时,共轭对称性实际上转变为对称性,即:
$\forall u, v \in V, \enspace \langle u, v \rangle = \langle v, u \rangle$

而由以上定义,我们可以快速得到以下性质:

1. 对于每个取定的 $u \in V$,将 $ v $ 变为 $\langle v, u \rangle$ 的函数是 $ V $ 到 $\mathbf{F}$ 的线性映射.

2. $\forall u \in V, \enspace \langle 0, u \rangle = \langle u, 0 \rangle = 0$.

3. $\forall u, v, w \in V, \enspace \langle u, v + w \rangle = \langle u, v \rangle + \langle u, w \rangle$.

4. $\forall \lambda \in \mathbf{F}, \enspace \forall u, v \in V, \enspace \langle u, \lambda v \rangle = \overline{\lambda} \langle u, v \rangle$.

其实从以上的定义与性质可以发现,实内积空间上的内积与我们之后要提到的双线性函数有着密不可分的联系——
实线性空间上的正定对称双线性函数实际上就是该空间上的一个内积,在此先按下不表。

\vspace{2ex} 

内积定义完成后,便可由该内积确定一个相应的范数:对于 $v \in V$,$v$ 的范数(记作 $ \lVert v \rVert $)
定义为 $ \lVert v \rVert = \sqrt{\langle v, v \rangle}$. \enspace 并且具有以下性质:

1. $\forall v \in V, \enspace \left\lVert v \right\rVert = 0 \Leftrightarrow v = 0$.

2. $\forall v \in V, \enspace \forall \lambda \in \mathbf{F}, 
\enspace \left\lVert \lambda v \right\rVert  = \left\lvert \lambda \right\rvert \lVert v \rVert$.

上述性质留给读者自证,从中我们也能发现一个普遍原理:处理范数的平方通常比直接处理范数更容易。

\vspace{2ex} 

以下给出几个内积和范数的示例:

\begin{example}
    \textup{(1)}$\mathbf{F}^{n}$ 上的欧几里得内积定义为:
    \[\left\langle (w_1, \ldots, w_n), (z_1, \ldots, z_n)\right\rangle = w_1\overline{z_1} + \cdots + w_n\overline{z_n} = \vec{W}\overline{\vec{Z}}^{T}\]
    对应范数:
    \[\left\lVert (z_1, \ldots, z_n) \right\rVert  = \sqrt{\lvert z^2_1 \rvert + \cdots + \lvert z^2_n \rvert}\]

    \textup{(2)}定义在 $ \left[-1, 1\right] $ 上的连续实值函数构成的向量空间可定义内积如下:
    \[\left\langle f, g\right\rangle = \int_{-1}^1f(x)g(x)\mathrm{d}x\]
    对应范数:
    \[\left\lVert f \right\rVert = \sqrt{\int_{-1}^1(f(x))^2\mathrm{d}x}\]
\end{example}

\subsection{正交的定义,基于正交的性质}

以下给出一个关键定义:

\begin{definition}
    两个向量 $u, v \in V$ 称为正交的,如果 $\langle u, v\rangle = 0$.
\end{definition}

该定义中向量的次序是无关紧要的,因为 $\langle u, v\rangle = 0 \Leftrightarrow \langle v, u\rangle = 0$. 

\vspace{2ex} 

那么正交的定义关键在何处呢?以下给出 $R^{n}$ 空间上夹角的定义以供理解
(证明良定义需要用到 Cauchy-Schwarz 不等式):

\begin{definition}
    设 $u, v \in R^{n}$,则 $u, v$ 的夹角 $ \theta $ 为 
    $ \theta = \arccos \frac{\langle u, v\rangle}{\lVert u \rVert \lVert v \rVert}$.
\end{definition}

那么我们可以发现,当两向量正交时,它们的夹角就是 $\frac{\pi}{2}$,
也就是我们在几何中常说的垂直,它能将我们导向一些重要的定理。

\vspace{2ex} 

让我们从一些简单的结果开始研究正交性,比如正交性与 0 的关系:

1. 0 正交与 $V$ 中的任意向量.

2. 0 是 $V$ 中唯一一个与自身正交的向量.

然后是熟悉的勾股定理在内积空间上的推广:

\begin{theorem}
    设 $u, v$ 是 $V$ 中的正交向量,则 $\lVert u + v \rVert^2 = \lVert u \rVert^2 + \lVert v \rVert^2 $ 
\end{theorem}

注意勾股定理的逆定理仅在实内积空间上成立。

\vspace{2ex} 

借助于正交的性质,我们能够简化很多与内积相关的计算,
进而会很自然的思考这样一个问题:一个向量能否分解两个互相正交的向量?

从而便引进了正交分解:

\begin{theorem}
    设 $u, v \in V$ 且 $v \neq 0$. 令 $ c = \frac{\langle u, v\rangle}{\lVert v \rVert^2}, 
    \enspace w = u - \frac{\langle u, v\rangle}{\lVert v \rVert^2}v$. 则 $\langle w, v\rangle = 0$ 
    且 $u = cv + w$
\end{theorem}

而通过正交分解,我们可以证明一个数学中最重要的不等式(之一):Cauchy-Schwarz 不等式。

\begin{theorem}
    设 $u, v \in V$. 则 $\left\lvert \left\langle u, v\right\rangle \right\rvert \leqslant \lVert u \rVert\lVert v \rVert$. 
    等号成立当且仅当 $u, v$ 之一是另一个的标量倍.
\end{theorem}

也可以通过引入参数,利用二次三项式的判别式证明。

\vspace{2ex}

借助 Cauchy-Schwarz 不等式,我们可以得到三角不等式:

\begin{theorem}
    设 $u, v \in V$. 则 $\lVert u, v \rVert \leqslant \lVert u \rVert + \lVert v \rVert$.
    等号成立当且仅当 $u, v$ 之一是另一个的非负标量倍.
\end{theorem}

其几何解释就是俗称的三角形两边之和小于第三边。

\vspace{2ex}

另一个与几何相关的结论就是平行四边形恒等式:

\begin{theorem}
    设 $u, v \in V$. 则 $ \lVert u + v \rVert^{2} + \lVert u - v \rVert^{2} = 2(\lVert u \rVert^{2} + \lVert v \rVert^{2})$.
\end{theorem}

其几何解释为任意的平行四边形两对角线的长度的平方和等于四边长度的平方和。

\vspace{2ex}


以下为另外几个与内积有关的恒等式,我们会在证明正规算子和自伴算子的性质时运用到它们:

\begin{example}
    证明下列式子成立:

    \textup{(1)} $\textbf{F} = \textbf{R}$ 时:
    \[(i) \enspace \langle u, v\rangle  = \frac{1}{4}\left( \lVert u + v \rVert^2 - \lVert u - v \rVert^2\right);\]
    \[(ii) \enspace \langle Tu, v\rangle + \langle Tv, u\rangle = 
    \frac{1}{2}\left(\langle T(u + v), u + v\rangle - \langle T(u - v), u - v\rangle\right).\]

    \textup{(2)} $\textbf{F} = \textbf{C}$ 时:
    \[(i) \enspace \langle u, v\rangle  = \frac{1}{4}\left( (\lVert u + v \rVert^2 - \lVert u - v \rVert^2) + i(\lVert u + iv \rVert^2 - \lVert u - iv \rVert^2) \right);\]
    \begin{align*}
        (ii) \enspace \langle Tu, v\rangle = & \frac{1}{4}  ((\langle T(u + v), u + v\rangle - \langle T(u - v), u - v\rangle) \\ 
                                             & + i(\langle T(u + iv), u + iv\rangle + \langle T(u - iv), u - iv\rangle)).
    \end{align*}

\end{example}


\section{标准正交基}

这一节我们将沿着正交的路径接着往下走,看看如果整个向量组乃至一组基都是单位化的且互相正交的话,会有怎样的性质。
我们也会讲解如何获取这样一组基的算法,并介绍 Riesz 表示定理,其揭示了线性泛函和内积的深刻联系。

\begin{definition}
    标准正交(规范正交):如果一个向量组的每个向量范数都是 1 且与其他向量正交
    则称这个向量组是标准正交(规范正交)的。
\end{definition}

在本书的剩余部分中我们都称此性质为标准正交。

由以上定义,我们得出:$ V $ 上的向量组 $ e_1, \ldots , e_n $ 是标准正交的,如果
\[ \langle e_j, e_k \rangle = \delta _{jk} = 
\begin{cases}
    1, \enspace j = k; \\
    0, \enspace j \neq k.
\end{cases}\]

标准正交组的优势在于处理其线性组合的范数很方便。

\begin{theorem}
    若 $e_1, \ldots , e_m$ 是 $ V $ 中的标准正交向量组,则对 $\forall a_1, \ldots a_m \in \mathbf{F}$ 均有
    \[ \lVert a_1e_1 + \cdots + a_ne_n\rVert^2 = \lvert a_1 \rvert^2 + \cdots + \lvert a_n \rvert^2.\]
\end{theorem}

反复使用勾股定理即可证明。该定理也有一个重要推论:

\begin{theorem}
    任何标准正交向量组都是线性无关的.
\end{theorem}

令其线性组合为 0 向量即证。

\begin{example}
    设 $e_1, \ldots , e_m$ 是 $ V $ 的标准正交组. 设 $ v \in V $. 证明
    \[ \lVert v \rVert^2 = \lvert \langle v, e_1\rangle \rvert^2 + \cdots + \lvert \langle v, e_m\rangle \rvert^2 \]
    当且仅当 $ v \in \mathrm{span}(e_1, \ldots , e_m).$
\end{example}

\vspace{2ex}

既然标准正交组都是线性无关的,很自然我们就会想到在线性空间中最有用的线性无关组:基。
也就有了标准正交基的定义。

\begin{definition}
    $ V $ 的标准正交基是 $ V $ 中的标准正交向量组构成的基.
\end{definition}

而由向量组确定为基的等价条件,易知长度为 $\mathrm{dim} \enspace V$ 的标准正交向量组都是
$ V $ 的标准正交基。    

标准正交基的优势就在于表出向量的表出系数可以提前确定。

\begin{theorem}
    设 $e_1, \ldots e_n$ 是 $ V $ 标准正交基且 $ v \in V$. 则
    \[ v = \langle v, e_1 \rangle e_1 + \cdots + \langle v, e_n \rangle e_n \]
    且
    \[ \lVert v \rVert^2 = \lvert \langle v, e_1\rangle \rvert^2 + \cdots + \lvert \langle v, e_m\rangle \rvert^2. \]
\end{theorem}

此为例 19.3 的特例。


标准正交基的性质十分美妙,但我们取出一组基使得其恰好是标准正交基是十分困难的,
所幸前人已经研究出了一套算法,可以将所有线性无关组转变为标准正交组,且张成空间相同。

\begin{theorem}
    Gram-Schmidt 过程:设 $v_1, \ldots ,v_n$ 是 $ V $ 中的线性无关向量组.
    设 $e_1 = v_1/ \lVert v_1 \rVert$. 对于 $ j = 2, \ldots , m$,定义 $ e_j $ 如下:
    \[ e_j = \frac{v_j - \langle v_j, e_1 \rangle e_1 - \cdots - \langle v_j, e_{j - 1} \rangle e_{j - 1} }{\lVert v_j - \langle v_j, e_1 \rangle e_1 - \cdots - \langle v_j, e_{j - 1} \rangle e_{j - 1} \rVert}.\]
    则 $e_1, \ldots , e_m $ 是 $ V $ 中的标准正交组,使得对 $ j = 1, \ldots , m $ 有
    \[ \mathrm{span} \enspace (v_1, \ldots, v_j) = \mathrm{span} \enspace (e_1, \ldots, e_j) \]
\end{theorem}

证明前半部分使用归纳法,后半部分证明两向量组等价即可。

让我们简单运用一下 Gram-Schmidt 过程。
\begin{example}
    求 $\mathcal{P}_2(\mathbf{R})$ 的一组标准正交基,内积定义为 $\langle p, q \rangle = \int_{-1}^1 p(x)q(x)\mathrm{d}x$. 
\end{example}

不难发现,Gram-Schmidt 过程实际上可以分成两部分:

1. 正交化:定义 $ u_1 = v_1 , \enspace u_j = v_j - \langle v_j, e_1 \rangle e_1 - \cdots - \langle v_j, e_{j - 1} \rangle e_{j - 1},
\enspace j = 2, \ldots , m$,此时 $u_1, \ldots , u_m$ 已经互相正交。

2. 单位化:$ e_j = \frac{u_j}{\lVert u_j \rVert} , \enspace j = 1, \ldots , m$,从而有 $\lVert e_j \rVert = 1, \enspace j = 1, \ldots , m$

Gram-Schmidt 过程可以说是线性代数计算较为困难的方面之一,也是应试经常考察的方面,需要多加注意。

借助 Gram-Schmidt 过程,显然我们可以得到以下结论:

1. 每个有限维内积空间都有标准正交基;

2. 设 $ V $ 是有限的. 则 $ V $ 中每个标准正交向量组都可以扩充成 $ V $ 的标准正交基.

\vspace{2ex}

在此我们先打住,回忆一下内积的定义,其本质上就是一个函数,它把 $ V $ 中元素的每个有序对 $(u, v)$ 都映射成一个数 
$ \langle u, v \rangle \in \mathbf{F}$,而我们也很熟悉一个把 $ V $ 中元素映射成一个数的函数,即所谓的线性泛函。
那么这两者之间是否有某种联系?我们先借助几个例子观察一下。

\begin{example}
    \textup{(1)} 定义如下的函数 $\varphi$: $\mathbf{F}^{3} \rightarrow \mathbf{F}$
    \[\varphi(z_1, z_2, z_3) = 2z_1 - 5z_2 + z_3\]
    是
\end{example}

\section{正交补}

\vspace{2ex} 
\centerline{\heiti \Large 内容总结}

\vspace{2ex} 

\centerline{\heiti \Large 习题}
\vspace{2ex} 
{\kaishu }
\begin{flushright}
    \kaishu

\end{flushright}
\centerline{\heiti A组}
\begin{enumerate}
	\item 
\end{enumerate}
\centerline{\heiti B组}
\begin{enumerate}
	\item 
\end{enumerate}
\centerline{\heiti C组}
\begin{enumerate}
	\item 
\end{enumerate}
\input{./专题/第二十讲 内积空间上的算子(I).tex}
\input{./专题/第二十一讲 内积空间上的算子(II).tex}
\input{./专题/第二十二讲 极分解与奇异值分解.tex}
\input{./专题/第二十三讲 实空间上的算子.tex}
\input{./专题/第二十四讲 行列式(II).tex}
\input{./专题/第二十五讲 线性代数与解析几何基础.tex}
\input{./专题/第二十六讲 二次型.tex}
\input{./专题/第二十七讲 线性代数与多元微积分.tex}
\end{document}