\chapter{线性映射基本定理}

在上一讲的讨论中我们定义了线性映射的基本概念,讨论了由其定义直接引出的性质. 本节我们将深入讨论线性映射像空间与核空间之间的关联,从而引出我们目前为止最核心的概念——同构,因为同构使得我们研究的抽象层次更上一层,为我们在下一讲中在这抽象的制高点获得最具象的表达形式——矩阵作铺垫.

\section{线性映射的秩}

通过对线性映射像的求解的讨论我们有$\im \sigma=\sigma(V_1)=\spa(\sigma(\alpha_1),\sigma(\alpha_2),\ldots,\sigma(\alpha_n))$. 我们基于此定义线性映射的秩:
\begin{definition}
    设$\sigma\in \mathcal{L}(V_1,V_2)$,如果$\sigma(V_1)$是$V_2$的有限维子空间,则$\sigma(V_1)$的维数称为$\sigma$的秩,记作$r(\sigma)$,即$r(\sigma)=\dim \sigma(V_1)$.
\end{definition}

这一定义是平凡的,简单理解线性映射的秩即为线性映射像空间的维数.

\section{线性映射基本定理}

这一定理是线性代数最重要的定理之一,因其重要性也被冠以(有限维线性空间)线性映射基本定理的名号:
\begin{theorem}[线性映射基本定理]\label{thm:6:线性映射基本定理}
    设$\sigma \in \mathcal{L}(V_1,V_2)$,若$\dim V_1=n$,则
    \[r(\sigma)+\dim\ker\sigma=n.\]
\end{theorem}
简而言之,这一定理表明:线性映射的秩(或者说线性映射像空间维数)与核空间维数之和等于出发空间的维数. 这一定理的证明非常重要,在之后的很多讨论中还会用到这一思想,因此我们给出详细的证明并阐述其中的思想:

\begin{proof}
    证明的思路和\hyperref[thm:4:维数公式]{线性空间维数公式}的证明思路类似,即``设小扩大''.

    我们设$\dim\ker\sigma=k$,并设$\ker\sigma$的一组基为$\alpha_1,\alpha_2,\ldots,\alpha_k$. 我们将其扩充为$V_1$的一组基,记为$\alpha_1,\alpha_2,\ldots,\alpha_k,\alpha_{k+1},\ldots,\alpha_n$.

    根据定理要证明的等式和前述假设,我们只需证$r(\sigma)=n-k$,即证明像空间维数为$n-k$. 我们知道像空间为$\spa(\sigma(\alpha_1),\sigma(\alpha_2),\ldots,\sigma(\alpha_n))$,其中根据我们的假设,$\sigma(\alpha_1)=\sigma(\alpha_2)=\cdots=\sigma(\alpha_k)=0$(因为它们是核空间的基),因此像空间为$\spa(\sigma(\alpha_{k+1}),\ldots,\sigma(\alpha_n))$. 我们只需证明这一向量组是线性无关的即可,因为这样这$n-k$个向量就可以构成像空间的一组基,从而证明了$r(\sigma)=n-k$.

    我们设$c_{k+1}\sigma(\alpha_{k+1})+\cdots+c_n\sigma(\alpha_n)=0$,即
    \[\sigma(c_{k+1}\alpha_{k+1}+\cdots+c_n\alpha_n)=0\]
    故$c_{k+1}\alpha_{k+1}+\cdots+c_n\alpha_n \in \ker\sigma$,因此可以被$\alpha_1,\alpha_2,\ldots,\alpha_k$线性表示. 于是有
    \[c_{k+1}\alpha_{k+1}+\cdots+c_n\alpha_n=c_1\alpha_1+\cdots+c_k\alpha_k\]
    即
    \[c_1\alpha_1+\cdots+c_k\alpha_k-c_{k+1}\alpha_{k+1}-\cdots-c_n\alpha_n=0\]
    由于$\alpha_1,\alpha_2,\ldots,\alpha_n$是$V_1$的一组基,因此$c_1=\cdots=c_k=c_{k+1}=\cdots=c_n=0$,故$\sigma(\alpha_{k+1}),\ldots,\sigma(\alpha_n)$线性无关,命题得证.
\end{proof}

事实上这一定理也被称为线性映射``维数公式'',但为了与\hyperref[thm:4:维数公式]{线性空间维数公式}区分,本讲义中我们称这一定理为线性映射基本定理. 读者可以比较一下两个``维数公式''的证明,二者都使用了``设小扩大''的思想,都将要证明的结论转化为证明一组向量是线性无关的,但其中证明线性无关的方法略有不同,读者可以仔细体会.

下面我们给出一个证明思想上类似的例子供读者练习:
\begin{example}
    设$\sigma$为有限维线性空间$V$上的线性变换,$W$是$V$的子空间,证明:
    \[\dim\sigma(W)+\dim(\ker\sigma \cap W)=\dim W.\]
\end{example}

\begin{proof}
    与\autoref{thm:6:线性映射基本定理}证明类似,我们``设小扩大''. 设$\dim W=n,\enspace\dim\ker\sigma\cap W=k$,设$\ker\sigma\cap W$的一组基为$\alpha_1,\alpha_2,\ldots,\alpha_k$,我们将其扩充为$W$的一组基,记为
    \[\alpha_1,\alpha_2,\ldots,\alpha_k,\alpha_{k+1},\ldots,\alpha_n.\]

    根据定理要证明的等式和前述假设,我们只需证$\dim\sigma(W)=n-k$. 我们知道像空间为$\sigma(W)=\spa(\sigma(\alpha_1),\sigma(\alpha_2),\ldots,\sigma(\alpha_n))$,其中根据我们的假设,$\sigma(\alpha_1)=\sigma(\alpha_2)=\cdots=\sigma(\alpha_k)=0$,因此像空间为$\spa(\sigma(\alpha_{k+1}),\ldots,\sigma(\alpha_n))$. 我们只需证明这一向量组是线性无关的即可,因为这样这$n-k$个向量就可以构成像空间的一组基,从而证明了$\dim\sigma(W)=n-k$.

    我们设
    \[c_{k+1}\sigma(\alpha_{k+1})+\cdots+c_n\sigma(\alpha_n)=0,\]
    即$\sigma(c_{k+1}\alpha_{k+1}+\cdots+c_n\alpha_n)=0$,故$c_{k+1}\alpha_{k+1}+\cdots+c_n\alpha_n\in\ker\sigma$,因此可以被$\alpha_1,\alpha_2,\ldots,\alpha_k$线性表示. 于是有
    \[c_{k+1}\alpha_{k+1}+\cdots+c_n\alpha_n=c_1\alpha_1+\cdots+c_k\alpha_k,\]
    即
    \[c_1\alpha_1+\cdots+c_k\alpha_k-c_{k+1}\alpha_{k+1}-\cdots-c_n\alpha_n=0,\]
    由于$\alpha_1,\alpha_2,\ldots,\alpha_n$是$W$的一组基,因此$c_1=\cdots=c_k=c_{k+1}=\cdots=c_n=0$,故$\sigma(\alpha_{k+1}),\ldots,\sigma(\alpha_n)$线性无关,命题得证.
\end{proof}

基于线性映射基本定理,我们可以得到如下定理:
\begin{theorem}\label{thm:6:双射等价条件}
    对$\sigma \in \mathcal{L}(V_1,V_2)$且$\dim V_1=\dim V_2=n$,下列条件等价:
    \begin{enumerate}
        \item \label{item:6:双射等价条件:1}
              $\ker \sigma=\{0\}$;

        \item \label{item:6:双射等价条件:2}
              $\sigma$为单射;

        \item $\sigma$为满射;

        \item $\sigma$为双射(可逆);

        \item $r(\sigma)=n$.
    \end{enumerate}
\end{theorem}

我们需要注意的是,上述 \ref*{item:6:双射等价条件:1} 与 \ref*{item:6:双射等价条件:2} 等价不是基于线性映射基本定理得到的,而是在前述\autoref{thm:5:单射与核空间} 中已经证明的. 其余等价性的证明也是非常简单,只需要简单套用维数公式即可.

实际上,线性映射基本定理还隐藏着一个我们之前以及介绍过的结论,即不可能存在从低维空间到高维空间的满射. 利用反证法,假设存在这样的映射$\sigma:V_1\to V_2$,则核空间维数$\dim\ker\sigma=n-r(\sigma)=\dim V_1-\dim V_2<0$,这显然是不合理的. 当然这一结论有一对称形式也成立,即不存在高维空间到低维空间的单射,证明类似,不再赘述.

还需注意的是,这一定理前提要求是有限维空间上的线性变换,因为我们可以给出如下例子:

\begin{example}
    设$V$是全体定义在实数域,取值于实数域的连续函数关于一般的函数加法和数乘构成的线性空间,$\sigma:V\to V$定义为$\sigma(f)=f'$,即求导变换,则$\sigma$是线性变换,且显然$\sigma$是满射(因为任意连续函数$g$一定黎曼可积,所以一定能在$V$中找到原函数使得原函数的导数为$g$),但$\sigma$不是单射,例如对于$g(x)=2x$有$f(x)=x^2+C$($C$为任意常数)都可以有$f'(x)=g(x)$. 因此我们发现这里定义在无限维线性空间$V$中的线性变换使得上面的定理中单射满射不等价.
\end{example}

\section{像与核的进一步讨论}

关于线性变换的像和核有很多的包含关系或等式等结论,实际上很多问题都来源于线性映射基本定理及其推论,本节我们主要探讨这一话题.

我们首先说明几个重要的原则:
\begin{enumerate}
    \item 解决此类问题大多需要综合利用维数公式及其推论,需要将题给条件转化为合适的等价表述然后解决;

    \item 注意集合相等的证明方式,实际上就是两个集合互相包含. 实际上很多时候一边的包含是显然的,只需证明另一边;

    \item 时刻注意线性映射的像和核的定义,线性空间的交、和与直和的概念,例如看到像需要想到其存在原像,看到和与直和要想到将向量分拆等.
\end{enumerate}

接下来我们看一些经典的结论(已知$V$为有限维线性空间,$\sigma\in \mathcal{L}(V,V)$),其中结论 \ref*{item:6:像与核的进一步讨论:1} 最为常见:

\begin{enumerate}
    \item \label{item:6:像与核的进一步讨论:1}
          若$\sigma$为幂等变换(即$\sigma^2=\sigma$)有$V=\ker\sigma\oplus\im \sigma$;

          \begin{proof}
            回忆直和的证明方法,我们这里利用先证明和为直和(即交为$\{0\}$)再证等号成立的方法. 设$\alpha\in\ker\sigma\cap\im \sigma$,则$\sigma(\alpha)=0$,且存在$\beta\in V$使得$\sigma(\beta)=\alpha$,因此利用$\sigma^2=\sigma$有
            \[0=\sigma(\alpha)=\sigma(\sigma(\beta))=\sigma^2(\beta)=\sigma(\beta)=\alpha,\]
            即$\ker\sigma\cap\im \sigma=\{0\}$,因此和为直和. 又由\autoref{thm:6:线性映射基本定理} 可知,$\dim V=\dim\ker\sigma+\dim\im \sigma$,因此$V=\ker\sigma\oplus\im \sigma$.
          \end{proof}

    \item 关于核空间,我们有如下定理,这一定理在之后讨论矩阵标准形的时候非常有用:
          \begin{theorem} \label{thm:6:核空间性质}
              我们有如下关于核空间增长与停止增长的性质:
              \begin{enumerate}
                  \item $\{0\}=\ker \sigma^0\subseteq\ker \sigma^1\subseteq\cdots\subseteq \ker \sigma^k\subseteq\ker \sigma^{k+1}\subseteq\cdots$;

                  \item 设$m$是非负整数使得$\ker \sigma^m=\ker \sigma^{m+1}$,则
                        \[\ker \sigma^m=\ker \sigma^{m+1}=\ker \sigma^{m+2}=\ker \sigma^{m+3}=\cdots\]

                  \item 令$n=\dim V$,则$\ker \sigma^n=\ker \sigma^{n+1}=\ker \sigma^{n+1}=\cdots$.
              \end{enumerate}
          \end{theorem}

          \begin{proof}
            \begin{enumerate}
                \item 设$i>j\geqslant 0$,则$\forall\alpha\in\ker\sigma^j$,即$\sigma^j(\alpha)=0$,则$\sigma^i(\alpha)=\sigma^{i-j}(\sigma^j(\alpha))=0$,即$\alpha\in\ker\sigma^i$,因此$\ker\sigma^j\subseteq\ker\sigma^i$,故$\ker \sigma^0\subseteq\ker \sigma^1\subseteq\cdots\subseteq \ker \sigma^k\subseteq\ker \sigma^{k+1}\subseteq\cdots$.

                这一点表明核空间随着线性变换的幂次增长而增长(至少不减),下面一点将说明这一不减序列一旦某个包含符号可以取等号,那么此后的项都相等.

                \item 任取$k>0$,由(1)可知$\ker \sigma^{m+k}\subseteq\ker \sigma^{m+k+1}$,故只需证$\ker \sigma^{m+k+1}\subseteq\ker\sigma^{m+k}$. 事实上,设$\alpha\in\ker \sigma^{m+k+1}$,则$0=\sigma^{m+k+1}(\alpha)=\sigma^{m+1}(\sigma^k(\alpha))$,即$\sigma^k(\alpha)\in\ker\sigma^{m+1}$. 又$\ker \sigma^m=\ker \sigma^{m+1}$,则$\sigma^k(\alpha)\in\ker\sigma^m\implies\sigma^m(\sigma^k(\alpha))=0\implies\sigma^{m+k}(\alpha)=0\implies\alpha\in\ker\sigma^{m+k}$,故$\ker \sigma^{m+k+1}\subseteq\ker\sigma^{m+k}$,因此$\ker \sigma^{m+k+1}=\ker\sigma^{m+k}$,故$\ker \sigma^m=\ker \sigma^{m+1}=\ker \sigma^{m+2}=\ker \sigma^{m+3}=\cdots$.

                \item 由上一点知我们只需证$\ker \sigma^n=\ker \sigma^{n+1}$. 反证法,若$\{0\}=\ker\sigma^0\subsetneqq\ker\sigma^1\subsetneqq\cdots\subsetneqq\ker\sigma^{n+1}$,则这一递增链条每处严格包含于的维数必然增加1,因此$\dim\ker\sigma^{n+1}\geqslant n+1>n$,但我们知道$\ker\sigma^{n+1}$是$V$的子空间,因此矛盾!故命题成立.
            \end{enumerate}
          \end{proof}
          对于像空间而言也有类似于\autoref{thm:6:核空间性质} 的定理,证明方法也是类似的,我们放在习题中供读者思考.

    \item 存在正整数$m$使得$V=\im \sigma^m+\ker\sigma^m$(和前述性质思想类似,我们放在习题中供读者思考);

    \item 下列条件等价:
          \begin{enumerate}
              \item $V=\ker\sigma\oplus\im \sigma$;

              \item $\ker\sigma \cap \im \sigma=\{0\}$;

              \item $\ker\sigma=\ker\sigma^2$;

              \item $\im \sigma=\im \sigma^2$;

              \item $r(\sigma^2)=r(\sigma)$.
          \end{enumerate}

          \begin{proof}
            \begin{enumerate}
                \item $(1)\implies(2)$:由直和的定义显然;

                \item $(2)\implies(3)$:由\autoref{thm:6:核空间性质} 可知$\ker\sigma\subseteq\ker\sigma^2$,又任取$\alpha\in\ker\sigma^2$,则
                \[0=\sigma^2(\alpha)=\sigma(\sigma(\alpha)),\]故$\sigma(\alpha)\in\ker\sigma\cap\im\sigma$,即$\sigma(\alpha)=0$,因此$\alpha\in\ker\sigma$,故$\ker\sigma^2\subseteq\ker\sigma$,因此$\ker\sigma=\ker\sigma^2$;

                \item $(3)\implies(4)$:令$\alpha\in\im\sigma^2$,故存在$\beta\in v$使得$\alpha=\sigma^2(\beta)=\sigma(\sigma(\beta))$,即$\alpha\in\im\sigma$,因此$\im\sigma^2\subseteq\im\sigma$,又由(3)知
                \[\dim\im\sigma^2=n-\dim\ker\sigma^2=n-\dim\ker\sigma=\dim\im\sigma,\]
                因此$\im\sigma^2=\im\sigma$;

                \item $(4)\implies(5)$:根据线性映射的秩的定义(等于像空间维数)显然;

                \item $(5)\implies(1)$:利用先证明和为直和(即交为$\{0\}$)再证等号成立的方法. 事实上$r(\sigma^2)=r(\sigma)$即$\dim\im\sigma^2=\dim\im\sigma$,又$(3)\implies(4)$证明了$\im\sigma^2\subseteq\im\sigma$,故$\im \sigma=\im \sigma^2$. 事实上,设$\beta_1,\cdots,\beta_s$为$\im\sigma$的一组基,则
                \[\im\sigma^2=\spa(\sigma(\beta_1),\cdots,\sigma(\beta_s)),\]
                由$\dim\im\sigma^2=\dim\im\sigma$可知$\sigma(\beta_1),\cdots,\sigma(\beta_s)$是$\im\sigma^2$的一组基,由$\im \sigma=\im \sigma^2$可知这也是$\im\sigma$的一组基.$\forall\alpha\in\ker\sigma\cap\im\sigma$,设
                \[\alpha=k_1\beta_1+\cdots+k_s\beta_s,\]
                由于
                \[0=\sigma(\alpha)=k_1\sigma(\beta_1)+\cdots+k_s\sigma(\beta_s),\]
                由$\sigma(\beta_1),\cdots,\sigma(\beta_s)$是一组基可知$k_1=\cdots=k_s=0$,因此$\alpha=0$,故$\ker\sigma\cap\im\sigma=\{0\}$,因此和为直和. 又由\autoref{thm:6:线性映射基本定理} 可知,$\dim V=\dim\ker\sigma+\dim\im \sigma$,因此$V=\ker\sigma\oplus\im \sigma$.
            \end{enumerate}
          \end{proof}

    \item $\dim(\ker\sigma+\im \sigma) \geqslant \dfrac{n}{2}$,等号成立充要条件为$\ker\sigma=\im \sigma$.

          \begin{proof}
              这一结论的证明需要结合两个维数公式. 事实上,由线性空间维数公式有
              \[\dim(\ker\sigma+\im \sigma)=\dim\ker\sigma+\dim\im \sigma-\dim(\ker\sigma \cap \im \sigma)=n-\dim(\ker\sigma \cap \im \sigma)\]
              因此只需证明$\dim(\ker\sigma+\im \sigma) \geqslant \dfrac{n}{2}$.

              我们用反证法,我们知道$\ker\sigma\cap \im \sigma$是$\ker\sigma$和$\im\sigma$的子空间,因此
              \begin{gather*}
                  \dim(\ker\sigma\cap \im \sigma) \leqslant \dim\ker\sigma \\
                  \dim(\ker\sigma\cap \im \sigma) \leqslant \dim\im\sigma
              \end{gather*}
              故若$\dim(\ker\sigma+\im \sigma)>\dfrac{n}{2}$,则有
              \[\dim\ker\sigma+\dim\im \sigma>n\]
              与线性映射基本定理矛盾,因此$\dim(\ker\sigma+\im \sigma) \geqslant \dfrac{n}{2}$成立.

              接下来我们讨论取等条件. 充分性显然,因为此时
              \begin{gather*}
                  \dim\ker\sigma=\dim\im\sigma=\dfrac{n}{2} \\
                  \ker\sigma\cap \im \sigma=\ker\sigma=\im\sigma
              \end{gather*}
              故
              \[\dim(\ker\sigma+\im \sigma)=n-\dim(\ker\sigma \cap \im \sigma)=\dfrac{n}{2}\]
              成立.

              接下来我们讨论必要性. 由
              \[\dim(\ker\sigma+\im \sigma)=n-\dim(\ker\sigma \cap \im \sigma)\]
              可知$\dim(\ker\sigma\cap \im \sigma)=\dfrac{n}{2}$,由$\ker\sigma\cap \im \sigma$是$\ker\sigma$和$\im\sigma$的子空间可知
              \begin{align*}
                  \dim\ker\sigma & \geqslant\dfrac{n}{2} \\
                  \dim\im \sigma & \geqslant\dfrac{n}{2}
              \end{align*}
              又由线性映射基本定理,$\dim\ker\sigma+\dim\im \sigma=n$,因此
              \[\dim\ker\sigma=\dim\im \sigma=\dfrac{n}{2}\]
              即子空间维数与原空间相等,故必有$\ker\sigma=\im \sigma=\ker\sigma\cap \im \sigma$成立(回顾线性空间$U\subseteq V$且$\dim U=\dim V$则$U=V$).
          \end{proof}
\end{enumerate}

\section{可逆与同构}

同构是直至目前线性代数中最重要的概念,本节中我们只讨论基本的概念和性质,在下一讲中我们将结合线性映射矩阵表示深入探讨同构的深层意义.

\subsection{线性空间同构的概念}

\begin{definition}[同构]
    如果由线性空间$V_1(\mathbf{F})$到$V_2(\mathbf{F})$存在一个线性双射$\sigma$,则称$V_1(\mathbf{F})$和$V_2(\mathbf{F})$是\keyterm{同构的},记作$V_1(\mathbf{F}) \cong V_2(\mathbf{F})$. $\sigma$称为$V_1(\mathbf{F})$到$V_2(\mathbf{F})$的一个\keyterm{同构映射}[isomorphism].
\end{definition}

根据定义我们发现,同构映射实际上就是线性双射. 关于同构的概念,我们有以下几点需要强调:
\begin{enumerate}
    \item 特别注意:同构是线性空间之间的关系,同构映射才是描述线性映射的;

    \item 事实上,同构也是一种等价关系,这一点很容易验证,读者可以自行尝试(可能传递性略有困难,实际上只需说明线性双射复合后仍是线性双射即可);

    \item 同构映射的逆映射也是同构映射,即线性双射的逆映射仍然是线性双射. 除此之外,两个同构映射的复合也是同构的. 这两个性质证明是容易的,我们放在习题中供读者验证.

    \item 对同构映射$\sigma$,$V_1$中向量组$ \alpha_1,\alpha_2,\ldots,\alpha_m $与$V_2$中对应的$ \sigma(\alpha_1),\sigma(\alpha_2),\ldots,\sigma(\alpha_m) $有相同的线性相关性.

          \begin{proof}
              我们已知一般的线性映射将线性相关的向量组映射为线性相关的向量组,因此对于同构映射,我们只需证明它能将线性无关的向量组映射为线性无关的向量组即可.

              设$V_1$中$\alpha_1,\alpha_2,\ldots,\alpha_m$线性无关,我们考察$\sigma(\alpha_1),\sigma(\alpha_2),\ldots,\sigma(\alpha_m)$的线性相关性. 设
              \[c_1\sigma(\alpha_1)+c_2\sigma(\alpha_2)+\cdots+c_m\sigma(\alpha_m)=0,\]
              即
              \[\sigma(c_1\alpha_1+c_2\alpha_2+\cdots+c_m\alpha_m)=0.\]
              因为$\sigma$是线性双射,因此$\sigma$首先必须是单射,因此$\ker\sigma=\{0\}$,所以
              \[c_1\alpha_1+c_2\alpha_2+\cdots+c_m\alpha_m=0\]
              由$\alpha_1,\alpha_2,\ldots,\alpha_m$线性无关,故$c_1=c_2=\cdots=c_m=0$,即$\sigma(\alpha_1),\sigma(\alpha_2),\ldots,\sigma(\alpha_m)$线性无关,证毕.
          \end{proof}

          这一结论比一般的线性映射更强,对于一般的线性映射只有将线性相关的向量组映射为线性相关的向量组,无法保证将线性无关的向量组映射为线性无关的向量组,但同构映射可以保证,因为它是线性双射. 这一性质也是本质的,因为双射具有``一一对应''的属性,因此直觉也告诉我们,线性空间的基在线性双射(同构映射)下的像应当对应于像空间的一组基.

          我们可以更进一步得到下面的结论:
          \begin{theorem}\label{thm:6:同构保秩}
              设$\sigma$是$V_1$到$V_2$的同构映射,$S_1=\{\alpha_1,\alpha_2,\ldots,\alpha_m\}$是$V_1$的任意一组向量,$S_2=\{\sigma(\alpha_1),\sigma(\alpha_2),\ldots,\sigma(\alpha_m)\}$,则$r(S_1)=r(S_2)$,即同构映射保持映射前后向量组秩不变.
          \end{theorem}
          \begin{proof}
              反证法. 假设$r(S_1)\neq r(S_2)$,我们从以下两方面导出矛盾:
              \begin{enumerate}
                  \item 若$r(S_1)>r(S_2)$,取$S_1$的极大线性无关组,记为$S_1'$,则$r(S_1')=r(S_1)>r(S_2)$. 又$S_1'$在$\sigma$下的像$S_2'$为$S_2$的子向量组,因此$r(S_2')\leqslant r(S_2)$. 但我们有同构映射保持线性无关性,因此$r(S_2')=r(S_1')=r(S_1)>r(S_2)$,矛盾!因此这种情况不可能;

                  \item 若$r(S_1)<r(S_2)$,取$S_2$的极大线性无关组,记为$S_2'$,则$r(S_2')=r(S_2)>r(S_1)$. 如前所述,同构映射的逆仍为同构映射,考虑$\sigma$的逆$\sigma^{-1}$,$S_2'$在$\sigma^{-1}$下的像$S_1'$为$S_1$的子向量组,因此$r(S_1')\leqslant r(S_1)$. 但我们有同构映射保持线性无关性,因此$r(S_1')=r(S_2')=r(S_2)>r(S_1)$,矛盾!因此这种情况不可能.
              \end{enumerate}
          \end{proof}
\end{enumerate}

我们讨论几个经典的一一对应的例子.
\begin{enumerate}
    \item 第一个例子是坐标映射:在有限维向量空间的向量的坐标一节中,我们说明了一个向量在一组基下坐标唯一,而一个坐标对应唯一一个向量,并且也证明了坐标运算的线性性,因此坐标映射是同构映射,并且是经典的同构映射. 它可以建立起任何一个$n$维线性空间$V(\mathbf{F})$与几何向量空间$\mathbf{F}^n$之间的一一对应(同构映射),即任意$n$维线性空间$V(\mathbf{F})\cong\mathbf{F}^n$. 这一点之后会强调多次,需牢记;

    \item 第二个例子将在下一将线性映射矩阵表示中描述并证明,目前我们只给出结论,读者不必惊恐于不理解其中的记号,因为下一讲的核心任务之一就是证明存在这一同构映射:若$\dim V_1(\mathbf{F})=m$,$\dim V_2(\mathbf{F})=n$,则$\mathcal{L}(V_1,V_2) \cong \mathbf{F}^{m \times n}$.
\end{enumerate}

\subsection{同构的等价条件}

下面我们给出同构的等价条件:
\begin{theorem}\label{thm:6:同构的等价条件}
    两个线性空间$V_1(\mathbf{F})$和$V_2(\mathbf{F})$同构的充要条件是它们的维数相等.
\end{theorem}

\begin{proof}
    必要性:设$V_1(\mathbf{F})$和$V_2(\mathbf{F})$同构,即存在线性双射(故至少是单射)$\sigma:V_1\to V_2$. 由线性映射基本定理,
    \[\dim V_1=\dim\ker\sigma+\dim\im\sigma=\dim\im\sigma=\dim V_2.\]
    故必要性成立.

    下证明充分性,即证两维数相等的线性空间之间存在线性双射. 设$\dim V_1=\dim V_2=n$,设$V_1$的一组基为$\alpha_1,\alpha_2,\ldots,\alpha_n$,$V_2$的一组基为$\beta_1,\beta_2,\ldots,\beta_n$,根据\autoref{thm:5:线性映射构造} 可知,我们可以定义线性映射$\sigma:V_1\to V_2$,使得
    \begin{equation}\label{eq:6:构造同构}
        \sigma(\alpha_1)=\beta_1,\sigma(\alpha_2)=\beta_2,\ldots,\sigma(\alpha_n)=\beta_n.
    \end{equation}
    接下来只需证明$\sigma$是线性双射即可. 事实上$\sigma$是单射是显然的,因为若$\sigma(\alpha)=0$,其中$\alpha\in V_1$,则$\alpha$可以写为$\alpha=k_1\alpha_1+k_2\alpha_2+\cdots+k_n\alpha_n$,则有
    \[\sigma(\alpha)=\sigma(k_1\alpha_1+k_2\alpha_2+\cdots+k_n\alpha_n)=k_1\sigma(\alpha_1)+k_2\sigma(\alpha_2)+\cdots+k_n\sigma(\alpha_n)=0,\]
    又由\autoref{eq:6:构造同构} 以及$\beta_1,\beta_2,\ldots,\beta_n$线性无关可知$k_1=k_2=\cdots=k_n=0$,因此$\alpha=0$,即$\sigma$是单射. 由\autoref{thm:6:双射等价条件}(或直接根据线性映射基本定理)可知,$\sigma$是线性双射,证毕.
\end{proof}

我们需要指出,同构是目前为止最重要的概念. 它统一了前面所学的所有主干内容,将线性空间可以按维数划分为不同的等价类,并将抽象再升一层,表明线性空间最本质的特点在于维数,因为我们可以通过同构建立起对所有维数相同的线性空间之间的一一对应. 更重要的是我们可以通过坐标映射建立起任何一个$n$维线性空间$V(\mathbf{F})$与几何向量空间$\mathbf{F}^n$之间的同构映射,从而遮蔽所有线性空间自身基的特色(例如有的线性空间中的元素是矩阵、函数、数列等),进而可以将所有对有限维线性空间的研究转为对简单向量空间的研究. 从此以后的大部分研究中,我们再提到线性空间,我们只需要说出线性空间的维数,就相当于给出了几乎所有的信息. 或许目前对上面的说法的认识还不够深刻,但在下一讲中我们将通过线性映射矩阵表示的讨论进一步加深理解.

下面我们通过几个例题来应用同构的等价条件,也同时进一步了解几个常见的同构的例子:
\begin{example}
    指出下面各组内的两个线性空间是否同构,若同构可以进一步思考同构映射的构造:
    \begin{enumerate}
        \item 最高次不超过$n-1$的多项式构成的线性空间$\mathbf{R}[x]_n$与$\mathbf{R}^n$;

        \item 全体复数在实数域上的线性空间$\mathbf{C}(\mathbf{R})$与$\mathbf{R}^2$;

        \item 全体二元复向量$\mathbf{C}^2$在实数域上构成的线性空间$\mathbf{C}^2(\mathbf{R})$与$\mathbf{R}[x]_4$;

        \item 全体二元复向量$\mathbf{C}^2$在复数域上构成的线性空间$\mathbf{C}^2(\mathbf{C})$与$\mathcal{L}(\mathbf{R}^4,\mathbf{R})$.
    \end{enumerate}
\end{example}

\begin{solution}
    \begin{enumerate}
        \item 同构,因为二者维数均为$n$. 同构映射非常简单,因为$\mathbf{R}[x]_n$在基$\{1,x,\cdots,x^{n-1}\}$下的坐标就在$\mathbf{R}^n$中,因此同构映射就是这一坐标映射:
        \[\sigma:\mathbf{R}[x]_n\to\mathbf{R}^n,\quad a_0+a_1x+\cdots+a_{n-1}x^{n-1}\mapsto(a_0,a_1,\cdots,a_{n-1}).\]
        我们很容易验证这一映射是线性双射,因此是同构映射.

        \item 同构,因为二者维数均为$2$(回忆\autoref{ex:3:不同数域的维数}). 这一同构映射同上一小问理,$\mathbf{C}(\mathbf{R})$在基$\{1,i\}$下的坐标就在$\mathbf{R}^2$中,因此同构映射就是这一坐标映射:
        \[\sigma:\mathbf{C}(\mathbf{R})\to\mathbf{R}^2,\quad a+bi\mapsto(a,b).\]
        事实上这就是将复数在二维平面中的表示,我们很容易验证这一映射是线性双射,因此是同构映射..

        \item 同构,因为二者维数均为$4$. 这一同构映射也非常简单,因为$\mathbf{C}^2(\mathbf{R})$在基
        \[\{(1,0),(i,0),(0,1),(0,i)\}\]
        下的坐标和$\mathbf{R}[x]_4$在基$\{1,x,x^2,x^3\}$下的坐标都在$\mathrm{R}^n$中,可以将它们一一对应,因此同构映射就是这一坐标映射:
        \[\sigma:\mathbf{C}^2(\mathbf{R})\to\mathbf{R}[x]_4,\quad (a+bi,c+di)\mapsto a+bx+cx^2+dx^3.\]
        我们很容易验证这一映射是线性双射,因此是同构映射.

        \item 不同构,因为$\mathbf{C}^2(\mathbf{C})$的维数为$2$,而$\mathcal{L}(\mathbf{R}^4,\mathbf{R})$的维数为$4$.
    \end{enumerate}
\end{solution}

\vspace{2ex}
\centerline{\heiti \Large 内容总结}

\vspace{2ex}
\centerline{\heiti \Large 习题}

\vspace{2ex}
{\kaishu }
\begin{flushright}
    \kaishu

\end{flushright}

\centerline{\heiti A组}
\begin{enumerate}
    \item 证明:同构映射的逆、复合仍然是同构映射.
\end{enumerate}

\centerline{\heiti B组}
\begin{enumerate}
    \item 设$\sigma(p(x))=p'(x)$(求导),$\forall p(x) \in \mathbf{R}[x]_n$.
          \begin{enumerate}
              \item 证明:$\sigma$是$\mathbf{R}[x]_n$上的线性变换;

              \item 求$\sigma$的值域和$r(\sigma)$,说明$\sigma$是否可逆;

              \item 求$\sigma$的核及其维数;

              \item 求$r(\sigma)+\dim\ker\sigma$,问:$\mathbf{R}[x]_n=\ker\sigma+\im \sigma$是否成立.
          \end{enumerate}

    \item 设$V$为有限维线性空间,$T\in \mathcal{L}(V,V)$且$T$不是恒等变换也不是零变换,问:下列情况是否可能发生?如果可能请举例,不可能请说明理由.
          \begin{enumerate}
              \item $\im T \cap \ker T = \{0\}$;

              \item $\im T \subseteq \ker T$;

              \item $\ker T = \im T$;

              \item $\ker T \subseteq \im T$.
          \end{enumerate}

    \item 若$\sigma_1,\sigma_2\in \mathcal{L}(V,V)$,判断下列说法是否正确,正确请给出证明,反之给出反例:
          \begin{enumerate}
              \item 由$r(\sigma)+\dim\ker\sigma=n$可知$V=\ker\sigma+\im \sigma$;

              \item 若有$\im T \cap \ker T = \{0\}$,则$V=\ker\sigma+\im \sigma$成立;

              \item 因为$\forall \alpha \in V$有$(\sigma_1+\sigma_2)(\alpha)=\sigma_1(\alpha)+\sigma_2(\alpha)$,所以$(\sigma_1+\sigma_2)(V)=\sigma_1(V)+\sigma_2(V)$;

              \item $(I-\sigma)(V)+\sigma(V)=V$,其中$I$为恒等映射.
          \end{enumerate}

    \item 已知$V$为有限维线性空间,$\sigma\in \mathcal{L}(V,V)$,且$\ker\sigma=\im \sigma$,证明:
          \begin{enumerate}
              \item $n$为偶数;

              \item 存在$V$的一组基$\alpha_1,\ldots,\alpha_n$使得
                    \[\sigma(\alpha_1,\ldots,\alpha_n)=(\alpha_1,\ldots,\alpha_n)\begin{pmatrix}
                            0 & E_{\frac{n}{2}} \\ 0 & 0
                        \end{pmatrix}.\]
          \end{enumerate}

    \item 设$V(\mathbf{R})$是线性空间,$\sigma$是$V(\mathbf{R})$到$\mathbf{R}^3$的同构映射,且$\sigma(\alpha_1)=(1,0,1),\enspace\sigma(\alpha_2)=(-2,1,0),\enspace\sigma(\alpha_3)=(-3,2,1),\enspace\sigma(\alpha_4)=(1,1,2)$.
          \begin{enumerate}
              \item $\alpha_1$在$\spa(\alpha_2,\alpha_3)$中吗?

              \item 设$W_1=\spa(\alpha_1,\alpha_2),\enspace W_2=\spa(\alpha_3,\alpha_4)$,求$W_1\cap W_2$.
          \end{enumerate}

    \item 设$c_1,c_2,\ldots,c_n$是$n$个互异的实常数. 证明:$\mathbf{R}[x]_n$到$\mathbf{R}$的一个映射$\sigma$:
          \[\sigma(p(x))=(p(c_1),p(c_2),\ldots,p(c_n))\]
          是$\mathbf{R}[x]_n$到$\mathbf{R}$的一个同构映射.

    \item 设$\sigma$和$\tau$分别为有限维线性空间$U\to V$和$V\to W$的线性映射,证明
    \[\dim\ker\sigma+\dim(\im\sigma\cap\ker\sigma)=\dim\ker(\tau\sigma).\]
\end{enumerate}

\centerline{\heiti C组}
\begin{enumerate}
    \item 设$V$是一个$n$维线性空间,$V=W_1\oplus W_2,\enspace\sigma\in \mathcal{L}(V,V)$. 证明:$\sigma$可逆$\iff V=\sigma(W_1)+\sigma(W_2)$.

    \item 设$V_1,V_2,V_3$分别为$m,n,s$维线性空间,$\sigma\in \mathcal{L}(V_1,V_2),\enspace\tau\in \mathcal{L}(V_2,V_3)$,则
          \[r(\sigma)+r(\tau)-n \leqslant r(\tau\sigma) \leqslant \min(r(\sigma),r(\tau)).\]

    \item 设$V_1$是有线维线性空间,$\sigma,\tau\in \mathcal{L}(V_1,V_2)$,则
          \[r(\sigma+\tau) \leqslant r(\sigma)+r(\tau).\]

          事实上前两题的结论在下一章节矩阵的秩中都会涉及,此处有兴趣的同学可以尝试从线性映射的角度理解这两个秩不等式. 由于这是教材中小字部分内容,一般而言不在考察范围,如果出现且无法找到合适方式,可以考虑化为矩阵进行证明.

    \item 设$\sigma\in \mathcal{L}(V,V)$,$\dim V_1=n$,且$\sigma^2=\sigma$,$I$是$V$上的恒等变换. 证明:
          \begin{enumerate}
              \item $(I-\sigma)(V) \in \ker\sigma$;

              \item $r(I-\sigma)+r(\sigma)=n$.
          \end{enumerate}

    \item 已知$V$为有限维线性空间,$\sigma\in \mathcal{L}(V,V)$,且$\sigma^2=\theta$(零映射). 证明:
          \begin{enumerate}
              \item $\sigma$的像空间维数不超过$\dfrac{n}{2}$;

              \item 设$A$是$\sigma$在某组基下的矩阵,则方程组$AX=0$的基础解系至少有$\dfrac{n}{2}$个解.
          \end{enumerate}

    \item 设$\mathbf{K} \subseteq \mathbf{F} \subseteq \mathbf{E}$是三个数域,已知$\mathbf{F}$作为$\mathbf{K}$上的线性空间是$n$维的,$\mathbf{E}$作为$\mathbf{F}$上的线性空间是$m$维的,证明:$\mathbf{E}$作为$\mathbf{K}$上的线性空间是$mn$维的.
\end{enumerate}
