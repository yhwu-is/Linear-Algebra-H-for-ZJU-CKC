\chapter{史海拾遗} \label{chap:史海拾遗}

通过前面十余讲的讨论,我们已经将线性代数的一个核心问题——有关线性方程组的解的一般理论完成,其意已尽. ``历史是一面镜子,它照亮现实,也照亮未来. '' 我们很有必要抓住这一时机来完整讨论有关于线性代数的历史,循着历代数学家的脚步重新审视所学的内容,再次感受其中逻辑的自然与顺畅. 很多时候,知道一件事情为什么、怎么来的更为可贵. 另一方面,我们也将在史海中搜寻整个数学大厦中与线性代数紧密相连的部分,开始我们后一阶段更多``未竟之美''的讨论.

提示:本讲中可能出现大量未学习的内容,事实上很多是后续会学习的,也有部分是非常前沿的介绍. 读者可以留个印象,日后再回过头来看,一定会感觉无比亲切.

\section{起点:初等代数}

\subsection{初等代数简介}

代数的英语为 algebra ,源于阿拉伯语单字``al-jabr''(本义为``重聚''),出自《代数学》(阿拉伯语:al-Kitāb al-muḫtaṣar fī ḥisāb al-ğabr wa-l-muqābala)这本书的书名上,意指移项和合并同类项之计算的摘要,其为波斯回教数学家花拉子米于820年所著.

事实上,初见代数一词我们脑海中便会浮现出小学、初中阶段老师反复强调的``用字母表示数''的思想,``一元一次方程''、``合并同类项''、``因式分解''等熟悉的词汇也会出现在眼前. 事实也是如此,初等代数的由来正是用字母表示数后,得到了一系列方程和多项式的有趣问题.

亚历山大港的丢番图(Dióphantos ho Alexandreús,公元200--284),是罗马时代的数学家. 大部分有关丢番图生平的信息来源于5世纪时希腊人梅特罗多勒斯(Metrodorus)在其文集中收录的一篇具有数学谜题性质的《丢番图墓志铭》:
\begin{quote}
    \kaishu
    坟中安葬着丢番图.

    多么令人惊讶,它忠实地记录了所经历的道路.

    上帝给予的童年占六分之一,

    又过十二分之一,两颊长胡,

    再过七分之一,点燃起结婚的蜡烛.

    五年之后天赐贵子,

    可怜迟到的宁馨儿,享年仅及其父之半,便进入冰冷的墓.

    悲伤只有用数论的研究去弥补,

    又过四年,他也走完了人生的旅途.
\end{quote}
据此列一元一次方程可知,丢番图享寿84岁,于33岁时成婚,38岁时生子,80岁时丧子. 丢番图作著的丛书《算术》(Arithmetica)处理求解代数方程组的问题,其中有不少已经遗失,但他的研究在数论中占有重要地位,如丢番图方程、丢番图几何、丢番图逼近等都是数学里的重要领域. 我们简要展开丢番图方程的讨论:
\begin{definition}
    形如
    \[a_1x_1^{b_1}+a_2x_2^{b_2}+\cdots+a_nx_n^{b_n}=c,\enspace a_i,b_i,c\in\mathbf{Z}\enspace(i=1,2,\ldots,n)\]
    的方程称为丢番图方程(或不定方程).
\end{definition}

简而言之,丢番图方程就是未知数只能使用整数的整数系数多项式等式,虽然定义看起来十分简单,学过初等数论的同学应当有些熟悉(事实上在之后的讨论中我们可以看到初等代数与初等数论之间是紧密联系的),但可以说这一方程在数学史上留下了浓墨重彩的一笔. 后来当法国数学家费马研究《算术》一书时,对其中某个方程颇感兴趣并认为其无解,说他对此``已找到一个绝妙的证明'',但却没有记录下来,直到三个世纪后才出现完整的证明. 我们这里简要介绍这一定理,即费马大定理:
\begin{theorem}
    丢番图方程$x^n+y^n-z^n=0$在$n>2$时无正整数解.
\end{theorem}

自费马提出猜想的三百余年以来,无数数学家为证明费马大定理而费尽心血,直至1995年,英国数学家安德鲁·怀尔斯(Andrew Wiles)最终给出了证明. 这一证明使用了代数数论、代数几何等大量现代数学工具. 如果读者有兴趣自行搜索费马大定理证明的历史,我们可以看到对这一看似简单定理的证明的不懈追求推动了现代数学的发展,可谓是意义重大. 或许在数学史中这些中间过程不过是短短的一行文字描述,但这就是人类探索真理的历程的缩影.

或许阅读这一讲义的很多同学都来源于计算机相关专业,我们这里还可以简要介绍丢番图方程与理论计算机的关联. 1900年,希尔伯特提出丢番图问题的可解答性为他在巴黎的国际数学家大会演说中所提出的23个重要数学问题的第十题. 这个问题是对于任意多个未知数的整系数不定方程,要求给出一个可行的算法,使得借助于它通过有限次运算,可以判定该方程有无整数解.

第十问题的解决是众人集体的智慧结晶. 其中美国数学家马丁·戴维斯(Martin Davis)、希拉里·普特南(Hilary Putnam)和朱莉娅·罗宾逊(Julia Robinson)做出了突出的贡献. 而最终的结果,是由俄国数学家尤里·马季亚谢维奇(Yuri Matiyasevich)于1970年所完成的:不可能存在一个算法能够判定任何丢番图方程是否有解. 这一问题涉及到``可计算性''的问题,相关的讨论读者在学习计算理论后将有进一步的了解. 想必读者听闻过罗素悖论或理发师悖论,或者是图灵停机问题,这些都与可计算性密不可分. 实际上,``可计算性''与人类逻辑与知识的边界密切相关——这显然是个异常宏大的主题,留待后人不断探索.

\subsection{西方初等代数发展史简述}

我们回到对于初等代数历史的讨论——刚刚我们显然有些跳脱了,但这些讨论对于了解数学之美也是必要的. 在古代西方,还有几个重要的时间节点值得提及:
\begin{enumerate}
    \item 公元前1800年左右,旧巴比伦斯特拉斯堡泥板书中记述其寻找著二次椭圆方程的解法;

    \item 公元前1600年左右,普林顿322号泥板书中记述了以巴比伦楔形文字写成的勾股数列表;

    \item 公元前800年左右,印度数学家包德哈亚那在其著作《包德哈尔那绳法经》中以代数方法找到了勾股数,给出了一些二次方程的几何解法,且找出了两组丢番图方程组的正整数解;

    \item 公元前300年左右,在《几何原本》的第二卷里,欧几里德给出了有正实数根之二次方程的解法,使用尺规作图的几何方法;

    \item 公元前100年左右,写于古印度的巴赫沙里手稿中使用了以字母和其他符号写成的代数标记法,且包含有三次与四次方程,多达五个未知量的线性方程之代数解,二次方程的一般代数公式,以及不定二次方程与方程组的解法.
\end{enumerate}

此为丢番图之前的初等代数发展重要节点,蕴含着古人朴素的智慧. 丢番图之后,499年,印度数学家阿耶波多在其所著之阿耶波多书里以和现代相同的方法求得了线性方程的自然数解,描述不定线性方程的一般整数解,给出不定线性方程组的整数解,而描述了微分方程;
628年,印度数学家婆罗摩笈多在其所著之梵天斯普塔释哈塔中,介绍了用来解不定二次方程的宇宙方法,且给出了解线性方程和二次方程的规则. 他发现二次方程有两个根,包括负数和无理数根.

此后便迎来一个更为重要的时间节点. 820年,代数(algebra)一词出现,其描述于波斯数学家花拉子米所著之完成和平衡计算法概要中对于线性方程与二次方程系统性的求解方法. 花拉子米常被认为是``代数之父'',其大多数的成果简化后会被收录在书籍之中,且成为现在代数所用的许多方法之一. 990年左右,波斯阿尔卡拉吉在其所著之al-Fakhri中更进一步地以扩展花拉子米的方法论来发展代数,加入了未知数的整数次方及整数开方. 他将代数的几何运算以现代的算术运算代替,定义了单项式并给出了任两个单项式相乘的规则.

此后,初等代数的发展逐步向着现代符号体系与研究方法发展,逐渐演化为了两个方向的问题的讨论:
\begin{enumerate}
    \item 未知数更多的一次方程组的解;

    \item 未知数次数更高的高次方程的解.
\end{enumerate}

前者与我们的主角:线性代数相关,而后者则引发了另一个学科——抽象代数的开端. 初等代数学逐步解决了2、3、4次方程求解问题,这些方程的解都可用系数的四则运算与根式运算给出,即可用根式解这些方程,此时初等代数也因此而达到顶峰. 但当时的数学家们继续探索
5次与5次以上方程的解也试图用根式解出这些方程,经过200余年却无重要进展,直到19世纪抽象代数的发展才有了转机,后续我们也将介绍这其中的故事.

\subsection{中国初等代数发展史简述}

在本节的最后,我们将视角转向东方,总结古代中国人在初等代数学中作出的贡献. 相信读者都十分熟悉这一问题:
\begin{quote}
    \kaishu
    今有雉兔同笼,上有三十五头,下有九十四足,问雉兔各几何?
\end{quote}

这是《孙子算经》(不晚于473年)中提出的著名的鸡兔同笼问题. 在《孙子算经》中还提出了读者在初等数论中就已十分熟悉的``中国剩余定理'',直至现代的密码学研究也无法离开这一重要定理.

实际上,早在《孙子算经》出现前500年左右(公元前100年左右),中国古代数学名著《九章算术》中便处理了代数方程的问题. 其中的``方程章''是世界上最早的系统研究代数方程的专门论著. 它在世界数学历史上最早创立了多元一次方程组的筹式表示方法,以及它的多种求解方法. 《九章算术》把这些线性方程组的解法称为``方程术'',其实质相当于现今的高斯消元法(早于高斯约1900年).

除去线性方程组的贡献,在高次方程方面,中国古代也有相当丰富的成果. 625年左右,中国数学家王孝通在《缉古算经》中找出了三次方程的数值解;1247年,南宋数学家秦九韶在《数书九章》中用秦九韶算法解一元高次方程. 1248年,金朝数学家李冶的《测圆海镜》利用天元术将大量几何问题化为一元多项式方程,是一部几何代数化的代表作. 1300年左右,中国数学家朱世杰处理了多项式代数,发明四元术解答了多达四个未知数的多项式方程组,发明非线性多元方程的消元法,将相关多项式进行乘法、加法和减法运算,逐步消元,将多元非线性方程组化为单个未知数的高次多项式方程;并以数值解出了288个四次、五次、六次、七次、八次、九次、十次、十一次、十二次和十四次多项式方程.

\section{演化:线性代数的产生与发展}

如前所述,初等代数经过数个世纪的发展逐渐演化为了两个大的方向:未知数更多的一次方程组和未知数次数更高的高次方程. 在这两个方向上的发展,使得代数学发展到高等代数的阶段,上面两个方向简而言之就是现在大家熟悉的线性方程组理论(线性代数)和多项式理论(以致后来的抽象代数). 本节我们主要讨论前者,后者我们将在下一节中讨论.

\subsection{行列式与Cramer法则的引入}

在这一部分,我们首先将重点介绍线性方程组理论的开山鼻祖——莱布尼茨. 莱布尼茨(Gottfried Wilhelm Leibniz,1646--1716),德国自然科学家、数学家、物理学家、历史学家和哲学家,和牛顿同为微积分的创建人. 他博览群书,涉猎百科,对丰富人类的科学知识宝库做出了不可磨灭的贡献,行列式与线性方程组理论是他留给人类的财富中很小但很重要的一部分.

莱布尼茨的第一个大的贡献便是引入了新符号. 莱布尼兹首先创立了采用两个记号的双标码记法,他在方程中使用系数
10,11,12;20,21,22;30,31,32,因为两个数字各有所指,所以相当于现代数学中方程系数符号的下标,即相当于$a_{10},a_{11},\ldots$的下标. 莱布尼兹在1693年给洛必达的一封信中给出了一个方程组:
\[\begin{cases}
        10+11x+12y=0 \\
        20+21x+22y=0 \\
        30+31x+32y=0
    \end{cases}\]

事实上,这一方程组有两个未知量和三个方程,当常数项不全为0时,这是一个非齐次线性方程组. 然后莱布尼茨首先对第一行和第二行消去变量$y$,有
\[10\cdot 22+11\cdot 22x-12\cdot 20-12\cdot 21x=0,\]
然后对第一行和第三行消去变量$y$,有
\[10\cdot 32+11\cdot 32x-12\cdot 30-12\cdot 31x=0,\]
对上述两式消去$x$,有
\[10\cdot 21\cdot 32+11\cdot 22\cdot 30+12\cdot 20\cdot 31-12\cdot 21\cdot 30-11\cdot 20\cdot 32-10\cdot 22\cdot 31=0,\]
事实上,这一式与现在我们所熟知的行列式形式
\[\begin{vmatrix}
        10 & 11 & 12 \\
        20 & 21 & 22 \\
        30 & 31 & 32
    \end{vmatrix}=0\]
完全一致. 回顾线性方程组有解的条件,即$(10,20,30)^\mathrm{T}$可以由$(11,21,31)^\mathrm{T},(12,22,32)^\mathrm{T}$线性表示,因此上面行列式等于0是方程组有解的必要条件,即莱布尼茨通过消元法解出了现在线性方程组有解的一个必要条件.

进一步地,莱布尼茨用记号$\overline{1\cdot 2\cdot 3\cdot 4}$表示现在的四阶行列式. 莱布尼兹说式$\overline{1\cdot 2\cdot 3\cdot 4}$由$4!=24$项组成,这些项可以由其中某一项指数的所有置换而得到. 这里为了叙述的完整性,我们先给出置换的相关定义:
\begin{definition}[置换]
    一个集合$S$的\term{置换}\index{zhihuan@置换 (permutation)}是一个从$S$到$S$的双射$p:S\to S$.
\end{definition}

\begin{example}
    设$S=\{1,2,3\}$,定义$p(1)=2,p(2)=3,p(3)=1$是$S$的一个置换,因为它是从$S$到$S$的双射. 我们通常将其记为$p_1=\begin{pmatrix}
            1 & 2 & 3 \\
            2 & 3 & 1
        \end{pmatrix}$,上面一行是按顺序排列的$S$的元素,下一行是按置换后的顺序排列的$S$的元素.

    同理$p_2=\begin{pmatrix}
            1 & 2 & 3 \\
            1 & 3 & 2
        \end{pmatrix}$,$p_3=\begin{pmatrix}
            1 & 2 & 3 \\
            1 & 2 & 3
        \end{pmatrix}$都是$S$的置换.
\end{example}

由此我们发现,对于有限集合$S$而言,其上的置换就是集合中元素多次对换位置,这里所谓的对换就是将两个元素交换. 当一个置换可以表示成连续偶数个对换时,称其为偶置换,否则称其为奇置换. 例如,$p_1$是偶置换,因为我们要先交换1,2得到2,1,3然后交换1,3得到结果. 同理,$p_2$是奇置换,$p_3$是偶置换.

莱布尼茨在文章中表示,由11,22,33,44的第二位数通过偶置换得到的那些项有相同的符号(取正号),其余取相反的符号(负号). 本质上,莱布尼兹知道现代一个行列式的一个组合定义,区别仅在于根据奇偶置换所确定的符号规则被逆序数代替,并且柯西也将这一低阶行列式的情形扩展为了一般的$n$阶情形,我们将在介绍柯西时给出相关的结论,届时我们也将进一步完善上面对于奇偶置换的讨论.

除此之外,基于这一``行列式''的定义此莱布尼茨也给出了最原始的Cramer法则,因此可以称为这一方向理论的鼻祖. 然而,莱布尼茨的很多工作都是后来(1850年左右)才被人们发现的,所以他的方法对后来其他数学家提出的规则几乎没有影响. 事实上,同时代的日本数学家关孝和在其著作《解伏元题法》中也提出了行列式的概念与算法. 而在Cramer法则上,麦克劳林和Cramer的工作更早被人们认识到.

麦克劳林(Maclaurin,1698--1746)是18世纪英国最具有影响的数学家之一. 他自幼聪慧勤奋,11岁便步入大学校门,17岁就以有关引力研究的论文获硕士学位,19岁受聘为阿伯丁马里沙尔学院数学教授,21岁当选为英国皇家学会会员. 麦克劳林最为读者熟知的贡献想必是麦克劳林级数展开式,实际上他还有几何学等方面其他贡献. 线性代数方面,在他1748年的遗著《代数论著》(\emph{A Treatise of Algebra})中,麦克劳林最先开创了用行列式的方法来求解含2个、3个和4个未知量的联立线性方程组. 遗憾的是,麦克劳林没能进一步给出一个明确的法则来确定符号. 虽然,书中的记法不太好,符号变化的规则又比较模糊,但它确实就是我们今天所使用的Cramer法则.

事实上,现在我们所熟知的Cramer法则是由瑞士数学家加布里埃尔·克莱姆(Gabriel Cramer,1704--1752)在1750年的著作《线性代数分析导言》(\emph{Introduction à l'analyse des lignes courbes algébriques})中给出的. 为了确定经过5个点的一般二次曲线的系数,他引入了这一著名的法则,并且因其符号上更为简洁明了的优越性而被人们所接受. 事实上,克莱姆最著名的工作是在1750年发表关于代数曲线方面的权威之作. 他最早证明一个第$n$度的曲线是由$n(n + 3)/2$个点来决定的.

\subsection{线性方程组与行列式的进一步研究}

在前人工作的基础上,关于线性方程组以及行列式的理论有了更快的发展. 裴蜀(E. Bézout,1730--1783),法国数学家. 曾在海军学校和皇家炮兵学校任教,主要从事代数方程理论的研究并取得一系列的成果. 1764年,裴蜀发表论文提出了行列式中项的构成规则和符号的形成规则. 他给出了行列式的一个循环构造规律,同时用不同于莱布尼兹、克莱姆的方法,给出了项的构成规则和符号确定规则. 他所作的成就对后来行列式理论的奠基和发展起着非常重要的作用. 同时,裴蜀在该文中证明了含$n$个未知量的$n$个齐次线性方程组有非零解的条件是其``结式''(系数行列式)等于零,跳出了前人对于求解方程组计算问题的讨论,转向对一般理论的讨论.

在行列式的发展史上,第一个对行列式理论做出连贯的逻辑的阐述,即把行列式理论与线性方程组求解相分离的人,是法国数学家范德蒙(A-T. Vandermonde,1735--1796)——他不仅把行列式应用于解线性方程组,而且对行列式理论本身进行了开创性研究. 范德蒙自幼在父亲的指导下学习音乐,但对数学有浓厚的兴趣,后来终于成为法兰西科学院院士. 他给出了用二阶子式和它们的余子式来展开行列式的法则,这跳出了前人从线性方程组角度研究行列式的范畴,因此就对行列式本身这一点来说,他是这门理论的奠基人. 当然,范德蒙还有一个读者十分熟知的工作,便是计算了范德蒙行列式,这一行列式对于后续的研究有非常重要的地位.

除此之外,范德蒙的工作也得到了进一步的推广. 1772年,拉普拉斯在一篇论文中证明了范德蒙提出的一些规则,并推广了他的展开行列式的方法,便有了大家熟知的按多行(多列)展开的拉普拉斯定理. 1779年,裴蜀(正是前面所介绍的,实际上这里介绍的数学家很多都有工作的交织)发表了一篇《代数方程的一般理论》的文章,这篇论文给出了解决非齐次线性方程组的方法,这个方法是他在克莱姆、范德蒙和拉普拉斯行列式理论基础上的总结. 除此之外,裴蜀在论文中还有其他很多关于行列式理论的发现:他改进了拉普拉斯展开式的另一个改进形式;得出了行列式的两行或两列相同则结果为零的结论;并结合线性方程的消元法得出了著名的``裴蜀定理''等.

接下来对行列式理论做了可谓``大一统''工作的是著名数学家柯西——是的,又是他,一个和欧拉、高斯一样无处不在的数学家. 1812年,柯西率先使用了双下标的方式表示方程组系数(即$a_{11}$这样的有两个数字组成的下标),有趣的是柯西当年还没有使用双竖线的方式表示行列式,而是采用$S(\pm a_{11}a_{22}\cdots a_{nn})$的形式. 现在为人熟知的双竖线的表示形式是后文将要介绍的矩阵论创始人凯莱在1841年率先使用的. 事实上,柯西给出了与现代行列式定义几乎完全一致的版本. 为了展开叙述柯西的理论,我们在此进一步讨论有关于置换的概念. 我们来看一个简单的置换
\[p=\begin{pmatrix}
        1 & 2 & 3 & 4 & 5 \\
        2 & 1 & 4 & 5 & 3
    \end{pmatrix},\]
我们可以将得到上述置换的过程分为两步. 第一步我们对1和2进行置换,得到
\[p_1=\begin{pmatrix}
        1 & 2 & 3 & 4 & 5 \\
        2 & 1 & 3 & 4 & 5
    \end{pmatrix},\]
然后我们对3,4,5进行置换,但保持上一步中已经改变的1和2不变,这一过程可以写为
\[p_2=\begin{pmatrix}
        1 & 2 & 3 & 4 & 5 \\
        1 & 2 & 4 & 5 & 3
    \end{pmatrix}.\]
就可以得到$p$. 事实上这接续的两步和映射的复合运算含义完全一致. 回忆$h=g\circ f$,$h(x)=g(f(x))$是先对$x$作用$f$然后作用$g$,这里实际上也是对$1,2,3,4,5$先做了$p_1$的置换然后做了$p_2$的置换,即$p=p_1\circ p_2$,写成乘法形式即为
\[\begin{pmatrix}
        1 & 2 & 3 & 4 & 5 \\
        2 & 1 & 4 & 5 & 3
    \end{pmatrix}=\begin{pmatrix}
        1 & 2 & 3 & 4 & 5 \\
        1 & 2 & 4 & 5 & 3
    \end{pmatrix}\begin{pmatrix}
        1 & 2 & 3 & 4 & 5 \\
        2 & 1 & 3 & 4 & 5
    \end{pmatrix}.\]
实际上,上面的置换$p_1$相当于将1变为了2,然后2又变回了1,这构成了一个长度为2的循环. $p_2$相当于将3变为了4,4变为了5,然后5又变回了3,因此构成了一个长度为3的循环. 因此上式也可以改写为
\[p=(3,4,5)(1,2)\]
其中$(3,4,5)$就表示从3到4,4到5,然后5到3的循环(事实上$(4,5,3),(5,3,4)$同理表示同一个循环),$(1,2)$也是同理. 事实上,因为$3,4,5$和$1,2$之间没有元素是重复的,因此可以称它们是不相交的. 事实上我们有如下定理,我们不加证明地直接给出:
\begin{theorem}
    $S$上的任意置换$p$都可以表示为长度$\geqslant 2$且不相交的循环的乘积,且这一分解式在不考虑循环顺序(即上面所说的$(3,4,5)$和$(4,5,3)$实则表示一个循环)下是唯一确定的.
\end{theorem}

事实上,我们在前面讲的对换(即两个元素交换顺序)就是长度为2的循环. 关于对换,我们也有一个重要的结论:
\begin{theorem}\label{thm:16:对换乘积}
    $S$上的任意置换$p$都可以表示为对换的乘积.
\end{theorem}

实际上这一结论是很直观的,$1,2,\ldots,n$的任意置换实际上都可以通过两两交换顺序得到. 最愚蠢的方式就是反过来思考如何从任意置换反推到$1,2,\ldots,n$,然后全部顺序调换即可. 反推的方式就是首先找到1的位置,然后一直向左对换到最左端,然后开始找2,对换到第二个位置,以此类推,因此这一结论是显然正确的. 但接下来的证明将会从另一个角度给出更为丰富的结果:

\begin{proof}

\end{proof}

事实上在介绍莱布尼茨的工作时我们就介绍了莱布尼茨利用奇数或偶数次对换作为依据确定行列式定义中每一个排列的乘积前的符号,这里我们给出严谨的关于符号的说明:
\begin{theorem}[置换的符号] \label{thm:16:置换的符号}
    设$p$是$S$上的一个置换,将$p$分解为对换的乘积:
    \[p=p_1\cdots p_k,\]
    则称
    \[\tau(p)=(-1)^k\]
    为$p$的\term{符号}(亦称符号差或奇偶性),它由置换$p$唯一确定且不依赖于对换分解的方法. 此外任取$q,r$也为$S$上的置换,则有
    \[\tau(qr)=\tau(q)\tau(r).\]
\end{theorem}

\begin{proof}

\end{proof}

基于这一定理,我们可以有如下合理的定义:
\begin{definition}
    若$\tau(p)=1$,则称$p$为$S$上的偶置换;若$\tau(p)=-1$,则称$p$为$S$上的奇置换.
\end{definition}

事实上我们在之前也定义了奇置换和偶置换,即从顺序排列的数列经过奇数还是偶数次对换可以得到最终的置换出发进行定义. 两个定义实际上是统一的,统一性由下面这一定理保证:
\begin{theorem}\label{thm:16:置换符号计算公式}
    设$S$上的置换$p$分解为长为$l_1,l_2,\ldots,l_m$的互不相交的循环的乘积,则
    \[\tau(p)=(-1)^{\sum\limits_{k=1}^m(l_k-1)}.\]
\end{theorem}

因为对换是长度为2的循环,事实上每一个$l_i-1$都等于1. 如果上述定理成立,那么奇数次的对换将会使得置换符号为$-1$,因为$-1$的奇数次方,偶置换则会得到符号为$-1$的偶数次方,即为1,因此两个定义统一.

\begin{proof}
    事实上,根据\autoref{thm:16:置换的符号},我们有
    \[\tau(p)=\tau(p_1)\cdots\tau(p_m),\]
    根据\autoref{thm:16:对换乘积} 的证明我们知道,$p_i$可以被写为$l_i-1$个对换的乘积,因此我们有$\tau(p_i)=(-1)^{l_k-1}$,因此有
    \[\tau(p)=(-1)^{\sum\limits_{k=1}^m(l_k-1)}.\]
\end{proof}

接下来我们就要看如何将上述置换的分解与符号的理论用于计算行列式. 柯西从$n$个数$a_1,\ldots,a_n$出发,作乘积$a_1\cdots a_n$,然后类似于范德蒙行列式,作所有不同元之间的差的积$\displaystyle\prod_{1\leqslant i\leqslant j\leqslant n}(a_j-a_i)$,得到乘积
\begin{equation}\label{eq:16:柯西行列式}
    a_1\cdots a_n\prod_{1\leqslant i\leqslant j\leqslant n}(a_j-a_i)
\end{equation}
柯西对这个乘积中各项所含的幂改写成第二个下标,例如把$a_2^3$改写为$a_{23}$,把这样改写后得到的表达式定义为一个行列式,记作$S(\pm a_{11}a_{22}\cdots a_{nn})$.

我们以$n=3$为例展示上面的过程. 根据柯西描述的算法,乘积为
\begin{align*}
               & a_1a_2a_3(a_2-a_1)(a_3-a_1)(a_3-a_2)                                                                               \\
    =          & a_1a_2^2a_3^3-a_1^2a_2a_3^3+a_1^3a_2a_3^2-a_1a_2^3a_3^2+a_1^2a_2^3a_3-a_1^3a_2^2a_3                                \\
    \triangleq & a_{11}a_{22}a_{33}-a_{12}a_{21}a_{33}+a_{13}a_{21}a_{32}-a_{11}a_{23}a_{32}+a_{12}a_{23}a_{31}-a_{13}a_{22}a_{31}.
\end{align*}

事实上这与今天我们熟悉的三阶行列式计算公式完全一致,事实上$n$阶都是一致的. 柯西天才地用一个很简短的抽象公式将前人找到的规律描述了出来,同时也发明了双下标的表示,将行列式可以写成$n\times n$的矩形方阵形式,并且沿用至今.

事实上,由\autoref{eq:16:柯西行列式} 展开得到的式子中的项不难看出都可以写成如下形式:
\[\prod a_{1p(1)}\ldots a_{np(n)}\]
其中$p$是集合$S=\{1,2,\ldots,n\}$上的一个置换,因为乘积前面的$a_1\cdots a_n$会保证每一个数都出现,而后面的乘积由排列组合的知识可知只有$a_1,\ldots,a_n$的次数分别为$0,1,\ldots,n-1$的一个置换才会留下来且前面的系数为1或$-1$. 接下来便有一个问题,即前面的系数究竟是1还是$-1$. 柯西使用的方法与前面介绍的几乎完全一致!他就是通过计算$(-1)^{\sum\limits_{k=1}^m(l_k-1)}$这一方式判断的,其中$l_k$是上述置换$p$进行循环分解后各项的长度. 基于此,我们有行列式定义如下:
\begin{definition}
    $n$阶行列式
    \[\begin{vmatrix}
            a_{11} & a_{12} & \cdots & a_{1n} \\
            a_{21} & a_{22} & \cdots & a_{2n} \\
            \vdots & \vdots & \ddots & \vdots \\
            a_{n1} & a_{n2} & \cdots & a_{nn}
        \end{vmatrix}=\sum_{p\in S_n}\tau(p)\prod_{i=1}^na_{ip(i)},\]
    其中$S_n$是集合$S=\{1,2,\ldots,n\}$上置换的全体,$\tau(p)$是置换$p$的符号.
\end{definition}

在很多教科书上,这一定义也被称为逆序数定义,这是因为置换的符号实际上也可以视为所谓``逆序对''个数的体现. 那么何为逆序对呢?其实一对数就是$(a_i,a_j)$,其中$i<j$,当$a_i>a_j$时,我们称这一对数为逆序对. 例如,对于置换
\[p=\begin{pmatrix}
        1 & 2 & 3 \\
        3 & 1 & 2
    \end{pmatrix},\]
$(3,1)$和$(3,2)$都是逆序对,因为3在1前面但比1大,3在2前面但比2大. 而数对$(1,2)$则是顺序对,因为1在2前面且比2小,这与原先$1,2,3$的排列顺序是统一的.

基于逆序对的定义我们可以有奇置换和偶置换的另一个定义:
\begin{definition}
    我们称逆序数为奇数的置换为奇置换,逆序数为偶数的置换为偶置换.
\end{definition}

显然我们也必须要求这一定义和前面的定义是一致的,事实上我们有如下定理:
\begin{theorem}
    任意置换$p$的逆序数的奇偶性与其可以被分解为对换乘积的个数的奇偶性相同.
\end{theorem}

\begin{proof}

\end{proof}

由这一定理以及\autoref{thm:16:置换符号计算公式} 我们知道逆序数为奇数时,置换符号为$-1$,逆序数为偶数时,置换符号为1,因此这一定义与前面的定义是一致的.

由此我们可以看出,行列式的逆序数定义实际上最开始来源于莱布尼茨、克莱姆等人最朴素的从低阶出发的探索,它们找到了一些规律,这些规律由天才数学家柯西进行抽象总结,得到具有普适性的方法,变成了上面沿用至今的严格定义. 而后人又结合置换、逆序数等理论进行重新叙述,再嵌套一层抽象,最终上面完整的叙述逻辑. 在这里我们看到一个很抽象甚至初看没有什么道理的定义是如何自然演化而来的,实际上是数学理论螺旋式进步以及多个理论(可能对应很多条数学发展支线)在教育家们的手中进行合理排列后所呈现的状态.

更进一步地,1815年,柯西发表了一篇关于行列式理论的基础性文章. 在这篇文章中它不仅用这个名字代替了几个旧的术语,也在文章中给出了系统的一般行列式乘法定理,证明了新组的行列式是原来两个组的行列式的乘积. 在这篇论文中,柯西第一次论述了包括一个给定的矩阵的伴随矩阵的思想,以及通过展开任何行或者列来计算行列式的步骤,完善了范德蒙和拉普拉斯的工作,给出了严谨的证明. 在柯西的行列式的工作中,还涉及到对称矩阵以及相似变换等问题. 在柯西1826年的《微积分在几何中的应用教程》中,讨论了后续学习中将要介绍的一些二次型理论,以及实对称矩阵特征值均为实数(后续会讲解)等重要结论. 除此之外,柯西在相似行列式的研究中,证明了大家熟知的相似变换有相同的特征值的结论. 由此可见,柯西这一数学天才对于后世的影响是无比深远的,从记号层面的革新,到行列式展开、行列式乘法等理论的大一统,以及现在大家熟知的结论的证明,都能看出柯西贡献的突出与伟大.

继柯西之后在行列式理论方面最高产的人就是德国数学家雅可比(C. Jacobi,1804--1851),他引进了函数行列式,即``Jacobi 行列式''(读者学习多元微积分时会十分熟悉这一名词),指出函数行列式在多重积分的变量替换中的作用,给出了函数行列式的导数公式. 雅可比的著名论文《论行列式的形成和性质》标志着行列式系统理论的建成. 事实上,行列式在数学分析、几何学、线性方程组理论、二次型理论等多方面的应用,促使行列式理论自身在19世纪也得到了很大发展. 整个19世纪都有行列式的新结果. 除了一般行列式的大量定理之外,还有许多有关特殊行列式的其他定理都相继得到.

最后笔者要在此说明一点. 我们在讲义中介绍了行列式最常见的三种定义方式. 事实上,按照历史的发展脉络,的确是现在看来最不直观的逆序数定义的思想首先出现的. 《大学数学:代数与几何》中给出的公理化定义有很强的几何直观性,也与列向量组的线性相关性等有很直接的关联,但实际上几何直观源于后来拉格朗日发现行列式和以其列向量构成的四面体的体积之间的关系,是远在莱布尼茨的思想出现之后才讨论的. 而关于行列式展开的定义根据上面的讨论也知道,是范德蒙、拉普拉斯和柯西接力提出并给出严谨证明才得到的.

事实上,笔者在编写讲义的时候就发现,无论从哪个定义出发定义行列式都是显得``毫无道理''的,因为完全缺乏引入,这不像之前研究线性空间那般自然(因为我们发现了方程组行向量间的线性相关性影响了解的唯一性,并且线性空间也有高中学习的平面向量的直观作为基础),行列式的定义直接丢出来会显得非常笨拙而且没有道理,但我们研究其性质会发现它和可逆、矩阵的秩甚至以后的特征值理论有很强的关联,行列式仿佛成为了一个无头但有尾的理论,这可能也是为什么《线性代数应该这样学》完全抛弃了行列式来讲述线性代数——因为这非常难引入且不是必要的. 但笔者还是希望保留行列式这一具有重要历史地位的理论,并且它对于之后的很多研究都有重要意义,所以笔者选择在史海拾遗中从历史的角度提供一种``直观''——它来源于数学家最开始对一些问题的研究. 因此我们详细地描述了莱布尼茨如何从消元法解线性方程组得到类似于现代行列式的定义的过程. 尽管这是低维的情况,但接下来在讲述柯西的工作时,$n$维的情况在逻辑上就像是自然的推广了(我们相信在历史中也是如此,前人对于线性方程组解的形式(如Cramer法则)和行列式的研究都构成了柯西研究的出发点,在此基础上柯西做了更进一步的抽象得到了现在的行列式定义,后人又结合了置换等概念将这一理论进一步形式化),这样我们也勉强算是梳理出了一个有引入的能更让初学者接受的行列式理论:至少我们从线性方程组的解的讨论起步——于是根据朝花夕拾的知识我们知道行列式一定和线性相关、矩阵的秩等概念有很强的关联,得到一些简单的结论,然后不断抽象直至今天呈现在读者面前的令人摸不着头脑的理论,虽然非常冗长,但相信能让读者接受这一概念的引入也是比较自然的.

数学不是魔术,不是从无到有的魔法,其发展历程必然是螺旋式上升的过程,很多不直观的概念可能只是因为历经数百年很多数学家不断改进而使人很难看出当年原本自然的想法源于何处. 希望读者读完本讲后再看教材时,能体会到每一个概念、每一个定理背后的历史厚重感,它们都是历代数学家在前人肩膀上不断总结、创新而来的,这些想法或是沉稳的推进,也可能是天才的智慧. 也许在未来很多年以后的教材上,有一个全新的概念或者结论,它可能只是短短的一行描述,但那也许凝结着现在正阅读着这段文字的你未来很多年研究的心血——这也许就是一种价值的实现.

\subsection{矩阵理论的发展}

随着线性方程组和行列式理论的建立和发展,在行列式基础之上的矩阵理论发展非常迅速. ``矩阵''这个词是由西尔维斯特首先使用的,他是为了将数字的矩形阵列区别于行列式而发明了这个术语. 而实际上,矩阵这个课题在诞生之前就已经发展的很好了. 从行列式的大量工作中明显的表现出来,不管行列式的值是否与问题有关,方阵本身都可以研究和使用,矩阵的许多基本性质也是在行列式的发展中建立起来的. 在逻辑上,矩阵的概念应先于行列式的概念,然而在历史上次序正好相反.

虽然矩阵一词是西尔维斯特率先发明的,但英国数学家凯莱(A. Cayley,1821--1895)一般被公认为是矩阵论的创立者,因为在西尔维斯特创用矩阵术语以前,凯莱对于矩阵的有关概念及其性质就有所研究. 1843年,凯莱即己研究三阶以上的高阶矩阵的行列式理论(\emph{On the theory of determinants}),L. Gegenbauer、M-Lecat、L. H. Rice等在这个领域又进行了扩展. 1846年,凯莱定义了转置矩阵以及对称矩阵,与现代的定义完全一致. 在1855--1858年间,凯莱在矩阵方面做了许多开创性的工作. 1855年,凯莱注意到在线性方程组中使用矩阵是非常方便的,因而引进矩阵以简化记号,这就有了现在我们使用的阶梯矩阵等术语以及$AX=\vec{b}$的记号.

1858年,凯莱发表了重要文章《矩阵论的研究报告》(\emph{A memoir on the theory of matrices}). 在该研究报告中,凯莱系统地阐述了矩阵的理论体系,如矩阵概念的引入、相关概念和运算的定义,使得矩阵从零散的知识发展为系统完善的理论体系. 凯莱定义了矩阵加法和数乘运算,并且从变换的复合引入了矩阵乘法的运算法则,也给出了一些特殊矩阵例如零矩阵、单位矩阵等,同时也说明了两个矩阵相乘不符合交换律,但也着重强调了矩阵乘法是可结合的. 除此之外,凯莱也引入了逆矩阵的概念. 凯莱在文章中采用单个的符号表示矩阵,证明了矩阵$A$可逆时,方程$AX=\vec{b}$的解可以写为$X=A^{-1}\vec{b}$,并且也给出了矩阵可逆时
\[A^{-1}=\frac{1}{|A|}A^*.\]
凯莱还利用一般的代数运算和矩阵运算的相似性得出了矩阵的一些结论. 例如当行列式为零时矩阵不可逆,零矩阵不可逆,两个非零矩阵乘积可以为零矩阵等结论. 除此之外,凯莱在文章中采用单个的符号表示矩阵,推出了方阵的特征多项式的形式,并说明了特征多项式的根就是特征值的重要结论. 除此之外,凯莱也证明了我们后面要详细介绍的``哈密顿-凯莱''定理的一部分,这被称为``矩阵理论中最著名的理论之一''.

凯莱第一个把矩阵作为独立的概念提出来,并作为独立的理论加以研究. 可以说,《矩阵论的研究报告》的公开发表,标志着矩阵理论作为一个独立数学分支的诞生. 但我们之前也提到,矩阵一词是西尔维斯特在研究方程的个数与未知量的个数不相同的线性方程组时最先使用的. 因此我们也很有必要接着介绍凯莱的挚友——西尔维斯特在矩阵理论方面的成果.

詹姆斯·约瑟夫·西尔维斯特(James Joseph Sylvester,1814--1897),1829年进入设在利物浦的皇家学会的学校学习,他学习努力,成绩突出,曾因解决了美国抽彩承包人提出的一个排列问题而得到500美元的数学奖金.1846年西尔维斯特进入内殿(Inner Temple)法学协会,并于1850年取得律师资格. 在这期间他和同时进入林肯法律学会的凯莱建立起了深厚的友谊. 他们在从事法律业务的间隙,经常在一起交流数学研究的成果. 西尔维斯特一生致力于纯数学的研究,他和凯莱、哈密顿等人一起开创了自牛顿以来英国纯粹数学的繁荣局面. 西尔维斯特的成就主要在代数方面,在代数方程论、数论等诸领域都有重要的贡献. 西尔维斯特一生创造过许多数学名词,流传至今的如矩阵、判别式等都是他首先使用的.

1850年,西尔维斯特在研究方程的个数与未知量的个数不相同的线性方程组时,由于无法使用行列式(因为行列式要求行列数相同),所以引入了矩阵一词来表由$m$行$n$列元素组成的矩形阵列,西尔维斯特也引入了对角矩阵、数量矩阵等概念.1879年弗罗伯纽斯给出矩阵的秩的概念后,1884年,西尔维斯特给出了零性的概念和零性律:他把矩阵的阶数与秩的差叫做矩阵的零性,并说明了两个(而且可以推广为任意有限数目)矩阵乘积的零性不能比任意因子的零性小,也不能比组成这一乘积的因子的零性之和大. 西尔维斯特的这一零性律现在应当叙述为:
\[r(A)+r(B)-n\leqslant r(AB)\leqslant\min\{r(A),r(B)\},\]
这是矩阵理论中关于矩阵乘积的秩的一个重要定理. 除此之外,西尔维斯特和凯莱也就矩阵方程
\[A_1XB_1+\cdots+A_kXB_k=C\]
和
\[AX-XB=O,\]
这里篇幅有限我们不再展开叙述. 事实上,在之后的二次型理论中,我们还会学习到所谓的西尔维斯特惯性定理,这也是非常核心的理论.

在矩阵论发展史上,弗罗贝尼乌斯(G. Frobenius, 1849--1917)的贡献是不可磨灭的. 1870年左右,群论成为数学研究的主流之一,弗罗贝尼乌斯在柏林时就受到库默尔和克罗内克的影响,对抽象群理论产生兴趣并从事这方面的研究,发表了多篇有价值的论文. 1892年,他重返柏林大学任数学教授. 1893年当选为柏林普鲁士科学院院士. 弗罗贝尼乌斯的主要数学贡献在群论方面,在行列式、矩阵、双线性型以及代数结构方面也有出色的工作. 矩阵论方面,1878年,弗罗贝尼乌斯引进了西尔维斯特$\lambda$矩阵的行列式因子、不变因子和初等因子等概念(在一般的高等代数教材中都会由此引入讨论若当标准形),给出了正交矩阵、相似矩阵、合同矩阵等概念(与现在的定义是完全一致的),指出了各种不同类型的矩阵的关系. 1894年,他又对1878年的不变因子和初等因子理论做了更深入的工作,进一步整理了维尔斯特拉斯不变因子和初等因子的理论. 1879年,弗罗贝尼乌斯引进了矩阵的秩的概念: 矩阵的秩就是矩阵中非零子式的最大阶数. 他也引进了行列式秩的定义:如果一个行列式的所有$r+1$阶子式为0,但至少有一个$r$阶子式不为零,那么就称$r$为行列式的秩.

在上述三位数学家的工作下,矩阵论中的一个核心问题:矩阵约化与分解不断地有了新的突破. 我们将在后续章节用大量的篇幅介绍这一主题——因此这里也许可以算是承上启下的一节. 1854年,约当研究了矩阵化为标准型的问题. 1892年,梅茨勒(H. Metzler)引进了矩阵的超越函数概念并将其写成矩阵的幂级数的形式. 傅立叶、西尔和庞加莱的著作中还讨论了无限阶矩阵问题,这主要是适用方程发展的需要而开始的. 事实上,矩阵由最初作为一种工具经过两个多世纪的发展,现在已成为独立的一门数学分支——矩阵论. 而矩阵论又可分为矩阵方程论、矩阵分解论和广义逆矩阵论等矩阵的现代理论. 矩阵及其理论现已广泛地应用于现代科技的各个领域,相信很多工科读者在将来的学习中将会大量运用这一方面的结果.

\subsection{线性代数的应用:解析几何的发展}

线性代数的发展与解析几何的发展有着密切的联系,应该说二者间在数学发展史上来看是互相促进的关系. 一方面,从希腊时代到1600年几何统治着数学,代数居于附庸的地位. 而解析几何为确立代数在数学界的地位铺平了道路. 1600年以后,代数才从几何统治的桎梏下解放出来,成为一门独立的基础数学科目,占据了它在数学中应有的地位. 另一方面,我们接下来也将会展开很多用线性代数知识解决几何问题的实例.

\subsection{线性空间与线性映射的角度}

前面的讨论我们一直讨论的是线性方程组与行列式关联的历史,本讲义中最重视的线性空间和线性映射理论被搁置了. 这一节我们将重点放在这一部分,供

物理学的发展带动了向量理论及向量空间的发展,而向量理论和向量空间的发展也打开了新的数学前景. 当今数学意义上的向量及向量空间的概念内容丰富,形式多样.

向量空间第一个具体定义的是由皮亚诺(G. Peano,1858--1932)在1888年的《几何计算》中给出,但是影响并不广泛. 直到1918年外尔(Hermann Weyl,1885--1955)的工作,使得人们重新认识到了皮亚诺公理化定义的重要性. 大约在1920年左右,分析学家巴拿赫、哈恩和维纳等对皮亚诺的向量空间做了进一步的研究,并引起了广泛的影响,随之而产生了如赋范向量空间、希尔伯特空间等等.

\section{推广:线性代数之后的线性和代数}

\subsection{泛函分析}

\subsection{抽象代数}

本小节我们接着前面初等代数发展为高等代数的两个方向之二,继续讨论有关于高次方程与多项式理论的历史.

\section{进阶:(线性)代数的进一步发展}

\subsection{抽象代数}

\subsection{泛函分析}

\subsection{抽象代数}

\section{进阶:本世纪的线性代数}

在这一节中,我们就几个专题探讨线性代数中的一些概念在上世纪末到本世纪的过程中的发展,主要探讨矩阵论而非线性空间理论的部分. 这是因为矩阵论的发展相对而言比较接地气,大部分从结果上看非常简捷明快. 本节提及的很多结论,在证明中都有众多数学分支如草蛇灰线穿行其中. 但是,一个初步的展现未必需要完整的表述,省略一些证明也不妨碍读者领略现代矩阵论研究的风采. 至于从线性代数真正意义上``生发开去''的东西,我们将留在下一节讨论.

\subsection{正定性:从数到矩阵,以及本世纪的矩阵论}

二次型在这份讲义当中尚未被置于中心地位. 往早期讲,它的研究缘起于费马大定理,而往晚近讲,二次型的代数理论研究几近一个完全的分支. 我们在此抛开Witt理论之类的东西不谈,暂且只就谈正定性及其与矩阵论的关系,对于二次空间(quadratic space)的研究,及二次型与二次互反律(quadratic reciprocity law)的关系等等主题,读者自可参考志村五郎(Goro Shimura)的书《二次型的算术》(\textit{Arithmetic of Quadratic Forms}).

首先考虑这个问题:如何判断一个多项式是非负的?从初中开始,我们就知道有配方法:如果能把一个多项式拆成一堆非负项的和,那么它当然就是非负的,这个方法在二次型正定性的讨论中也被多次用到. 很自然的,下面这个问题就被提出了:

\begin{quote}
    \kaishu
    是不是任意一个非负多项式都能被拆成一系列多项式的平方和?
\end{quote}

答案是否定的. 一个典型的反例在 Motzkin 1967 年出版的一本关于代数-几何不等式的书中给出:
\[ P(x, y, z) = x^6 + y^4z^2 + y^2z^4 -3x^2y^2z^2 \]

希尔伯特第十七问题就是这个问题的一个推广:
\begin{quote}
    \kaishu
    是不是任意一个非负多项式都能被拆成一系列有理函数的平方和?
\end{quote}

这样的推广终归是成立了,它的证明由 E. Artin 在 1927 年给出,C. N. Delzell 在 1984 年甚至给出了一个算法以构造这样的平方和. 但是,故事到这里还没有结束,把``数''推广到``矩阵'',有趣的事情发生了:

\begin{theorem}[Helton, 2002]\label{thm:16:helton2002}
    对称的矩阵多项式是半正定的,当且仅当它能被拆成一组矩阵多项式的平方和.
\end{theorem}

当然,我们需要澄清一些概念:

\begin{definition}
    一个 $n$ 元的矩阵单项式是一个形如 $am_1m_2 \cdots m_k$ 的连乘积,其中 $a \in \R$,$m_i$ 为矩阵变元 $x_j \enspace(j \leqslant n)$ 或其转置. 一个 $n$ 元的矩阵多项式是有限个 $n$ 元矩阵单项式的和.
\end{definition}

\begin{definition}
    一个 $n$ 元的矩阵多项式 $Q$ 的值域 $\im Q$ 被定义为

    \[ \bigcup_{m = 1}^\infty\left\{Q(A_1, A_2, \ldots, A_n) \mid A_i \in \mathbf{M}_m(\R)\right\} \]

    即在代入任意 $n$ 个 $m$ 阶方阵之后所能得到的结果.
\end{definition}

\begin{definition}
    称一个矩阵多项式 $Q$ 是对称的,如果所有 $\im Q$ 中的矩阵都是对称的;称它是半正定的,如果所有 $\im Q$ 中的矩阵都是半正定的.
\end{definition}

\begin{definition}
    所谓矩阵多项式的平方和,指的是如下形式的有限和:
    \[ P(x_1, x_2, \ldots, x_n) = \sum_{i = 1}^k h_i(x_1, x_2, \ldots, x_n)h_i^\mathrm{T}(x_1, x_2, \ldots, x_n) \]
    其中 $h_i$ 均为 $n$ 元矩阵多项式. 显然,这个平方和一定是半正定的.
\end{definition}

\autoref{thm:16:helton2002} 的证明已经超出了这份讲义所能探讨的范畴. 其原论文足有二十页,证明过程可谓相当详尽,有兴趣的读者可自行查阅.

下面我们再给出一个将数推广到矩阵的相关问题,它在某种意义上也与正定性有关,而且也是本世纪的成果. 最小二乘法在本讲义前面很早的地方就有介绍,同样将其推广到矩阵,就要求我们推广距离的概念:

\begin{definition}
    考虑 $n$ 阶正定矩阵 $A, B$,定义它们之间的迹度量距离(trace metric distance)为:
    \[ \delta(A, B) = \sqrt{\sum_{i = 1}^n \log^2\lambda_i(A^{-1}B)} \]
    其中 $\lambda_i(M)$ 表示矩阵 $M$ 的第 $i$ 个特征值.
\end{definition}

\begin{definition}
    $k$ 个矩阵 $A_1, A_2, \ldots, A_k$ 的 Karcher 均值(Karcher mean)指的是正定矩阵 $X$,使得 $X$ 到所有 $A_i$ 的迹度量距离的平方和最小,将其记作 $\sigma(A_1, A_2, \ldots, A_k)$
\end{definition}

这个概念由 Karcher 在 1973 年先引入到任意度量空间中,这里呈现的是 Moakher 在 2005 年将其引入到矩阵中的版本. 在 2006 年,Bhatia 和 Holbrook 合作的论文呈现了它的一系列性质,并给出了一个猜想. 这个猜想是对下面这个简单事实的推广:

\begin{lemma}
    考虑 $x_1, x_2, \ldots, x_n, y_1, y_2, \ldots, y_n \in \R$,且 $\forall i \leqslant n,\enspace x_i \leqslant y_i$. 记 $x, y$ 分别为使得
    \[ \sum_{i = 1}^n (x - x_i)^2, \quad \sum_{i = 1}^n (y - y_i)^2 \]
    取得极小值的 $x$ 和 $y$,则 $x \leqslant y$.
\end{lemma}

很显然,对吧?他们猜测,Karcher 均值也有这种美好的单调性,不过,首先要在正定矩阵上引进序关系:

\begin{definition}
    如果 $B - A$ 正定,则 $A \leqslant B$.
\end{definition}

\begin{theorem}[Bhatia-Holbrook, 2006, Lawson-Lim, 2011]
    考虑 $A_1, A_2, \ldots, A_n, B_1, B_2, \ldots, B_n$ 为 $m$ 阶正定矩阵,如果 $\forall i \leqslant n,\enspace A_i \leqslant B_i$,则
    \[ \sigma(A_1, A_2, \ldots, A_n) \leqslant \sigma(B_1, B_2, \ldots, B_n) \]
\end{theorem}

等等,为什么是定理?正如我们已经暗示的,这个结果由 Lawson 和 Lim 在 2011 年成功证明了,其证明用到了 Loewner-Heinz 全局非正曲率度量空间(Loewner-Heinz globally nonpositively curved metric spaces)的一些性质,事实上这就是正定矩阵集合全体的复杂性带来的.

提及特征值,我们不妨再看看另外一个不那么幸运的结果,它是一个被证伪的猜想.

\begin{definition}
    称一个方阵 $A$ 的谱半径(spectral radius)为其特征值的绝对值的最大者,记为 $\rho(A)$.
\end{definition}

这个定义是非常有用的,如果我们注意到以下两个定理:

\begin{theorem}
    对于可对角化的矩阵 $U$,以下式子成立:
    \[ \lVert Uv \rVert \leqslant \rho(U) \lVert v \rVert \]
\end{theorem}

这是谱定理的一个直接推论. 注意,如果 $U$ 不是可对角化的,那么这个式子不成立,读者可以自行尝试构造反例,应当不难.

\begin{theorem}
    \[ \rho(A_1A_2\cdots A_k) \leqslant \rho(A_1)\rho(A_2)\cdots \rho(A_k) \]
\end{theorem}

这是 Gelfand 公式的一个直接推论. 也就是说,如果我们限制了所有矩阵 $A_i$ 都是可对角化的(例如在许多物理过程中),可以通过谱半径的乘积来估计乘积的谱半径的上界,这对于很多连续进行线性变换的系统的估计都是非常关键的. 而且,下面这个定义也一样重要:

\begin{definition}[广义谱半径]
    设 $\Sigma$ 为有限个 $m$ 阶方阵的集合,称其广义谱半径(generalized spectral radius)为
    \[ \rho(\Sigma) = \varlimsup_{k \to \infty} \rho_k(\Sigma) \]
    其中
    \[ \rho_k(\Sigma) = \max\{\rho(A_1A_2 \cdots A_k) \mid A_i \in \Sigma,\enspace i \leqslant k\} \]
    $\varlimsup$ 意指上极限,即最大的聚点.
\end{definition}

这个定义的物理意义可以类似地理解:当我们有一大堆可能的线性变换的时候,进行趋近于无穷次的随机选取变换,最终结果的范数最大值很可能落在哪里. 当然,很容易看出来,对于任意的 $k$,都有 $\rho_k(\Sigma) \leqslant \rho(\Sigma)$. 这个倒霉猜想由 Lagarias 和 Wang 在 1995 年提出,被称为有限性猜想:

\begin{quote}
    \kaishu
    对于任意 $\Sigma$,存在 $k$ 使得 $\rho_k(\Sigma) = \rho(\Sigma)$.
\end{quote}

也就是说,广义谱范数可以在有限步内抵达. 但不幸的是,这个猜想在七年后被 Bousch 和 Mairesse 证伪了,证伪的方法事实上是直接构造了一个 $\Sigma$ 的例子. 但是,他们的构造在此如要呈现将花费过多的笔墨,因为他们完全是从迭代函数系统(iterative function system)出发完成构造的,要用到关于 Lyapunov 指数的一些知识. 当然,对于读完这份讲义的读者,这些内容并不会太困难,不妨找来他们的论文一探究竟.

顺便一提,证明对于绝对值最小的特征值来说类似的猜想不成立特别简单,构造的例子如下,读者不妨自行思考一下为什么.
\[
    \begin{bmatrix}
        2 & 0 \\ 0 & 1/2
    \end{bmatrix}, \quad \begin{bmatrix}
        1/3 & 0 \\ 0 & 3
    \end{bmatrix}
\]

\subsection{线性方程组的解:快一点,再快一点!}

这一节讨论的主要是如何求解一个线性方程组
\[ A\vec{x} = \vec{b} \]

当然,读者会说,高斯消元法嘛,第一讲就已经讲过了,如果 $A$ 可逆那就是 $\vec{x} = A^{-1}\vec{b}$ 呗. 也不算错,那如果 $A$ 干脆不是一个方阵呢?确切的解自然不存在,最小二乘解也由 Penrose-Moore 逆给出:

\begin{definition}
    考虑 $m \times n$ 矩阵 $A$,它的 Penrose-Moore 逆为一个 $n \times m$ 矩阵 $A^\dagger$ 满足以下条件:

    \begin{enumerate}
        \item $A A^\dagger A = A$

        \item $A^\dagger A A^\dagger = A^\dagger$

        \item $A^\dagger A$ 和 $AA^\dagger$ 都是对称(或者厄米)的
    \end{enumerate}

    这样的矩阵存在且唯一.
\end{definition}

当然,不难写出近似的精度的定义:

\begin{definition}
    称 $\widetilde{\vec{x}}$ 是 $A\vec{x} = \vec{b}$ 的一个精度为 $\varepsilon$ 的近似解,如果:
    \[ \lVert \widetilde{\vec{x}} - A^\dagger \vec{b} \rVert_A \leqslant \varepsilon \lVert A^\dagger \vec{b} \rVert_A \]
    其中
    \[ \lVert \vec{x} \rVert_A = \sqrt{\vec{x}^\mathrm{T}A\vec{x}} \]
    称为由 $A$ 诱导的范数.
\end{definition}

这是对于一般的矩阵的结果. 通常情况下,我们只要处理实对称矩阵,甚至还能伴随一个更强的条件:

\begin{definition} \index{ruozhuduijiao@弱主对角占优 (weakly diagonally dominant)}
    称矩阵 $(a_{ij})$ 是\term{弱主对角占优}的,如果
    \[ a_{ii} \geqslant \sum_{j \neq i} |a_{ij}| \]
    对于任意 $i$ 成立.
\end{definition}

实对称的主对角占优矩阵对应的线性方程组求解当然可以使用直接求逆的方法,但是由于在处理图论问题以及求解一些椭圆型偏微分方程时,它往往以非常大的规模出现,所以哪怕是非线性时间的算法,也显得比较缓慢了,而到达近线性的突破到现在还不到十年:

\begin{theorem}[Spielman-Teng, 2014]
    存在一个在任意精度 $\varepsilon$ 下求解实对称的主对角占优矩阵 $A$ 对应的方程组 $Ax = b$ 的随机算法,其时间复杂度为
    \[ O(m \log^c n \log(1 / \varepsilon)) \]
    其中 $m$ 为 $A$ 中的非零元个数,$c$ 为一常数.
\end{theorem}

我们无意在此详述这个算法如何运作. 事实上,稍有图论基础的读者应不难看懂原论文. 它的核心在于,使用好的图采样来稀疏化,进而优化求解.

\subsection{随机矩阵:吸引了陶哲轩的未解之谜}

既然提到随机算法,那不妨也介绍一下随机矩阵的相关研究. 这就不像前面几个领域在本世纪内的研究较为松散,反而是如火如荼,哪怕是将近十年的主要结果罗列一番也破费精力. 由于笔者对概率论不甚熟悉,也不曾深究这些结论的证明,在此,我们仅罗列一些结果,有些较新的结果的准确性可能需要明眼的读者自行甄别.

首先,何谓随机矩阵?我们这里只探讨最简单的模型:Bernoulli 矩阵.

\begin{definition}
    称一个矩阵为 Bernoulli 矩阵,如果它的每一个元素都是独立同分布、成功概率 $p = 1/2$ 的 Bernoulli 随机变量.
\end{definition}

第一个问题是,这玩意有多大概率是奇异的?简单观察一下,不难猜想,奇异性的来源也就是行或者列相同,因此,我们给出以下猜想:

\begin{theorem}[Tikhomirov, 2020] \label{thm:16:tik2020}
    一个 $n$ 阶 Bernoulli 矩阵奇异的概率为:
    \[ P_n = \left( \frac{1}{2} + o(1) \right)^n \]
\end{theorem}

这个结果的证明几经波折. 原始的猜想早在上世纪就被正式阐述过,早在 1967 年,Koml\'os 就已经表明,$\displaystyle\lim_{n \to \infty} P_n = 0$. 直到 1995 年,Kahn 等人才给出第一个指数级别的估计 $P_n = O(0.999^n)$. 在 2007 年,陶哲轩和 Van Vu 给出了 $P_n = O(0.958^n)$ 的估计并在第二年获得了如下突破:

\begin{theorem}[Tao-Vu, 2007]
    一个 $n$ 阶 Bernoulli 矩阵奇异的概率为:
    \[ P_n = \left( \frac{3}{4} + o(1) \right)^n \]
\end{theorem}

下一个突破性进展在 2010 年,Bourgain 等人进一步压下了上界:

\begin{theorem}[Bourgain, 2010]
    一个 $n$ 阶 Bernoulli 矩阵奇异的概率为:
    \[ P_n = \left( \frac{1}{\sqrt{2}} + o(1) \right)^n \]
\end{theorem}

最终形成了我们看到的\autoref{thm:16:tik2020}. 事实上,Tikhomirov 证明的结果比这个还要广一点:

\begin{theorem}[Tikhomirov, 2020]
    一个 $n$ 阶,成功概率为 $p \in (0, 1/2]$ 的 Bernoulli 矩阵奇异的概率为:
    \[ P_n = \left( 1 - p + o(1) \right)^n \]
\end{theorem}

Jain 等人在同年也给出了另一个关于任意有限支集的分布代替独立同分布的结果.

另外的问题还有:
\begin{enumerate}
    \item 对于高斯分布的矩阵,结果如何?

    \item 最大特征值,也就是谱半径的分布如何?(Tracy-Widom 规则及其推广)

    \item 所有特征值的分布如何?(Circular law conjecture,由陶哲轩在 2010 年证明)

    \item 范数和逆的范数的分布如何?(Spielman-Teng 猜想,2002 年被形式化提出,Littlewood-Offord 问题的反问题,陶哲轩和 Van Vu 在 2009 年给出了一个初步结果)

    \item 其它非独立同分布的随机模型如何?(关于随机带矩阵(random band matrix),Khorunzhiy 猜想在 2010 年被 Sodin 证明)
\end{enumerate}

\subsection{机器学习!}

当然咯,提到本世纪的线性代数,也不能不提及机器学习的一些发展. 我们仅就几个特定问题,讨论线性代数方法在机器学习中的应用. 这里提及的很多结论会比前面的结果更加轻松,亦可供对此感兴趣的读者仔细参详,甚至自行写出证明.

我们要谈的第一个问题叫做压缩感知(compressed sensing). 给定一个稀疏信号和有限次的测量,测量的次数很可能远少于要复原的变量的个数,这个时候,我们需要多少次测量才能以某个精度复原出原始信号?最初的结果来自于 Donoho 在 2006 年的一篇论文,在 Google 学术上,这篇论文被引次数在笔者写下这一小节时已经达到了惊人的 33115 次,很可能是本世纪引用次数最高的一篇数学论文. 我们下面来介绍一下这篇论文的主要结果,首先需要给出几个定义:

\begin{definition}
    一个向量 $x$ 的 $l_p$-范数定义为:
    \[ \lVert x \rVert_p = \left( \sum_{i}|x_i|^p \right)^{\frac{1}{p}} \]
    其中 $x_i$ 为其分量.
\end{definition}

\begin{definition}
    称一个向量是 $l_p$-稀疏的,如果它满足:
    \[ \lVert x \rVert_p < R \]
    其中 $R$ 为给定正常数,$0 < p \leqslant 1$.
\end{definition}

\section{未来:从线性代数出发能望到多远}

\begin{quote}

    \kaishu
    对于模或者其它具备某种线性操作的数学结构, 例如稍后要介绍的 Abel 范畴中的对象, 复形及其上同调的研究是通称为同调代数的学科的经典内容; 这是本真意义的``线性代数''的一个真子集, 也是本书核心.

    \begin{flushright}
        \kaishu
        ——李文威《代数学方法》卷二:线性代数
    \end{flushright}

\end{quote}

到底什么是线性代数?想必大部分读者都看过那张线性代数九宫格. 当然,手抓饼也可以是线性代数. 但是在此,为了使得我们的讨论不致过于冗长,准李文威老师的思路,我们以线性性作为基准,以此衡量何谓线性代数的发展. 一切事关线性性的探讨都可以被称为线性代数,而我们在这里仅取一瓢之水,希望读者窥一斑可见全豹.

\subsection{模作为线性空间的推广}

当然,我们说线性性,很显然的问题就是,什么样的东西是一个足以保线性性的东西?一个域上的线性空间,那当然可以. 它自身有 Abel 群结构,对于域的加法和数乘作用相容——等等,为什么一定要是一个域呢?域中的除法,以及域的交换性,这些东西看起来对线性性并没有多少意义. 如果你注意到了这一点,那么恭喜你,你已经能够写出模的定义了:

\begin{definition}
    置 $\langle R : +, \cdot \rangle$ 为一环,$\langle M : + \rangle$ 为一 Abel 群. 称 $M$ 为 $R$ 上的左模,如果我们能给它赋予一个映射 $R \times M \to M$ ,以左乘记,满足以下条件:

    \begin{itemize}
        \item $r(m_1 + m_2) = rm_1 + rm_2,\enspace \forall m_1, m_2 \in M,\enspace r \in R$

        \item $(r_1 + r_2)m = r_1m + r_2m,\enspace \forall m \in M,\enspace r_1, r_2 \in R$

        \item $(r_1r_2)m = r_1(r_2m),\enspace \forall m \in M,\enspace r_1, r_2 \in R$

        \item $1_Rm = m,\enspace \forall m \in M$
    \end{itemize}

    对称地,我们也可以定义右模. 事实上,右模无外乎反环 $R^{\mathrm{op}}$ 上的左模.
\end{definition}

对于模理论,深究起来亦破费一番功夫. 这里,我们只提示一个在笔者看来非常有用的性质:

\begin{theorem}[Freyd-Mitchel, 1964]
    任意小的 Abel 范畴都可以满、忠实且正合地嵌入到某个环 $R$ 的左模范畴中去.
\end{theorem}

深入介绍这个定理会让这一节变得相当复杂,或许需要单开一本书来讲. 但是,单从字面意义上讲,读者不难看出,这个意思就是,对 Abel 范畴的研究可以在某个环 $R$ 的左模范畴中完成,而事实上另有一个嵌入定理将另一个更广泛的结构嵌入到 Abel 范畴中:

\begin{theorem}[Quillen-Gabriel, 1972]
    任意一个小的 Quillen 正合范畴都可以被保正合地嵌入到 Abel 范畴中.
\end{theorem}

也就是说,$R$ 的左模范畴可以成为研究一类更广泛的结构的样板,这是其在范畴论中不可或缺的意义之一. 至于模自身的理论,也不可谓不庞大,而且在代数几何等领域的现代发展中发挥着重要的作用. 不过,为了让读者少听点天书,我们还是进入下一个主题吧. 如果读者有兴趣,自可参阅各种现代代数学发展相关的资料.

\subsection{张量积:一个过渡性章节}

张量积的构造其实无外乎线性映射的推广:

\begin{definition}
    置 $V, W, Z$ 为线性空间,称 $V \times W \to Z$ 是双线性的(bilinear),如果它关于 $V$ 和关于 $W$ 都是线性的.
\end{definition}

\begin{definition} \label{thm:16:tensorprod}
    $V$ 和 $W$ 的张量积 $V \otimes W$ 无外乎一种具备泛性质的双线性映射 $\varphi$,即满足对于任意双线性映射 $h$,存在唯一线性映射 $\widetilde h$ 使得下图交换:

    \begin{center}
        \begin{tikzcd}
            V \times W \ar[r, "\varphi"] \ar[rd, "h"] & V \otimes W \ar[d, dashed, "\widetilde h"] \\
            & Z
        \end{tikzcd}
    \end{center}

    也就是说,$\widetilde h \circ \varphi = h$.
\end{definition}

当然,为了确立这个定义的正确性,我们要保证这个空间 $V \otimes W$ 的唯一性. 事实上,它就相当于积空间商掉一些部分:

\begin{align*}
    R = \spa ( & (v_1 + v_2, w) - (v_1, w) - (v_2, w), \\
               & (v, w_1 + w_2) - (v, w_1) - (v, w_2), \\
               & (sv, w) - s(v, w),                    \\
               & (v, sw) - s(v, w))
\end{align*}

于是:
\[ V \otimes W \cong (V \times W) / R \]
不难验证这样给出的 $V \otimes W$ 满足\autoref{thm:16:tensorprod}.

\subsection{从代数到单子:往程序设计范式前进}

\subsection{线性化:群表示的艺术}

我们已经提及,线性性是一个非常好的性质. 对于一般的群,这种性质都是不可能存在的. 但是,如何给一个非线性的结构赋予线性结构呢?我们先看一个事实.

\begin{lemma}
    $\mathcal{L}(V)$ 关于复合操作构成一个群.
\end{lemma}

这个结果的验证应该颇为简单. 随后,我们自然想到,如果研究一个群到这个群的群同态,即那些保群乘法结构的映射,那么我们不久相当于把对群的研究线性化了吗?这样,我们就有了以下定义:

\begin{definition}
    从群 $G$ 到线性空间 $V$ 上的表示(representation)意指一个从 $G$ 到 $\mathcal{L}(V)$ 的群同态.
\end{definition}

通常地,我们会研究实表示和复表示,也就是把 $V$ 取为 $\R^n(\R)$ 和 $\C^n(\C)$. 我们将表示空间的维数称为表示的维数. 对表示论的研究很大程度上是对特征标的研究:

\begin{definition}
    考虑有限维表示 $\rho: G \to \mathcal{L}(V)$,$V$ 是域 $\mathbf{F}$ 上的向量空间,表示 $\rho$ 的特征标被定义为:
    \begin{align*}
        \chi_\rho \colon & G \to F                 \\
                         & g  \mapsto  \tr \rho(g)
    \end{align*}
\end{definition}

当然,对表示论的介绍也足以撑起一本巨著,在此,我们仅提示它在对群论研究中的一些意义,罗列一些用这种方式能够轻松证明,而用通常方法无法证明的结果:
\begin{itemize}
    \item Burnside 定理:所有 $p^aq^b$ 阶的群都是可解群. 这个结果就笔者所知尚未有不使用表示论做出的证明,事实上,表示论给出的证明相当简短.

    \item Feit-Thompson 定理:所有奇数阶群都是可解群. 这个结果的证明非常厚,笔者没看过,但是被其厚度震撼过.

    \item 有限单群分类定理,或称宏伟定理(the Grand theorem). 这是上世纪到本世纪群论最重要的结果之一,其主要部分证明就是应用了表示论引入的特征标理论.

    \item Ore 猜想:有限非 Abel 单群中的任意元素都是换位子. 这个猜想提出于 1951 年,在 2010 年,Liebeck 等人应用李型单群的 Deligne-Lusztig 不可约特征标理论将其完全攻克.
\end{itemize}

此外,表示论在许多物理学领域中发挥着关键性的作用. 一个典型的例子就是现代粒子物理学,其主要工具之一就是李群的表示论. 在量子物理中,哈密顿量的对称群的表示也能够引出多重态(multiplet)的概念,其中不可约的表示部分表征了系统的可能能级. 如果对于这方面的研究感兴趣,笔者(作为一个不怎么懂物理的人)推荐参考 GTM267,B. C. Hall 所著的《写给数学家的量子理论》(\textit{Quantum Theory for Mathematicians})第 17 章的内容.

这里我们另外指明很漂亮的一点,群的表示可以对应地用来建构多面体. 给定三维空间中的一个基向量 $v$,对其应用某有限群中各个元素的矩阵表示,我们可以很自然地得出一个多面体——通过这种方式,我们可以自然得到五种正多面体,它们是多面体点群的三维表示生成的.

如果读者对多面体足够熟悉,那么很自然的一个发现是,我们也可以取三个向量和原点出发构筑一个四面体,然后对这个四面体进行对称操作,即应用矩阵表示得到最终的结果. 事实上,在正多面体的情形下,我们得到的就是一个面的顶点、棱心、面心和原点构成的四面体. 因此,如果将其进行推广,取任意一个它的变形,都可以得到一个具备对应对称性的多面体. 这种方式往往也是计算机中表示一个具备某种对称性的多面体的方法,它的好处是可以利用群表示的方式节省所需的存储空间.

\subsection{拓扑向量空间:从布尔巴基学派的遗产走出}

\subsection{仿射簇,以及代数几何的问题}

\section*{附:本讲义未竟专题概览}

\section*{参考资料}

\begin{enumerate}
    \item \href{https://zh.wikipedia.org/wiki/%E4%BB%A3%E6%95%B0}{维基百科:代数}

    \item \href{https://zhuanlan.zhihu.com/p/574858845}{知乎:代数发展史}

    \item
\end{enumerate}
\vspace{2ex}
\centerline{\heiti \Large 内容总结}

\vspace{2ex}
\centerline{\heiti \Large 习题}

\vspace{2ex}
{\kaishu 如果我们想要预见数学的将来,适当的途径是研究这门科学的历史和现状.}
\begin{flushright}
    \kaishu
    ——庞加莱
\end{flushright}

\centerline{\heiti A组}
\begin{enumerate}
    \item
\end{enumerate}

\centerline{\heiti B组}
\begin{enumerate}
    \item
\end{enumerate}

\centerline{\heiti C组}
\begin{enumerate}
    \item
\end{enumerate}
