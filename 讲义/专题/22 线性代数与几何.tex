\chapter{线性代数与几何}

解析几何很大程度上是线性代数发展的初衷,在研究点线面以及几何体时,将集体的几何问题抽象化为代数问题使其方便解决与计算,即是解析几何的主要思想. 本节我们将会从线性代数的角度探究解析几何的一些基本概念与方法. 此在线性代数课程的考察中也会有少部分的解析几何内容,但内容较浅,主要考察点、直线、平面等之间的关系.

\section{欧氏空间上的运算}

我们也已经基本掌握了模、内积、夹角等在内积空间中的基本概念,在此我们引入一些在先前的学习中接触较少的概念.
\begin{definition}{点积}{} \index{dianji@点积 (dot product)}
    \term{点积}是在三维欧氏空间中对两个向量的运算,用$\vec{a}\cdot\vec{b}$表示. 两向量点积得到的数值等于两向量模长的乘积与两向量夹角的余弦的乘积.
\end{definition}
特别的,三维欧氏空间中的向量点积$(a_1,a_2,a_3)\cdot(b_1,b_2,b_3)$可以表示为\[a_1b_1+a_2b_2+a_3b_3\]
由点积的计算,我们可以很方便地得到两向量夹角的余弦,即\[\cos\theta=\frac{\vec{a}\cdot\vec{b}}{|\vec{a}||\vec{b}|}\]
\begin{definition}{叉乘}{} \index{chacheng@叉乘 (cross product)}
    \term{叉乘}是在三维欧氏空间中对两个向量的运算,用$\vec{a}\times\vec{b}$表示. 两向量叉乘得到的向量垂直于两向量,方向遵循右手定则,其模长为两向量的模的乘积与两向量夹角的正弦的乘积.
\end{definition}
由定义可知,叉乘仅在三维欧氏空间中有定义,且叉乘的结果是一个向量,而不是一个数. 关于叉乘向量的计算有另一种更常用的用行列式表示的计算方法,即
\[(a_1,a_2,a_3)\times(b_1,b_2,b_3)=\begin{vmatrix}
        \vec{i} & \vec{j} & \vec{k} \\
        a_1     & a_2     & a_3     \\
        b_1     & b_2     & b_3
    \end{vmatrix}\]
其中$\vec{i},\vec{j},\vec{k}$为三维欧氏空间的自然基.

在解析几何中,叉乘的一个重要应用是求解与两向量垂直的向量.
\begin{definition}{混合积}{} \index{hunheji@混合积 (mixed product)} \index{biaoliangsancongji@标量三重积 (scalar triple product)}
    \term{混合积}(或称\term{标量三重积},不同于\term{矢量三重积})是三维欧氏空间中对三个向量的运算,用$[\vec{a},\vec{b},\vec{c}]$表示,等价于$(\vec{a}\times\vec{b})\cdot\vec{c}$.
\end{definition}
混合积的几何意义是以$\vec{a},\vec{b},\vec{c}$为邻边的平行六面体的体积,可以用行列式表示为
\[[(a_1,a_2,a_3),(b_1,b_2,b_3),(c_1,c_2,c_3)]=\begin{vmatrix}
        a_1 & a_2 & a_3 \\
        b_1 & b_2 & b_3 \\
        c_1 & c_2 & c_3
    \end{vmatrix}\]
同时读者也不难验证 $ (\vec{a}\times\vec{b})\cdot\vec{c} = \vec{a}\cdot(\vec{b}\times\vec{c}) $. 其应用之一是可以用来判断三个向量是否共面.

\section{点、直线、平面的表示}

一个点在欧氏空间中可以用一个向量来表示. 在三维欧氏空间中,我们可以用三个实数来表示一个点的坐标.

\subsection{平面的方程}

平面是欧氏空间中的一个基本几何对象,我们有多种代数方法来表示平面.

平面的一般方程是平面的一种最基本的表示方法,即$Ax+By+Cz+D=0$. 平面的一般方程十分简洁,但是我们很难由此方程得到平面的几何性质,因此我们还需要考虑其他的表示方法. 例如,一个平面由平面上一点与平面上两个不共线的向量来表示. 假设已知平面上一点$P(x_0,y_0,z_0)$和平面上两个不共线的向量$\vec{u}=(a,b,c)$和$\vec{v}=(d,e,f)$,则平面上的任意一点$Q(x,y,z)$都满足$\overrightarrow{PQ}$与$\vec{u}$和$\vec{v}$线性相关,即
\[\overrightarrow{PQ}=k_1\vec{u}+k_2\vec{v}\]
化为坐标形式即为
\[\begin{cases}
        x=x_0+k_1a+k_2d \\
        y=y_0+k_1b+k_2e \\
        z=z_0+k_1c+k_2f
    \end{cases}\]
这就是平面的参数方程,其中$k_1,k_2$是参数.

此外,平面还可以由平面上一点和平面的法向量来表示. 假设已知平面上一点$P(x_0,y_0,z_0)$和平面的法向量$\vec{n}=(A,B,C)$,则平面上的任意一点$Q(x,y,z)$都满足向量$\overrightarrow{PQ}$与$\vec{n}$垂直,即点积为0. 由此可得其方程为\[A(x-x_0)+B(y-y_0)+C(z-z_0)=0\]这种表示方法称为\term{点法式}.

我们发现这跟平面的一般方程十分相似,实际上,我们可以直接通过平面的一般方程得到平面的法向量.

在得到一张由其他方式表示的平面时,我们往往也会将其转化为一般式或点法式,以便于我们计算其与其他几何对象的关系. 例如,得到一个由平面上一点与平面上两不共线的向量表示的平面,则可以通过求两向量的叉积得到平面的法向量,从而得到平面的点法式.

\begin{example}{}{}
    若已知一个平面上有三点$A(1,2,0),\enspace B(0,1,-1),\enspace C(1,1,1)$,求该平面的一般方程.
\end{example}

\subsection{直线的方程}

直线在欧氏空间中也是一个基本对象,同样有多种代数方法可以表示直线.

首先直线可以用某两张平面的交表示. 假设有两相交平面的方程,联立可得直线方程
\[\begin{cases}
        A_1x+B_1y+C_1z+D_1=0 \\
        A_2x+B_2y+C_2z+D_2=0
    \end{cases}\]
即为直线的一般方程. 这种联立方程的表示方法最为基本,但是不够简洁,大多情况下也不够直观. 所以更多情况下我们希望在表示中可以直观体现直线的一些特征. 因此,可以用直线上的一个点和直线的方向(即方向向量)来确定一条直线.

假设已知直线上的一点$A_0(x_0,y_0,z_0)$和直线的方向向量$\vec{l}=(a,b,c)$,则直线上的任意一点$A(x,y,z)$都满足$\overrightarrow{AA_0}$与$\vec{l}$平行,用具体的方程则表示为
\[\frac{x-x_0}{a}=\frac{y-y_0}{b}=\frac{z-z_0}{c}\]
其中$a,b,c$不为零. 这种表示方法称为\term{点向式}.

如果我们对上述式子进行替换,令\[t=\frac{x-x_0}{a}=\frac{y-y_0}{b}=\frac{z-z_0}{c}\]
则可得
\[\begin{cases}
        x=x_0+at \\
        y=y_0+bt \\
        z=z_0+ct
    \end{cases}\]
这样就得到了直线的参数方程,其中$t$为参数.

当然还有以两点确定一条直线的表示方法,我们可以轻松地算出直线的方向向量,然后用点向式或参数方程来表示. 最后可以得出方程
\[\frac{x-x_1}{x_2-x_1}=\frac{y-y_1}{y_2-y_1}=\frac{z-z_1}{z_2-z_1}\]

那么如何实现从一般方程到点向式或参数方程的转换呢?最简单的方法是求解线性方程组再用两点表示或者参数表示,但是这样的方法比较麻烦,事实上我们可以利用法向量进行转换. 假设两平面的一般方程为$A_1x+B_1y+C_1z+D_1=0$与$A_2x+B_2y+C_2z+D_2=0$,则可以得到两平面的法向量分别为$\vec{n}_1=(A_1,B_1,C_1),\enspace\vec{n}_2=(A_2,B_2,C_2)$,因为该直线在两张平面内,所以直线与两个法向量都垂直,所以$\vec{n}_1\times\vec{n}_2$即为直线的方向向量. 再求出一般方程的一个解(即直线上一点)即可得到直线的点向式与参数方程.

\section{平面与直线间的位置关系}

对于三维欧氏空间中的几何对象,我们主要需要研究平行、相交与重合等关系. 我们可以通过平面与直线的方程来判断.

\subsection{线与线的位置关系}

线与线之间的位置关系判断主要依靠它们的方向向量. 如果两条直线的方向向量平行,则两条直线平行或重合,此时再判断两直线是否存在公共点,若联立方程有解,说明两直线重合,否则两条直线平行. 如果两条直线的方向向量不平行,则还需要判断两条直线是否共面,若共面则说明两条直线相交,否则两条直线异面. 此时以两直线方程联立方程组,若有解则说明存在交点,否则说明两条直线异面.

\begin{example}{}{}
    已知直线$L_1=\begin{cases}
            x+y+z-1=0 \\
            x-2y+2=0
        \end{cases},\enspace L_2=\begin{cases}
            x=2t  \\
            y=t+a \\
            z=bt+1
        \end{cases}$,试确定$a,b$的值使得$L_1,L_2$是:
    \begin{enumerate}
        \item 平行直线;

        \item 异面直线.
    \end{enumerate}
\end{example}

\subsection{线与面的位置关系}

线与面的位置关系首先需要判断线的方向向量与平面的法向量的关系. 如果方向向量与法向量平行,则说明线与面垂直. 如果两者垂直,则说明该直线与平面平行或者在平面内,只需再判断直线上的点是否在平面内即可.

此外还有一些对于平面不同表示形式的方法. 例如,假设已知直线的方向向量与平面上两个不平行的向量,则可以对这三个向量做混合积,如果混合积为零,则说明三个向量共面,即直线与平面平行或者在平面内.

\subsection{面与面的位置关系}

面与面的位置关系主要依靠两个平面的法向量来判断. 如果两个平面的法向量平行,则说明两个平面平行或重合,再判断两平面是否存在公共点. 若两法向量垂直,则两平面也垂直.

\subsection{平面与点的位置关系}
一个空间中的平面 $\Gamma$ 可以以一个方程表示:
\[
A x + B y + C z + D = 0
\]
它的一个法向量是 $\vec{n} = (A, B, C)$,因此我们说一个向量 $\vec{a}$ 在平面 $\Gamma$ 中当且仅当
\[
\langle \vec{n}, \vec{a} \rangle = -D
\]
一个简单的观察是,$D$ 表征了平面和原点之间的距离. 实际上,平面和原点之间的距离就是
\[
d_{O\text{-}\Gamma} = \frac{\abs{D}}{\sqrt{A^2 + B^2 + C^2}}
\]
这并不难估计,只需注意到
\[
\langle \vec{n}, \frac{-D}{A^2 + B^2 + C^2}\vec{n} \rangle = -D
\]
即可. 另一种计算方式也可由截距入手,读者可自行计算直角四面体的体积,这留作一个习题. 有了这个式子之后,我们就能考虑任意一点到平面的距离. 这种考虑的第一个步骤是考虑怎么将某个点``移作''原点,实际上,设这个点的向量表示为 $\vec{b}$,我们就有一个新的以内积的形式作的表示:
\[
\left\langle \vec{n}, \vec{a} + \vec{b} \right\rangle = -D
\]
利用线性性将其写开,并移项,得到新的平面的常数项:
\[
D' = \langle \vec{n}, \vec{b} \rangle + D
\]
将此式代入到已有的方程中,就能得到:
\[
d_{\vec{b}\text{-}\Gamma} = \frac{\abs{\langle \vec{n}, \vec{b} \rangle + D}}{\sqrt{A^2 + B^2 + C^2}}
\]
或者,如果你宁愿为了得到一个更为优雅的方程而多加一些标记,把 $\vec{b}$ 记作 $(x_0, y_0, z_0)$,则可以得到结果:
\[
d_{\vec{b}\text{-}\Gamma} = \frac{\abs{A x_0 + B y_0 + C z_0 + D}}{\sqrt{A^2 + B^2 + C^2}}
\]
另一种同样利用内积的推导方式是利用 Cauchy-Schwartz 不等式,同样见诸习题.

另一个式子是球方程. 就球而言,我们需要考虑的事情其实很少. 一个圆心为 $\vec{a}$,半径为 $r$ 的球 $C$ 中的点 $\vec{x}$ 满足方程:
\[
\langle \vec{x} - \vec{a}, \vec{x} - \vec{a} \rangle = r^2
\]
如果你乐意的话,也可以将其展开:
\[
\langle \vec{x}, \vec{x} \rangle - 2 \langle \vec{x}, \vec{a} \rangle + \langle \vec{a}, \vec{a} \rangle - r^2 = 0
\]
它是一个二次的形式. 对它而言,我们更关注的是它的另外一个性质:其切点的向量表示就是切面的一个法向量,在后面的讨论中,这将是至关重要的一点. 因为相切关系的平移不变性,我们给出以下结果:

\begin{theorem}{}{}
球 $\langle \vec{x}, \vec{x} \rangle = r^2$ 与平面 $\langle \vec{n}, \vec{x} \rangle = -D$ 在点 $\vec{t}$ 处相交,且总共有且仅有一个交点(即相切)当且仅当:
\begin{enumerate}
    \item $\langle \vec{t}, \vec{t} \rangle = r^2, \quad \langle \vec{n}, \vec{t} \rangle = -D$,即 $\vec{t}$ 既在圆上,也在平面上;
    \item 存在 $\lambda \in \R$ 使得 $\vec{t} = \lambda \vec{n}$ 成立,即 $\vec{t}$ 和 $\vec{n}$ 共线.
\end{enumerate}
\end{theorem}

这个命题的证明与命题本身一样显而易见,只要利用二次方程的求根公式即可,与圆和直线相切的推导别无二致,请读者自行完成证明.

\section{射影空间与仿射变换}

在不考虑内积的情况下,在向量空间中的几何性质当中,实际上只有共线性和共线点之间的比例是特殊的. 在此,我们定义

\begin{definition}{}{}
    称一个保持共线性和定比分点性质的映射为仿射变换(affine transformation).
\end{definition}

一个重要的结论如下:
\begin{theorem}{}{}
    $\R^n \to \R^m$ 的仿射变换全体就是所有形如
    \[
    F(u) = A u + v
    \]
    的映射全体
\end{theorem}

这个结论告诉我们,仿射变换就是线性变换的一种推广. 它实际上就是线性变换和平移操作的复合. 这个结论的证明并不复杂:

\begin{proof}
    零点的像平移、基的像线性变换(利用共线性和定比分点性质推导出线性变换的定义).
\end{proof}

% 例子:平移、旋转、拉伸、斜截

但是,这样的记号多少有点不够精致. 我们知道的线性变换有了非常丰富的性质,所以我们也希望能够把仿射变换做成某种线性映射. 考虑一个更加形式性的解法:
\[
A u + v = \begin{pmatrix}
    A & v \\
    0 & 1
\end{pmatrix} \begin{pmatrix}
    u \\ 1
\end{pmatrix}
\]
作为分块矩阵,它们自然地能够满足仿射变换的需求,但是它将一个 $\R^n \to \R^m$ 的仿射变换表示为了一个 $\R^{n + 1} \to \R^{m + 1}$ 的线性变换. 但是,这样的表示依然是不够充分的. 这是因为我们``要求''向量为一个末尾为 $1$ 的向量,而矩阵的最后一行也被确定为一个末尾为 $1$ 的行向量. 基于此限制的线性变换是``过度受限''的. 因此,我们希望更多地以几何的方式解读这些限制,而不是以代数的方式将其强加到空间之上.

回顾对平面方程的讨论. 我们业已提及,一个平面的表达式就是
\[
\langle \vec{n}, \vec{a} \rangle = -D
\]
注意到,这里的法向量 $\vec{n}$ 是没有正规化的. 也就是说,实际上平面方程由两种东西决定:一是 $\vec{n}$ 的坐标,一是 $-D$ 的取值,而且实际上平面方程就相当于
\[
\langle (\vec{n}, D), (\vec{a}, 1) \rangle = 0
\]
而且左边的向量似乎可以在相差一个倍数的意义下表示同一个平面. 因此,我们定义:
\begin{definition}{}{}
    称空间 $\R^{n + 1}$ 关于如下等价关系构成的等价类为 $n$ 维实射影空间:
    \[
    x \sim y \iff \exists k \in \R, x = k y
    \]
    将其构成的集合全体记作 $\mathbf{PR}^n$.
\end{definition}

现在,我们完成了从代数到几何最关键的一步:通过一个等价关系,原来的线性空间变成了一个看上去相当奇怪的结构. 但是,它把捉了几何上重要的``比值''关系. 接下来的部分当中,我们就需要讨论实射影空间所具备的结构特征.

注意到,在射影空间之间,存在一个事实上很平凡的嵌入关系:
\begin{lemma}{}{}
    $\mathbf{PR}^{n - 1}$ 能够被自然地嵌入到 $\mathbf{PR}^n$ 当中,此嵌入映射如下:
    \[
    [x] \mapsto [(x, 0)]
    \]
\end{lemma}

证明只需要验证这个映射的良定性即可. 不难发现,这个映射实际上是一个单射. 而且在去掉所有形如 $[(x, 0)]$ 的等价类之后,剩下的所有等价类就是所有形如 $[(x, 1)]$ 的等价类. 也就是说:
\begin{theorem}{}{}
    实射影空间 $\mathbf{PR}^n$ 无非是 $\R^{n - 1}$ 和 $\mathbf{PR^{n - 1}}$ 的无交并.
\end{theorem}

通过一个更简单的想法可以更好地理解这样的结构:射影直线 $\mathbf{PR}^1$ 就是一个 $\R$ 并上一个``无穷远点'',射影平面 $\mathbf{PR}^2$ 就是一个 $\R^2$ 并上一个无穷远处的射影直线. 推而广之,一个射影空间 $\mathbf{PR}^n$ 就是一个 $R^{n - 1}$ 并上一个无穷远处的子射影空间 $\mathbf{PR}^{n - 1}$.

那么,对于我们的目的而言,这有什么用呢?实际上,可以表明:

\begin{theorem}{}{}
    一个 $\R^n \to \R^m$ 的仿射变换可以被写成一个 $\mathbf{PR}^n \to \mathbf{PR}^m$ 的线性映射的商,且子空间 $\mathbf{PR}^{m - 1}$ 的原像包含于子空间 $\mathbf{PR}^n$ 当中.
\end{theorem}

也就是说,我们现在的操作如下图所示:
\[
\begin{tikzcd}
    \R^n \dar \rar & \R^{n + 1} \dar \rar & \mathbf{PR}^n \dar \rar & \R^n \dar \\
    \R^m \rar & \R^{m + 1} \rar & \mathbf{PR}^m \rar & \R^m
\end{tikzcd}
\]

\section{曲面上的标架}

考虑曲面 $S$ 的参数化,将其写成下面的形式:
\[
\begin{cases}
x = x(u, v) \\
y = y(u, v) \\
z = z(u, v)
\end{cases}
\]

考虑导数
\[
\vec{t}_u = (x_{u}, y_{u}, z_{u}), \quad \vec{t}_v = (x_{v}, y_{v}, z_{v})
\]

对微积分有所了解的读者应该不难看出,这两个向量是曲面在某个点的切向量. 考虑一个``好的''参数化,即在任意点这样的两个切向量线性无关. 这样的参数网对于满足一些可微性条件的曲面来说总是存在的,在此不做证明. 我们定义:
\begin{definition}
    称切向量 $\vec{t}_u |_{\vec{t}_0}, \vec{t}_v |_{\vec{t}_0}$ 所张成的向量空间为曲面 $S$ 在 $\vec{t_0}$ 点的切平面,记作 $T_{\vec{t}_0} S$.
\end{definition}
切平面的单位法向量称为这个点处曲面的法向量,记作 $\vec{n}$. 不难发现,在曲面上的任意点上,$(\vec{t}_u, \vec{t}_v, \vec{n})$ 张成三维空间,这被称为曲面上在这个点处的自然标架.

在自然标架下,我们不难考虑曲面上某一片小区域的面积,通过一点点微积分的操作,我们可以得出:
\[
\mathrm{d} s^2 = |\vec{t}_u|^2 \mathrm{d} u^2 + 2 \vec{t}_u \cdot \vec{t}_v \mathrm{d} u \mathrm{d} v + |\vec{t}_v|^2 \mathrm{d} v^2
\]
我们定义:
\begin{align*}
    E &= |\vec{t}_u|^2 \\
    F &= \vec{t}_u \cdot \vec{t}_v \\
    G &= |\vec{t}_v|^2
\end{align*}
不难发现,$E > 0, G > 0, E G - F^2 > 0$,我们可以将其写成矩阵的形式:
\[
\begin{pmatrix}
    E & F \\
    F & G
\end{pmatrix}
\]
这种形式通常被称作曲面的第一基本形式. 不难发现,它给定了曲面上的长度和面积.

考虑坐标变换
\[
\begin{cases}
    \tilde{u} = \tilde{u}(u, v) \\
    \tilde{v} = \tilde{v}(u, v)
\end{cases}
\]
我们探讨在此变换下(如果这个变换之后的结果也是``好的''),第一基本形式会发生什么样的变化. 记变换之后的第一基本形式为:
\[
\begin{pmatrix}
    \tilde{E} & \tilde{F} \\
    \tilde{F} & \tilde{G}
\end{pmatrix}
\]
可以证明:
\begin{theorem}{}{}
    变换前后的第一标准形式与原来的第一标准形式有如下关系:
    \[
    \begin{pmatrix}
        \tilde{E} & \tilde{F} \\
        \tilde{F} & \tilde{G}
    \end{pmatrix}
     =
    J^\mathrm{T} \begin{pmatrix}
        E & F \\
        F & G
    \end{pmatrix} J
    \]
    其中 $J$ 为 Jacobi 矩阵,形式如下:
    \[
    \begin{pmatrix}
        \dfrac{\partial u}{\partial \tilde{u}} & \dfrac{\partial u}{\partial \tilde{v}} \\ \\
        \dfrac{\partial v}{\partial \tilde{u}} &
        \dfrac{\partial v}{\partial \tilde{v}}
    \end{pmatrix}
    \]
\end{theorem}

证明基于 Jacobi 矩阵的含义来完成.

\begin{proof}
    将 $\mathrm{d} s^2$ 按照一阶的形式展开.
\end{proof}

我们会意识到,这个``乘上一个可逆矩阵及其转置''的形式是非常重要的. 在后文中,我们会称之为矩阵的相合标准型. 由此,读者亦可窥见此定义的重要性.

\section{二次曲面及其分类}

下面我们考虑一族简单的曲面,令
\[
S: a x^2 + b y^2 + c z^2 + 2 d y z + 2 e x z + 2 f x y + g x + h y + i z = 1
\]
其中关于 $x, y, z$ 都是二次的,称作二次曲面. 下面我们想问的是,如何对二次曲面完成分类. 注意到,上面的方程也可以被写成:
\[
S: \begin{pmatrix}
    x & y & z
\end{pmatrix}
\begin{pmatrix}
    a & f & e \\
    f & b & d \\
    e & d & c
\end{pmatrix}
\begin{pmatrix}
    x \\ y \\ z
\end{pmatrix} + \begin{pmatrix}
    g & h & i
\end{pmatrix} \begin{pmatrix}
    x \\ y \\ z
\end{pmatrix} = 1
\]

一般地,前面的形式 $v^\mathrm{T} A v$ 被称为``二次型'',在后面我们会花费一整讲的时间来讨论有关于它的理论. 但是在此之前,我们先思考如何对二次曲面进行分类.

上面的方程可以简记作:
\[
v^\mathrm{T} A v + b^\mathrm{T} v = 1
\]

首先,我们考虑对 $A$ 进行谱分解. 由于 $A$ 是实对称矩阵,所以它可以被写成 $Q^\mathrm{T} \Lambda Q$ 的形式,其中 $\Lambda$ 为对角矩阵,$Q$ 为正交矩阵. 我们知道,正交矩阵对应的是一个等距同构. 所以,我们将其先写成:
\[
v^\mathrm{T} Q^\mathrm{T} \Lambda Q v + b^\mathrm{T} Q^\mathrm{T} Q v = 1
\]
于是,将 $Q v$ 置为 $\tilde v$,则能得到方程:
\[
\tilde v^\mathrm{T} \Lambda \tilde v^\mathrm{T} + \tilde b^\mathrm{T} \tilde v' = 1
\]
其中 $\tilde b = Q^\mathrm{T} b$.

此时我们对应到二次曲面,实际上是通过一次等距同构消掉了 $d, e, f$ 三个系数. 我们得到的方程就变成了
\[
S: ax^2 + by^2 + cz^2 + g x + h y + i z = 1
\]
通过我们熟知的配方法,我们可以把左式变成
\[
S: a(x - \frac{g}{2a})^2 - \frac{g^2}{4a} + b(y - \frac{h}{2b})^2 - \frac{h^2}{4b} + c(z - \frac{i}{2c})^2 - \frac{i^2}{4c} = 1
\]
也就是说,通过一个平移变换,就能将原式变成
\[
a x^2 + b y^2 + c z^2 = d
\]
的形式.

总结上述讨论,我们可以得到以下定理:

\begin{theorem}{}{}
    任意二次曲面经过等距变换都可以得到满足如下曲面方程的曲面:
    \[
    a x^2 + b y^2 + c z^2 = d
    \]
\end{theorem}

注意到一点,这里的 $a, b, c$ 是原先的矩阵
\[
\begin{pmatrix}
    a & f & e \\
    f & b & d \\
    e & d & c
\end{pmatrix}
\]
的特征值. 也就是说,对于一个二次曲面的方程,这个矩阵的特征值几乎决定了它的几何结构,这在后面的分类讨论给出每种曲面的形式之后将能够更清晰地展现. 一个有意思的观察是,对于二次曲线,我们也能得出类似的结果,这留作一个习题,请读者自行完成分类. 而在后面的学习中,我们会发现,对于更高维度空间中的二次曲面,这依然是一个重要的分类依据.

\begin{summary}

\end{summary}

\begin{exercise}
    \exquote[笛卡尔]{我决心放弃那个仅仅是抽象的几何. 这就是说,不再去考虑那些仅仅是用来练思想的问题. 我这样做,是为了研究另一种几何,即目的在于解释自然现象的几何.}

    \begin{exgroup}
        \item 用等体积法计算原点到平面的距离. 考虑 $A, B, C$ 均非零的平面 $A x + B y + C z + D = 0$,计算:
        \begin{enumerate}
            \item 平面在 $x, y, z$ 轴上的截距以及三个截点与平面构成的直角四面体的体积.
            \item 三个截点围成的三角形的面积.(提示:运用 Helen 公式或者秦九韶公式).
            \item 给出平面到原点的距离公式,并与文中的公式比对.
        \end{enumerate}

        \item 用 Cauchy-Schwartz 不等式证明点到平面的距离公式.
    \end{exgroup}

    \begin{exgroup}
        \item 考虑平面曲线 \[A x^2 + B y^2 + C x y + D x + E y + F = 0\] 的分类.(提示:类比上面对空间二次曲面的分类,找出变换方式和判据)
    \end{exgroup}

    \begin{exgroup}
        \item
    \end{exgroup}
\end{exercise}
