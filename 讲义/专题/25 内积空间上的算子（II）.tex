\chapter{内积空间上的算子(II)}

前面我们对正规算子和自伴算子做了相当充分的工作,从这章开始我们准备对一般的算子做些工作. 

\section{正交矩阵和酉矩阵}

本节我们将唤醒一些沉睡的记忆,如果你已经忘了过渡矩阵或矩阵的相似,可以移步到前面的章节再回顾一下. 如果你还在这的话,那么坐稳,我们马上开始. 

\vspace{2ex}

\subsection{定义}

为了更好地引进正交矩阵和酉矩阵,我们有必要把共轭转置说的更清楚些. 
共轭转置有着以下的运算性质,虽然都是看起来很显然的事情,此处还是稍稍赘述一下:

设有矩阵 $ A $, $ B $ 和数 $ \lambda \in \mathbf{C}$,则

\begin{enumerate}    
    \item $ (\overline{A + B})^{\mathrm{T}} = \overline{A}^{\mathrm{T}} + \overline{B}^{\mathrm{T}} $;
    
    \item $ \overline{(AB)}^{\mathrm{T}} = \overline{B}^{\mathrm{T}} \overline{A}^{\mathrm{T}} $;
    
    \item $ (\overline{\lambda A})^{\mathrm{T}} = \overline{\lambda} \enspace \overline{A}^{\mathrm{T}} $;
    
    \item $ \overline{\overline{A}^{\mathrm{T}}}^{\mathrm{T}} = A $. 
\end{enumerate}

共轭转置说清楚后,便可以由此定义正交矩阵和酉矩阵. 

\begin{definition} \keyterm{酉矩阵} \keyterm{正交矩阵} 
    在复数域(实数域)上,矩阵 $ A $ 满足 $ \overline{A}^{\mathrm{T}} A = E $( $ {A}^{\mathrm{T}} A = E $ ),
    则矩阵 $ A $ 被称为酉矩阵(正交矩阵) 
\end{definition}

而如何刻画正交矩阵和酉矩阵的性质呢?下面的一个定理揭示了其与标准正交基的关系,可以从中窥得一些性质.  

\begin{theorem}
    设 $ (e_1, e_2, \ldots , e_n) $ 是复(实)内积空间 $ V $ 上的标准正交基,$ (f_1, f_2, \ldots , f_n) $ 是 $ V $ 上的一组基,
    从 $ (e_1, e_2, \ldots , e_n) $ 到 $ (f_1, f_2, \ldots , f_n) $ 的过渡矩阵为 $ A $. 则 $ (f_1, f_2, \ldots , f_n) $ 是标准正交基的
    充要条件是 $ A $ 为酉矩阵(正交矩阵). 
\end{theorem}

以下仅针对复内积空间的情况进行证明. 

\begin{proof}
    由过渡矩阵的定义,$ (f_1, f_2, \ldots , f_n) $ = $ (e_1, e_2, \ldots , e_n)A $,$ A = (a_{ij})_{n \times n} $. 

    由矩阵乘法的运算,可以得到
    \[ f_i = \sum_{j = 1}^{n} a_{ji}e_j , \enspace f_k = \sum_{j = 1}^{n} a_{jk}e_j. \]

    对两者做内积,有
    \[
    \langle f_i, f_k \rangle = \left\langle \sum_{j = 1}^{n} a_{ji}e_j, \sum_{j = 1}^{n} a_{jk}e_j \right\rangle
    = \sum_{j = 1}^{n} a_{ji}\overline{a_{jk}} 
    \]

    注意到 $ a_{ji}, j = 1, \ldots , n $ 是 $ A^{\mathrm{T}} $ 的第 $ i $ 行的元素,
    $ \overline{a_{jk}}, j = 1, \ldots , n $ 是 $ \overline{A} $ 的第 $ k $ 列的元素.
   
    定义 $ B = A^{\mathrm{T}}\overline{A} = (b_{ik})_{n \times n} $,则 $ \langle f_i, f_k \rangle = b_{ik} $. 

    必要性:如果 $ (f_1, f_2, \ldots , f_n) $ 是一组标准正交基,则
    \[
        b_{ik} = \langle f_i, f_k \rangle = 
        \begin{cases}
            1, & i = k \\
            0, & i \neq k 
        \end{cases}    
    \]

    由此可知 $ B = E $, $ \overline{B} = \overline{A}^{\mathrm{T}} A = \overline{E} = E $,即 $ A $ 是酉矩阵. 
    
    充分性:将必要性证明推理过程倒写即可. 

\end{proof}

如果这条定理中的 $ (e_1, e_2, \ldots , e_n) $ 取为该空间的自然基,就会有 $ (f_1, f_2, \ldots , f_n) = A $,我们便可以不太严谨地
得到如下的这个结论

\begin{theorem}
    矩阵 $ A $ 是酉矩阵(正交矩阵)等价于其列向量构成标准正交基. 
\end{theorem}

证明是平凡的,就交给你自己验证了. 

那提到了过渡矩阵,我们也就不得不提与之息息相关的一个等价关系——相似了. 
相信你已经回忆起来,相似实际上是同一个算子在不同基下的矩阵表示之间的关系,
实现这个变化正是依赖于两组基之间的过渡矩阵. 而我们的主线正是依靠基变换实现的,
只不过我们现在用的都是标准正交基,在基变换上也要有所升级. 所以,
让我们先定义两个特殊一点的相似关系:

\begin{definition}
    \begin{enumerate}

        \item \keyterm{酉相似}:复内积空间上,若 $ B = P^{-1}AP = \overline{P}^{\mathrm{T}}AP $,
        则称矩阵 $ A $ 与矩阵 $ B $ 酉相似. 

        \item \keyterm{正交相似}:实内积空间上,若 $ B = P^{-1}AP = {P}^{\mathrm{T}}AP $,
        则称矩阵 $ A $ 与矩阵 $ B $ 正交相似. 
    \end{enumerate}
\end{definition}

它俩的特殊之处你可能一下子没看出来,不过没关系,我们可以先回到它们对应的算子上去看看. 

\subsection{等距同构}

由之前一章,我们知道,算子与其伴随在同一组标准正交基下的矩阵表示是互为共轭对称的,所以设对应的算子是 $ S $,
则其应该满足 $ S^*S = I $. 那么这个性质能将我们导向何处呢?

考虑两侧同时作用向量 $ u $,再与 向量 $ v $ 做内积,那么我们得到了如下的式子:
\[ \langle S^*Su, v \rangle = \langle u, v \rangle. \]
再结合伴随的定义,稍微做个变换,就有了下面这个美妙的结果:
\[ \langle Su, Sv \rangle = \langle u, v \rangle. \]
也就是说,这个算子 $ S $ 同时作用在两个向量上的话不改变它们的内积. 
更进一步的话,如果取 $ v = u $,我们就能得到最终的结果:
\[ \lVert Su \rVert = \lVert u \rVert \]
算子 $ S $ 保持范数. 

\begin{definition}
    \keyterm{等距同构} 算子 $ S \in \mathcal{L}(V) $ 称为等距同构,如果 $ \forall v \in V $,
    都有 $ \lVert Su \rVert = \lVert u \rVert $.  
\end{definition}

注意我们这里虽然使用了共轭对称,但是从伴随的角度上来说复内积空间和实内积空间其实是一样的,
也就是说等距同构的概念在这两类空间上是一致的,只不过刻画上会有所差距,之后会有所介绍. 此外,
也常称实内积空间上的等距同构为正交算子,复内积空间上的等距同构称为酉算子. 

让我们看道简单的例题加深一下对等距同构的印象. 
\begin{example}
    设 $ \lambda_1, \ldots , \lambda_n $ 都是模为 1 的标量,
    $ e_1, \ldots , e_n $ 是 $ V $ 的标准正交基,$ S \in \mathcal{L}(V) $ 
    满足 $ Se_j = \lambda_je_j $,证明 $ S $ 是等距同构. 
\end{example}

然后来介绍一下等距同构的等价条件,虽然很多,但大部分都是我们刚才推理过程中已经得到的结果. 

\begin{theorem}
    设 $ S \in \mathcal{L}(V) $ ,则以下条件等价:
    \begin{enumerate}
        \item $ S $ 是等距同构;
        
        \item 对所有 $ u, v \in V $ 均有 $ \langle Su, Sv \rangle = \langle u, v \rangle $;
        
        \item 对 $ V $ 中的任意标准正交向量组 $ e_1, \ldots , e_m $ 均有 $ Se_1, \ldots , Se_m $ 是标准正交的;
        
        \item $ V $ 有规范正交基 $ e_1, \ldots ,e_n $ 使得 $ Se_1, \ldots , Se_n $ 是标准正交基;
        
        \item $ SS^* = S^*S = I $; 
        
        \item $ S^* $ 是等距同构;
        
        \item $ S $ 是可逆的且 $ S^{-1} = S^* $. 
    \end{enumerate}
\end{theorem}

证明大部分都在上面的过程中证明过了,剩下的请大家自行验证. 

我们关注第 3 个条件,即等距同构将标准正交组映射成标准正交组. 那么其对应的矩阵,
即酉矩阵或正交矩阵也保有这样的性质,在基变换时它们能将标准正交基仍然变换成标准正交基,
这正是我们所希望看到的,也就是酉相似(正交相似)的特殊之处.

这样我们就可以用酉相似和正交相似对谱定理进行矩阵语言的刻画. 

\begin{theorem}
    \begin{enumerate}
        \item 复方阵 $ A $ 酉相似于对角矩阵的充要条件是 $ A $ 是正规矩阵;
        
        \item 实方阵 $ A $ 正交相似于对角矩阵的充要条件是 $ A $ 是实对称矩阵. 
    \end{enumerate}
\end{theorem}

继续关注第 5 个条件,很容易的就会发现,等距同构其实也是正规算子,那么其相较于
正规算子又有什么加强呢?在复内积空间下,结合 \ref{例 25.1},就会有下面这个优美的等价条件. 

\begin{theorem}
    设 $ V $ 是复内积空间,$ S \in \mathcal{L}(V) $. 则以下条件等价:
    \begin{enumerate}
        \item $ S $ 是等距同构
        
        \item $ V $ 有一个由 $ S $ 的特征向量构成的标准正交基,相应的特征值的绝对值均为 1. 
    \end{enumerate}
\end{theorem}

几何意义其实是相当直观的,所有绝对值为 1 的复数作用在模为 1 的向量上都不会改变其模,而只是
进行对称与旋转(事实上旋转也可以表为对称,之后会有所介绍). 这在几何上的运用是相当广泛的.  

\section{正定矩阵}

在数学分析课程中,我们常常会讨论多元函数的极值,极值的刻画依赖的正是矩阵是否有定,正定还是负定还是半正定还是半负定. 

那么,何为正定矩阵?这一定义在不同的数域下仍然有差异. 

\begin{definition} 
    \keyterm{正定矩阵} 
    \begin{enumerate}
        \item 实数域:对 $ n $ 阶实对称矩阵 $ M $,若对于所有非零实系数向量 $ z $,均有
        $ z^{T}Mz > 0 $,则称矩阵 $ M $ 为正定矩阵;
        
        \item 复数域:对 $ n $ 阶 Hermite 矩阵 $ M $,若对于所有非零向量 $ z $,
        $ z^{H}Mz > 0 $,则称矩阵 $ M $ 为正定矩阵. 
    \end{enumerate}
\end{definition}

复数域上的定义合理性是由“对于 Hermite 矩阵 $ M $,$ z^{H}Mz $ 必为实数”保证的. 

由实数域上的正定矩阵的定义,我们可以发现其与二次型的相关性,我们也可以利用从二次型
中所学来判定实正定矩阵. 

\begin{theorem}
    设 $ A $ 为 $ n $ 阶实对称矩阵,则以下条件等价:
    \begin{enumerate}
        \item $ A $ 是正定矩阵;
        
        \item $ A $ 的正惯性指数为 $ n $,即 $ A \simeq E $;
        
        \item 存在可逆矩阵 $ P $,使得 $ A = P^{T}P $;
        
        \item $ A $ 的 $ n $ 个特征值 $ \lambda_1, \lambda_2, \ldots, \lambda_n $ 都大于零. 
    \end{enumerate}
\end{theorem}

对应的复数域版本相信大家也很容易就能够联想得到,只需要将转置变为共轭转置即可. 

以下是一些更深层次地判别矩阵是否正定的条件,同时它们也是正定矩阵的一些重要的性质. 

\begin{theorem}
    $ A $ 是 $ n $ 阶的 Hermite 矩阵. 以下条件等价. 
    \begin{enumerate}
        \item $ A $ 是正定矩阵;
        
        \item 双线性函数 $ \langle x, y \rangle = x^{H}Ay $ 
        定义了一个 $ \mathcal{C}^{n} $ 上的一个内积.
        事实上, $ \mathcal{C}^{n} $ 所有内积都可视作由某个正定矩阵以此方式得到;

        \item $ A $ 是向量 $ x_1, \ldots , x_n \in \mathcal{C}^{k} $ 构成的
        Gram 矩阵. 即 $ A = B^{H}B $,其中 $ B $ 未必是方阵,但一定是单的,并且
        这种分解方式不唯一.  
        
        \item \keyterm{Cholesky decomposition} 存在唯一的下三角矩阵 $ L $,
        其主对角元均为正数,使得 $ A = LL^{H} $

        \item \keyterm{Sylvester's criterion} $ A $ 的所有顺序主子式均为正.
        (但对于半正定矩阵而言,顺序主子式非负不能推出矩阵半正定)
    \end{enumerate}
\end{theorem} 

像 2 就道出了正定矩阵和内积之间的关系,3 和 4 给出了正定矩阵的一些分解方式. 

现在让我们跳开去,先去看看算子上的事情,不过出于更实用的原因,我们研究
半正定矩阵对应的算子. 

\begin{definition}
    \keyterm{正算子}
    设算子 $ T \in \mathcal{L}(V) $,如果 $ T $ 是自伴的且 $ \forall v \in V $
    均有 $ \langle Tv, v \rangle \geqslant 0 $. 
\end{definition}

但对于正算子的定义似乎和对于半正定矩阵的定义方向完全不同,前者依托内积,
后者则是依托二次型. 不过,若是你还记得我们曾经提到过内积本身是一种正定
齐次双线性函数,以及二次型可以通过双线性函数引入,就可以捕获这其中的相关之处. 
我们接下来进行推导. 

\begin{proof}
    设 $ V $ 的一组标准正交基为 $ e = (e_1, e_2, \ldots, e_n) $,任取向量 $ \alpha \in V $,
    设其在 $ e $ 下的坐标为 $ x = (x_1, x_2, \ldots, x_n)^{T} $. 设正算子 $ T \in \mathcal{L}(V) $
    在 $ e $ 下的矩阵为 $ A = (a_{ij})_{n \times n}$,则 
    \begin{align*}
        \langle Tv, v \rangle 
        & = \langle Tex, ex \rangle = \langle eAx, ex \rangle \\ 
        & = \langle 
            (e_1, e_2, \ldots ,e_n)
            \begin{pmatrix}
                & a_{11} & a_{12} & \ldots & a_{1n} \\
                & a_{21} & a_{22} & \ldots & a_{2n} \\
                & \vdots & \vdots &        & \vdots \\
                & a_{n1} & a_{n2} & \ldots & a_{nn} 
            \end{pmatrix}
            \begin{pmatrix}
                x_1 \\
                x_2 \\
                \vdots \\
                x_n
            \end{pmatrix} ,
            (e_1, e_2, \ldots ,e_n)
            \begin{pmatrix}
                x_1 \\
                x_2 \\
                \vdots \\
                x_n
            \end{pmatrix}
        \rangle \\
        & = \langle \sum_{i = 1}^{n}e_{i}\sum_{j = 1}^{n}a_{ij}x_{j}, \sum_{i = 1}^{n}x_{i}e_{i} \rangle
        = \overline{x_1}\sum_{j = 1}^{n}a_{1j}x_{j} + \overline{x_2}\sum_{j = 1}^{n}a_{2j}x_{j} + \cdots + \overline{x_n}\sum_{j = 1}^{n}a_{nj}x_{j} \\
        & = (\overline{x_1}, \overline{x_2}, \ldots, \overline{x_n})
        \begin{pmatrix}
            \sum_{j = 1}^{n}a_{1j}x_{j} \\
            \sum_{j = 1}^{n}a_{2j}x_{j} \\
            \vdots \\
            \sum_{j = 1}^{n}a_{nj}x_{j}
        \end{pmatrix}
        = x^{H}Ax. 
    \end{align*}

    此对应复内积空间的情形,实内积空间的情形也就显然了. 
\end{proof}

我们定义出的正算子虽然名为正算子,但我们类比的时候它其实是类似于非负数,
非负数很重要的一种运算就是开方运算. 类似的,我们也可以定义算子的平方根.

\begin{definition}
    \keyterm{平方根} 算子 $ R $ 被称为算子 $ T $ 的平方根,如果 $ R^{2} = T $. 
\end{definition}

以下是正算子的刻画. 

\begin{theorem}
    设 $ T \in \mathcal{L}(V) $. 则以下条件等价. 
    \begin{enumerate}
        \item $ T $ 是正的;
        
        \item $ T $ 是自伴的且 $ T $ 的所有特征值非负;
        
        \item $ T $ 有正的平方根;
        
        \item $ T $ 有自伴的平方根;
        
        \item 存在算子 $ R \in \mathcal{L}(V) $ 使得 $ T = R^{*}R $. 
    \end{enumerate}
\end{theorem}

从这里我们进一步加深类比. 

\vspace{2ex}
\centerline{\heiti \Large 内容总结}

\vspace{2ex}

\centerline{\heiti \Large 习题}
\vspace{2ex}
{\kaishu }
\begin{flushright}
    \kaishu

\end{flushright}
\centerline{\heiti A组}
\begin{enumerate}
    \item 证明:上三角的酉矩阵必为对角矩阵. 
    
    \item 证明:任一 $ n $ 级可逆复矩阵 $ A $ 一定可以被唯一分解成 $ A = PB $,
    其中 $ P $ 是 $ n $ 级酉矩阵,$ B $ 是主对角元均为正实数的 $ n $ 级上三角矩阵. 
\end{enumerate}
\centerline{\heiti B组}
\begin{enumerate}
    \item
\end{enumerate}
\centerline{\heiti C组}
\begin{enumerate}
    \item
\end{enumerate}
