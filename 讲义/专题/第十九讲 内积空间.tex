\chapter{内积空间}

\section{内积和范数}
\subsection{内积和范数的定义及性质}
前面研究的所有空间都是线性空间,只注重于线性结构,忽视了向量的度量性质,如向量的长度、夹角等。
但度量性质恰是在分析、几何问题中不可缺少的。故从此章起,我们引入度量的概念,将线性空间推广为内积空间。

\vspace{2ex} 

内积的引入始于我们曾在高中研究过的 $\mathbf{R}^{2}$ 与 $\mathbf{R}^{3}$ 上的向量点积,范数则是始于向量的长度概念。
内积即是点积性质的推广,本质上就是一个函数,它把 $ V $ 中元素的每个有序对 $(u, v)$ 都映射成一个数 
$ \langle u, v \rangle \in \mathbf{F}$,并且具有以下性质:

1. 正定性:$\forall v \in V, \enspace \langle v, v \rangle \geqslant 0, \enspace \langle v, v \rangle = 0 \Leftrightarrow v = 0$; 

2. 第一个位置的加性:$\forall u, v, w \in V, \enspace \langle u + v, w \rangle = \langle u, w \rangle + \langle v, w \rangle$;

3. 第一个位置的齐性:$\forall \lambda \in \mathbf{F}, \enspace \forall u, v \in V, 
\enspace \langle \lambda u, v \rangle = \lambda \langle u, v \rangle$;

4. 共轭对称性:$\forall u, v \in V, \enspace \langle u, v \rangle = \overline{\langle v, u \rangle}$.

每个实数都等于它的复共轭,所以在处理实向量空间时,共轭对称性实际上转变为对称性,即:
$\forall u, v \in V, \enspace \langle u, v \rangle = \langle v, u \rangle$

而由以上定义,我们可以快速得到以下性质:

1. 对于每个取定的 $u \in V$,将 $ v $ 变为 $\langle v, u \rangle$ 的函数是 $ V $ 到 $\mathbf{F}$ 的线性映射.

2. $\forall u \in V, \enspace \langle 0, u \rangle = \langle u, 0 \rangle = 0$.

3. $\forall u, v, w \in V, \enspace \langle u, v + w \rangle = \langle u, v \rangle + \langle u, w \rangle$.

4. $\forall \lambda \in \mathbf{F}, \enspace \forall u, v \in V, \enspace \langle u, \lambda v \rangle = \overline{\lambda} \langle u, v \rangle$.

其实从以上的定义与性质可以发现,实内积空间上的内积与我们之后要提到的双线性函数有着密不可分的联系——
实线性空间上的正定对称双线性函数实际上就是该空间上的一个内积,在此先按下不表。

\vspace{2ex} 

内积定义完成后,便可由该内积确定一个相应的范数:对于 $v \in V$,$v$ 的范数(记作 $ \lVert v \rVert $)
定义为 $ \lVert v \rVert = \sqrt{\langle v, v \rangle}$. \enspace 并且具有以下性质:

1. $\forall v \in V, \enspace \left\lVert v \right\rVert = 0 \Leftrightarrow v = 0$.

2. $\forall v \in V, \enspace \forall \lambda \in \mathbf{F}, 
\enspace \left\lVert \lambda v \right\rVert  = \left\lvert \lambda \right\rvert \lVert v \rVert$.

上述性质留给读者自证,从中我们也能发现一个普遍原理:处理范数的平方通常比直接处理范数更容易。

\vspace{2ex} 

以下给出几个内积和范数的示例:
\begin{example}
    \textup{(1)}$\mathbf{F}^{n}$ 上的欧几里得内积定义为:
    \[\left\langle (w_1, \ldots, w_n), (z_1, \ldots, z_n)\right\rangle = w_1\overline{z_1} + \cdots + w_n\overline{z_n} = \vec{W}\overline{\vec{Z}}^{T}\]
    对应范数:
    \[\left\lVert (z_1, \ldots, z_n) \right\rVert  = \sqrt{\lvert z^2_1 \rvert + \cdots + \lvert z^2_n \rvert}\]

    \textup{(2)}定义在 $ \left[-1, 1\right] $ 上的连续实值函数构成的向量空间可定义内积如下:
    \[\left\langle f, g\right\rangle = \int_{-1}^1f(x)g(x)\mathrm{d}x\]
    对应范数:
    \[\left\lVert f \right\rVert = \sqrt{\int_{-1}^1(f(x))^2\mathrm{d}x}\]
\end{example}

\subsection{正交的定义,基于正交的性质}

以下给出一个关键定义:
\begin{definition}
    两个向量 $u, v \in V$ 称为正交的,如果 $\langle u, v\rangle = 0$.
\end{definition}
该定义中向量的次序是无关紧要的,因为 $\langle u, v\rangle = 0 \Leftrightarrow \langle v, u\rangle = 0$. 

那么正交的定义关键在何处呢?以下给出 $R^{n}$ 空间上夹角的定义以供理解
(证明良定义需要用到 Cauchy-Schwarz 不等式):
\begin{definition}
    设 $u, v \in R^{n}$,则 $u, v$ 的夹角 $ \theta $ 为 
    $ \theta = \arccos \frac{\langle u, v\rangle}{\lVert u \rVert \lVert v \rVert}$.
\end{definition}

那么我们可以发现,当两向量正交时,它们的夹角就是 $\frac{\pi}{2}$,
也就是我们在几何中常说的垂直,它能将我们导向一些重要的定理。

\vspace{2ex} 

让我们从一些简单的结果开始研究正交性,比如正交性与 0 的关系:

1. 0 正交与 $V$ 中的任意向量.

2. 0 是 $V$ 中唯一一个与自身正交的向量.

\clearpage % 此处进行了强制分页.

然后是熟悉的勾股定理在内积空间上的推广:
\begin{theorem}
    设 $u, v$ 是 $V$ 中的正交向量,则 $\lVert u + v \rVert^2 = \lVert u \rVert^2 + \lVert v \rVert^2 $ 
\end{theorem}

注意勾股定理的逆定理仅在实内积空间上成立。

\vspace{2ex} 

借助于正交的性质,我们能够简化很多与内积相关的计算,
进而会很自然的思考这样一个问题:一个向量能否分解两个互相正交的向量?

从而便引进了正交分解:
\begin{theorem}
    设 $u, v \in V$ 且 $v \neq 0$. 令 $ c = \frac{\langle u, v\rangle}{\lVert v \rVert^2}, 
    \enspace w = u - \frac{\langle u, v\rangle}{\lVert v \rVert^2}v$. 则 $\langle w, v\rangle = 0$ 
    且 $u = cv + w$
\end{theorem}

而通过正交分解,我们可以证明一个数学中最重要的不等式(之一):Cauchy-Schwarz 不等式。
\begin{theorem}
    设 $u, v \in V$. 则 $\left\lvert \left\langle u, v\right\rangle \right\rvert \leqslant \lVert u \rVert\lVert v \rVert$. 
    等号成立当且仅当 $u, v$ 之一是另一个的标量倍.
\end{theorem}
也可以通过引入参数,利用二次三项式的判别式证明。

\vspace{2ex}

借助 Cauchy-Schwarz 不等式,我们可以得到三角不等式:
\begin{theorem}
    设 $u, v \in V$. 则 $\lVert u, v \rVert \leqslant \lVert u \rVert + \lVert v \rVert$.
    等号成立当且仅当 $u, v$ 之一是另一个的非负标量倍.
\end{theorem}
其几何解释就是俗称的三角形两边之和小于第三边。

\vspace{2ex}

另一个与几何相关的结论就是平行四边形恒等式:
\begin{theorem}
    设 $u, v \in V$. 则 $ \lVert u + v \rVert^{2} + \lVert u - v \rVert^{2} = 2(\lVert u \rVert^{2} + \lVert v \rVert^{2})$
\end{theorem}


\section{标准正交基}

\section{正交补}

\vspace{2ex} 
\centerline{\heiti \Large 内容总结}

\vspace{2ex} 

\centerline{\heiti \Large 习题}
\vspace{2ex} 
{\kaishu }
\begin{flushright}
    \kaishu

\end{flushright}
\centerline{\heiti A组}
\begin{enumerate}
	\item 
\end{enumerate}
\centerline{\heiti B组}
\begin{enumerate}
	\item 
\end{enumerate}
\centerline{\heiti C组}
\begin{enumerate}
	\item 
\end{enumerate}