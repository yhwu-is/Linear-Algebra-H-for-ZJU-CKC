\chapter{行列式计算进阶}

\section{化三角形法}

\begin{example}
    计算行列式$D_{n+1}=\begin{vmatrix}
            1      & a_{1}       & a_{2}       & \cdots & a_n       \\
            1      & a_{1}+b_{1} & a_{2}       & \cdots & a_n       \\
            1      & a_{1}       & a_{2}+b_{2} & \cdots & a_n       \\
            \vdots & \vdots      & \vdots      & \ddots & \vdots    \\
            1      & a_{1}       & a_{2}       & \cdots & a_n+b_{n}
        \end{vmatrix}$.
\end{example}

解析:观察行列式的特点,主对角线下方的元素与第 1 行元素对应相同,故用第 1 行的 $-1$ 倍加到下面各行便可使主对角线下方的元素全部变为0,即化为上三角形.

\begin{solution}
    将该行列式第 1 行的 $-1$ 倍分别加到第 $2,3,\ldots,n+1$ 行上去,可得
    \[ D_{n+1}=\begin{vmatrix}
            1 & a_{1} & a_{2} & \ldots & a_n   \\
              & b_{1} &       &        &       \\
              &       & b_{2} &        &       \\
              &       &       & \ddots &       \\
              &       &       &        & b_{n}
        \end{vmatrix}=\prod_{i=1}^n b_i \]
\end{solution}

\section{连加法}

\begin{example}
    计算行列式$D_n=\begin{vmatrix}
            x_1-m  & x_2    & \cdots & x_n    \\
            x_1    & x_2-m  & \cdots & x_n    \\
            \vdots & \vdots & \ddots & \vdots \\
            x_1    & x_2    & \cdots & x_n-m
        \end{vmatrix}$.
\end{example}

\begin{solution}
    \begin{align*}
        D_n & =\begin{vmatrix}
                   \displaystyle\sum_{i=1}^{n} x_i-m & x_2    & \cdots & x_n    \\
                   \displaystyle\sum_{i=1}^{n} x_i-m & x_2-m  & \cdots & x_n    \\
                   \vdots                            & \vdots & \ddots & \vdots \\
                   \displaystyle\sum_{i=1}^{n} x_i-m & x_2    & \cdots & x_n-m
               \end{vmatrix} \\
            & =\left(\sum_{i=1}^{n} x_i-m\right)
        \begin{vmatrix}
            1      & x_2    & \cdots & x_n    \\
            1      & x_2-m  & \cdots & x_n    \\
            \vdots & \vdots & \ddots & \vdots \\
            1      & x_2    & \cdots & x_n-m
        \end{vmatrix}                                   \\
            & =\left(\sum_{i=1}^{n} x_i-m\right)
        \begin{vmatrix}
            1 & x_2 & \cdots & x_n \\
              & -m  &        &     \\
              &     & \ddots &     \\
              &     &        & -m
        \end{vmatrix}                                                \\
            & =(-m)^{n-1}\left(\sum_{i=1}^{n} x_1-m\right)
    \end{align*}
\end{solution}

\section{滚动消去法}

当行列式每两行的值比较接近时,可采用让邻行中的某一行减或者加上另一行的若干倍, 这种方法叫滚动消去法.

\begin{example}
    计算行列式$D_n=\begin{vmatrix}
            1      & 2      & 3      & \cdots & n-1    & n      \\
            2      & 1      & 2      & \cdots & n-2    & n-1    \\
            3      & 2      & 1      & \cdots & n-3    & n-2    \\
            \vdots & \vdots & \vdots & \ddots & \vdots & \vdots \\
            n-1    & n-2    & n-3    & \cdots & 1      & 2      \\
            n      & n-1    & n-2    & \cdots & 2      & 1
        \end{vmatrix},\enspace n \geqslant 2$.
\end{example}

\begin{solution}
    从最后一行开始每行减去上一行
    \begin{align*}
        D_n & =\begin{vmatrix}
                   1      & 2      & 3      & \cdots & n-1    & n      \\
                   1      & -1     & -1     & \cdots & -1     & -1     \\
                   1      & 1      & -1     & \cdots & -1     & -1     \\
                   \vdots & \vdots & \vdots & \ddots & \vdots & \vdots \\
                   1      & 1      & 1      & \cdots & -1     & -1     \\
                   1      & 1      & 1      & \cdots & 1      & -1
               \end{vmatrix}=\begin{vmatrix}
                                 1      & 2      & 3      & \cdots & n-1    & n      \\
                                 2      & 0      & 0      & \cdots & 0      & -2     \\
                                 2      & 2      & 0      & \cdots & 0      & -2     \\
                                 \vdots & \vdots & \vdots & \ddots & \vdots & \vdots \\
                                 2      & 2      & 2      & \cdots & 0      & -2     \\
                                 1      & 1      & 1      & \cdots & 1      & -1
                             \end{vmatrix} \\
            & =\begin{vmatrix}
                   1      & 2      & 3      & \cdots & n-1    & n+1    \\
                   2      & 0      & 0      & \cdots & 0      & 0      \\
                   2      & 2      & 0      & \cdots & 0      & 0      \\
                   \vdots & \vdots & \vdots & \ddots & \vdots & \vdots \\
                   2      & 2      & 2      & \cdots & 0      & 0      \\
                   1      & 1      & 1      & \cdots & 1      & 0
               \end{vmatrix}=2^{n-2}
        \begin{vmatrix}
            1      & 2      & 3      & \cdots & n-1    & n+1    \\
            1      & 0      & 0      & \cdots & 0      & 0      \\
            1      & 1      & 0      & \cdots & 0      & 0      \\
            \vdots & \vdots & \vdots & \ddots & \vdots & \vdots \\
            1      & 1      & 1      & \cdots & 1      & 0
        \end{vmatrix}                      \\
            & =(-1)^{n+1}(n+1) 2^{n-2}
    \end{align*}
\end{solution}

\section{降阶法}

将高阶行列式化为低阶行列式再求解.

\begin{example}
    解行列式$D_n=\begin{vmatrix}
            x     & -1    &        &         &         \\
                  & x     & \ddots &         &         \\
                  &       & \ddots & -1      &         \\
                  &       &        & x       & -1      \\
            a_{0} & a_{1} & \cdots & a_{n-2} & a_{n-1}
        \end{vmatrix}$
\end{example}

\begin{solution}
    按最后一行展开,得
    \begin{align*}
        D_n & =\sum_{i=0}^{n-1}(-1)^{n+i+1}a_i\begin{vmatrix} A & O \\ O & B \end{vmatrix}
        =\sum_{i=0}^{n-1}(-1)^{n+i+1}a_i|A||B|                                                        \\
            & =\sum_{i=0}^{n-1}(-1)^{n+i+1}a_i(x^i)\left((-1)^{n-1-i}\right) =\sum_{i=0}^{n-1}a_i x^i
    \end{align*}

    其中$A\in \mathbf{M}_i(\mathbf{R}),\enspace B\in \mathbf{M}_{n-1-i}(\mathbf{R})$,
    \[ A=\begin{pmatrix}
            x & -1 &        &    &    \\
              & x  & \ddots &    &    \\
              &    & \ddots & -1 &    \\
              &    &        & x  & -1 \\
              &    &        &    & x
        \end{pmatrix},\enspace
        B=\begin{pmatrix}
            -1 &    &        &    &    \\
            x  & -1 &        &    &    \\
               & x  & \ddots &    &    \\
               &    & \ddots & -1 &    \\
               &    &        & x  & -1
        \end{pmatrix} \]
\end{solution}

\begin{example}
    解行列式$D_n=\begin{vmatrix}
            \lambda & a      & a      & a      & \cdots & a      \\
            b       & \gamma & \beta  & \beta  & \cdots & \beta  \\
            b       & \beta  & \gamma & \beta  & \cdots & \beta  \\
            \vdots  & \vdots & \vdots & \vdots & \ddots & \vdots \\
            b       & \beta  & \beta  & \beta  & \cdots & \gamma
        \end{vmatrix}$
\end{example}

\begin{solution}
    从第$n$行到第3行,每行都减去上一行;再从第3列到第$n$列,每列都加到第2列,得

    \begin{align*}
        D_n & =
        \begin{vmatrix}
            \lambda & a            & a            & a            & \cdots & a            \\
            b       & \gamma       & \beta        & \beta        & \cdots & \beta        \\
            0       & \beta-\gamma & \gamma-\beta & 0            & \cdots & 0            \\
            0       & 0            & \beta-\gamma & \gamma-\beta & \cdots & 0            \\
            \vdots  & \vdots       & \vdots       & \vdots       & \ddots & \vdots       \\
            0       & 0            & 0            & 0            & \cdots & \gamma-\beta
        \end{vmatrix}             \\
            & =\begin{vmatrix}
                   \lambda & (n-1)a            & a            & a            & \cdots & a            \\
                   b       & \gamma+(n-2)\beta & \beta        & \beta        & \cdots & \beta        \\
                   0       & 0                 & \gamma-\beta & 0            & \cdots & 0            \\
                   0       & 0                 & \beta-\gamma & \gamma-\beta & \cdots & 0            \\
                   \vdots  & \vdots            & \vdots       & \vdots       & \ddots & \vdots       \\
                   0       & 0                 & 0            & 0            & \cdots & \gamma-\beta
               \end{vmatrix} \\
            & =\begin{vmatrix}
                   \lambda & (n-1)a            \\
                   b       & \gamma+(n-2)\beta
               \end{vmatrix} \cdot \begin{vmatrix}
                                       \gamma-\beta & 0            & \cdots & 0            \\
                                       \beta-\gamma & \gamma-\beta & \cdots & 0            \\
                                       \vdots       & \vdots       & \ddots & \vdots       \\
                                       0            & 0            & \cdots & \gamma-\beta
                                   \end{vmatrix}           \\
            & =(\lambda \gamma+\lambda(n-2)\beta-(n-1)ab)(\gamma-\beta)^{n-2}
    \end{align*}
\end{solution}

\section{升阶法}

升阶法就是把 $n$ 阶行列式增加一行一列变成 $n+1$ 阶行列式,再通过性质化简算出结果,这种计算行列式的方法叫做升阶法或加边法. 升阶法的最大特点就是要找每行或每列相同的因子, 那么升阶之后, 就可以利用行列式的性质把绝大多数元素化为0,这样就达到简化计算的效果.

\begin{example}
    解行列式 $D=\begin{vmatrix}
            0      & 1      & 1      & \cdots & 1      & 1      \\
            1      & 0      & 1      & \cdots & 1      & 1      \\
            1      & 1      & 0      & \cdots & 1      & 1      \\
            \vdots & \vdots & \vdots & \ddots & \vdots & \vdots \\
            1      & 1      & 1      & \cdots & 0      & 1      \\
            1      & 1      & 1      & \cdots & 1      & 0
        \end{vmatrix}$.
\end{example}

\begin{solution}
    使行列式 $D$ 变成 $n+1$ 阶行列式, 即
    \[ D=\begin{vmatrix}
            1      & 1      & 1      & \cdots & 1      & 1      \\
            0      & 0      & 1      & \cdots & 1      & 1      \\
            0      & 1      & 0      & \cdots & 1      & 1      \\
            \vdots & \vdots & \vdots & \ddots & \vdots & \vdots \\
            0      & 1      & 1      & \cdots & 0      & 1      \\
            0      & 1      & 1      & \cdots & 1      & 0
        \end{vmatrix} \]

    再将第一行的 $-1$ 倍加到其他各行, 得:
    \[ D =\begin{vmatrix}
            1      & 1  & \cdots & 1  \\
            -1     & -1 &        &    \\
            \vdots &    & \ddots &    \\
            -1     &    &        & -1
        \end{vmatrix} \]

    从第二列开始, 每列乘以 $-1$ 加到第一列, 得:
    \begin{align*}
        D & =\begin{vmatrix}
                 -(n-1) & 1  & \cdots & 1  \\
                        & -1 &        &    \\
                        &    & \ddots &    \\
                        &    &        & -1
             \end{vmatrix} \\
          & =(-1)^{n+1}(n-1)
    \end{align*}
\end{solution}

\section{数归/递推法}

\begin{example} \label{ex:14:递推法}
    计算行列式$D_n=\begin{vmatrix}
            \cos \beta & 1            &        &              &              \\
            1          & 2 \cos \beta & \ddots &              &              \\
                       & 1            & \ddots & 1            &              \\
                       &              & \ddots & 2 \cos \beta & 1            \\
                       &              &        & 1            & 2 \cos \beta
        \end{vmatrix}$.
\end{example}

\begin{solution}
    \begin{align*}
        D_1 & =\cos\beta                               \\
        D_2 & =\begin{vmatrix}
                   \cos\beta & 1          \\
                   1         & 2\cos\beta
               \end{vmatrix}=2\cos^2\beta-1=\cos2\beta
    \end{align*}

    猜想$D_n=\cos n\beta$. 数学归纳证明:

    假设当 $n=k$ 时,结论成立,即 $D_{k}=\cos k \beta$. 现证当 $n=k+1$ 时,结论也成立. $ n=k+1 $ 时,
    \[ D_{k+1} = \begin{vmatrix}
            \cos \beta & 1            &        &              &              \\
            1          & 2 \cos \beta & \ddots &              &              \\
                       & 1            & \ddots & 1            &              \\
                       &              & \ddots & 2 \cos \beta & 1            \\
                       &              &        & 1            & 2 \cos \beta
        \end{vmatrix} \]

    将 $D_{k+1}$ 按最后一行展开, 得
    \begin{align*}
        D_{k+1}={} & (-1)^{k+1+k+1} \cdot 2 \cos \beta
        \begin{vmatrix}
            \cos \beta & 1            &        &              &              \\
            1          & 2 \cos \beta & \ddots &              &              \\
                       & 1            & \ddots & 1            &              \\
                       &              & \ddots & 2 \cos \beta & 1            \\
                       &              &        & 1            & 2 \cos \beta
        \end{vmatrix} \\
                   & +(-1)^{k+1+k}
        \begin{vmatrix}
            \cos \beta & 1            & 0            & \cdots & 0      \\
            1          & 2 \cos \beta & 1            & \cdots & 0      \\
            0          & 1            & 2 \cos \beta & \cdots & 0      \\
            \vdots     & \vdots       & \vdots       & \ddots & \vdots \\
            0          & 0            & 0            & \cdots & 1
        \end{vmatrix}       \\
        ={}        & 2\cos\beta D_k-D_{k-1}
    \end{align*}

    而$D_{k}=\cos k \beta,\enspace
        D_{k-1}=\cos (k-1)\beta = \cos (k \beta-\beta) = \cos k \beta \cos \beta+\sin k \beta \sin \beta$. 所以有
    \begin{align*}
        D_{k+1} & = 2 \cos \beta D_{k}-D_{k-1}                                               \\
                & =2 \cos \beta \cos k \beta-\cos k \beta \cos \beta-\sin k \beta \sin \beta \\
                & =\cos k \beta \cos \beta-\sin k \beta \sin \beta                           \\
                & =\cos (k+1) \beta
    \end{align*}

    则证得$D_n=\cos n\beta,\enspace n\in \mathbf{N}$.
\end{solution}

下面介绍常系数线性递推数列,为了方便,只介绍二阶情况. 如果$D_n$满足关系式
\[ aD_n+bD_{n-1}+cD_{n-2}=0 \]
解特征方程
\[ ar^2+br+c=0 \]
会有三种根的情况.
\begin{enumerate}
    \item $\Delta>0$, 有两个不等的实根$r_1, r_2$,则有
          \[ D_n=C_1r_1^n+C_2r_2^n \]

    \item $\Delta=0$, 有重实根$r$,则有
          \[ D_n=(C_1+nC_2)r^n \]

    \item $\Delta<0$, 有共轭复根$r=\cos\beta\pm \i\sin\beta$,则有
          \[ D_n=C_1\cos n\beta + C_2\sin n\beta \]
\end{enumerate}
以上式子中的$C_1,C_2$均为任意常数,可以令$n=1,2$获得.

所以其实\autoref{ex:14:递推法} 也可以使用递推式求得答案(留作习题证明略). 不过一般遇到的还是特征根为实数的情况比较多,给出一道练习例题:

\begin{example}
    计算行列式$D_n=
        \begin{vmatrix}
            9 & 5 &        &   &   \\
            4 & 9 & \ddots &   &   \\
              & 4 & \ddots & 5 &   \\
              &   & \ddots & 9 & 5 \\
              &   &        & 4 & 9
        \end{vmatrix}$.
\end{example}

\begin{solution}
    按第一列展开, 得
    \[ D_n=9 D_{n-1}-20 D_{n-2} \]
    即 $ D_n-9 D_{n-1}+20 D_{n-2}=0 $.

    作特征方程
    \[ x^{2}-9 x+20=0 \]
    解得 $ x_1=4,\enspace x_2=5 $. 则
    \[ D_n=A \cdot 4^n+B \cdot 5^n \]

    当 $n=1$ 时, $9=4A+5B$;

    当 $n=2$ 时,$61=16A+25B$.

    解得$A=-4,\enspace B=5$,所以
    \[ D_n=5^{n+1}-4^{n+1} \]
\end{solution}

\section{硬拆法}

\begin{example}
    计算行列式$D_n=\begin{vmatrix}
            1-a_{1} & a_{2}   &        &           &         \\
            -1      & 1-a_{2} & \ddots &           &         \\
                    & -1      & \ddots & a_{n-1}   &         \\
                    &         & \ddots & 1-a_{n-1} & a_{n}   \\
                    &         &        & -1        & 1-a_{n}
        \end{vmatrix}$.
\end{example}

\begin{solution}
    把第一列的元素看成两项的和进行拆列, 得
    \begin{align*}
        D_n= & \begin{vmatrix}
                   1-a_{1} & a_{2}   &        &           &         \\
                   -1      & 1-a_{2} & \ddots &           &         \\
                   0 + 0   & -1      & \ddots & a_{n-1}   &         \\
                   0 + 0   &         & \ddots & 1-a_{n-1} & a_{n}   \\
                   0 + 0   &         &        & -1        & 1-a_{n}
               \end{vmatrix}         \\
        =    & \begin{vmatrix}
                   1  & a_{2}   &        &           &         \\
                   -1 & 1-a_{2} & \ddots &           &         \\
                      & -1      & \ddots & a_{n-1}   &         \\
                      &         & \ddots & 1-a_{n-1} & a_{n}   \\
                      &         &        & -1        & 1-a_{n}
               \end{vmatrix} + \begin{vmatrix}
                                   -a_{1} & a_{2}   &        &           &         \\
                                          & 1-a_{2} & \ddots &           &         \\
                                          & -1      & \ddots & a_{n-1}   &         \\
                                          &         & \ddots & 1-a_{n-1} & a_{n}   \\
                                          &         &        & -1        & 1-a_{n}
                               \end{vmatrix}
    \end{align*}

    上面第一个行列式的值为 1(从第 1 行开始,每一行依次加到下一行),所以
    \[ \begin{aligned}
            D_n & =1-a_{1}\begin{vmatrix}
                              1-a_{2} & a_{3}   &        &           &         \\
                              -1      & 1-a_{3} & \ddots &           &         \\
                                      & -1      & \ddots & a_{n-1}   &         \\
                                      &         & \ddots & 1-a_{n-1} & a_{n}   \\
                                      &         &        & -1        & 1-a_{n}
                          \end{vmatrix} \\
                & =1-a_{1} D_{n-1} \cdot
        \end{aligned} \]

    这个式子对任何 $n \geqslant 2$ 都成立, 因此有
    \begin{align*}
        D_n & =1-a_{1} D_{n-1}                                          \\
            & =1-a_{1}(1-a_{2} D_{n-2})                                 \\
            & =\cdots                                                   \\
            & =1-a_{1}+a_{1} a_{2}+\cdots+(-1)^n a_{1} a_{2} \cdots a_n \\
            & =1+\sum_{i=1}^{n}(-1)^i \prod_{j=1}^i a_j
    \end{align*}
\end{solution}

\section{箭形行列式}

\begin{example}
    计算行列式$\begin{vmatrix}
            a_1    & 1   & 1   & \cdots & 1   \\
            1      & a_2                      \\
            1      &     & a_3                \\
            \vdots &     &     & \ddots       \\
            1      &     &     &        & a_n
        \end{vmatrix}$,其中$a_i\neq 0\enspace(i=1,2,\ldots,n)$.
\end{example}

\begin{solution}
    \[ \text{原式}=\begin{vmatrix}
            a_1-\displaystyle\sum_{i=2}^n\frac{1}{a_i} & 1   & 1   & \cdots & 1   \\
            0                                          & a_2                      \\
            0                                          &     & a_3                \\
            \vdots                                     &     &     & \ddots       \\
            0                                          &     &     &        & a_n
        \end{vmatrix} = \left(\sum_{i=2}^n\frac{1}{a_i}\right) \left(\prod_{j=2}^na_j\right) \]
\end{solution}

\section{Vandermonde 行列式}

\begin{example}
    求行列式 $D_n=\begin{vmatrix}
            1         & 1         & \cdots & 1         \\
            x_1       & x_2       & \cdots & x_n       \\
            x_1^{2}   & x_2^{2}   & \cdots & x_n^{2}   \\
            \vdots    & \vdots    & \ddots & \vdots    \\
            x_1^{n-2} & x_2^{n-2} & \cdots & x_n^{n-2} \\
            x_1^{n}   & x_2^{n}   & \cdots & x_n^{n}
        \end{vmatrix}$.
\end{example}

\begin{solution}
    考虑构造一个$n+1$阶的Vandermonde行列式.
    \[ f(x)=\begin{vmatrix}
            1         & 1         & \cdots & 1         & 1       \\
            x_1       & x_2       & \cdots & x_n       & x       \\
            x_1^{2}   & x_2^{2}   & \cdots & x_n^{2}   & x^{2}   \\
            \vdots    & \vdots    & \ddots & \vdots    & \vdots  \\
            x_1^{n-2} & x_2^{n-2} & \cdots & x_n^{n-2} & x^{n-2} \\
            x_1^{n-1} & x_2^{n-1} & \cdots & x_n^{n-1} & x^{n-1} \\
            x_1^{n}   & x_2^{n}   & \cdots & x_n^{n}   & x^{n}
        \end{vmatrix} \]

    将 $f(x)$ 按第 $n+1$ 列展开, 得
    \[ f(x)=A_{1, n+1}+A_{2, n+1} x+\cdots+A_{n, n+1} x^{n-1}+A_{n+1, n+1} x^{n} \]
    其中 $x^{n-1}$ 的系数为
    \[ A_{n, n+1}=(-1)^{n+(n+1)} D_n=-D_n \]

    又根据 Vandermonde 行列式的结果知
    \[ f(x)=(x-x_1)(x-x_2)\cdots(x-x_n) \prod_{1 \leqslant j<i \leqslant n}(x_i-x_j) \]
    由上式可求得 $x^{n-1}$ 的系数为
    \[ -(x_1+x_2+\cdots+x_n) \prod_{1 \leqslant j<i \leqslant n}(x_i-x_j) \]
    故有
    \[ D_n=(x_1+x_2+\cdots+x_n) \prod_{1 \leqslant j<i \leqslant n}(x_i-x_j) \]
\end{solution}

\section{$^*$利用$|E_m-AB|=|E_n-BA|$} \label{sec:14:利用}

\begin{example}
    求行列式 $\begin{vmatrix}
            0      & 2a_1   & 3a_1   & \cdots & na_1     \\
            a_2    & a_2    & 3a_2   & \cdots & na_2     \\
            a_3    & 2a_3   & 2a_3   & \cdots & na_3     \\
            \vdots & \vdots & \vdots & \ddots & \vdots   \\
            a_n    & 2a_n   & 3a_n   & \cdots & (n-1)a_n \\
        \end{vmatrix}$.
\end{example}

\begin{solution}
    \[ \text{原式}=\prod_{i=1}^na_i \begin{vmatrix}
            0      & 2      & 3      & \cdots & n      \\
            1      & 1      & 3      & \cdots & n      \\
            1      & 2      & 2      & \cdots & n      \\
            \vdots & \vdots & \vdots & \ddots & \vdots \\
            1      & 2      & 3      & \cdots & n-1    \\
        \end{vmatrix} \]

    注意到
    \begin{align*}
        \begin{vmatrix}
            0      & 2      & 3      & \cdots & n      \\
            1      & 1      & 3      & \cdots & n      \\
            1      & 2      & 2      & \cdots & n      \\
            \vdots & \vdots & \vdots & \ddots & \vdots \\
            1      & 2      & 3      & \cdots & n-1
        \end{vmatrix}
         & = \begin{vmatrix}(-1)\left(E_n-\begin{pmatrix}
                1      & 2      & 3      & \cdots & n      \\
                1      & 2      & 3      & \cdots & n      \\
                1      & 2      & 3      & \cdots & n      \\
                \vdots & \vdots & \vdots & \ddots & \vdots \\
                1      & 2      & 3      & \cdots & n
            \end{pmatrix}\right)\end{vmatrix} \\
         & =(-1)^n\begin{vmatrix}E_n-
                      \begin{pmatrix}
                1 \\1\\1\\\vdots\\1
            \end{pmatrix}\begin{pmatrix}1 & 2 & 3 & \cdots & n\end{pmatrix}\end{vmatrix}
    \end{align*}

    而 \[ \begin{vmatrix}E_n-\begin{pmatrix}
                1 \\1\\1\\\vdots\\1
            \end{pmatrix}\begin{pmatrix}1 & 2 & 3 & \cdots & n\end{pmatrix}\end{vmatrix}
        =1-\begin{pmatrix}1 & 2 & 3 & \cdots & n\end{pmatrix}
        \begin{pmatrix}1 \\ 1 \\ 1 \\ \vdots \\ 1\end{pmatrix}
        =-\frac{n^2+n-2}{2} \]

    所以原式$\displaystyle =(-1)^{n+1}\frac{n^2+n-2}{2}\prod_{i=1}^na_i$.
\end{solution}

\section{Laplace定理}
我们在\autoref{def:13:递归式定义} 中讲述了行列式一行(一列)展开的方式,这里我们讲解一个更加一般的展开方式:即按照$k$行($k$列)展开.

为了描述这一定理,我们需要首先将单个元素的余子式和代数余子式的概念推广到子式的余子式和代数余子式. 为此,我们先给出一个定义:
\begin{definition}
    $n$阶行列式$|A|$中任意取定$k$行、$k$列$(1\leqslant k<n)$,记为$i_1,\cdots,i_k$行,$j_1,\cdots,j_k$列,位于这些行和列的交叉处的$k^2$个元素所构成的$k$阶子式称为$|A|$的一个$k$阶子式,这一$k$阶子式记为$A\begin{pmatrix}
            i_1 & \cdots & i_k \\
            j_1 & \cdots & j_k
        \end{pmatrix}$.

    划去子式所在的取定的$k$行、$k$列,剩下的元素所构成的$n-k$阶行列式称为这个$k$阶子式的余子式,记为
    \[A\begin{pmatrix}
            i_1' & \cdots & i_{n-k}' \\
            j_1' & \cdots & j_{n-k}'
        \end{pmatrix},\]
    其中$\{i_1',\cdots,i_{n-k}'\}=\{1,\cdots,n\}\setminus\{i_1,\cdots,i_k\}$,$\{j_1',\cdots,j_{n-k}'\}=\{1,\cdots,n\}\setminus\{j_1,\cdots,j_k\}$,且$i_1'<\cdots<i_{n-k}'$,$j_1'<\cdots<j_{n-k}'$. 它前面乘以$(-1)^{(i_1+\cdots+i_k)+(j_1+\cdots+j_k)}$所得的数称为这个$k$阶子式的代数余子式.
\end{definition}

事实上上面关于子式的定义与上一讲给出的完全一致,只是多了一个记号,而子式的余子式实际上也只是单个元素余子式到多行多列的自然扩展. 举个简单的例子,对于$3$阶行列式$|A|=\begin{vmatrix}
        a_{11} & a_{12} & a_{13} \\
        a_{21} & a_{22} & a_{23} \\
        a_{31} & a_{32} & a_{33}
    \end{vmatrix}$,取定第$1,3$行,第$1,2$列,那么这个$2$阶子式为$A\begin{pmatrix}
        1 & 3 \\
        1 & 2
    \end{pmatrix}=\begin{vmatrix}
        a_{11} & a_{12} \\
        a_{31} & a_{32}
    \end{vmatrix}$,它的余子式为$A\begin{pmatrix}
        2 \\ 3
    \end{pmatrix}=\begin{vmatrix}
        a_{23}
    \end{vmatrix}$,代数余子式为$(-1)^{(1+3)+(1+2)}a_{23}=-a_{23}$.

\begin{theorem}\label{thm:14:Laplace定理}
    在$n$阶行列式$|A|$中,取定$k$行:第$i_1,i_2,\ldots,i_k$行($1\leqslant i_1<i_2<\cdots<i_k\leqslant n$,且$1\leqslant k<n$),则这$k$行元素形成的所有$k$阶子式与它们自己的代数余子式的乘积之和等于$|A|$,即
    \begin{equation}\label{eq:14:Laplace定理}
        |A|=\sum_{1\leqslant j_1<j_2<\cdots<j_k\leqslant n}A\begin{pmatrix}
            i_1 & \cdots & i_k \\
            j_1 & \cdots & j_k
        \end{pmatrix}\cdot (-1)^{(i_1+\cdots+i_k)+(j_1+\cdots+j_k)}A\begin{pmatrix}
            i_1' & \cdots & i_{n-k}' \\
            j_1' & \cdots & j_{n-k}'
        \end{pmatrix}.
    \end{equation}

    若取定$k$列:第$j_1,j_2,\ldots,j_k$列($1\leqslant j_1<j_2<\cdots<j_k\leqslant n$,且$1\leqslant k<n$),则这$k$列元素形成的所有$k$阶子式与它们自己的代数余子式的乘积之和等于$|A|$.
\end{theorem}

下面的证明需要利用一些行列式逆序数定义的知识,感兴趣的读者可以参考史海拾遗一讲的描述,当然也可以略过这里的证明.

\begin{proof}
    我们只证明前一半按行展开的情况,后一半按列展开的情况可以类似证明. 事实上,$|A|$是$n!$项的代数和(事实上直接用按一行(一列)展开的定义结合数学归纳法很容易得到),而\autoref{eq:14:Laplace定理} 右边求和符号包含$C_n^k$项,然后$k$阶子式展开有$k!$项,$n-k$阶代数余子式有$(n-k)!$项,所以$C_n^k\cdot k!\cdot (n-k)!=n!$,所以\autoref{eq:14:Laplace定理} 等号左右项数相同.

    事实上,我们不难发现\autoref{eq:14:Laplace定理} 等号右端$n!$项是互不相同的(不是计算结果一定互不相同,是参与计算的行列式元素不同),又\autoref{eq:14:Laplace定理} 等号左右项数相同,故我们只需证明等号右侧每一项都是$|A|$展开中的一项即可.

    在\autoref{eq:14:Laplace定理} 右侧任取一项:
    \[(-1)^{\tau(\mu_1\cdots\mu_k)}a_{i_1\mu_1}\cdots a_{i_k\mu_k}(-1)^{(i_1+\cdots+i_k)+(j_1+\cdots+j_k)}(-1)^{\tau(\nu_1\cdots\nu_{n-k})}a_{i_1'\nu_1}\cdots a_{i_{n-k}'\nu_{n-k}'},\]
    其中$\mu_1,\cdots,\mu_k$为$j_1,\cdots,j_k$的一个排列,$\nu_1,\cdots,\nu_{n-k}$为$j_1',\cdots,j_{n-k}'$的一个排列,$\tau$代表置换的符号(在朝花夕拾一讲中有介绍).

    而在\autoref{eq:14:Laplace定理} 左侧有如下一项:
    \[(-1)^{\tau(i_1\cdots i_ki_1'\cdots i_{n-k}')+\tau(\mu_1\cdots\mu_k\nu_1\cdots\nu_{n-k})}a_{i_1\mu_1}\cdots a_{i_k\mu_k}a_{i_1'\nu_1}\cdots a_{i_{n-k}'\nu_{n-k}'},\]
    并且我们有
    \begin{align*}
         & (-1)^{\tau(i_1\cdots i_ki_1'\cdots i_{n-k}')+\tau(\mu_1\cdots\mu_k\nu_1\cdots\nu_{n-k})}                                                        \\
         & =(-1)^{\sum\limits_{r=1}^ki_r-\frac{k(1+k)}{2}}(-1)^{\tau(\mu_1\cdots\mu_k)+\tau(\nu_1\cdots\nu_{n-k})+\sum\limits_{r=1}^kj_r+\frac{k(1+k)}{2}} \\
         & =(-1)^{\tau(\mu_1\cdots\mu_k)+\tau(\nu_1\cdots\nu_{n-k})}(-1)^{(i_1+\cdots+i_k)+(j_1+\cdots+j_k)}.
    \end{align*}
    因此\autoref{eq:14:Laplace定理} 右侧任意一项都可以在左侧找到对应,证毕.
\end{proof}

我们用一个简单的例子来应用这一定理:
\begin{example}
    设$A$为$n$阶方阵,$B$为$m$阶方阵,$C$为$m\times n$矩阵,证明:
    \[ \begin{vmatrix}
            A & O \\
            C & B
        \end{vmatrix}=|A|\cdot|B|. \]
\end{example}

\begin{proof}
    将$\begin{vmatrix}
            A & O \\
            C & B
        \end{vmatrix}$按前$n$行展开,得到的所有可能$n$阶子式只有$A$不为0(其它子式都有全零列),且其代数余子式为
    \[(-1)^(1+2+\cdots+k)+(1+2+\cdots+k)|B|=|B|,\]
    由Laplace定理,原式$=|A|\cdot|B|$.
\end{proof}

因此这里给出了比教材179页例4更为简洁的证明方法,当前前提在于利用了一个证明起来更为复杂的定理.

\vspace{2ex}
\centerline{\heiti \Large 内容总结}

希望以上的技巧不需要在考试中用到.

\vspace{2ex}
\centerline{\heiti \Large 习题}

\vspace{2ex}
{\kaishu 我总是尽我的精力和才能来摆脱那种繁重而单调的计算. }
\begin{flushright}
    \kaishu
    —— 约翰·纳皮尔
\end{flushright}

\centerline{\heiti A组}
\begin{enumerate}
    \item $ D_n=\begin{vmatrix}
                  1      & 2      & 3      & 4      & \cdots & n-1    & n      \\
                  x      & 1      & 2      & 3      & \cdots & n-2    & n-1    \\
                  x      & x      & 1      & 2      & \cdots & n-3    & n-2    \\
                  x      & x      & x      & 1      & \cdots & n-4    & n-3    \\
                  \vdots & \vdots & \vdots & \vdots & \ddots & \vdots & \vdots \\
                  x      & x      & x      & x      & \cdots & 1      & 2      \\
                  x      & x      & x      & x      & \cdots & x      & 1      \\
              \end{vmatrix}$(提示:考虑滚动消去法).

    \item $ D_n=\begin{vmatrix}
                  a_1 & b_1 &     &        &         \\
                      & a_2 & b_2 &        &         \\
                      &     & a_3 & \ddots &         \\
                      &     &     & \ddots & b_{n-1} \\
                  b_n &     &     &        & a_n
              \end{vmatrix}$.

    \item $D_n=\begin{vmatrix}
                  a+b & ab  &        &     &     \\
                  1   & a+b & \ddots &     &     \\
                      & 1   & \ddots & ab  &     \\
                      &     & \ddots & a+b & ab  \\
                      &     &        & 1   & a+b \\
              \end{vmatrix}$.

    \item 用递推法解\autoref{ex:14:递推法}.

    \item (P188 第4题)解行列式 $ \begin{vmatrix}
                  a^{2} & (a+1)^{2} & (a+2)^{2} & (a+3)^{2} \\
                  b^{2} & (b+1)^{2} & (b+2)^{2} & (b+3)^{2} \\
                  c^{2} & (c+1)^{2} & (c+2)^{2} & (c+3)^{2} \\
                  d^{2} & (d+1)^{2} & (d+2)^{2} & (d+3)^{2}
              \end{vmatrix} $.

    \item (P189 第6题)设
          \[ D=\begin{vmatrix}
                  1+a_1  & 1      & \cdots & 1      \\
                  1      & 1+a_2  & \cdots & 1      \\
                  \vdots & \vdots & \ddots & \vdots \\
                  1      & 1      & \cdots & 1+a_n
              \end{vmatrix} \]
          \begin{enumerate}
              \item 用递推公式计算行列式$D$;

              \item 硬拆$D$为$2^n$个行列式,计算出结果.
          \end{enumerate}

    \item (P189 第5题(2))解行列式 $\begin{vmatrix}
                  a_{1}+a_{2}         & a_{2}+a_{3}         & \cdots & a_{n-1}+a_n         & a_n+a_{1}         \\
                  a_{1}^{2}+a_{2}^{2} & a_{2}^{2}+a_{3}^{2} & \cdots & a_{n-1}^{2}+a_n^{2} & a_n^{2}+a_{1}^{2} \\
                  \vdots              & \vdots              & \ddots & \vdots              & \vdots            \\
                  a_{1}^{n}+a_{2}^{n} & a_{2}^{n}+a_{3}^{n} & \cdots & a_{n-1}^{n}+a_n^{n} & a_n^{n}+a_{1}^{n}
              \end{vmatrix}$.

    \item (P188 第1题 (5)--(8))解行列式
          \begin{enumerate} \begin{multicols}{2}
                  \item $ D_1=\begin{vmatrix}
                          1 & 2 & 3 & 4 \\2&3&4&1\\3&4&1&2\\4&1&2&3
                      \end{vmatrix} $

                  \item $ D_2=\begin{vmatrix}
                          \lambda+2 & -1        & -1        & -1        \\
                          -1        & \lambda+2 & -1        & -1        \\
                          -1        & -1        & \lambda+2 & -1        \\
                          -1        & -1        & -1        & \lambda+2
                      \end{vmatrix} $

              \end{multicols} \begin{multicols}{2} % dirty hack

                  \item $ D_3=\begin{vmatrix}
                          1^{2} & 2^{2} & 3^{2} & 4^{2} \\
                          2^{2} & 3^{2} & 4^{2} & 5^{2} \\
                          3^{2} & 4^{2} & 5^{2} & 6^{2} \\
                          4^{2} & 5^{2} & 6^{2} & 7^{2}
                      \end{vmatrix} $

                  \item $ D_4=\begin{vmatrix}
                          3 & 2 & 0 & 0 \\
                          1 & 3 & 2 & 0 \\
                          0 & 1 & 3 & 2 \\
                          0 & 0 & 1 & 3
                      \end{vmatrix} $
              \end{multicols} \end{enumerate}

    \item (P190 第9题)解行列式
          \begin{enumerate}
              \item $D=\begin{vmatrix}
                            1      & 2      & \cdots & 2      & 2      \\
                            2      & 2      & \cdots & 2      & 2      \\
                            \vdots & \vdots & \ddots & \vdots & \vdots \\
                            2      & 2      & \cdots & n-1    & 2      \\
                            2      & 2      & \cdots & 2      & n\end{vmatrix}$;

              \item $^*$ $D=\begin{vmatrix}
                            1      & 2      & \cdots & n-1    & n      \\
                            2      & 3      & \cdots & n      & 1      \\
                            3      & 4      & \cdots & 1      & 2      \\
                            \vdots & \vdots & \ddots & \vdots & \vdots \\
                            n      & 1      & \cdots & n-2    & n-1
                        \end{vmatrix}$.
          \end{enumerate}

    \item (P190 第10题(2))证明$\begin{vmatrix}
                  a      & c      & c      & \cdots & c      \\
                  b      & a      & c      & \cdots & c      \\
                  b      & b      & a      & \cdots & c      \\
                  \vdots & \vdots & \vdots & \ddots & \vdots \\
                  b      & b      & b      & \cdots & a
              \end{vmatrix}=\dfrac{b(a-c)^{n}-c(a-b)^{n}}{b-c}$,其中 $b \neq c$, 等式左端是 $n$ 阶行列式.
\end{enumerate}

\centerline{\heiti B组}
\begin{enumerate}
    \item $^*$ 设$A,B,C,D$都是$n$阶方阵,且$AC=CA$,证明$\begin{vmatrix} A&B \\ C&D \end{vmatrix}=|AD-CB|$(为简化,可以只考虑$A$可逆的情况).

    \item $A\in \mathbf{F}^{m\times n}, B\in \mathbf{F}^{n\times m}$,证明$|E_m-AB|=|E_n-BA|$(即\autoref{sec:14:利用} 中的结论).

    \item $A\in \mathbf{F}^{m\times n}, B\in \mathbf{F}^{n\times m}$,证明$|\lambda E_m-AB|=\lambda^{m-n}|\lambda E_n-BA|$(为简化,$\lambda>0,\enspace m>n$).
\end{enumerate}

\centerline{\heiti C组}
\begin{enumerate}
    \item 解行列式
          \begin{multicols}{2} \begin{enumerate}
                  \item $D=\begin{vmatrix}
                                ax+by & ay+bz & az+bx \\
                                ay+bz & az+bx & ax+by \\
                                az+bx & ax+by & ay+bz
                            \end{vmatrix}$

                  \item $D=\begin{vmatrix}
                                x^2+1 & xy    & xz    \\
                                xy    & y^2+1 & yz    \\
                                xz    & yz    & z^2+1
                            \end{vmatrix}$
              \end{enumerate} \end{multicols}

    \item $^*$ 计算行列式$|2E-\alpha_1^\mathrm{T}\beta_1-\alpha_2^\mathrm{T}\beta_2|$,其中$\alpha_1=(a_1,a_2,\ldots,a_n),\enspace \beta_1=(b_1,b_2,\ldots,b_n),\enspace \alpha_2=(c_1,c_2,\ldots,c_n),\enspace \beta_2 = (d_1,d_2,\ldots,d_n)$.(提示:利用$|\lambda E_m-AB|=\lambda^{m-n}|\lambda E_n-BA|$)

    \item 已知$n$阶矩阵$A$满足
          \[ AA^\mathrm{T}=E,\enspace |A|=-1 \]
          求证:$|E+A|=0$.
\end{enumerate}
