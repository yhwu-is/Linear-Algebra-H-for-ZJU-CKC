\chapter*{未竟专题一\  \ 预备思想}
\addcontentsline{toc}{chapter}{未竟专题一\ \ 预备思想}

在第一讲中我们介绍了一些预备知识,但是在正式开始我们的学习旅程前,我希望在这里先讨论一些和知识本身关系不大的话题,也就是一些学习这门课的一些数学思想的准备,目的主要是给刚刚进入大学的同学一个思维上升的台阶,以便更好地接受接下来抽象的内容.

\section*{如何书写数学证明}
\addcontentsline{toc}{section}{如何书写数学证明}

\section*{代数结构的引入}
\addcontentsline{toc}{section}{代数结构的引入}

在第一讲中,

\section*{公理化思想与布尔巴基学派}
\addcontentsline{toc}{section}{公理化思想与布尔巴基学派}

\subsection*{公理化思想}

\subsection*{布尔巴基学派}

\vspace{2ex}
\centerline{\heiti \Large 内容总结}


\vspace{2ex}
\centerline{\heiti \Large 习题}

\vspace{2ex}
{\kaishu 教育不是灌输,而是点燃火焰。}
\begin{flushright}
    \kaishu
    ——苏格拉底
\end{flushright}

\centerline{\heiti A组}
\begin{enumerate}
    \item
\end{enumerate}

\centerline{\heiti B组}
\begin{enumerate}
    \item
\end{enumerate}

\centerline{\heiti C组}
\begin{enumerate}
    \item
\end{enumerate}
