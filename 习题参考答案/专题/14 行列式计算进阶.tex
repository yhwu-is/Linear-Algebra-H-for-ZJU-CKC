\section*{14 行列式计算进阶}
\addcontentsline{toc}{section}{14 行列式计算进阶}

\vspace{2ex}

\centerline{\heiti A组}
\begin{enumerate}
    \item 考虑滚动消去法,每行减下一行,得
    \[
    \begin{vmatrix}
    1-x&1&1&1&\ldots&1&1\\
    0&1-x&1&1&\ldots&1&1\\
    0&0&1-x&1&\ldots&1&1\\
    0&0&0&1-x&\ldots&1&1\\
    \vdots&\vdots&\vdots&\vdots&\ddots&\vdots&\vdots\\
    0&0&0&0&\ldots&1-x&1\\
    x&x&x&x&\ldots&x&1\\
    \end{vmatrix},\]
    第一行乘$(-x)$到第$n$行,得\[\begin{vmatrix}
    1-x&1&1&1&\ldots&1&1\\
    0&1-x&1&1&\ldots&1&1\\
    0&0&1-x&1&\ldots&1&1\\
    0&0&0&1-x&\ldots&1&1\\
    \vdots&\vdots&\vdots&\vdots&\ddots&\vdots&\vdots\\
    0&0&0&0&\ldots&1-x&1\\
    x^2&0&0&0&\ldots&0&1-x\\
    \end{vmatrix},\]
    按第$n$行展开,得原式等于$(1-x)^n+(-1)^{n+1}x^2T_{n-1}$,其中\[T_{n-1}=\begin{vmatrix}
    1&1&1&\ldots&1&1\\
    1-x&1&1&\ldots&1&1\\
    0&1-x&1&\ldots&1&1\\
    \vdots&\vdots&\vdots&\ddots&\vdots&\vdots\\
    0&0&0&\ldots&1&1\\
    0&0&0&\ldots&1-x&1\\
    \end{vmatrix},\]
    按第$n-1$行展开$T_{n-1}$,有$T_{n-1}=T_{n-2}+(-1)(1-x)T_{n-2}=xT_{n-2}$.由$T_2=\begin{vmatrix}
    1&1\\1-x&1\end{vmatrix}=x$,递推得$T_{n-1}=x^{n-2}$.代入可解得原式$=(1-x)^n-(-x)^n$

    \item 按第一列展开,得$D_n=\prod_{i=1}^na_i+(-1)^{n+1}\prod_{i=1}^nb_i$.

    \item 按第一列展开,得\[\begin{aligned}
    D_{n}&=(a+b) D_{n-1}-\left|\begin{array}{ccccccc}
    a b & 0 & 0 & 0 & \cdots & 0 & 0 \\
    1 & a+b & a b & 0 & \cdots & 0 & 0 \\
    0 & 1 & a+b & a b & \cdots & 0 & 0 \\
    \vdots & \vdots & \vdots & a+b & \ddots & \vdots & \vdots \\
    0 & 0 & 0 & 0 & \cdots & a+b & a b \\
    0 & 0 & 0 & 0 & \cdots & 1 & a+b
    \end{array}\right| \\
    &=(a+b) D_{n-1}-a b D_{n-2}
    \end{aligned}.\]
    \begin{enumerate}
        \item 法一:特征方程法. 有特征方程$r^2-(a+b)r+ab=(r-a)(r-b)=0$,先考虑$a\neq b$,则$D_n=C_1a^n+C_2b^n$,$D_{1}=a+b$,$D_{2}=a^{2}+a b+b^{2}$,则有
        \[\begin{cases}
            C_1a+C_2b=a+b\\
            C_1a^2+C_2b^2=a^2+ab+b^2
            \end{cases},\]
        考虑Cramer法则,首先判断系数行列式$\begin{vmatrix}
            a&b\\
            a^2&b^2
        \end{vmatrix}=ab(b-a)$,若$a=b=0$,原行列式为0;若$a=0,b\neq 0$,则有$C_2=1$,$D_n=b^n$;同理若$a=0,b\neq 0$,$D_n=a^n$.考虑$a\neq 0,b\neq 0$,则系数行列式非0,根据Cramer法则可以解得
        \[C_1=\frac{\begin{vmatrix}
            a+b&b\\
            a^2+ab+b^2&b^2
            \end{vmatrix}}{ab(b-a)}=\frac{a}{a-b},\quad
            C_2=\frac{\begin{vmatrix}
            a&a+b\\
            a^2&a^2+ab+b^2
            \end{vmatrix}}{ab(b-a)}=-\frac{b}{a-b}\]
        故$D_n=\displaystyle\frac{a^{n+1}-b^{n+1}}{a-b}$.考虑到$a$或$b$为0时也符合该公式,故合并为同一公式. 再考虑$a=b$,则$D_n=(C_1+C_2n)a^n$,有方程
        \[\begin{cases}
            (C_1+C_2)a=2a\\
            (C_1+2C_2)a^2=3a^2
            \end{cases},\]
        $a=0$,则$D_n=0$;$a\neq 0$,可以解得$C_1=C_2=1$,有$D_n=(n+1)a^n$. 考虑到$a=0$时$D_n$也符合该式,故合并.综上有
        \[D_n=\begin{cases}
            \displaystyle\frac{a^{n+1}-b^{n+1}}{a-b},&a\neq b\\
            (n+1)a^n,&a=b
            \end{cases}.\]

        \item 法二:递推式变形法. 递推式变形, 得
        \[\mathrm{D}_{n}-a D_{n-1}=b\left(D_{n-1}-a D_{n-2}\right),\]
        由于 $D_{1}=a+b, D_{2}=a^{2}+a b+b^{2}$,从而利用上述递推公式得
        \[\begin{aligned}
            \mathrm{D}_{n}-a D_{n-1}&=b\left(D_{n-1}-a D_{n-2}\right)\\
            &=b^{2}\left(D_{n-2}-a D_{n-3}\right)\\
            &=\cdots=b^{n-2}\left(D_{2}-a D_{1}\right)=b^{n}
            \end{aligned},\]
        故\[\begin{aligned}
            D_{n}&=a D_{n-1}+b^{n}=a\left(a D_{n-2}+b^{n-1}\right)+b^{n}\\
            &=\cdots=a^{n-1} D_{1}+a^{n-2} b^{2}+\cdots+a b^{n-1}+b^{n} \\
            &=a^{n}+a^{n-1} b+\cdots+a b^{n-1}+b^{n}\\
            &=\begin{cases}
            \displaystyle\frac{a^{n+1}-b^{n+1}}{a-b},&a\neq b\\
            (n+1)a^n,&a=b
            \end{cases}
            \end{aligned}.\]
    \end{enumerate}

    \item 略.

    \item \begin{enumerate}
        \item 法一:慢慢消去. \[\begin{aligned}
            \left|\begin{array}{llll}
            a^{2} & (a+1)^{2} & (a+2)^{2} & (a+3)^{2} \\
            b^{2} & (b+1)^{2} & (b+2)^{2} & (b+3)^{2} \\
            c^{2} & (c+1)^{2} & (c+2)^{2} & (c+3)^{2} \\
            d^{2} & (d+1)^{2} & (d+2)^{2} & (d+3)^{2}
            \end{array}\right|&=
            \left|\begin{array}{llll}
            a^{2} & 2a+1 & 4a+4 & 6a+9 \\
            b^{2} & 2b+1 & 4b+4 & 6b+9 \\
            c^{2} & 2c+1 & 4c+4 & 6c+9 \\
            d^{2} & 2d+1 & 4d+4 & 6d+9
            \end{array}\right|\\&=
            \left|\begin{array}{llll}
            a^{2} & 2a+1 & 4a+4 & 4 \\
            b^{2} & 2b+1 & 4b+4 & 4 \\
            c^{2} & 2c+1 & 4c+4 & 4 \\
            d^{2} & 2d+1 & 4d+4 & 4
            \end{array}\right|\\
            &=4\left|\begin{array}{llll}
            a^{2} & 2a & 4a & 1 \\
            b^{2} & 2b & 4b & 1 \\
            c^{2} & 2c & 4c & 1 \\
            d^{2} & 2d & 4d & 1
            \end{array}\right|=0
            \end{aligned}.\]
        \item 关注到行列式中只有$(a^2, b^2, c^2, d^2)^\mathrm{T}, (a, b, c, d)^\mathrm{T}$和$(1, 1, 1, 1)^\mathrm{T}$三个线性无关列向量,
        则一眼看出线性相关. 因此设原行列式为$|\alpha_1, \alpha_2, \alpha_3, \alpha_4|$,有$\alpha_4=\alpha_1-3\alpha_2+3\alpha_3$,因此原行列式值为0.
    \end{enumerate}

    \item \begin{enumerate}
        \item \[\begin{aligned}
        D_n&=\begin{vmatrix}
        D_{n-1}&a\\
        b&a_n
        \end{vmatrix},a=(0, \cdots, 0, -a_{n-1})^\mathrm{T}, b=(1, \cdots, 1)\\
        &=a_nD_{n-1}+a_{n-1}\begin{vmatrix}
        D_{n-2}&1_{n-2}\\
        1_{n-2}&1
        \end{vmatrix}, 1_{n-2}\text{表示由}(n-2)\text{个1构成的行/列向量}\\
        &=a_nD_{n-1}+a_{n-1}\begin{vmatrix}
        a_1&0&0&\ldots&0&1\\
        0&a_2&0&\ldots&0&1\\
        0&0&a_3&\ldots&0&1\\
        \vdots&\vdots&\vdots&\ddots&\vdots&\vdots\\
        0&0&0&\ldots&a_{n-2}&1\\
        0&0&0&\ldots&0&1\\
        \end{vmatrix}(1-n\text{列用第}n\text{列减})\\
        &=a_nD_{n-1}+\prod_{i=1}^{n-1}a_i
        \end{aligned},\]
        递推可得$\displaystyle D_n=\left(\prod_{k=1}^na_k\right)
        \left(1+\sum_{i=1}^n\frac{1}{a_i}\right)$.

        \item 将 1 写成 $1+0$, 将 $D$ 硬拆成 $2^{n}$ 个行列式, 只有如下的 $n+1$ 个 行列式非 0 :
        \[\begin{aligned}
        D_n =&\begin{vmatrix}
        1 & & & \\
        1 & a_{2} & & \\
        \vdots & & \ddots & \\
        1 & & & a_{n}
        \end{vmatrix}
        +\begin{vmatrix}
        a_{1} & 1 & \\
        & 1 & \\
        & \vdots & \ddots & \\
        & 1 & & a_{n}
        \end{vmatrix}+\cdots+\\
        &\begin{vmatrix}
        a_{1} & & &1 \\
        & a_{2} & & 1 \\
        & & \ddots & \vdots \\
        & & & 1
        \end{vmatrix}+\begin{vmatrix}
        a_{1} &  \\
        &a_{2}&\\
        & & \ddots & \\
        &&&a_n
        \end{vmatrix}\\
        =&\left(\prod_{k=1}^{n} a_{k}+\frac{1}{a_{2}} \prod_{k=1}^{n} a_{k}\right)\left(1+\sum_{i=1}^{n} \frac{1}{a_{i}}\right)
        \end{aligned}.\]
    \end{enumerate}

    \item 暴力拆成$2^n$个行列式,其中只有2个每列各不相同,其他都因为有相同列而为0. 因此有\[
    \begin{aligned}
    \text{原式}&=\begin{vmatrix}
    a_{1} & a_{2} & \cdots & a_{n-1} & a_{n} \\
    a_{1}^{2}& a_{2}^{2} & \cdots & a_{n-1}^{2} & a_{n}^{2} \\
    \vdots & & & & \vdots \\
    a_{1}^{n} & a_{2}^{n} & \cdots & a_{n-1}^{n} & a_{n}^{n}
    \end{vmatrix}+
    \begin{vmatrix}
    a_{2} & a_{3} & \cdots & a_{n} & a_{1} \\
    a_{2}^{2} & a_{3}^{2} & \cdots & a_{n}^{2} & a_{1}^{2} \\
    \vdots & & & & \vdots \\
    a_{2}^{n} & a_{3}^{n} & \cdots & a_{n}^{n} & a_{1}^{n}
    \end{vmatrix}\\
    &=[1+(-1)^{n-1}]\begin{vmatrix}
    a_{1} & a_{2} & \cdots & a_{n-1} & a_{n} \\
    a_{1}^{2}& a_{2}^{2} & \cdots & a_{n-1}^{2} & a_{n}^{2} \\
    \vdots & & & & \vdots \\
    a_{1}^{n} & a_{2}^{n} & \cdots & a_{n-1}^{n} & a_{n}^{n}
    \end{vmatrix}\\&=[1+(-1)^{n-1}]\prod_{i=1}^na_i
    \begin{vmatrix}
    1 & 1 & \cdots & 1 & 1 \\
    a_{1}& a_{2} & \cdots & a_{n-1} & a_{n} \\
    \vdots & & & & \vdots \\
    a_{1}^{n-1} & a_{2}^{n-1} & \cdots & a_{n-1}^{n-1} & a_{n}^{n-1}
    \end{vmatrix}\\
    &=[1+(-1)^{n-1}]\left(\prod_{i=1}^na_i\right)\prod_{1\leqslant i<j\leqslant n}(a_j-a_i)\\
    \end{aligned}.\]

    \item \begin{enumerate}
        \item 连加法得结果为160;
        \item 连加法得结果为$(\lambda-1)(\lambda+3)^3$;
        \item 如果对数据敏感能看出线性相关,可以直接由$\alpha_4-\alpha_1=3(\alpha_3-\alpha_2)$证得其线性相关,则为0. 也可以使用滚动消去法,如\[
        \begin{vmatrix}
        1&4&9&16\\
        3&5&7&9\\
        5&7&9&11\\
        7&9&11&13
        \end{vmatrix}=\begin{vmatrix}
        1&4&9&16\\
        3&5&7&9\\
        2&2&2&2\\
        2&2&2&2
        \end{vmatrix}=0.\]
        \item 当然是递推法的常见形式,但是只有4阶,所以可以暴算,转化为上三角行列式,如\[
        \begin{vmatrix}
        3&2&0&0\\
        0&\frac{7}{3}&2&0\\
        0&0&\frac{15}{7}&2\\
        0&0&0&\frac{31}{15}
        \end{vmatrix}=31.\]
    \end{enumerate}

    \item \begin{enumerate}
        \item 行列式中2比较多,用全是2的第二行去消,然后按第2行展开.\[D=\begin{vmatrix}
        -1 & 0 & \cdots & 0 & 0 \\
        2 & 2 & \cdots & 2 & 2 \\
        \vdots & & & & \vdots \\
        0 & 0 & \cdots & n-3 & 0 \\
        0 & 0 & \cdots & 0 & n-2
        \end{vmatrix}=
        =2\cdot(-1)\cdot(n-2)!=-2(n-2)!.\]

        \item \begin{enumerate}
            \item 法一:考虑滚动消去法,有
            \[\begin{aligned}
            D&=\begin{vmatrix}
            1 & 2 & 3 &\cdots & n-1 & n \\
            1 & 1 & 1 &\cdots & 1 & 1-n \\
            1 & 1 & 1 &\cdots & 1-n & 1 \\
            \vdots & & & & & \vdots \\
            1 & 1 & 1-n &\cdots & 1 & 1\\
            1 & 1-n & 1 &\cdots & 1 & 1
            \end{vmatrix}\\&=\begin{vmatrix}
            0 & n+1 & 2 &\cdots & n-2 & n-1 \\
            0 & n & 0 &\cdots & 0 & -n \\
            0 & n & 0 &\cdots & -n & 0 \\
            \vdots & & & & & \vdots \\
            0 & n & -n &\cdots & 0 & 0\\
            1 & 1-n & 1 &\cdots & 1 & 1
            \end{vmatrix}\\
            &=(-1)^{n+1}\begin{vmatrix}
            n+1 & 2 &\cdots & n-2 & n-1 \\
            n & 0 &\cdots & 0 & -n \\
            n & 0 &\cdots & -n & 0 \\
            \vdots & & & & \vdots \\
            n & -n &\cdots & 0 & 0
            \end{vmatrix}\\
            &=(-1)^{n+1}(-1)^{\frac{(n-2)(n-3)}{2}}
            \begin{vmatrix}
            n+1 & n-1 &\cdots & 3 & 2 \\
            n & -n &\cdots & 0 & 0 \\
            n & 0 &\cdots & 0 & 0 \\
            \vdots & & & & \vdots \\
            n & 0 &\cdots & 0 & -n
            \end{vmatrix}\\
            &=(-1)^{\frac{n^2-3n+8}{2}}\begin{vmatrix}
            \frac{n(n+1)}{2} & n-1 &\cdots & 3 & 2 \\
            0 & -n &\cdots & 0 & 0 \\
            0 & 0 &\cdots & 0 & 0 \\
            \vdots & & & & \vdots \\
            0 & 0 &\cdots & 0 & -n
            \end{vmatrix}=(-1)^{\frac{n^2-n+12}{2}}\frac{n^{n-1}(n+1)}{2}
            \end{aligned}.\]

            \item 法二:考虑连加法. 将第 $2,3, \cdots, n$ 列都加到第 1 列, 提出公因子 $\frac{1}{2} n(n+1)$, 再依次将第 $n-1$ 行乘 $(-1)$ 加到第 $n$ 行, $\cdots$, 第 2 行乘 $(-1)$ 加到第 3 行, 第 1 行乘 $(-1)$ 加到第 2 行, 然后对第 1 列展开, 得到一个 $n-1$ 阶行列式, 它的副对角元为 $1-$ $n$, 其余元素均为 1 . 再把它的各列加到第 1 列, 并把它的第 1 行乘 $(-1)$ 加到其 余各行, 得
            \[\begin{aligned}
            D &=\frac{n(n+1)}{2}\left|\begin{array}{ccccc}
            1 & 2 & \cdots & n-1 & n \\
            1 & 3 & \cdots & n & 1 \\
            1 & 4 & \cdots & 1 & 2 \\
            \vdots & \vdots & & \vdots & \vdots \\
            1 & 1 & \cdots & n-2 & n-1
            \end{array}\right| \\
            &=\frac{n(n+1)}{2}\left|\begin{array}{ccccc}
            1 & 2 & 3 & \cdots & n \\
            0 & 1 & 1 & \cdots & 1-n \\
            \vdots & \vdots & \vdots & & \vdots \\
            0 & 1 & 1-n & \cdots & 1 \\
            0 & 1-n & 1 & \cdots & 1
            \end{array}\right|_{n} \\
            &=\frac{n(n+1)}{2}
            \begin{vmatrix}
            1 & 1 & \cdots & 1 & 1-n \\
            \vdots & \vdots & & \vdots & 1 & \vdots \\
            1 & 1-n & \cdots & 1 & 1 \\
            1-n & 1 & \cdots & 1 & 1
            \end{vmatrix}
            \end{aligned},\]
            \[
            =\frac{n(n+1)}{2}\left|\begin{array}{ccccc}
            -1 & 1 & \cdots & 1 & 1-n \\
            0 & 0 & \cdots & -n & n \\
            \vdots & \vdots & & \vdots & \vdots \\
            0 & -n & \cdots & 0 & n \\
            0 & 0 & \cdots & 0 & n
            \end{array}\right|_{n-1},\]
            将上式先对第 1 列展开, 得到一个 $n-2$ 阶行列式, 再将它对最后一行展开, 得
            \[\begin{aligned}
            D &=\frac{-n(n+1)}{2} n\left|\begin{array}{ccccc}
            0 & 0 & \cdots & 0 & -n \\
            0 & 0 & \cdots & -n & 0 \\
            \vdots & \vdots & & \vdots & \vdots \\
            0 & -n & \cdots & 0 & 0 \\
            -n & 0 & \cdots & 0 & 0
            \end{array}\right|_{n-3} \\
            &=-\frac{n^{2}(n+1)}{2}(-1)^{\frac{(n-3)(n-4)}{2}}(-n)^{n-3} \\
            &=(-1)^{\frac{n(n-1)}{2} }\frac{(n+1) n^{n-1}}{2}
            \end{aligned}.\]
            事实上,两种方法得到的答案是等价的.
        \end{enumerate}
    \end{enumerate}

    \item 若 $b=c$, 则每行行和都相等, 考虑连加法, 把第 $2,3, \cdots, n$ 列都加到第 1 列, 提出公 因子 $a+(n-1) c$, 再将第 1 行乘 $(-1)$ 加到其余各行, 得到上三角行列式. 于是
    \[\begin{aligned}
    D_{n} &=[a+(n-1) c]\left|\begin{array}{ccccc}
    1 & c & c & \cdots & c \\
    1 & a & c & \cdots & c \\
    1 & c & a & \cdots & c \\
    \vdots & \vdots & \vdots & & \vdots \\
    1 & c & c & \cdots & a
    \end{array}\right| \\
    &=[a+(n-1) c]\left|\begin{array}{ccccc}
    1 & c & c & \cdots & c \\
    0 & a-c & 0 & \cdots & 0 \\
    0 & 0 & a-c & \cdots & 0 \\
    \vdots & \vdots & \vdots & & \vdots \\
    0 & 0 & 0 & \cdots & a-c
    \end{array}\right| \\
    &=[a+(n-1) c](a-c)^{n-1} .
    \end{aligned},\]
    若 $b \neq c$, 第 $i$ 行乘 $(-1)$ 加到第 $i-1$ 行 $(i$ 依次取 $n, n-1, \cdots, 2)$, 再对第 1 列 展开,
    \[D_{n}=\left|\begin{array}{ccccc}
    a-b & c-a & 0 & \cdots & 0 \\
    0 & a-b & c-a & \cdots & 0 \\
    \vdots & \vdots & \vdots & & \vdots \\
    0 & 0 & 0 & \cdots & c-a \\
    b & b & b & \cdots & a
    \end{array}\right|=(a-b)D_{n-1}+(-1)^{n+1}b(c-a)^{n-1},\]
    有了递推式,但是递推式中有$(-1)^{n+1}$,递推比较麻烦可能还需要处理. 如果用数归证明的话应该可以直接下手了,但是有更聪明的办法:取转置,有$D_n^\mathrm{T}=D_n$,同理有$D_n=(a-c)D_{n-1}+(-1)^{n+1}c(b-a)^{n-1}$.
\end{enumerate}

\centerline{\heiti B组}
\begin{enumerate}
    \item \begin{enumerate}
        \item 可以硬拆成8个行列式,其中只有2个非零,则得到
        \[D=\begin{vmatrix}
            ax&ay&az\\
            ay&az&ax\\
            az&ax&ay
            \end{vmatrix}+\begin{vmatrix}
            by&bz&bx\\
            bz&bx&by\\
            bx&by&bz\\
            \end{vmatrix}=(a+b)\begin{vmatrix}
            x&y&z\\
            y&z&x\\
            z&x&y\\
            \end{vmatrix}=(a^3+b^3)(3xyz-\sum x^2),\]
        另解:分解成两个矩阵相乘再各自求行列式.这其实是看出这是一种线性变换的本质之后的做法:
        \[D=\begin{vmatrix}
            \begin{pmatrix}
            a&b&0\\
            0&a&b\\
            b&0&a\\
            \end{pmatrix}
            \begin{pmatrix}
            x&y&z\\
            y&z&x\\
            z&x&y\\
            \end{pmatrix}
            \end{vmatrix}.\]
        \item 其实直接用三阶行列式的公式也不错,这里介绍基于硬拆的方法
        \[
        \begin{aligned}
        D&=x\begin{vmatrix}
        x&xy&xz\\
        y&y^2+1&yz\\
        z&yz&z^2+1
        \end{vmatrix}+\begin{vmatrix}
        1&xy&xz\\
        0&y^2+1&yz\\
        0&yz&z^2+1
        \end{vmatrix}\\
        &=x\begin{vmatrix}
        x&0&0\\
        y&1&0\\
        z&0&1
        \end{vmatrix}+\begin{vmatrix}
        y^2+1&yz\\
        yz&z^2+1
        \end{vmatrix}
        =\sum x^2+1
        \end{aligned}.\]
    \end{enumerate}

    \item 利用$|\lambda E_m-AB|=\lambda^{m-n}|\lambda E_n-BA|$
    \[\begin{aligned}
    |2E-\alpha_1^\mathrm{T}\beta_1-\alpha_2^\mathrm{T}\beta_2|
    &=\begin{vmatrix}
    2E-\begin{pmatrix}
    \alpha_1^\mathrm{T}&\alpha_2^\mathrm{T}
    \end{pmatrix}\begin{pmatrix}
    \beta_1\\\beta_2
    \end{pmatrix}
    \end{vmatrix}
    =2^{n-2}\begin{vmatrix}
    2E-\begin{pmatrix}
    \beta_1\\\beta_2
    \end{pmatrix}\begin{pmatrix}
    \alpha_1^\mathrm{T}&\alpha_2^\mathrm{T}
    \end{pmatrix}
    \end{vmatrix}\\
    &=2^{n-2}\begin{vmatrix}
    2-\beta_1\alpha_1^\mathrm{T}&-\beta_1\alpha_2^\mathrm{T}\\
    -\beta_2\alpha_1^\mathrm{T}&2-\beta_2\alpha_2^\mathrm{T}
    \end{vmatrix}\\
    &=2^{n-2}
    \left[\left(2-\sum_{i=1}^na_ib_i\right)
    \left(2-\sum_{i=1}^nc_id_i\right)
    -\left(\sum_{i=1}^na_id_i\right)
    \left(\sum_{i=1}^nb_ic_i\right)\right]
    \end{aligned}.
    \]

    \item 由条件有$|E+A|=|AA^\mathrm{T}+A|=|A(A^\mathrm{T}+E)|=|A||A^\mathrm{T}+E|=-|A^\mathrm{T}+E|$,然而,由$(E+A)^\mathrm{T}=E+A^\mathrm{T}$,有$|E+A|=|A^\mathrm{T}+E|$,综上两式就有$|E+A|=0$.
\end{enumerate}

\clearpage
