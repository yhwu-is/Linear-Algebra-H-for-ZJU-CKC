\section*{13 行列式(I)}
\addcontentsline{toc}{section}{13 行列式(I)}

\vspace{2ex}

\centerline{\heiti A组}
\begin{enumerate}
    \item 证明:\begin{enumerate}
        \item (线性性)直接将公理化定义用递归式对第i列展开:
        \begin{align*}
            & D(\alpha_1,\ldots,\lambda\alpha_{i}+\mu\beta_i,\ldots,\alpha_n)\\
            ={}& \sum_{k=1}^{n}(\lambda a_{ki}+\mu b_{ki})A_{ki}\\
            ={}& \lambda \cdot \sum_{k=1}^{n}a_{ki}A_{ki}+\mu \cdot \sum_{k=1}^{n}b_{ki}A_{ki}\\
            ={}& \lambda D(\alpha_1,\ldots,\alpha_{i},\ldots,\alpha_n)+\mu D(\alpha_1,\ldots,\beta_i,\ldots,\alpha_n)
        \end{align*}
        则线性性得证.
        \item (反对称性)使用数归方法证明:\\显然可得,$D(\alpha_1,\alpha_2)=-D(\alpha_2,\alpha_1)$,然后做出归纳假设:对于任意正整数$i,j$,$i,j \in [1,n-1]$且$i \neq j$,有:
        \[ D(\alpha_1,\ldots,\alpha_i,\ldots,\alpha_j,\ldots,\alpha_{n-1})=-D(\alpha_1,\ldots,\alpha_j,\ldots,\alpha_i,\ldots,\alpha_{n-1}) \]
        由此做出递推,对交换前后的行列式的首行做展开:
        \begin{align*}
            D(\alpha_1,\ldots,\alpha_{i},\ldots,\alpha_{j},\ldots,\alpha_n)
            & =\sum_{k=1}^{n}a_{1k}A_{1k}\\
            D(\alpha_1,\ldots,\alpha_{j},\ldots,\alpha_{i},\ldots,\alpha_n)
            & =\sum_{k=1}^{n}a'_{1k}A'_{1k}
        \end{align*}
        其中,除第$i,j$项外,由归纳假设,其余项都满足$a_{1k}=a'_{1k},A_{1k}=-A'_{1k}$,则有$a_{1k}A_{1k}=-a'_{1k}A'_{1k},k\neq i,j$.因此主要考察$a_{1i}A_{1i}+a_{1j}A_{1j}$与$a'_{1i}A'_{1i}+a'_{1j}A'_{1j}$这两项.首先有$a'_{1i}=a_{1j},a'_{1j}=a_{1i}$.然后将$A_{1i}$与$A'_{1j}$两项展开对比:
        \begin{align*}
            A_{1i}&=(-1)^{1+i}(\alpha'_{1},\ldots,\alpha'_{i-1},\alpha'_{i+1},\ldots,\alpha'_{j},\ldots,\alpha'_{n})\\
            A'_{1j}&=(-1)^{1+j}(\alpha'_{1},\ldots,\alpha'_{i-1},\alpha'_{j},\alpha'_{i+1},\ldots,\alpha'_{j-1},\alpha'_{j+1},\ldots,\alpha'_{n})
        \end{align*}
        式中的$\alpha'_k$表示原列向量去掉首行元素后剩余$n-1$个元素组成的新列向量.可以发现,$A'_{1j}$向左交换$j-(i+1)$次后与$A_{1i}$是绝对值一致的.则根据归纳假设,有$(-1)^{j-(i+1)}A'_{1j}=(-1)^{1+j-(1+i)}A_{1i}$,即有$A'_{1j}=-A_{1i}$,所以$a_{1i}A_{1i}+a_{1j}A_{1j}=-(a'_{1j}A'_{1j}+a'_{1i}A'_{1i})$.综上可证:
        \[ D(\alpha_1,\ldots,\alpha_{i},\ldots,\alpha_{j},\ldots,\alpha_n)=D(\alpha_1,\ldots,\alpha_{j},\ldots,\alpha_{i},\ldots,\alpha_n) \]
        \item (规范性)只需使用递归式定义,逐次展开即可.
    \end{enumerate}
    \item 使用倍加列变换的性质,可得$|\alpha_1+\alpha_2+\alpha_3,\alpha_1+3\alpha_2+9\alpha_3,\alpha_1+4\alpha_2+16\alpha_3|=6|\alpha_1,\alpha_2,\alpha_3|=12$
    \item \begin{enumerate}
        \item 首先由反对称矩阵的性质,有$A^T=-A$,其次由矩阵与行列式的性质,可得$|A^T|=|A|$,则可推出$|A|=|-A|=(-1)^n|A|$,又n为奇数,故$|A|=0$,矩阵$A$不可逆.
        \item 只需证明$AB$的秩小于$n$,即$|AB|=0$即可.$B$是奇数阶反对称矩阵,$|B|=0$,则$|AB|=|A||B|=0$得证.
    \end{enumerate}
    \item 根据伴随矩阵的性质,$\begin{pmatrix}
        A & O \\ O & B
    \end{pmatrix}^* = \begin{vmatrix}
        A & O \\ O & B
    \end{vmatrix} \cdot \begin{pmatrix}
        A & O \\ O & B
    \end{pmatrix}^{-1}$.其中,对$\begin{vmatrix}
        A & O \\ O & B
    \end{vmatrix}$做递归式展开,有$\begin{vmatrix}
        A & O \\ O & B
    \end{vmatrix}=\sum_{k=1}^{m}(-1)^{1+k}a_{1k}\begin{vmatrix}
        M_{1k} & O \\ O & B
    \end{vmatrix}$,依此逐次展开可得$\begin{vmatrix}
        A & O \\ O & B
    \end{vmatrix}=|A||B|=ab$.又$ab\begin{pmatrix}
        A & O \\ O & B
    \end{pmatrix}^{-1}=ab\begin{pmatrix}
        A^{-1} & O \\ O & B^{-1}
    \end{pmatrix}=\begin{pmatrix}
        b\cdot aA^{-1} & O \\ O & a\cdot bB^{-1}
    \end{pmatrix}$,最终可得$\begin{pmatrix}
        A & O \\ O & B
    \end{pmatrix}^* = \begin{pmatrix}
        bA^* & O \\ O & aB^*
    \end{pmatrix}$.

    与前一个式子的展开类似,$\begin{pmatrix}
        O & A \\ B & O
    \end{pmatrix}^*=\begin{vmatrix}
        O & A \\ B & O
    \end{vmatrix}\cdot \begin{pmatrix}
        O & A \\ B & O
    \end{pmatrix}^{-1}$,其中对前一步推导做少量修正后可得$\begin{vmatrix}
        O & A \\ B & O
    \end{vmatrix}=(-1)^{mn}|A||B|=(-1)^{mn}ab$,则$(-1)^{mn}ab\begin{pmatrix}
        O & A \\ B & O
    \end{pmatrix}^{-1}=(-1)^{mn}\begin{pmatrix}
        O & a\cdot bB^{-1} \\ b\cdot aA^{-1} & O
    \end{pmatrix}$,最终可得$\begin{pmatrix}
        O & A \\ B & O
    \end{pmatrix}^*=(-1)^{mn}\begin{pmatrix}
        O & aB^* \\ bA^* & O
    \end{pmatrix}$.
    \item 证明;\begin{enumerate}
        \item 对正整数$k$,有$(A^k)^*=(A^*)^k$.从而若$A^m=A$,则$(A^*)^m=(A^m)^*=A^*$.即$A$是幂等矩阵,则$A^*$也是幂等矩阵.同样,若$A^m=0$,则$(A^*)^m=(A^m)^*=0$.即$A$是幂零矩阵,则$A^*$也是幂零矩阵.
        \item $(A^*)^T=(A^T)^*=A^*$.从而若$A$是对称矩阵,则$A^*$也是对称矩阵.$(A^*)^T=(A^T)^*=(-A)^*=(-1)^{n-1}A^*$.从而若$A^*$为偶数阶时也为反对称矩阵,奇数阶时为对称矩阵.
    \end{enumerate}
    \item 对于上三角矩阵,只需看其伴随矩阵的下半部分元素,即$A_{ij},j>i$.对这些代数余子式,要么有一整行或整列为零,要么对角线存在零元素,则这些余子式均为零,伴随矩阵是上三角矩阵.\par 或者使用另法,当$A$是可逆矩阵,即$A$对角线元素不为零,则$A^*=|A|\cdot A^{-1}$,其中$A^{-1}$也是上三角矩阵,因此伴随矩阵$A^*$是上三角矩阵.
    \item $|A|=0$,则$r(A)<n$,故$r(A^*)=\begin{cases}
        1,&r(A)=n-1\\
        0,&r(A)<n-1
    \end{cases}$.当$r(A^*)=0$时,答案显然成立;当$r(A^*)=1$时,由矩阵秩的定义,任意两行(列)必成比例,否则秩大于1,矛盾.综上,原题得证.
    \item $(\alpha_1-\alpha_2-2\alpha_3,2\alpha_1+\alpha_2-\alpha_3,3\alpha_1+\alpha_2+2\alpha_3)=(\alpha_1,\alpha_2,\alpha_3)\begin{pmatrix}
        1 & 2 & 3 \\ -1 & 1 & 1 \\ -2 & -1 & 2
    \end{pmatrix}$,又$\begin{vmatrix}
        1 & 2 & 3 \\ -1 & 1 & 1 \\ -2 & -1 & 2
    \end{vmatrix}=12\neq 0$,因此该矩阵可逆,则$(\alpha_1-\alpha_2-2\alpha_3,2\alpha_1+\alpha_2-\alpha_3,3\alpha_1+\alpha_2+2\alpha_3)$与$(\alpha_1,\alpha_2,\alpha_3)$秩相等,因此这三个向量线性无关.
    \item 只要取$\alpha_3,\alpha_4 \in \mathbf{R}^4$,使得$D(\alpha_1,\alpha_2,\alpha_3,\alpha_4) \neq 0$即可.易得$\begin{vmatrix}
        1 & 1 & 0 & 0 \\
        2 & 4 & 0 & 0 \\
        1 & -1 & 1 & 0 \\
        -1 & 1 & 0 & 1
    \end{vmatrix}=\begin{vmatrix}
        1 & 1 \\
        2 & 4
    \end{vmatrix} \cdot \begin{vmatrix}
        1 & 0 \\
        0 & 1
    \end{vmatrix}=2 \neq 0$.于是可取$\alpha_3=(0,0,1,0)^T,\alpha_4=(0,0,0,1)^T$,能使得${\alpha_1,\alpha_2,\alpha_3,\alpha_4}$为$\mathbf{R}^4$的一组基.
\end{enumerate}

\centerline{\heiti B组}
\begin{enumerate}
    \item
\end{enumerate}

\centerline{\heiti C组}
\begin{enumerate}
    \item
\end{enumerate}

\clearpage
