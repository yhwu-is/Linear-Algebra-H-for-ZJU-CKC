\phantomsection
\section*{13 行列式}
\addcontentsline{toc}{section}{13 行列式}

\vspace{2ex}

\centerline{\heiti A组}
\begin{enumerate}
    \item 证明:\begin{enumerate}
              \item (线性性)直接将公理化定义用递归式对第$i$列展开:
                    \begin{align*}
                            & D(\alpha_1,\ldots,\lambda\alpha_{i}+\mu\beta_i,\ldots,\alpha_n)                                      \\
                        ={} & \sum_{k=1}^{n}(\lambda a_{ki}+\mu b_{ki})A_{ki}                                                      \\
                        ={} & \lambda \cdot \sum_{k=1}^{n}a_{ki}A_{ki}+\mu \cdot \sum_{k=1}^{n}b_{ki}A_{ki}                        \\
                        ={} & \lambda D(\alpha_1,\ldots,\alpha_{i},\ldots,\alpha_n)+\mu D(\alpha_1,\ldots,\beta_i,\ldots,\alpha_n)
                    \end{align*}
                    则线性性得证.

              \item (反对称性)使用数学归纳法证明. 显然,$D(\alpha_1,\alpha_2)=-D(\alpha_2,\alpha_1)$,然后做出归纳假设:对于任意正整数$i,j$,$1 \leqslant i, j \leqslant n - 1$且$i \neq j$,有:
                    \[ D(\alpha_1,\ldots,\alpha_i,\ldots,\alpha_j,\ldots,\alpha_{n-1})=-D(\alpha_1,\ldots,\alpha_j,\ldots,\alpha_i,\ldots,\alpha_{n-1}) \]
                    由此做出递推,对交换前后的行列式的首行做展开:
                    \begin{align*}
                        D(\alpha_1,\ldots,\alpha_{i},\ldots,\alpha_{j},\ldots,\alpha_n)
                         & =\sum_{k=1}^{n}a_{1k}A_{1k}   \\
                        D(\alpha_1,\ldots,\alpha_{j},\ldots,\alpha_{i},\ldots,\alpha_n)
                         & =\sum_{k=1}^{n}a'_{1k}A'_{1k}
                    \end{align*}
                    其中,除第$i,j$项外,由归纳假设,其余项都满足$a_{1k}=a'_{1k},A_{1k}=-A'_{1k}$,则有$a_{1k}A_{1k}=-a'_{1k}A'_{1k},k\neq i,j$. 因此主要考察$a_{1i}A_{1i}+a_{1j}A_{1j}$与$a'_{1i}A'_{1i}+a'_{1j}A'_{1j}$这两项. 首先有$a'_{1i}=a_{1j},a'_{1j}=a_{1i}$. 然后将$A_{1i}$与$A'_{1j}$两项展开对比:
                    \begin{align*}
                        A_{1i}  & =(-1)^{1+i}(\alpha'_{1},\ldots,\alpha'_{i-1},\alpha'_{i+1},\ldots,\alpha'_{j},\ldots,\alpha'_{n})                             \\
                        A'_{1j} & =(-1)^{1+j}(\alpha'_{1},\ldots,\alpha'_{i-1},\alpha'_{j},\alpha'_{i+1},\ldots,\alpha'_{j-1},\alpha'_{j+1},\ldots,\alpha'_{n})
                    \end{align*}
                    式中的$\alpha'_k$表示原列向量去掉首行元素后剩余$n-1$个元素组成的新列向量. 可以发现,$A'_{1j}$向左交换$j-(i+1)$次后与$A_{1i}$是绝对值一致的. 则根据归纳假设,有$(-1)^{j-(i+1)}A'_{1j}=(-1)^{1+j-(1+i)}A_{1i}$,即有$A'_{1j}=-A_{1i}$,所以$a_{1i}A_{1i}+a_{1j}A_{1j}=-(a'_{1j}A'_{1j}+a'_{1i}A'_{1i})$. 综上可证:
                    \[ D(\alpha_1,\ldots,\alpha_{i},\ldots,\alpha_{j},\ldots,\alpha_n)=D(\alpha_1,\ldots,\alpha_{j},\ldots,\alpha_{i},\ldots,\alpha_n) \]

              \item (规范性)只需使用递归式定义,逐次展开即可.
          \end{enumerate}

    \item 使用倍加列变换的性质,可得$|\alpha_1+\alpha_2+\alpha_3,\alpha_1+3\alpha_2+9\alpha_3,\alpha_1+4\alpha_2+16\alpha_3|=6|\alpha_1,\alpha_2,\alpha_3|=12$

    \item \begin{enumerate}
              \item 首先由反对称矩阵的性质,有$A^T=-A$,其次由矩阵与行列式的性质,可得$|A^T|=|A|$,则可推出$|A|=|-A|=(-1)^n|A|$,又n为奇数,故$|A|=0$,矩阵$A$不可逆.

              \item 只需证明$AB$的秩小于$n$,即$|AB|=0$即可. $B$是奇数阶反对称矩阵,$|B|=0$,则$|AB|=|A||B|=0$得证.
          \end{enumerate}

    \item 根据伴随矩阵的性质,$\begin{pmatrix}
                  A & O \\ O & B
              \end{pmatrix}^* = \begin{vmatrix}
                  A & O \\ O & B
              \end{vmatrix} \cdot \begin{pmatrix}
                  A & O \\ O & B
              \end{pmatrix}^{-1}$. 其中,对$\begin{vmatrix}
                  A & O \\ O & B
              \end{vmatrix}$做递归式展开,有$\begin{vmatrix}
                  A & O \\ O & B
              \end{vmatrix}=\displaystyle\sum_{k=1}^{m}(-1)^{1+k}a_{1k}\begin{vmatrix}
                  M_{1k} & O \\ O & B
              \end{vmatrix}$,依此逐次展开可得$\begin{vmatrix}
                  A & O \\ O & B
              \end{vmatrix}=|A||B|=ab$. 又$ab\begin{pmatrix}
                  A & O \\ O & B
              \end{pmatrix}^{-1}=ab\begin{pmatrix}
                  A^{-1} & O \\ O & B^{-1}
              \end{pmatrix}=\begin{pmatrix}
                  b\cdot aA^{-1} & O \\ O & a\cdot bB^{-1}
              \end{pmatrix}$,最终可得$\begin{pmatrix}
                  A & O \\ O & B
              \end{pmatrix}^* = \begin{pmatrix}
                  bA^* & O \\ O & aB^*
              \end{pmatrix}$.

          与前一个式子的展开类似,$\begin{pmatrix}
                  O & A \\ B & O
              \end{pmatrix}^*=\begin{vmatrix}
                  O & A \\ B & O
              \end{vmatrix}\cdot \begin{pmatrix}
                  O & A \\ B & O
              \end{pmatrix}^{-1}$,其中对前一步推导做少量修正后可得$\begin{vmatrix}
                  O & A \\ B & O
              \end{vmatrix}=(-1)^{mn}|A||B|=(-1)^{mn}ab$,则$(-1)^{mn}ab\begin{pmatrix}
                  O & A \\ B & O
              \end{pmatrix}^{-1}=(-1)^{mn}\begin{pmatrix}
                  O & a\cdot bB^{-1} \\ b\cdot aA^{-1} & O
              \end{pmatrix}$,最终可得$\begin{pmatrix}
                  O & A \\ B & O
              \end{pmatrix}^*=(-1)^{mn}\begin{pmatrix}
                  O & aB^* \\ bA^* & O
              \end{pmatrix}$.

    \item 证明;\begin{enumerate}
              \item 对正整数$k$,有$(A^k)^*=(A^*)^k$. 从而若$A^m=A$,则$(A^*)^m=(A^m)^*=A^*$. 即$A$是幂等矩阵,则$A^*$也是幂等矩阵. 同样,若$A^m=0$,则$(A^*)^m=(A^m)^*=0$. 即$A$是幂零矩阵,则$A^*$也是幂零矩阵.

              \item $(A^*)^T=(A^T)^*=A^*$. 从而若$A$是对称矩阵,则$A^*$也是对称矩阵$. (A^*)^T=(A^T)^*=(-A)^*=(-1)^{n-1}A^*$. 从而若$A^*$为偶数阶时也为反对称矩阵,奇数阶时为对称矩阵.
          \end{enumerate}

    \item 对于上三角矩阵,只需看其伴随矩阵的下半部分元素,即$A_{ij},j>i$. 对这些代数余子式,要么有一整行或整列为零,要么对角线存在零元素,则这些余子式均为零,伴随矩阵是上三角矩阵.\par 或者使用另法,当$A$是可逆矩阵,即$A$对角线元素不为零,则$A^*=|A|\cdot A^{-1}$,其中$A^{-1}$也是上三角矩阵,因此伴随矩阵$A^*$是上三角矩阵.

    \item $|A|=0$,则$r(A)<n$,故$r(A^*)=\begin{cases}
                  1, & r(A)=n-1 \\
                  0, & r(A)<n-1
              \end{cases}$. 当$r(A^*)=0$时,答案显然成立;当$r(A^*)=1$时,由矩阵秩的定义,任意两行(列)必成比例,否则秩大于1,矛盾. 综上,原题得证.

    \item $(\alpha_1-\alpha_2-2\alpha_3,2\alpha_1+\alpha_2-\alpha_3,3\alpha_1+\alpha_2+2\alpha_3)=(\alpha_1,\alpha_2,\alpha_3)\begin{pmatrix}
                  1 & 2 & 3 \\ -1 & 1 & 1 \\ -2 & -1 & 2
              \end{pmatrix}$,又$\begin{vmatrix}
                  1 & 2 & 3 \\ -1 & 1 & 1 \\ -2 & -1 & 2
              \end{vmatrix}=12\neq 0$,因此该矩阵可逆,则$(\alpha_1-\alpha_2-2\alpha_3,2\alpha_1+\alpha_2-\alpha_3,3\alpha_1+\alpha_2+2\alpha_3)$与$(\alpha_1,\alpha_2,\alpha_3)$秩相等,因此这三个向量线性无关.

    \item 只要取$\alpha_3,\alpha_4 \in \mathbf{R}^4$,使得$D(\alpha_1,\alpha_2,\alpha_3,\alpha_4) \neq 0$即可. 易得$\begin{vmatrix}
                  1  & 1  & 0 & 0 \\
                  2  & 4  & 0 & 0 \\
                  1  & -1 & 1 & 0 \\
                  -1 & 1  & 0 & 1
              \end{vmatrix}=\begin{vmatrix}
                  1 & 1 \\
                  2 & 4
              \end{vmatrix} \cdot \begin{vmatrix}
                  1 & 0 \\
                  0 & 1
              \end{vmatrix}=2 \neq 0$. 于是可取$\alpha_3=(0,0,1,0)^T,\alpha_4=(0,0,0,1)^T$,能使得${\alpha_1,\alpha_2,\alpha_3,\alpha_4}$为$\mathbf{R}^4$的一组基.
\end{enumerate}

\centerline{\heiti B组}
\begin{enumerate}
    \item 这三小问实际上都是对原行列式中的某一行进行了替换进行计算.
          \begin{enumerate}
              \item \begin{align*}
                        A_{21}+A_{22}+A_{23}+A_{24}
                         & = \begin{vmatrix}
                                 3 & 0  & 4  & 1 \\
                                 1 & 1  & 1  & 1 \\
                                 0 & -7 & 8  & 3 \\
                                 5 & 3  & -2 & 2
                             \end{vmatrix}
                        = \begin{vmatrix}
                              3 & -3 & 1  & -2 \\
                              1 & 0  & 0  & 0  \\
                              0 & -7 & 8  & 3  \\
                              5 & -2 & -7 & -3
                          \end{vmatrix}           \\
                         & = (-1)^{2+1} \begin{vmatrix}
                                            -3 & 1  & -1 \\
                                            -7 & 8  & 3  \\
                                            -2 & -7 & -3
                                        \end{vmatrix} \\
                         & = 148
                    \end{align*}

              \item \begin{align*}
                        A_{31}+A_{33}
                         & = 1A_{31}+0A_{32}+1A_{33}+0A_{34}
                        = \begin{vmatrix}
                              3 & 0 & 4  & 1 \\
                              2 & 3 & 1  & 4 \\
                              1 & 0 & 1  & 0 \\
                              5 & 3 & -2 & 2
                          \end{vmatrix}
                        = 3 \begin{vmatrix}
                                3 & 0 & 4  & 1 \\
                                2 & 1 & 1  & 4 \\
                                1 & 0 & 1  & 0 \\
                                5 & 1 & -2 & 2
                            \end{vmatrix}              \\
                         & = 3 \begin{vmatrix}
                                   3 & 0 & 1  & 1 \\
                                   2 & 1 & -1 & 4 \\
                                   1 & 0 & 0  & 0 \\
                                   5 & 3 & -7 & 2
                               \end{vmatrix}
                        = 3 \cdot (-1)^{3+1} \begin{vmatrix}
                                                 0 & 1  & 1 \\
                                                 1 & -1 & 4 \\
                                                 3 & -7 & 2
                                             \end{vmatrix} \\
                         & = -12
                    \end{align*}

              \item \begin{align*}
                        M_{41}+M_{42}+M_{43}+M_{44}
                         & = -A_{41}+A_{42}-A_{43}+A_{44}
                        = \begin{vmatrix}
                              3  & 0  & 4  & 1 \\
                              2  & 3  & 1  & 4 \\
                              0  & -7 & 8  & 3 \\
                              -1 & 1  & -1 & 1
                          \end{vmatrix}                  \\
                         & = \begin{vmatrix}
                                 3  & 3  & 1  & 4 \\
                                 2  & 5  & -1 & 6 \\
                                 0  & -7 & 8  & 3 \\
                                 -1 & 0  & 0  & 0
                             \end{vmatrix}
                        = {-1}^{4+1} \cdot (-1) \begin{vmatrix}
                                                    3  & 1  & 4 \\
                                                    5  & -1 & 6 \\
                                                    -7 & 8  & 3
                                                \end{vmatrix} \\
                         & = -78
                    \end{align*}
          \end{enumerate}

    \item 有问题,之后可能会考虑修改题目,此处略.

    \item \begin{align*}
              \lvert A+B^{-1} \rvert ={} & \lvert B^{-1}BA+B^{-1}E \rvert = \lvert B^{-1} \rvert \cdot \lvert BA+E \rvert                  \\
              ={}                        & \frac{1}{2} \lvert BA+A^{-1}A \rvert = \frac{1}{2} \lvert B+A^{-1} \rvert \cdot \lvert A \rvert \\
              ={}                        & \frac{3}{2} \lvert A^{-1}+B \rvert = 3
          \end{align*}

    \item 正交矩阵满足 $AA^{\mathrm{T}} = A^{\mathrm{T}}A = E$,所以 $\lvert AA^{\mathrm{T}} \rvert = \lvert A \rvert^2 = \lvert E \rvert = 1$. 而 $\lvert A \rvert < 0$,所以 $\lvert A \rvert = -1$.
          \[\lvert E+A \rvert = \lvert AA^{\mathrm{T}}+A \rvert = \lvert A \rvert \cdot \lvert A^{\mathrm{T}}+E \rvert = -\lvert (A+E)^{\mathrm{T}} \rvert = -\lvert E+A \rvert.\]
          故 $\lvert E+A \rvert = 0$.

          以上两道题都多次运用了 $E = AA^{-1} = A^{-1}A$ 的技巧,请大家留意.

    \item \begin{enumerate}
              \item 由于行列式
                    \[ D = \begin{vmatrix}
                            a_{11}     & a_{12}     & a_{13}     & \cdots & a_{1n}     \\
                            a_{11}     & a_{12}     & a_{13}     & \cdots & a_{1n}     \\
                            a_{21}     & a_{22}     & a_{23}     & \cdots & a_{2n}     \\
                            \vdots     & \vdots     & \vdots     & \ddots & \vdots     \\
                            a_{n-1, 1} & a_{n-1, 2} & a_{n-1, 3} & \cdots & a_{n-1, n}
                        \end{vmatrix} = 0\]
                    而 $M_1, -M_2, \ldots, (-1)^{n-1}M_n$ 恰是 $D$ 的第一行元素的代数余子式,所以将 $D$ 按第一行展开,可知\[a_{11}M_1+a_{12}(-M_2)+\cdots+a_{1n}(-1)^{n-1}M_n=0.\]
                    而 $D$ 的其他行元素与第一行元素的代数余子式乘积之和为 0,于是结论成立.

              \item 因为 $r(A) = n-1$,所以 $M_1, -M_2, \ldots, (-1)^{n-1}M_n$,不全为 0. 且该方程组解空间维数为 1,$M_1, -M_2, \ldots, (-1)^{n-1}M_n$ 正是该方程组的非零解,结论成立.
          \end{enumerate}

    \item $A^* = \lvert A \rvert A^{-1} = 2A^{-1}, B^* = \lvert B \rvert B^{-1} = B^{-1}$,故 \[\lvert 2A^*B^*-A^{-1}B^{-1} \rvert = \lvert 4A^{-1}B^{-1}-A^{-1}B^{-1} \rvert = \lvert 3A^{-1}B^{-1} \rvert = \dfrac{3^n}{\lvert A \rvert \lvert B \rvert} = \dfrac{3^n}{2}.\]

    \item \begin{enumerate}
              \item 设 $A = (a_{ij})$,则 $A^* = (A_{ji})$($A^*$ 的表达式). 而 $A^T = A^*$,有 $a_{ij} = A_{ij}, \forall i, j = 1, 2, \ldots, n$. 而 $A$ 非零,故 $\exists a_{kl} \neq 0$,从而有 \[ \lvert A \rvert = a_{k1}A_{k1}+\cdots+a_{kn}A_{kn} = a_{k1}^2+\cdots+a_{kn}^2 > 0.\]

              \item $\lvert A \rvert > 0$ 有 $A$ 是可逆的,故 $A^* = \lvert A \rvert A^{-1} = A^{\mathrm{T}}$,从而 $A^{\mathrm{T}}A = \lvert A \rvert E$. 两侧取行列式有 $\lvert A \rvert^2 = \lvert A \rvert^n $,结合 $\lvert A \rvert > 0$ 有 $\lvert A \rvert = 1$.

              \item $\lvert A \rvert = 1$,故 $A^{-1} = A^{\mathrm{T}}$,$A$ 是正交矩阵.

              \item \begin{align*}
                        \lvert E-A \rvert & = \lvert AA^{\mathrm{T}}-AE \rvert = \lvert A \rvert \lvert A^{\mathrm{T}}-E\rvert = \lvert A^{\mathrm{T}}-E\rvert = \lvert (A-E)^{\mathrm{T}} \rvert \\
                                          & = \lvert A-E \rvert = (-1)^n\lvert E-A \rvert.
                    \end{align*}
                    $n$ 为奇数,则 $\lvert E-A \rvert = -\lvert E-A \rvert$,即 $\lvert E-A \rvert = 0$.
          \end{enumerate}

    \item $r(A) = n-1$,则 $\lvert A \rvert = 0$ 且 $AX = 0$ 的解空间的维数为 1. 而考虑 $AA^* = \lvert A \rvert E = 0$,且 $\exists A_{ij} \neq 0$,所以 $(A_{i1}, A_{i2}, \ldots , A_{in})^{\mathrm{T}}$ 是所求的基础解系.

    \item 设 \[A = \begin{pmatrix}
                  a_{11} & a_{12} & \cdots & a_{1n} \\
                  a_{21} & a_{22} & \cdots & a_{2n} \\
                  \vdots & \vdots & \ddots & \vdots \\
                  a_{n1} & a_{n2} & \cdots & a_{nn} \\
              \end{pmatrix}\]
          则 \[A^* = \begin{pmatrix}
                  A_{11}     & A_{21}     & \cdots & A_{n-1, 1}   & A_{n1}     \\
                  A_{12}     & A_{22}     & \cdots & A_{n-1, 2}   & A_{n2}     \\
                  \vdots     & \vdots     & \ddots & \vdots       & \vdots     \\
                  A_{1, n-1} & A_{2, n-1} & \cdots & A_{n-1, n-1} & A_{n, n-1} \\
                  A_{1n}     & A_{2n}     & \cdots & A_{n-1, n}   & A_{nn}     \\
              \end{pmatrix}\]
          注意到目标行列式是 $A^*$ 中元素 $A_{nn}$ 的代数余子式,也就是 $(A^*)^*$ 中 $(n, n)$ 位置的元素. 由 {例13.9(4)} $(A^*)^* = \lvert A \rvert^{n-2}A$ 可知结论成立. % FIXME: xref

    \item 设 $f(x) = c_0+c_1x+c_2x^2+\cdots+c_{n-1}x^{n-1}$,则有
          \[\begin{cases} \begin{aligned}
                      c_0+c_1a_1+c_2a_1^2+\cdots+c_{n-1}a_1^{n-1} & = b_1,            \\
                      c_0+c_1a_2+c_2a_2^2+\cdots+c_{n-1}a_2^{n-1} & = b_2,            \\
                                                                  & \vdotswithin{ = } \\
                      c_0+c_1a_n+c_2a_n^2+\cdots+c_{n-1}a_n^{n-1} & = b_n.            \\
                  \end{aligned} \end{cases}\]
          注意此处我们研究的对象是 $(c_0, c_1, \ldots, c_{n-1})^{\mathrm{T}}$. 因为系数行列式
          \[D = \begin{vmatrix}
                  1      & a_1    & \cdots & a_1^{n-1} \\
                  1      & a_2    & \cdots & a_2^{n-1} \\
                  \vdots & \vdots & \ddots & \vdots    \\
                  1      & a_n    & \cdots & a_n^{n-1} \\
              \end{vmatrix} = \prod_{1 \leqslant j < i \leqslant n} (a_i-a_j) \neq 0.\]
          所以由 Cramer 法则,上述关于 $(c_0, c_1, \ldots, c_{n-1})^{\mathrm{T}}$ 的方程组有唯一解,所以满足条件的多项式函数 $f$ 是唯一存在的.

    \item 令 $A = (\alpha_1, \alpha_2, \ldots, \alpha_n), B = \begin{pmatrix}
                  \alpha_1^{\mathrm{T}}\alpha_1 & \alpha_1^{\mathrm{T}}\alpha_2 & \cdots & \alpha_1^{\mathrm{T}}\alpha_n \\
                  \alpha_2^{\mathrm{T}}\alpha_1 & \alpha_2^{\mathrm{T}}\alpha_2 & \cdots & \alpha_2^{\mathrm{T}}\alpha_n \\
                  \vdots                        & \vdots                        & \ddots & \vdots                        \\
                  \alpha_n^{\mathrm{T}}\alpha_1 & \alpha_n^{\mathrm{T}}\alpha_2 & \cdots & \alpha_n^{\mathrm{T}}\alpha_n \\
              \end{pmatrix}$,显然 $B = A^{\mathrm{T}}A$,由 {11.4 秩不等式第 4 个} 有 $r(B) = r(A)$. 进而 $n$ 维向量组 $(\alpha_1, \alpha_2, \ldots, \alpha_n)$ 线性无关等价于 $r(A) = n$,也就等价于 $r(B) = n$,进而等价于 $\lvert B \rvert \neq 0$,命题得证. % FIXME: xref

    \item $(\implies)$ 记 $A = (\alpha_1, \alpha_2, \ldots, \alpha_n), \varepsilon_i = (0, \ldots, 1, 0, \ldots, 0)^{\mathrm{T}}$(第 $i$ 个为 1),$E = (\varepsilon_1, \varepsilon_2, \ldots, \varepsilon_n)$. 因为 $\alpha_1, \alpha_2, \ldots, \alpha_n$ 是 $n$ 维线性无关的向量组,故 $\lvert A \rvert \neq 0$,即 $A$ 可逆,$A = EA, AA^{-1} = E$,\[(\alpha_1, \alpha_2, \ldots, \alpha_n) = (\varepsilon_1, \varepsilon_2, \ldots, \varepsilon_n)A, (\varepsilon_1, \varepsilon_2, \ldots, \varepsilon_n) = (\alpha_1, \alpha_2, \ldots, \alpha_n)A^{-1}.\] 即 $\alpha_1, \alpha_2, \ldots, \alpha_n$ 与 $\varepsilon_1, \varepsilon_2, \ldots, \varepsilon_n$ 等价. $\varepsilon_1, \varepsilon_2, \ldots, \varepsilon_n$ 为 $\mathbf{R}^n$ 的一个基,能表出任一 $n$ 维向量,故 $\alpha_1, \alpha_2, \ldots, \alpha_n$ 能表出 $\mathbf{R}^n$ 中任一 $n$ 维向量.

          $(\impliedby)$ 若 $\alpha_1, \alpha_2, \ldots, \alpha_n$ 能表出任一 $n$ 维向量,则 $\varepsilon_i = \lambda_{i1}\alpha_1+\cdots+\lambda_{in}\alpha_n, i = 1, 2, \ldots, n$. 即
          \[E = (\varepsilon_1, \varepsilon_2, \ldots, \varepsilon_n) = (\alpha_1, \alpha_2, \ldots, \alpha_n)\begin{pmatrix}
                  \lambda_{11} & \lambda_{21} & \cdots & \lambda_{n1} \\
                  \lambda_{11} & \lambda_{21} & \cdots & \lambda_{n1} \\
                  \vdots       & \vdots       & \ddots & \vdots       \\
                  \lambda_{11} & \lambda_{21} & \cdots & \lambda_{n1} \\
              \end{pmatrix}.\]
          进而 $\lvert \alpha_1, \alpha_2, \ldots, \alpha_n \rvert \begin{vmatrix}
                  \lambda_{11} & \lambda_{21} & \cdots & \lambda_{n1} \\
                  \lambda_{11} & \lambda_{21} & \cdots & \lambda_{n1} \\
                  \vdots       & \vdots       & \ddots & \vdots       \\
                  \lambda_{11} & \lambda_{21} & \cdots & \lambda_{n1} \\
              \end{vmatrix} = 1$,所以 $\lvert \alpha_1, \alpha_2, \ldots, \alpha_n \rvert \neq 0$,$\alpha_1, \alpha_2, \ldots, \alpha_n$ 线性无关.

    \item $A$ 的前 $s$ 列组成的 $s$ 阶子式为Vandermonde行列式
          \[D = \begin{vmatrix}
                  1      & a      & a^2    & \cdots & a^{n-1}    \\
                  1      & a^2    & a^4    & \cdots & a^{2(n-1)} \\
                  \vdots & \vdots & \vdots & \ddots & \vdots     \\
                  1      & a^s    & a^{2s} & \cdots & a^{s(n-1)} \\
              \end{vmatrix}.\]
          由于当 $0 < r < n$ 时,$a^r \neq 1$,因此 $a, a^2, \ldots, a^s$ 两两不同,进而 $D \neq 0$,于是 $r(A) \geqslant s$. 又因为 $A$ 的行数是 $s$,所以 $r(A) \leqslant s$. 从而 $r(A) = s$.

    \item \begin{enumerate}
              \item 线性变换的验证省略.

              \item $(\implies)$ 若 $\lvert AB \rvert = 0$,则 $\lvert A \rvert = 0$ 或 $\lvert B \rvert = 0$,故 $A$ 或 $B$ 不可逆. 不妨假设 $A$ 不可逆,则存在 $X_0 \neq 0$ 使得 $AX_0 = 0$,$T(X_0) = AX_0B = 0$. 但 $T$ 是可逆的,所以 $T$ 是单射,$T(X) = 0 \Leftrightarrow X = 0$,矛盾.

                    $(\impliedby)$ $\lvert AB \rvert \neq 0$ 则 $A, B$ 可逆,故 $T(X) = AXB$ 的逆映射为 $T^{-1}(X) = A^{-1}XB^{-1}$.
          \end{enumerate}

    \item 因为 $r(A) < n, A_{11} \neq 0$,所以 $r(A) = n-1$,进而由 {例13.9 (6)} 可知 $r(A^*) = 1$,所以有 % FIXME: xref
          \[A^* = \begin{pmatrix} a_1 \\ a_2 \\ \vdots \\ a_n \end{pmatrix} (\lambda_1, \lambda_2, \ldots, \lambda_n).\]
          设 $(\lambda_1, \lambda_2, \ldots, \lambda_n) \begin{pmatrix} a_1 \\ a_2 \\ \vdots \\ a_n \end{pmatrix} = k$,则
          \begin{align*}
              (A^*)^2 & = \begin{pmatrix}a_1 \\ a_2 \\ \vdots \\ a_n \end{pmatrix}
              (\lambda_1, \lambda_2, \ldots, \lambda_n)
              \begin{pmatrix} a_1 \\ a_2 \\ \vdots \\ a_n \end{pmatrix}
              (\lambda_1, \lambda_2, \ldots, \lambda_n)                                        \\
                      & = \begin{pmatrix} a_1 \\ a_2 \\ \vdots \\ a_n \end{pmatrix} \cdot k
              \cdot (\lambda_1, \lambda_2, \ldots, \lambda_n) = k
              \begin{pmatrix} a_1 \\ a_2 \\ \vdots \\ a_n \end{pmatrix}
              (\lambda_1, \lambda_2, \ldots, \lambda_n) = kA^*
          \end{align*}

    \item $\forall a_i \in \mathbf{R}, i \in \mathbf{Z}_{+}$ 满足:若 $i \neq j$,则 $a_i \neq a_j$,考虑以Vandermonde行列式的形式进行排布. 设
          \[ A = \begin{pmatrix}
                  1         & 1         & \cdots & 1         & \cdots \\
                  a_1       & a_2       & \cdots & a_k       & \cdots \\
                  a_1^2     & a_2^2     & \cdots & a_k^2     & \cdots \\
                  \vdots    & \vdots    & \ddots & \vdots    &        \\
                  a_1^{n-1} & a_2^{n-1} & \cdots & a_k^{n-1} & \cdots
              \end{pmatrix}\]
          考虑 $A$ 的任意 $n$ 阶主子式 $D_n$,其均构成Vandermonde行列式,又 $i \neq j$ 有 $a_i \neq a_j$,所以值均不为 0,也就是说任意 $n$ 个向量都线性无关,其个数也恰好为 $n$,构成 $V$ 的一组基,命题得证.
\end{enumerate}

\centerline{\heiti C组}
\begin{enumerate}
    \item 对原矩阵进行分块初等变换化为上三角块矩阵后进行计算. 因为 $\lvert A \rvert \neq 0$,所以 $A$ 可逆,进而有如下变换:
          \[\begin{pmatrix}
                  E        & O \\
                  -CA^{-1} & E
              \end{pmatrix} \begin{pmatrix}
                  A & B \\
                  C & D \\
              \end{pmatrix} = \begin{pmatrix}
                  A & B          \\
                  O & D-CA^{-1}B
              \end{pmatrix},\]
          所以
          \begin{align*}
                  & \begin{vmatrix}
                        A & B \\
                        C & D \\
                    \end{vmatrix}
              = \begin{vmatrix}
                    E        & O \\
                    -CA^{-1} & E \\
                \end{vmatrix}
              \begin{vmatrix}
                  A & B \\
                  C & D \\
              \end{vmatrix}       \\
              ={} & \begin{vmatrix}
                        A & B          \\
                        O & D-CA^{-1}B
                    \end{vmatrix}
              = \lvert A \rvert \lvert D-CA^{-1}B \rvert = \lvert AD-ACA^{-1}B \rvert.
          \end{align*} 由于 $AC = CA$,所以有 $ACA^{-1} = CAA^{-1} = C$,所以
          \[\begin{vmatrix}
                  A & B \\
                  C & D \\
              \end{vmatrix} = \lvert AD-CB \rvert.\]

    \item 这道题目我们利用了一个分块矩阵作为中间“桥梁”使得其通过分块初等变换之后能分别得到两个方向上的结果. 考虑矩阵 $\begin{pmatrix}
                  A                   & \alpha \\
                  -\beta^{\mathrm{T}} & 1
              \end{pmatrix}$,有
          \[\begin{pmatrix}
                  A                   & \alpha \\
                  -\beta^{\mathrm{T}} & 1
              \end{pmatrix} \begin{pmatrix}
                  E                  & O \\
                  \beta^{\mathrm{T}} & E
              \end{pmatrix} = \begin{pmatrix}
                  A+\alpha \beta^{\mathrm{T}} & \alpha \\
                  O                           & 1
              \end{pmatrix},\]
          所以
          \[\begin{vmatrix}
                  A                   & \alpha \\
                  -\beta^{\mathrm{T}} & 1
              \end{vmatrix} = \begin{vmatrix}
                  A                   & \alpha \\
                  -\beta^{\mathrm{T}} & 1
              \end{vmatrix} \begin{vmatrix}
                  E                  & O \\
                  \beta^{\mathrm{T}} & E
              \end{vmatrix} = \begin{vmatrix}
                  A+\alpha \beta^{\mathrm{T}} & \alpha \\
                  O                           & 1
              \end{vmatrix} = \lvert A+\alpha \beta^{\mathrm{T}} \rvert.\]
          另一方面,
          \[\begin{pmatrix}
                  E                        & O \\
                  \beta^{\mathrm{T}}A^{-1} & 1
              \end{pmatrix} \begin{pmatrix}
                  A                   & \alpha \\
                  -\beta^{\mathrm{T}} & 1
              \end{pmatrix} = \begin{pmatrix}
                  A & \alpha                           \\
                  O & 1+\beta^{\mathrm{T}}A^{-1}\alpha
              \end{pmatrix}.\]
          注意 $\beta^{\mathrm{T}}A^{-1}\alpha$ 的最终结果是一个数. 进而
          \[\begin{vmatrix}
                  A                   & \alpha \\
                  -\beta^{\mathrm{T}} & 1
              \end{vmatrix} = \begin{vmatrix}
                  E                        & O \\
                  \beta^{\mathrm{T}}A^{-1} & 1
              \end{vmatrix} \begin{vmatrix}
                  A                   & \alpha \\
                  -\beta^{\mathrm{T}} & 1
              \end{vmatrix} = \begin{vmatrix}
                  A & \alpha                           \\
                  O & 1+\beta^{\mathrm{T}}A^{-1}\alpha
              \end{vmatrix} = \lvert A \rvert(1+\beta^{\mathrm{T}}A^{-1}\alpha).\]
          所以有
          \[\lvert A+\alpha \beta^{\mathrm{T}} \rvert = \lvert A \rvert(1+\beta^{\mathrm{T}}A^{-1}\alpha)\]

    \item 依旧是对分块矩阵做初等分块变换. 考虑到
          \begin{align*}
              \left(\begin{pmatrix}
                        E & E \\
                        O & E
                    \end{pmatrix}
              \begin{pmatrix}
                  A & B \\
                  B & A
              \end{pmatrix} \right)
              \begin{pmatrix}
                  E & -E \\
                  O & E
              \end{pmatrix}
               & = \begin{pmatrix}
                       A+B & A+B \\
                       B   & A
                   \end{pmatrix}
              \begin{pmatrix}
                  E & -E \\
                  O & E
              \end{pmatrix}      \\
               & = \begin{pmatrix}
                       A+B & O   \\
                       B   & A-B
                   \end{pmatrix},
          \end{align*}
          所以有 \begin{align*}
              \begin{vmatrix}
                  A & B \\
                  B & A
              \end{vmatrix}
               & = \begin{vmatrix}
                       E & E \\
                       O & E
                   \end{vmatrix}
              \begin{vmatrix}
                  A & B \\
                  B & A
              \end{vmatrix}
              \begin{vmatrix}
                  E & -E \\
                  O & E
              \end{vmatrix}
              = \begin{vmatrix}
                    A+B & O   \\
                    B   & A-B
                \end{vmatrix}    \\
               & =
              \lvert (A+B)(A-B) \rvert = \lvert A+B \rvert \lvert A-B \rvert
          \end{align*}

    \item 首先有 $\begin{vmatrix}
                  \lvert A \rvert & \lvert B \rvert \\
                  \lvert C \rvert & \lvert D \rvert
              \end{vmatrix} = \lvert A \rvert \lvert D \rvert-\lvert B \rvert \lvert C \rvert$.
          \begin{enumerate}
              \item 若 $\lvert A \rvert \neq 0$,即 $A$ 可逆. 因为矩阵的初等变换不改变矩阵的秩,所以由
                    \[\begin{pmatrix}
                            E        & O \\
                            -CA^{-1} & E
                        \end{pmatrix} \begin{pmatrix}
                            A & B \\
                            C & D
                        \end{pmatrix} \begin{pmatrix}
                            E & -A^{-1}B \\
                            O & E        \\
                        \end{pmatrix} = \begin{pmatrix}
                            A & O          \\
                            O & D-CA^{-1}B
                        \end{pmatrix},\]
                    条件 $r\left(\begin{pmatrix}
                                A & B \\
                                C & D
                            \end{pmatrix}\right) = n$ 以及 $A$ 可逆,可以得到
                    \[D-CA^{-1}B = O.\]
                    即若 $A$ 可逆,则 $D = CA^{-1}B$,并且
                    \[\begin{vmatrix}
                            \lvert A \rvert & \lvert B \rvert \\
                            \lvert C \rvert & \lvert D \rvert
                        \end{vmatrix} = \lvert A \rvert \lvert CA^{-1}B \rvert-\lvert B \rvert \lvert C \rvert = \lvert A \rvert \lvert C \rvert \lvert A^{-1} \rvert \lvert B \rvert-\lvert B \rvert \lvert C \rvert = 0.\]

              \item 若 $\lvert A \rvert = 0$,只需证 $\lvert B \rvert \lvert C \rvert = 0$. 若 $\lvert B \rvert \neq 0$,则由
                    \[\begin{pmatrix}
                            E        & O \\
                            -DB^{-1} & E
                        \end{pmatrix} \begin{pmatrix}
                            A & B \\
                            C & D
                        \end{pmatrix} \begin{pmatrix}
                            E        & O \\
                            -B^{-1}A & E
                        \end{pmatrix} = \begin{pmatrix}
                            O          & B \\
                            C-DB^{-1}A & O
                        \end{pmatrix},\]
                    有 $C-DB^{-1}A = O$. 注意到 $\lvert A \rvert = 0$,故 \[\lvert C \rvert = \lvert DB^{-1}A \rvert = \lvert D \rvert \lvert B^{-1} \rvert \lvert A \rvert = 0.\] 同理可证若 $\lvert C \rvert \neq 0$,则 $\lvert B \rvert = 0$.
          \end{enumerate}
          综上,结论成立.

    \item \begin{enumerate}
              \item \label{item:13:B:5:1}
                    采用反证法. 设 $\lvert A \rvert = 0$,则线性方程组 $AX = 0$ 有非零解,设为 $X_0 = (x_1, x_2, \ldots, x_n)^{\mathrm{T}}$,记
                    \[\lvert x_k \rvert = \max \{\lvert x_1 \rvert, \lvert x_2 \rvert, \ldots, \lvert x_n \rvert\}.\]
                    由 $X_0 \neq 0$ 可知 $\lvert x_k \rvert > 0$,考虑 $AX = 0$ 的第 $k$ 个方程,有 $\displaystyle\sum_{j=1}^n a_{kj}x_j = 0$,于是
                    \[\lvert a_{kk} \rvert \lvert x_k \rvert = \lvert -\displaystyle\sum_{j \neq k}a_{kj}x_j \rvert \leqslant \displaystyle\sum_{j \neq k}\lvert a_{kj} \rvert \lvert x_k \rvert.\]
                    约去 $\lvert x_k \rvert$ 后可得 $\lvert a_{kk} \rvert \leqslant \displaystyle\sum_{j \neq k} \lvert a_{kj} \rvert$,这与条件矛盾. 所以 $\lvert A \rvert \neq 0$.

              \item 构造实函数
                    \[f(t) = \begin{vmatrix}
                            a_{11}  & ta_{12} & ta_{13} & \cdots & ta_{1n} \\
                            ta_{21} & a_{22}  & ta_{23} & \cdots & ta_{2n} \\
                            ta_{31} & ta_{32} & a_{33}  & \cdots & ta_{3n} \\
                            \vdots  & \vdots  & \vdots  & \ddots & \vdots  \\
                            ta_{n1} & ta_{n2} & ta_{n3} & \cdots & a_{nn}  \\
                        \end{vmatrix}\]
                    由于 $A$ 是实矩阵,所以 $f(t)$ 是关于 $t$ 的一个实系数多项式(连续)函数,同时
                    \[f(0) = a_{11}a_{22}\cdots a_{nn} > 0.\] 当$t \in [0, 1]$ 时,还有
                    \[a_{ii} > \sum_{j \neq i} \lvert a_{ij} \rvert \leqslant \sum_{j \neq i} \lvert ta_{ij} \rvert.\]
                    由 \ref*{item:13:B:5:1} 可知 $f(t)$ 在 $[0, 1]$ 上非零,由连续函数的介值定理可知 $f(1) > 0$,即 $\lvert A \rvert > 0$.

              \item 此为直接推论不再赘述.
          \end{enumerate}

    \item \begin{enumerate}
              \item 设 $\begin{pmatrix}
                            A & C \\
                            O & B
                        \end{pmatrix}$ 的伴随矩阵为 $\begin{pmatrix}
                            X & Y \\
                            Z & W
                        \end{pmatrix}$. 而 $\begin{vmatrix}
                            A & C \\
                            O & B
                        \end{vmatrix} = \lvert A\rvert \lvert B \rvert$,所以有
                    \[\begin{pmatrix}
                            A & C \\
                            O & B
                        \end{pmatrix} \begin{pmatrix}
                            X & Y \\
                            Z & W
                        \end{pmatrix} = \lvert A\rvert \lvert B \rvert \begin{pmatrix}
                            E & O \\
                            O & E
                        \end{pmatrix}.\]
                    得到方程组
                    \[\begin{cases}
                            AX+CZ = \lvert A \rvert \lvert B \rvert E \\
                            AY+CW = O                                 \\
                            BZ    = O                                 \\
                            BW    = \lvert A \rvert \lvert B \rvert E
                        \end{cases}\]
                    考虑一般情况,我们不再单独讨论 $B$ 是否等于 $O$. 所以 $Z = O$, $X = \lvert B \rvert A^*$, $W = \lvert A \rvert B^*$, $Y = -A^*CB^*$. 即 \[\begin{pmatrix}
                            A & C \\
                            O & B
                        \end{pmatrix}^* = \begin{pmatrix}
                            \lvert B \rvert A^* & -A^*CB^*            \\
                            O                   & \lvert A \rvert B^*
                        \end{pmatrix}\]

              \item 若 $A$ 可逆,则可以通过以下的初等分块变换将其化为上三角块矩阵.
                    \[\begin{pmatrix}
                            E        & O \\
                            -CA^{-1} & E
                        \end{pmatrix} \begin{pmatrix}
                            A & B \\
                            C & D
                        \end{pmatrix} = \begin{pmatrix}
                            A & B          \\
                            O & D-CA^{-1}B
                        \end{pmatrix}.\]
                    两侧取伴随有
                    \begin{align*}
                            & \left(\begin{pmatrix}
                                        E        & O \\
                                        -CA^{-1} & E
                                    \end{pmatrix}
                        \begin{pmatrix}
                            A & B \\
                            C & D
                        \end{pmatrix}\right)^*
                        = \begin{pmatrix}
                              A & B \\
                              C & D
                          \end{pmatrix}^*
                        \begin{pmatrix}
                            E        & O \\
                            -CA^{-1} & E
                        \end{pmatrix}^*            \\
                        ={} & \begin{pmatrix}
                                  A & B          \\
                                  O & D-CA^{-1}B
                              \end{pmatrix}^*
                        = \begin{pmatrix}
                              \lvert D-CA^{-1}B \rvert A^* & -A^*B(D-CA^{-1}B)^*            \\
                              O                            & \lvert A \rvert (D-CA^{-1}B)^*
                          \end{pmatrix}
                    \end{align*}
                    而
                    \[\begin{pmatrix}
                            E        & O \\
                            -CA^{-1} & E
                        \end{pmatrix}^* \begin{pmatrix}
                            E        & O \\
                            -CA^{-1} & E
                        \end{pmatrix} = \begin{vmatrix}
                            E        & O \\
                            -CA^{-1} & E
                        \end{vmatrix} \begin{pmatrix}
                            E & O \\
                            O & E
                        \end{pmatrix} = \begin{pmatrix}
                            E & O \\
                            O & E
                        \end{pmatrix}\]
                    所以
                    \begin{align*}
                            & \begin{pmatrix}
                                  A & B \\
                                  C & D
                              \end{pmatrix}^*                                                                         \\
                        ={} & \begin{pmatrix}
                                  \lvert D-CA^{-1}B \rvert A^* & -A^*B(D-CA^{-1}B)^*            \\
                                  O                            & \lvert A \rvert (D-CA^{-1}B)^*
                              \end{pmatrix}
                        \begin{pmatrix}
                            E        & O \\
                            -CA^{-1} & E
                        \end{pmatrix}                                                                                \\
                        ={} & \begin{pmatrix}
                                  \lvert D-CA^{-1}B \rvert A^*+A^*B(D-CA^{-1}B)^*CA^{-1} & -A^*B(D-CA^{-1}B)^*            \\
                                  -\lvert A \rvert (D-CA^{-1}B)^*CA^{-1}                 & \lvert A \rvert (D-CA^{-1}B)^*
                              \end{pmatrix} &
                    \end{align*}
          \end{enumerate}

    \item \begin{enumerate}
              \item 因为 $n=2$ 时 $(A^*)^* = A$,所以 $B = (B^*)^* = A^*$,而 $B$ 的伴随矩阵是唯一的,所以存在唯一的2阶方阵 $A = B^*$ 使得 $A^* = B$.

              \item $(\impliedby)$ 由 {例13.9(6)} 可得. % FIXME: xref

                    $(\implies)$ \begin{enumerate}
                        \item $r(B) = n$ 时,若存在 $A$ 使得 $A^* = B$,则由 $(A^*)^* = \lvert A \rvert^{n-2}A$,有
                              \[A = \dfrac{1}{\lvert A \rvert^{n-2}}(A^*)^* = \dfrac{1}{\lvert A \rvert^{n-2}}B^* = \dfrac{1}{\lvert A \rvert^{n-2}}\lvert B \rvert B^{-1},\]
                              而
                              \[\lvert B \rvert = \lvert A^* \rvert = \lvert A \rvert^{n-1}, d\]
                              代入上式可得
                              \[A = \lvert A \rvert B^{-1} = \sqrt[n-1]{\lvert B \rvert} B^{-1}\]
                              从而满足 $A^* = B$ 的矩阵 $A$ 存在,且有 $n-1$ 个.

                        \item $r(B) = 1$ 时,存在可逆矩阵 $P, Q$ 使得
                              \[B = P\begin{pmatrix}
                                      1 & O \\
                                      O & O \\
                                  \end{pmatrix}Q.\]
                              若存在 $A$ 满足 $A^{*} = B$,则 $r(A) = n-1$,从而存在可逆矩阵 $G, H$ 使得
                              \[A = G\begin{pmatrix}
                                      0 & O       \\
                                      O & E_{n-1}
                                  \end{pmatrix}H,\]
                              则
                              \[A^* = H^*\begin{pmatrix}
                                      0 & O       \\
                                      O & E_{n-1}
                                  \end{pmatrix}^*G^* = H^*\begin{pmatrix}
                                      1 & O \\
                                      O & O \\
                                  \end{pmatrix}G^* = \lvert HG \rvert H^{-1}\begin{pmatrix}
                                      1 & O \\
                                      O & O \\
                                  \end{pmatrix}G^{-1},\]
                              由 $A^* = B$ 可得
                              \[\lvert HG \rvert H^{-1}\begin{pmatrix}
                                      1 & O \\
                                      O & O \\
                                  \end{pmatrix}G^{-1} = P\begin{pmatrix}
                                      1 & O \\
                                      O & O \\
                                  \end{pmatrix}Q,\]
                              即
                              \[\lvert HG \rvert\begin{pmatrix}
                                      1 & O \\
                                      O & O \\
                                  \end{pmatrix} = HP\begin{pmatrix}
                                      1 & O \\
                                      O & O \\
                                  \end{pmatrix}QG,\]
                              记 $C = HP$, $D = QG$,且分块为 $C = \begin{pmatrix}
                                      C_{11} & C_{12} \\
                                      C_{21} & C_{22}
                                  \end{pmatrix}$, $D = \begin{pmatrix}
                                      D_{11} & D_{12} \\
                                      D_{21} & D_{22}
                                  \end{pmatrix}$,其中 $C_{22}, D_{22}$ 是 $n-1$ 阶矩阵,则
                              \[\lvert HG \rvert\begin{pmatrix}
                                      1 & O \\
                                      O & O \\
                                  \end{pmatrix} = \begin{pmatrix}
                                      C_{11} & C_{12} \\
                                      C_{21} & C_{22}
                                  \end{pmatrix} \begin{pmatrix}
                                      1 & O \\
                                      O & O \\
                                  \end{pmatrix} \begin{pmatrix}
                                      D_{11} & D_{12} \\
                                      D_{21} & D_{22}
                                  \end{pmatrix} = \begin{pmatrix}
                                      C_{11}D_{11} & C_{11}D_{12} \\
                                      C_{21}D_{11} & C_{21}D_{12}
                                  \end{pmatrix},\]
                              于是
                              \[\lvert HG \rvert = C_{11}D_{11}, C_{11}D_{12} = O, C_{21}D_{11} = O, C_{21}D_{12} = O.\]
                              因为 $H, G$ 可逆,所以 $C_{11} \neq 0, D_{11} \neq 0$,于是 $C_{21} = O = D_{12}$. 从而
                              \begin{align*}
                                  A & = G\begin{pmatrix}
                                             0 & O       \\
                                             O & E_{n-1}
                                         \end{pmatrix}H
                                  = Q^{-1}D\begin{pmatrix}
                                               0 & O       \\
                                               O & E_{n-1}
                                           \end{pmatrix}CP^{-1}      \\
                                    & = Q^{-1}\begin{pmatrix}
                                                  D_{11} & O      \\
                                                  D_{21} & D_{22}
                                              \end{pmatrix}
                                  \begin{pmatrix}
                                      0 & O       \\
                                      O & E_{n-1}
                                  \end{pmatrix}
                                  \begin{pmatrix}
                                      C_{11} & C_{12} \\
                                      O      & C_{22}
                                  \end{pmatrix} P^{-1}               \\
                                    & = Q^{-1} \begin{pmatrix}
                                                   0 & O            \\
                                                   O & D_{22}C_{22}
                                               \end{pmatrix} P^{-1}.
                              \end{align*}
                              又
                              \[C_{11}D_{11} = \lvert HG \rvert = \lvert CP^{-1}Q^{-1}D \rvert = \dfrac{1}{\lvert PQ \rvert}\lvert DC \rvert.\]
                              接下来转为求 $\lvert DC \rvert$. 而
                              \[DC = \begin{pmatrix}
                                      D_{11} & O      \\
                                      D_{21} & D_{22}
                                  \end{pmatrix} \begin{pmatrix}
                                      C_{11} & C_{12} \\
                                      O      & C_{22}
                                  \end{pmatrix} = \begin{pmatrix}
                                      D_{11}C_{11} & D_{11}C_{12}              \\
                                      D_{21}C_{11} & D_{21}C_{12}+D_{22}C_{22}
                                  \end{pmatrix}.\]
                              考虑初等分块变换
                              \begin{align*}
                                  \begin{pmatrix}
                                      1                  & O       \\
                                      -D_{21}D_{11}^{-1} & E_{n-1}
                                  \end{pmatrix}DC
                                   & = \begin{pmatrix}
                                           1                  & O       \\
                                           -D_{21}D_{11}^{-1} & E_{n-1}
                                       \end{pmatrix}
                                  \begin{pmatrix}
                                      D_{11}C_{11} & D_{11}C_{12}              \\
                                      D_{21}C_{11} & D_{21}C_{12}+D_{22}C_{22}
                                  \end{pmatrix} \\
                                   & = \begin{pmatrix}
                                           D_{11}C_{11} & D_{11}C_{12} \\
                                           O            & D_{22}C_{22}
                                       \end{pmatrix},
                              \end{align*}
                              故
                              \[C_{11}D_{11} = \dfrac{1}{\lvert PQ \rvert}\lvert DC \rvert = \dfrac{1}{\lvert PQ \rvert}D_{11}C_{11} \lvert D_{22}C_{22} \rvert,\]
                              从而
                              \[\dfrac{1}{\lvert PQ \rvert} \lvert D_{22}C_{22} \rvert = 1,\]
                              即
                              \[\lvert D_{22}C_{22} \rvert = \lvert PQ \rvert.\]
                              命题得证.

                        \item $r(B) = 0$ 则是平凡情况,其是所有 $r \leqslant n-2$ 矩阵的伴随矩阵.
                    \end{enumerate}

              \item 由 $r(B^*) = r(A) = 1$ 可知 $r(B) = 2$. $B^*B = \lvert B \rvert E = 0$,由此可知 $B$ 的列向量为方程组 $B^*X = 0$ 的解,其基础解系为
                    \[\alpha_1 = (-1, 1, 0)^{\mathrm{T}}, \alpha_2 = (-1, 0, 1)^{\mathrm{T}}.\]
                    令 $B = (\alpha_1, \alpha_2, \alpha_3)$,其中 $\alpha_3 = k_1\alpha_1+k_2\alpha_2 = (k_1+k_2, -k_1, -k_2)^{\mathrm{T}}$. 由 $BB^* = 0$ 解得 $k_1 = k_2 = 1$,从而
                    \[B = \begin{pmatrix}
                            -1 & -1 & 2  \\
                            1  & 0  & -1 \\
                            0  & 1  & -1
                        \end{pmatrix}.\]
          \end{enumerate}
\end{enumerate}

\clearpage
