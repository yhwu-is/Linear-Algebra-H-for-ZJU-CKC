\phantomsection
\section*{7 线性映射矩阵表示(I)}
\addcontentsline{toc}{section}{7 线性映射矩阵表示(I)}

\vspace{2ex}

\centerline{\heiti A组}
\begin{enumerate}
    \item 教材 P154/9,此处略.
    \item 教材 P154/10,此处略.
\end{enumerate}

\centerline{\heiti B组}
\begin{enumerate}
    \item $ T(\beta_1, \beta_2, \ldots, \beta_n) = (\beta_1, \beta_2, \ldots, \beta_n)A $,其中
          \[ A = \begin{pmatrix}
                    &   &        &   & a_1    \\
                  1 &   &        &   & a_2    \\
                    & 1 &        &   & a_3    \\
                    &   & \ddots &   & \vdots \\
                    &   &        & 1 & a_n
              \end{pmatrix} \]
          是 $ T $ 关于基 $ B $ 的表示矩阵.

          $ T $ 是同构 $ \iff T $ 是双射 $ \iff r(T) = n $,满秩. 所以当 $ a_1 \neq 0 $ 时,$ r(T) = n $ 满秩,此时 $ T $ 是同构映射.

    \item \begin{enumerate}
              \item

              \item 方法一:设 $ (\sigma(f_1), \sigma(f_2), \sigma(f_3)) = (f_1, f_2, f_3) A $,则由
                    \[ (\sigma(f_1), \sigma(f_2), \sigma(f_3)) = (1, x, x^2) \begin{pmatrix}
                            2 & 0 & 1 \\
                            0 & 1 & 1 \\
                            2 & 0 & 1
                        \end{pmatrix} = (1, x, x^2) \begin{pmatrix}
                            1  & 1 & 0 \\
                            -1 & 0 & 1 \\
                            0  & 1 & 2
                        \end{pmatrix} A \]
                    可得
                    \begin{align*}
                        A & = \begin{pmatrix}
                                  1  & 1 & 0 \\
                                  -1 & 0 & 1 \\
                                  0  & 1 & 2
                              \end{pmatrix}^{-1}
                        \begin{pmatrix}
                            2 & 0 & 1 \\
                            0 & 1 & 1 \\
                            1 & 0 & 1
                        \end{pmatrix}          \\
                          & = \begin{pmatrix}
                                  -1 & -2 & 1  \\
                                  2  & 2  & -1 \\
                                  -1 & -1 & 1
                              \end{pmatrix}
                        \begin{pmatrix}
                            2 & 0 & 1 \\
                            0 & 1 & 1 \\
                            1 & 0 & 1
                        \end{pmatrix}
                        = \begin{pmatrix}
                              -1 & -2 & -2 \\
                              3  & 2  & 3  \\
                              -1 & -1 & -1
                          \end{pmatrix}
                    \end{align*}

                    方法二:通过待定系数法解方程组
                    \[ \begin{cases}
                            \sigma(f_1) = -f_1 + 3 f_2 - f_3   \\
                            \sigma(f_2) = -2 f_1 + 2 f_2 - f_3 \\
                            \sigma(f_3) = -2 f_1 + 3 f_2 - f_3
                        \end{cases} \]
                    解得
                    \[ (\sigma(f_1), \sigma(f_2), \sigma(f_3)) = (f_1, f_2, f_3) \begin{pmatrix}
                            -1 & -2 & -2 \\
                            3  & 2  & 3  \\
                            -1 & -1 & -1
                        \end{pmatrix} \]

              \item \begin{align*}
                        \sigma(f) & = \sigma\left((1, x, x^2)
                        \begin{pmatrix} 1 \\ 2 \\ 3 \end{pmatrix}\right)= \sigma\left((f_1, f_2, f_3)
                        \begin{pmatrix}
                            1  & 1 & 0 \\
                            -1 & 0 & 1 \\
                            0  & 1 & 2
                        \end{pmatrix}^{-1} \begin{pmatrix} 1 \\ 2 \\ 3 \end{pmatrix}\right) \\
                                  & = (\sigma(f_1), \sigma(f_2), \sigma(f_3))
                        \begin{pmatrix}
                            1  & 1 & 0 \\
                            -1 & 0 & 1 \\
                            0  & 1 & 2
                        \end{pmatrix}^{-1} \begin{pmatrix} 1 \\ 2 \\ 3 \end{pmatrix}        \\
                                  & = (1, x, x^2)
                        \begin{pmatrix}
                            2 & 0 & 1 \\
                            0 & 1 & 1 \\
                            1 & 0 & 1
                        \end{pmatrix}
                        \begin{pmatrix}
                            1  & 1 & 0 \\
                            -1 & 0 & 1 \\
                            0  & 1 & 2
                        \end{pmatrix}^{-1}
                        \begin{pmatrix} 1 \\ 2 \\ 3 \end{pmatrix}                           \\
                                  & = (1, x, x^2)
                        \begin{pmatrix} -4 \\ 3 \\ 2 \end{pmatrix}                          \\
                                  & = 2x^2 + 3x - 4
                    \end{align*}

                    或也可采用待定系数法求出 $ f = -2 f_1 + 3 f_2 $,所以 $ \sigma(f) = -2 \sigma(f_1) + 3 \sigma(f_2) = 2x^2 + 3x - 4 $.
          \end{enumerate}

    \item \begin{enumerate}
              \item $ \forall X, Y \in \mathbf{M}_2(\mathbf{R}),\enspace k_1, k_2 \in \mathbf{R} $,有
                    \begin{align*}
                        \varphi(k_1 X + k_2 Y) & = A(k_1 X + k_2 Y)B               \\
                                               & = k_1 A X B + k_2 A Y B           \\
                                               & = k_1 \varphi(X) + k_2 \varphi(Y)
                    \end{align*}
                    所以 $ \varphi $ 是 $ \mathbf{M}_2(\mathbf{R}) $ 上的线性映射.

              \item 证明:易知 $ B $ 可逆. 当 $ \lambda = -1 $ 时,$ A $ 可逆. 故 $ \varphi $ 可逆,$ \varphi^{-1}(X) = A^{-1} X B^{-1} $.

              \item $ \lambda = -1 $ 时,取 $ V $ 的一组基 $ \alpha_1 = \begin{pmatrix} 1 & 0 \\ 0 & 0 \end{pmatrix}, \alpha_2 = \begin{pmatrix} 0 & 1 \\ 0 & 0 \end{pmatrix}, \alpha_3 = \begin{pmatrix} 0 & 0 \\ 1 & 0 \end{pmatrix}, \alpha_4 = \begin{pmatrix} 0 & 0 \\ 0 & 1 \end{pmatrix} $,有
                    \begin{gather*}
                        \begin{aligned}
                            \im \sigma & = \spa(\varphi(\alpha_1), \varphi(\alpha_2), \varphi(\alpha_3), \varphi(\alpha_4))                                \\
                                       & = \spa\left(\begin{pmatrix} 1 & 2 \\ -1 & -2 \end{pmatrix}, \begin{pmatrix} -1 & -1 \\ 1 & 1 \end{pmatrix}\right)
                        \end{aligned} \\
                        \ker \sigma = \spa\left(\begin{pmatrix} 2 & -3 \\ 0 & 1 \end{pmatrix}, \begin{pmatrix} 1 & 0 \\ 1 & 0 \end{pmatrix}\right)
                    \end{gather*}
                    (步骤略,答案不唯一)

              \item 取 $ \begin{pmatrix} 1 & 2 \\ -1 & -2 \end{pmatrix}, \begin{pmatrix} -1 & -1 \\ 1 & 1 \end{pmatrix}, \alpha_3, \alpha_4 $ 即可. 此时矩阵为$ \begin{pmatrix}
                            2 & -2 & -1 & 0 \\
                            4 & -2 & 0  & 1 \\
                            0 & 1  & 0  & 0 \\
                            0 & 0  & 0  & 0
                        \end{pmatrix} $. (答案不唯一)
          \end{enumerate}

    \item 我们可以类比线性空间
          \[ W = \{ x \in \mathbf{R}^4 \mid x_1 + x_2 + x_3 = 0 \} \]
          $ W $ 的基只需求解线性方程组 $ x_1 + x_2 + x_3 = 0 $ 即可,得到基础解系为 $ (-1, 0, 1, 0)^{\mathrm{T}},\allowbreak (-1, 1, 0, 0)^{\mathrm{T}},\allowbreak (0, 0, 0, 1)^{\mathrm{T}} $.

          换回 $ V $,即有基为 $ A_1 = \begin{pmatrix} -1 & 0 \\ 1 & 0 \end{pmatrix}, A_2 = \begin{pmatrix} -1 & 1 \\ 0 & 0 \end{pmatrix}, A_3 = \begin{pmatrix} 0 & 0 \\ 0 & 1 \end{pmatrix} $. 而
          \begin{align*}
                         & \sigma(A_1) = \begin{pmatrix} -2 & 1 \\ 1 & 0 \end{pmatrix} = A_1 + A_2                      \\
                         & \sigma(A_2) = \begin{pmatrix} -2 & 1 \\ 1 & 0 \end{pmatrix} = A_1 + A_2                      \\
                         & \sigma(A_3) = \begin{pmatrix} -2 & 1 \\ 1 & 0 \end{pmatrix} = 2 A_3                          \\
              \implies{} & \sigma(A_1 + A_2) = 2(A_1 + A_2),\enspace \sigma(A_1 - A_2) = 0,\enspace \sigma(A_3) = 2 A_3
          \end{align*}
          取基 $ A_1 - A_2, A_1 + A_2, A_3 $,有
          \[ (\sigma(A_1 - A_2), \sigma(A_1 + A_2), \sigma(A_3)) = (A_1 - A_2, A_1 + A_2, A_3) = \begin{pmatrix}
                  0 & 0 & 0 \\
                  0 & 2 & 0 \\
                  0 & 0 & 2
              \end{pmatrix} \]
          为对角矩阵.

    \item \begin{enumerate}
              \item \label{item:7:B:5:1}
                    求核空间,即求使得 $ T(f(x)) $ 为零矩阵的 $ f(x) $ 构成的空间. 设 $ f(x) = ax^3 + bx^2 + cx + d $,则有
                    \[ \begin{cases}
                            f(0) = d = 0             \\
                            f(1) = a + b + c + d = 0 \\
                            f(-1) = -a + b -c + d = 0
                        \end{cases} \]
                    解方程,令 $ a = t $ 有
                    \[ \begin{cases}
                            a = t  \\
                            b = 0  \\
                            c = -t \\
                            d = 0
                        \end{cases} \implies f(x) = t(x^3 - x) \]
                    故 $ N(T) = \spa(x^3 - x) $.

                    求像空间,取 $ \mathbf{R}[x]_4 $ 的自然基 $ 1, x, x^2, x^3 $.
                    \[ \begin{matrix}
                            T(1) = \begin{pmatrix} 1 & 1 \\ 1 & 1 \end{pmatrix}   & T(x) = \begin{pmatrix} 0 & 1 \\ -1 & 0 \end{pmatrix}   \\
                            T(x^2) = \begin{pmatrix} 0 & 1 \\ 1 & 0 \end{pmatrix} & T(x^3) = \begin{pmatrix} 0 & 1 \\ -1 & 0 \end{pmatrix}
                        \end{matrix} \]
                    求它们的极大线性无关组. 我们发现 $ T(x) = T(x^3) $,故先舍弃 $ T(x^3) $,然后令
                    \begin{align*}
                                   & k_1 T(1) + k_2 T(x) + k_3 T(x^2) = 0                                             \\
                        \implies{} & \begin{pmatrix} k_1 & k_1 + k_2 + k_3 \\ k_1 - k_2 + k_3 & k_1 \end{pmatrix} = 0 \\
                        \implies{} & \begin{cases}
                                         k_1 = 0             \\
                                         k_1 + k_2 + k_3 = 0 \\
                                         k_1 - k_2 + k_3 = 0 \\
                                     \end{cases}                                           \\
                        \implies{} & k_1 = k_2 = k_3 = 0
                    \end{align*}
                    故 $ T(1), T(x), T(x^2) $ 线性无关. 故 $ R(T) = \spa\left(\begin{pmatrix} 1 & 1 \\ 1 & 1 \end{pmatrix}, \begin{pmatrix} 0 & 1 \\ -1 & 0 \end{pmatrix}, \begin{pmatrix} 0 & 1 \\ 1 & 0 \end{pmatrix}\right) $.

              \item 由 \ref*{item:7:B:5:1},$ \dim N(T) = 1 $,$ \dim R(T) = 3 $. 又有 $ \dim \mathbf{R}[x]_4 = 4 $,所以成立
                    \[ \dim \mathbf{R}[x]_4 = \dim N(T) + \dim R(T) \]
          \end{enumerate}

    \item \begin{enumerate}
              \item
                    对非齐次线性方程组$AX=\xi_1$,\\
                    $\bar{A}=\begin{pmatrix}
                            1  & -1 & -1 & -1 \\
                            -1 & 1  & 1  & 1  \\
                            0  & -4 & -2 & -2
                        \end{pmatrix}
                        \rightarrow\begin{pmatrix}
                            1 & -1 & -1       & -1       \\
                            0 & 1  & \frac 12 & \frac 12 \\
                            0 & 0  & 0        & 0
                        \end{pmatrix}
                        \rightarrow\begin{pmatrix}
                            1 & 0 & -\frac 12 & -\frac 12 \\
                            0 & 1 & \frac 12  & \frac 12  \\
                            0 & 0 & 0         & 0
                        \end{pmatrix}$,则\\
                    $\xi_2=C_1\begin{pmatrix}
                            \frac 12  \\
                            -\frac 12 \\
                            1
                        \end{pmatrix} + \begin{pmatrix}
                            -\frac 12 \\
                            \frac 12  \\
                            0
                        \end{pmatrix}=\dfrac 12\begin{pmatrix}
                            C_1 - 1  \\
                            -C_1 + 1 \\
                            2C_1
                        \end{pmatrix}$(其中$C_1$为任意常数).\\
                    $A^2=\begin{pmatrix}
                            2  & 2  & 0 \\
                            -2 & -2 & 0 \\
                            4  & 4  & 0
                        \end{pmatrix}$,对齐次线性方程组$A^2X=\xi_1$,\\
                    $\bar{B}=\begin{pmatrix}
                            A^2 & \xi_1
                        \end{pmatrix}=\begin{pmatrix}
                            2  & 2  & 0 & 1  \\
                            -2 & -2 & 0 & 1  \\
                            4  & 4  & 0 & -2
                        \end{pmatrix}
                        \rightarrow\begin{pmatrix}
                            1 & 1 & 0 & -\frac 12 \\
                            0 & 0 & 0 & 0         \\
                            0 & 0 & 0 & 0
                        \end{pmatrix}$,\\
                    则$A^2X=\xi_1$的通解\\
                    $\xi_3=C_2\begin{pmatrix}
                            -1 \\
                            1  \\
                            0
                        \end{pmatrix}+C_3\begin{pmatrix}
                            0 \\
                            0 \\
                            1
                        \end{pmatrix}+\begin{pmatrix}
                            -\frac 12 \\
                            0         \\
                            0
                        \end{pmatrix}=\begin{pmatrix}
                            -C_2 - \frac 12 \\
                            C_2             \\
                            C_3
                        \end{pmatrix}$(其中$C_2, C_3$为任意常数).
              \item
                    因为$|\xi_1,\xi_2,\xi_3|=\dfrac 12\begin{vmatrix}
                            -1 & C_1 - 1  & -C_2 - \frac 12 \\
                            1  & -C_1 + 1 & C_2             \\
                            -2 & 2C_1     & C_3
                        \end{vmatrix}=-\dfrac 12\neq 0$,\\
                    所以$\xi_1,\xi_2,\xi_3$线性无关.
          \end{enumerate}

    % \item \begin{enumerate}
    %           \item 对新的一组基,使用过渡矩阵进行表达如下:
    %                 \[ (\beta_1, \beta_2, \beta_3) = (\alpha_1, \alpha_2, \alpha_3) \begin{pmatrix}
    %                         2 & 1 & -1 \\
    %                         1 & 1 & 1  \\
    %                         3 & 2 & 1
    %                     \end{pmatrix} = (\alpha_1, \alpha_2, \alpha_3) C \]
    %                 其中 $ C $ 是可逆矩阵,且
    %                 \[ (\alpha_1, \alpha_2, \alpha_3) = (\beta_1, \beta_2, \beta_3) C^{-1} \]
    %                 将上式代入已知条件得
    %                 \[ \sigma\left((\beta_1, \beta_2, \beta_3) C^{-1}\right) = \left((\beta_1, \beta_2, \beta_3) C^{-1}\right) A \]
    %                 容易验证(只需利用线性变换和矩阵的等价性然后利用矩阵乘法结合律即可)上式左端等于 $ \left(\sigma(\beta_1, \beta_2, \beta_3)\right) C^{-1} $,所以
    %                 \[ \left(\sigma(\beta_1, \beta_2, \beta_3)\right) C^{-1} = (\beta_1, \beta_2, \beta_3)(C^{-1} A) \]
    %                 从而得 $ \sigma(\beta_1, \beta_2, \beta_3) = (\beta_1, \beta_2, \beta_3)(C^{-1}AC) $,故 $ \sigma $ 关于基 $ \beta_1, \beta_2, \beta_3 $ 下对应的矩阵 $ B $ 为
    %                 \[ B = C^{-1}AC = \begin{pmatrix}
    %                         2 & 1 & -1 \\
    %                         1 & 1 & 1  \\
    %                         3 & 2 & 1
    %                     \end{pmatrix}^{-1} \begin{pmatrix}
    %                         1 & 2 & -1 \\
    %                         2 & 1 & 0  \\
    %                         3 & 0 & 1
    %                     \end{pmatrix} \begin{pmatrix}
    %                         2 & 1 & -1 \\
    %                         1 & 1 & 1  \\
    %                         3 & 2 & 1
    %                     \end{pmatrix} = \begin{pmatrix}
    %                         2 & 0 & 1  \\
    %                         0 & 2 & 1  \\
    %                         3 & 1 & -1
    %                     \end{pmatrix} \]

    %           \item $ \sigma $ 的值域是 $ A $ 的列向量组的极大线性无关组,由于 $ A $ 的第 1 列可以由第 2 列和第 3 列线性表示,从而 $ \sigma(V) = \spa(2\alpha_1 + \alpha_2, -\alpha_1 + \alpha_3) $.

    %                 $ \ker \sigma $ 是线性方程组 $ AX = \vec{0} $ 的解空间,从而 $ \ker \sigma = \spa(\alpha_1 - 2\alpha_2 - 3\alpha_3) $.

    %           \item 由于 $ \alpha_1 $ 不能由 $ 2\alpha_1 + \alpha_2 $ 和 $ -\alpha_1 + \alpha_3 $ 线性表示,可以把 $ \sigma(V) $ 的基扩充为 $ V $ 的基 $ \alpha_1, 2\alpha_1 + \alpha_2, -\alpha_1 + \alpha_3 $. $ \sigma $ 在这个基下对应的矩阵是
    %                 \[ \begin{pmatrix}
    %                         1 & 2 & -1 \\
    %                         0 & 1 & 0  \\
    %                         0 & 0 & 1
    %                     \end{pmatrix}^{-1} \begin{pmatrix}
    %                         1 & 2 & -1 \\
    %                         2 & 1 & 0  \\
    %                         3 & 0 & 1
    %                     \end{pmatrix} \begin{pmatrix}
    %                         1 & 2 & -1 \\
    %                         0 & 1 & 0  \\
    %                         0 & 0 & 1
    %                     \end{pmatrix} = \begin{pmatrix}
    %                         0 & 0 & 0  \\
    %                         2 & 5 & -2 \\
    %                         3 & 6 & -2
    %                     \end{pmatrix} \]

    %           \item 由于 $ \alpha_1, \alpha_2 $ 不能由 $ \alpha_1 - 2\alpha_2 - 3\alpha_3 $ 线性表示,可以把 $ \ker \sigma $ 的基扩充为 $ V $ 的基 $\alpha_1, \alpha_2, \alpha_1 - 2\alpha_2 - 3\alpha_3 $. $ \sigma $ 在这个基下对应的矩阵是
    %                 \[ \begin{pmatrix}
    %                         1 & 0 & 1  \\
    %                         0 & 1 & -2 \\
    %                         0 & 0 & -3
    %                     \end{pmatrix}^{-1} \begin{pmatrix}
    %                         1 & 2 & -1 \\
    %                         2 & 1 & 0  \\
    %                         3 & 0 & 1
    %                     \end{pmatrix} \begin{pmatrix}
    %                         1 & 0 & 1  \\
    %                         0 & 1 & -2 \\
    %                         0 & 0 & -3
    %                     \end{pmatrix} = \begin{pmatrix}
    %                         2 & 2 & 0 \\
    %                         0 & 1 & 0 \\
    %                         1 & 0 & 0
    %                     \end{pmatrix} \]
    %       \end{enumerate}

    \item \begin{enumerate}
              \item 因为 $A^k=\begin{pmatrix}\lambda_1^k & 0 \\ 0 & \lambda_2^k\end{pmatrix}$,所以 $f(A)=a_mA^m+a_{m-1}A^{m-1}+\cdots+a_1A+a_0E = \begin{pmatrix}f(\lambda_1) & 0 \\ 0 & f(\lambda_2)\end{pmatrix}$.
              \item $A=PBP^{-1}$,则 $A^2=(PBP^{-1})(PBP^{-1})=PB^2P^{-1}$. 由归纳法得 $A^k=PB^kP^{-1}$,于是
                    \begin{align*}
                        f(A) & = a_mPB^mP^{-1}+a_{m-1}PB^{m-1}P^{-1}+\cdots+a_0                         \\
                             & =P\begin{pmatrix}f(\lambda_1) & 0 \\ 0 & f(\lambda_2)\end{pmatrix}P^{-1} \\
                             & =Pf(B)P^{-1}
                    \end{align*}
          \end{enumerate}
\end{enumerate}

\centerline{\heiti C组}
\begin{enumerate}
    \item
\end{enumerate}

\clearpage
