\section*{2013-2014学年线性代数I(H)期末}
\addcontentsline{toc}{section}{2013-2014学年线性代数I(H)期末}

\begin{center}
    任课老师:统一命卷\hspace{4em} 考试时长:120分钟
\end{center}

\begin{enumerate}
    \item [一、](10分)映射 $T:\mathbf{R^3} \to \mathbf{R^2}$ 由 $T(x_1,x_2,x_3)=(2x_1+3x_2-x_3,x_1+x_3)$ 定义.
    \begin{enumerate}[label=(\arabic*)]
        \item 证明 $T$ 是线性映射.
        \item 给出 $T$ 关于 $\mathbf{R^3}$ 和 $\mathbf{R^2}$ 的标准基的矩阵表示.
        \item 给出 $T$ 的核 $\mathrm{Ker}(T)$ 的一组基.
    \end{enumerate}
    \item [二、](10分)记 $V=\{f:\mathbf{R}\to \mathbf{R}|f$ 可导 $\}$,即 $V$ 是由实数导自身的全体可导函数所构成的集合.
    \begin{enumerate}[label=(\arabic*)]
        \item 试给出 $V$ 上加法和数乘运算,使 $V$ 成为实线性空间,并写出 $V$ 中的零向量.
        \item 记 $S=\{f_1,f_2,f_3\}$,其中
        \[f_1(x)=x;f_2(x)=\sin x;f_3(x)=e^x,\forall x \in \mathbf{R}.\]
        证明 $S$ 是 $V$ 的线性无关子集.
    \end{enumerate}
    \item [三、](10分)设 $(\ ,\ )$ 是实线性空间 $V$ 的内积,$B=\{v_1,v_2,\cdots,v_n\}$ 是 $V$ 的一组单位正交基.
    \begin{enumerate}[label=(\arabic*)]
        \item 证明:对任意的 $x,y \in V$,有 $(x,y) = \sum\limits_{i=1}^n(x,v_i)(y,v_i).$
        \item 设 $x=\sum\limits_{i=1}^n(-1)^iv_i$ 和 $y=\sum\limits_{i=1}^nv_i$ 的夹角是 $\theta$,求 $\cos \theta$,并指出 $x$ 和 $y$ 是否正交.
    \end{enumerate}
    \item [四、](10分)设 $\alpha_1,\alpha_2,\alpha_3,\beta_1,\beta_2$ 都是 $4$ 维实线性空间 $\mathbf{R^4}$ 的列向量,已知 $4$ 阶行列式
    $\lvert (\alpha_1,\alpha_2,\alpha_3,\beta_1) \rvert=d_1,\lvert (\beta_2,\alpha_3,\alpha_2,\alpha_1) \rvert = d_2.$ 求下面 $4$ 阶方阵的行列式.
    \begin{enumerate}[label=(\arabic*)]
        \item $A=(3\alpha_1-100\alpha_2,7\alpha_2,\alpha_3,\beta_1).$
        \item $B=(5\beta_1+6\beta_2,\alpha_1,\alpha_2,\alpha_3).$
    \end{enumerate}
    \item [五、](10分)设域 $\mathbf{F}$ 上 $n$ 维线性空间 $V$ 的非零向量都是线性映射 $T:V\to V$ 的特征向量.
    \begin{enumerate}[label=(\arabic*)]
        \item 证明 $T$ 是数乘映射,即存在 $\lambda \in \mathbf{F}$,使得对于任何 $v \in V$,有 $T(v)=\lambda v.$
        \item 给出 $T$ 的秩和零度 ($T$ 的零度 $=\mathrm{dim}(\mathrm{Ker}(T))$).
    \end{enumerate}
    \item [六、](10分)令 $A=\begin{pmatrix}0 & -1 & -1 \\ 1 & 1 & 0 \\ 1 & x & y\end{pmatrix}$,其特征多项式 $f(\lambda) = \lambda^3+\lambda+2.$
    \begin{enumerate}[label=(\arabic*)]
        \item 求 $x$ 和 $y$ 的值.
        \item 若将 $A$ 看作实矩阵,$A$ 是否可对角化?为什么?
        \item 若将 $A$ 看作复矩阵,$A$ 是否可对角化?为什么?
    \end{enumerate}
    \item [七、](10分)设 $A$ 和 $B$ 分别是 $m\times n$ 和 $n\times m$ 矩阵.
    \begin{enumerate}[label=(\arabic*)]
        \item 设 $\lambda \neq 0$. 证明 $\lambda$ 是 $m\times m$ 矩阵 $AB$ 的特征值当且仅当 $\lambda$ 是 $n\times n$ 矩阵 $BA$ 的特征值.
        \item 证明 $I_m-AB$ 是可逆矩阵当且仅当 $I_n-BA$ 是可逆矩阵($I_m$ 是 $m$ 阶单位矩阵,$I_n$ 是 $n$ 阶单位矩阵).
    \end{enumerate}
    \item [八、](10分)设实二次型 $f(x_1,x_2,x_3) = 2x_1^2+x_2^2-4x_1x_2-4x_2x_3.$
    \begin{enumerate}[label=(\arabic*)]
        \item 求实对称矩阵 $A$,使 $f(x_1,x_2,x_3)=(x_1,x_2,x_3)A(x_1,x_2,x_3)^{\mathbf{T}}.$
        \item 求可逆矩阵 $P$,使 $P^\mathbf{T}AP$ 是 $A$ 的相合规范形.
        \item 给出 $f$ 的正惯性指数和负惯性指数,并指出 $f$ 是否正定或负定.
    \end{enumerate}
    \item [九、](20分)判断下面命题的真伪. 若它是真命题,给出一个简单证明;若它是伪命题,举一个具体的反例将它否定.
    \begin{enumerate}[label=(\arabic*)]
        \item 域 $\mathbf{F}$ 上所有 $n$ 阶不可逆方阵所构成的集合是 $n$ 阶矩阵空间 $M_n(\mathbf{F})$ 的子空间.
        \item 对称矩阵 $A$ 的伴随矩阵 $A^*$ 也是对称矩阵.
        \item 设 $A$ 是 $m\times n$ 矩阵,$I_m$ 是 $m$ 阶单位阵,$B=(A|I_m)$ 是 $A$ 的增广矩阵,则 $B$ 的秩 $r(B)=m.$
        \item 若 $\mathrm{dim}(V) = n,\mathrm{dim}(W) = m$,且 $n < m$,则任何线性映射 $T:V\to W$ 都不可能是满射.
    \end{enumerate}
\end{enumerate}

\newpage
