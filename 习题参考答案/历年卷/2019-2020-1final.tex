\section*{2019-2020学年线性代数I(H)期末}
\addcontentsline{toc}{section}{2019-2020学年线性代数I(H)期末}

\begin{center}
    任课老师:统一命卷\hspace{4em} 考试时长:120分钟
\end{center}

\begin{enumerate}
	\item[一、](10分)设 $D=|a_{ij}|=\begin{vmatrix}3 & 6 & 9 & 12 \\ 2 & 4 & 6 & 8 \\1 & 2 & 0 & 3 \\ 5 & 6 & 4 & 3\end{vmatrix}$,求$A_{41}+2A_{42}+3A_{44}$,这里 $A_{ij}$ 是元素 $a_{ij}$ 的代数余子式.
	\item[二、](10分)设 $A \in M_{m \times s}(\mathbf{R})$,且 $r(A)=r$,证明:存在矩阵 $B \in M_{s \times n}(\mathbf{R})$,且 $r(B)=\min\{s-r,\ n\}$,使得 $AB=0$.
	\item[三、](10分)设 $\alpha$ 为 $\mathbf{R}^3$ 中的非零向量,$\sigma(x)=(x,\ \alpha)\alpha$,这里 $(\cdot\ ,\ \cdot)$ 是 $\mathbf{R}^3$ 的标准内积.
    \begin{enumerate}[label=(\arabic*)]
        \item 证明:$\sigma$ 为 $\mathbf{R}^3$ 上的线性变换,并求其像空间;
        \item 设 $\alpha=(1,\ 0,\ -2)$,分别求 $\sigma$ 在基 $\mathbf{B}_1=\{(1,\ 0,\ 0),\ (0,\ 1,\ 0),\ (0,\ 0,\ 1)\}$ 和 $\mathbf{B}_2=\{(1,\ 1,\ 1),\ (1,\ 1,\ 0),\ (1,\ 0,\ 0)\}$ 下的矩阵.
    \end{enumerate}
	\item[四、](10分)
    \begin{enumerate}[label=(\arabic*)]
        \item 设 $A$ 为 $n$ 阶矩阵且 $E-A$ 可逆,证明:$A$ 与 $(E-A)^{-1}$ 相乘可交换;
        \item 设 $A$ 为 $n$ 阶实反对称矩阵且 $E+A$ 可逆,证明:$(E-A)(E+A)^{-1}$ 为正交矩阵,且 $-1$ 不为其特征值.
    \end{enumerate}
	\item[五、](10分)已知 $\alpha_1,\alpha_2,\ldots,\alpha_s$ 是齐次线性方程组 $AX=\vec{0}$ 的一组基础解系,向量组
    \[\beta_1=t_1\alpha_1+t_2\alpha_2,\ \beta_2=t_1\alpha_2+t_2\alpha_3,\ \ldots,\ \beta_{s-1}=t_1\alpha_{s-1}+t_2\alpha_s\]
    试问当实数 $t_1,t_2$ 满足何条件时,$AX=\vec{0}$ 有基础解系包含向量 $\beta_1,\beta_2,\ldots,\beta_{s-1}$,并写出该基础解系中的其余向量.
	\item[六、](15分)已知二次型 $X^{\mathrm{T}}AX=ax_1^2+ax_2^2+ax_3^2+2x_1x_2+2x_1x_3-2x_2x_3$ 的秩为 $2$.
    \begin{enumerate}[label=(\arabic*)]
        \item 求实数 $a$ 的值;
        \item 用正交变换 $X=QY$ 将 $X^{\mathrm{T}}AX$ 化为标准形,给出 $Q$,并求二次型的正、负惯性指数.
    \end{enumerate}
	\item[七、](15分)记 $X=\Bigg\{\begin{pmatrix}a_{11} & a_{12} & a_{13} \\ a_{21} & a_{22} & a_{23} \\ a_{31} & a_{32} & a_{33}\end{pmatrix} \in M_{3 \times 3}(\mathrm{R}) \Bigg| \sum\limits_{j=1}^3a_{1j}=\sum\limits_{j=1}^3a_{2j}=\sum\limits_{j=1}^3a_{3j}=\sum\limits_{j=1}^3a_{jj}\Bigg\}$,证明:
    \begin{enumerate}[label=(\arabic*)]
        \item $X$ 是 $M_{3 \times 3}(\mathrm{R})$ 的一个子空间,并求该子空间的维数;
        \item 对任意可逆矩阵 $A \in X$,$(1,\ 1,\ 1)^{\mathrm{T}}$ 是 $A$ 和 $A^{-1}$ 的特征向量;
        \item 对任意可逆矩阵 $A \in X$,$A^{-1} \in X$.
    \end{enumerate}
	\item[八、](20分)判断下列命题的真伪,若它是真命题,请给出简单的证明;若它是伪命题,给出理由或举反例将它否定.
    \begin{enumerate}[label=(\arabic*)]
        \item 设 $A_1,\ A_2,\ \dots,\ A_{n+1}$ 是任意 $n+1$ 个 $n$ 阶矩阵,必存在不全为 $0$ 的实数 $\lambda_1,\ \lambda_2,\ \dots,\ \lambda_{n+1}$,使得矩阵 $\lambda_1A_1+\lambda_2A_2+\dots+\lambda_{n+1}A_{n+1}$ 不可逆;
        \item 复数集 $\mathbf{C}$ 关于复数的加法与复数的乘法构成的复数域上的线性空间与 $\mathbf{C}^2$ 同构;
        \item 设 $x \in \mathbf{R}^n$,对任意 $\lambda \in \mathbf{R}$,$E+\lambda xx^{\mathrm{T}}$ 为正定矩阵;
        \item 若 $A,\ B$ 为 $n$ 阶上三角矩阵,且对角线上元素都相同,则 $A$ 与 $ B$ 相似.
    \end{enumerate}
\end{enumerate}

\clearpage
