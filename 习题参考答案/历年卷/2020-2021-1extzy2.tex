\section*{2020-2021学年线性代数I(H)小测2}
\addcontentsline{toc}{section}{2020-2021学年线性代数I(H)小测2(谈之奕老师)}

\begin{center}
    任课老师:谈之奕\hspace{4em} 考试时长:90分钟
\end{center}

\begin{enumerate}
	\item[一、](10分)称实矩阵$A=(a_{ij})$是整数矩阵,如果每个$a_{ij}$都是整数.设$M$是整数矩阵,且可逆(作为实矩阵).证明:$M$的逆矩阵也是整数矩阵的充要条件是$M$的行列式等于$\pm 1$.
	\item[二、](15分)设$B=\{v_1,v_2,v_3\}$是线性空间$V$的一组基,线性映射$\sigma:V\to V$定义如下:
	\[\sigma(v_1)=v_2+v_3,\sigma(v_2)=v_3,\sigma(v_3)=v_1-v_2.\]
    \begin{enumerate}[label=(\arabic*)]
        \item 求$\sigma$在基$B$下的矩阵;
        \item 证明:$B'=\{v_2,v_3+v_1,v_1-v_2\}$是$V$的另一组基;
        \item 求$\sigma$在基$B'$下的矩阵.
    \end{enumerate}
	\item[三、](15分)设
	\[P=\begin{pmatrix}
        1 & 1 & 0 \\ 0 & 1 & 0 \\ 0 & 0 & 0
    \end{pmatrix},\enspace Q=\begin{pmatrix}
        0 & 0 \\ 1 & 0
    \end{pmatrix},\]
    定义$\mathbf{R}^{3\times 2}$上映射$\sigma$:
    \[\sigma(A)=PAQ.\]
    \begin{enumerate}[label=(\arabic*)]
        \item 验证$\sigma$是线性映射;
        \item 求$\im\sigma$和$\ker\sigma$;
        \item 验证关于$\sigma$的维数公式.
    \end{enumerate}
	\item[四、](15分)求参数 $a,\ b$  的值,使得 $\begin{vmatrix}1 & 1 & 1 \\ x & y & z \\u & v & w\end{vmatrix}=1,\ \begin{vmatrix}1 & 2 & -5 \\ x & y & z \\u & v & w\end{vmatrix}=2,\ \begin{vmatrix}2 & 3 & b \\ x & y & z \\u & v & w\end{vmatrix}=a$ 都成立,并求$\begin{vmatrix}x & y & z \\ 1 & -1 & 5 \\u & v & w\end{vmatrix}$.
	\item[五、](15分)设在$\mathbf{F}[x]_3$中有两组基:
	\[(A)\alpha_1=1-x,\alpha_2=-x+x^2,\alpha_3=3x-2x^2;\]
    \[(B)\beta_1=4x+5x^2,\beta_2=-1,\beta_3=3x+4x^2.\]
    \begin{enumerate}[label=(\arabic*)]
        \item 求基$(A)$到基$(B)$的过渡矩阵;
        \item 设$\alpha$在基$(A)$下的坐标为$(1,1,-1)^{\mathrm{T}}$,求$\alpha$在基$(B)$下的坐标.
    \end{enumerate}
	\item[六、](10分)设 $A \in M_{m \times n}(\mathbf{F})$,$r(A)=r$,$k$ 是满足条件 $r \leq k \leq n$ 的任意整数,证明存在 $n$ 阶方阵 $B$,使得 $AB=0$,且 $r(A)+r(B)=k$.
	\item[七、](20分)判断下列命题的真伪,若它是真命题,请给出简单的证明;若它是伪命题,给出理由或举反例将它否定.
	\begin{enumerate}[label=(\arabic*)]
        \item 域$\mathbf{F}$上的全体$n$阶可逆矩阵构成$M_n(\mathbf{F})$的一个子空间;
        \item 设$A$和$B$都是可逆矩阵,则矩阵$\begin{pmatrix}
            O & B \\ A & C
        \end{pmatrix}$也是可逆矩阵;
        \item 可逆矩阵$A$的伴随矩阵$A^*$的行列式等于1;
        \item 若对于任何正整数$n$,方阵$A$(阶数大于1)的$n$次乘积$A^n$都是非零方阵,则$A$可逆.
    \end{enumerate}
\end{enumerate}

\newpage
