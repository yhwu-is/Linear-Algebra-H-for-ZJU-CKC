\section*{2021-2022学年线性代数I(H)小测}
\addcontentsline{toc}{section}{2020-2021学年线性代数I(H)小测(刘康生老师)}

\begin{center}
    任课老师:刘康生\hspace{4em} 考试时长:90分钟
\end{center}
\begin{enumerate}
	\item[一、] 设矩阵$A=\begin{pmatrix}
        a & -1 & 1 \\ -1 & a & -1 \\ 1 & -1 & a
    \end{pmatrix}$,$\beta=\begin{pmatrix}
        0 \\ 1 \\ 1
    \end{pmatrix}$.假设线性方程组$Ax=\beta$有解但解不唯一.
    \begin{enumerate}[label=(\arabic*)]
        \item 求$a$的值;
        \item 给出$Ax=\beta$的一般解.
    \end{enumerate}
	\item[二、]设
	\[P=\begin{pmatrix}
        1 & 1 & 0 \\ 0 & 1 & 0 \\ 0 & 0 & 0
    \end{pmatrix},\enspace Q=\begin{pmatrix}
        0 & 0 \\ 1 & 0
    \end{pmatrix},\]
    定义$\mathbf{R}^{3\times 2}$上映射$\sigma$:
    \[\sigma(A)=PAQ.\]
    \begin{enumerate}[label=(\arabic*)]
        \item 验证$\sigma$是线性映射;
        \item 求$\im\sigma$和$\ker\sigma$;
        \item 验证关于$\sigma$的维数公式.
    \end{enumerate}
	\item[三、]设$B$是$3\times 1$矩阵,$C$是$1\times 3$矩阵,证明:$r(BC)\leqslant 1$;反之,若$A$是秩为1的$3\times 3$矩阵,证明:存在$3\times 1$矩阵$B$和$1\times 3$矩阵$C$,使得$A=BC$.
	\item[四、]设$B=\{\beta_1,\beta_2,\cdots,\beta_n\}$是实数域$\mathbf{R}$上线性空间$V$的一组基,$T\in\mathcal{L}(V)$,$T(\beta_1)=\beta_2$,$T(\beta_2)=\beta_3$,$\cdots$,$T(\beta_{n-1})=\beta_n$,$T(\beta_n)=\sum\limits_{i=1}^{n}a_i\beta_i(a_i\in\mathbf{R})$.求$T$在$B$下的表示矩阵.在什么条件下$T$是同构映射?
	\item[五、]设$A^*$是$n$阶方阵$A$的伴随矩阵,求$A^*$的秩.
	\item[六、]判断下列命题的真伪,若它是真命题,请给出简单的证明;若它是伪命题,给出理由或举反例将它否定.
	\begin{enumerate}[label=(\arabic*)]
        \item 给定线性空间$V$的非零向量$v$和线性空间$W$的向量$w$,总存在线性映射$T:V\to W$,使得$T(v)=w$;
        \item 若线性方程组有$m$个方程,$n$个变量,且$m<n$,则这个方程组一定有非零解;
        \item 若方阵$A^3=0$,则$E+A$和$E-A$都是可逆矩阵;
        \item 若方阵$A^2=A$,则$E+A$和$E-A$都是可逆矩阵.
    \end{enumerate}
\end{enumerate}

\clearpage
