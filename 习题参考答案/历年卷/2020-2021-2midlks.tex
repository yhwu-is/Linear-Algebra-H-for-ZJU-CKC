\phantomsection
\section*{2020-2021学年线性代数II(H)期中}
\addcontentsline{toc}{section}{2020-2021学年线性代数II(H)期中(刘康生老师)}

\begin{center}
    任课老师:刘康生\hspace{4em} 考试时长:60分钟
\end{center}

\begin{enumerate}
	\item[一、]设$V=\mathbf{R}^{2\times 2}$,$W=\mathbf{R}^{3\times 2}$,$T\in\mathcal{L}(V,W)$由下面的矩阵乘法定义:
	\[T(A)=\begin{pmatrix}
        1 & 1 \\ 1 & 1 \\ 0 & 0
    \end{pmatrix}A,\enspace \forall A\in V.\]
    \begin{enumerate}[label=(\arabic*)]
        \item 求$T$的像空间与核空间;

        \item 求$V$和$W$的一组基,使得$T$在这两组基下的矩阵为$\begin{pmatrix}
            E_r & O \\ O & O
        \end{pmatrix}\in\mathbf{R}^{6\times 4}$,其中$E_r$为$r$阶单位矩阵,$r=\dim\im T$.
    \end{enumerate}
	\item[二、]设$V=\mathbf{R}^3$,$U=\{(x_1,x_2,x_3)\in V\mid x_1+x_2+x_3=0\}$,$\alpha_1=(1,1,1)$,求$f\in V'$使得
	\[f(\alpha_1)=1,f(\alpha)=0,\forall\alpha\in U.\]
	\item[三、]设$A\in\mathbf{R}^{n\times n}$满足$A^2=A$,证明:存在可逆矩阵$P$使得
	\[P^{-1}AP=\begin{pmatrix}
        E_r & O \\ O & O
    \end{pmatrix}\in\mathbf{R}^{n\times n},\]
    其中$r$为$A$的秩.
\end{enumerate}

\clearpage
