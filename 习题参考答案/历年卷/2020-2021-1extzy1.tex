\phantomsection
\section*{2020-2021学年线性代数I(H)小测1}
\addcontentsline{toc}{section}{2020-2021学年线性代数I(H)小测1(谈之奕老师)}

\begin{center}
    任课老师:谈之奕\hspace{4em} 考试时长:90分钟
\end{center}

\begin{enumerate}
	\item[一、](10分)求全部实数$a$,使线性方程组$\begin{cases}
        3x_1+2x_2+x_3=2 \\ x_1-x_2-2x_3=-3 \\ ax_1-2x_2+2x_3=6
    \end{cases}$的解集非空.
	\item[二、](10分)设 $\mathbf{R}^4$ 是 $4$ 维欧氏空间(标准内积),$\alpha=(1,1,1,1),\ \beta=(-1,-1,0,2),\ \gamma=(1,-1,0,0) \in \mathbf{R}^4$,求:
    \begin{enumerate}[label=(\arabic*)]
        \item 与 $\alpha,\ \beta,\ \gamma$ 都正交的一个单位向量 $\delta$;
        \item $||\alpha+\beta+\gamma+\delta||$.
    \end{enumerate}
	\item[三、](15分)记线性映射$\sigma$的核为$\ker\sigma$,像为$\im\sigma$.设$\sigma_1,\sigma_2:V\to V$是线性映射,证明:
	\begin{enumerate}[label=(\arabic*)]
        \item $\ker\sigma_1\subseteq\ker(\sigma_2\circ\sigma_1)$;
        \item $\im(\sigma_2\circ\sigma_1)\subseteq\im\sigma_2$.
    \end{enumerate}
	\item[四、](15分)设$\alpha_1,\alpha_2,\alpha_3$是线性空间$V$的一组基,$T\in\mathcal{L}(V)$,且$T(\alpha_1)=\alpha_1+\alpha_2$,$T(\alpha_2)=\alpha_1-\alpha_2$,$T(\alpha_3)=\alpha_1+2\alpha_2$.求$T$的像空间和核空间,以及$T$的秩.
	\item[五、](15分)设 $W$ 是线性方程组 $\begin{cases}
        x_1-x_2+4x_3-x_4=0 \\ x_1+x_2-2x_3+3x_4=0
    \end{cases}$ 的解空间,求 $W$ 的一组单位正交基,并将其扩充成 $\mathbf{R}^4$ 的单位正交基,这里 $\mathbf{R}$ 是实数域.
	\item[六、](15分)设 $\mathbf{R}[x]_4$ 是数域 $\mathbf{R}$ 上次数小于 $4$ 的多项式所构成的线性空间(约定零多项式次数为 $-\infty$)。$M_2(\mathbf{R})$ 是 $\mathbf{R}$ 上 $2$ 阶方阵所构成的线性空间,定义 $T : \mathbf{R}[x]_4 \to M_2(\mathbf{R})$ 如下,对 $f(x) \in \mathbf{R}[x]_4$,
    \[T(f(x))=\begin{pmatrix}f(0) & f(1) \\ f(-1) & f(0)\end{pmatrix}\]
    \begin{enumerate}
        \item 求出 $T$ 的核空间 $N(T)$ 和像空间 $R(T)$;
        \item 验证关于 $T$ 的维数公式.
    \end{enumerate}
	\item[七、](20分)判断下列命题的真伪,若它是真命题,请给出简单的证明;若它是伪命题,给出理由或举反例将它否定.
	\begin{enumerate}[label=(\arabic*)]
        \item 若$S$是线性空间$V$的线性相关子集,则$S$的每个向量都是$S$的其他向量的线性组合;
        \item 若线性映射$T:V\to W$的核是$K$,则$\dim V=\dim W+\dim K$;
        \item 线性空间$V$的任何子空间$W$都是某个映射$T:V\to V$的核;
        \item 在5维欧式空间$V$中,存在两组线性无关向量$S_1=\{v_1,v_2,v_3\}$和$S_2=\{w_1,w_2,w_3\}$,使其满足内积$\langle v_i,w_j\rangle=0$,其中$i,j=1,2,3$.
    \end{enumerate}
\end{enumerate}

\clearpage
